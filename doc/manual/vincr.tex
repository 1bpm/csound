\begin{comment}
\documentclass[10pt]{article}
\usepackage{fullpage, graphicx, url}
\setlength{\parskip}{1ex}
\setlength{\parindent}{0ex}
\title{vincr}
\begin{document}


\begin{tabular}{ccc}
The Alternative Csound Reference Manual & & \\
Previous & &Next

\end{tabular}

%\hline 
\end{comment}
\section{vincr}
vincr�--� Accumulates audio signals. \subsection*{Description}


 \emph{vincr}
 increments an audio variable of another signal, i.e. accumulates output. 
\subsection*{Syntax}


 \textbf{vincr}
 asig, aincr
\subsection*{Performance}


 \emph{asig}
 -- audio variable to be incremented 


 \emph{aincr}
 -- incrementing signal 


 \emph{vincr}
 (variable increment) and \emph{clear}
 are intended to be used together. \emph{vincr}
 stores the result of the sum of two audio variables into the first variable itself (which is intended to be used as an accumulator in polyphony). The accumulator variable can be used for output signal by means of \emph{fout}
 opcode. After the disk writing operation, the accumulator variable should be set to zero by means of \emph{clear}
 opcode (or it will explode). 
\subsection*{Examples}


  See the \emph{fout}
 opcode for an example. 
\subsection*{See Also}


 \emph{clear}

\subsection*{Credits}


 


 


\begin{tabular}{ccc}
Author: Gabriel Maldonado &Italy &1999

\end{tabular}



 


 New in Csound version 3.56
%\hline 


\begin{comment}
\begin{tabular}{lcr}
Previous &Home &Next \\
vibrato &Up &vlowres

\end{tabular}


\end{document}
\end{comment}
