\begin{comment}
\documentclass[10pt]{article}
\usepackage{fullpage, graphicx, url}
\setlength{\parskip}{1ex}
\setlength{\parindent}{0ex}
\title{follow}
\begin{document}


\begin{tabular}{ccc}
The Alternative Csound Reference Manual & & \\
Previous & &Next

\end{tabular}

%\hline 
\end{comment}
\section{follow}
follow�--� Envelope follower unit generator. \subsection*{Description}


  Envelope follower unit generator. 
\subsection*{Syntax}


 ar \textbf{follow}
 asig, idt
\subsection*{Initialization}


 \emph{idt}
 -- This is the period, in seconds, that the average amplitude of \emph{asig}
 is reported. If the frequency of \emph{asig}
 is low then \emph{idt}
 must be large (more than half the period of \emph{asig}
 ) 
\subsection*{Performance}


 \emph{asig}
 -- This is the signal from which to extract the envelope. 
\subsection*{Examples}


  Here is an example of the follow opcode. It uses the files \emph{follow.orc}
, \emph{follow.sco}
, and \emph{beats.wav}
. 


 \textbf{Example 1. Example of the follow opcode.}

\begin{lstlisting}
/* follow.orc */
; Initialize the global variables.
sr = 44100
kr = 4410
ksmps = 10
nchnls = 1

; Instrument #1 - play a WAV file.
instr 1
  a1 soundin "beats.wav"
  out a1
endin

; Instrument #2 - have another waveform follow the WAV file.
instr 2
  ; Follow the WAV file.
  as soundin "beats.wav"
  af follow as, 0.01

  ; Use a sine waveform.
  as oscil 4000, 440, 1
  ; Have it use the amplitude of the followed WAV file.
  a1 balance as, af

  out a1
endin
/* follow.orc */
        
\end{lstlisting}
\begin{lstlisting}
/* follow.sco */
; Just generate a nice, ordinary sine wave.
f 1 0 32768 10 1

; Play Instrument #1 for two seconds.
i 1 0 2
; Play Instrument #2 for two seconds.
i 2 2 2
e
/* follow.sco */
        
\end{lstlisting}


  To avoid zipper noise, by discontinuities produced from complex envelope tracking, a lowpass filter could be used, to smooth the estimated envelope. 
\subsection*{Credits}


 


 


\begin{tabular}{ccc}
Author: Paris Smaragdis &MIT, Cambridge &1995

\end{tabular}



 
%\hline 


\begin{comment}
\begin{tabular}{lcr}
Previous &Home &Next \\
fold &Up &follow2

\end{tabular}


\end{document}
\end{comment}
