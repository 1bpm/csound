\begin{comment}
\documentclass[10pt]{article}
\usepackage{fullpage, graphicx, url}
\setlength{\parskip}{1ex}
\setlength{\parindent}{0ex}
\title{zkcl}
\begin{document}


\begin{tabular}{ccc}
The Alternative Csound Reference Manual & & \\
Previous & &Next

\end{tabular}

%\hline 
\end{comment}
\section{zkcl}
zkcl�--� Clears one or more variables in the zk space. \subsection*{Description}


  Clears one or more variables in the zk space. 
\subsection*{Syntax}


 \textbf{zkcl}
 kfirst, klast
\subsection*{Performance}


 \emph{ksig}
 -- the input signal 


 \emph{kfirst}
 -- first zk or za location in the range to clear. 


 \emph{klast}
 -- last zk or za location in the range to clear. 


 \emph{zkcl}
 clears one or more variables in the zk space. This is useful for those variables which are used as accumulators for mixing k-rate signals at each cycle, but which must be cleared before the next set of calculations. 
\subsection*{Examples}


  Here is an example of the zkcl opcode. It uses the files \emph{zkcl.orc}
 and \emph{zkcl.sco}
. 


 \textbf{Example 1. Example of the zkcl opcode.}

\begin{lstlisting}
/* zkcl.orc */
; Initialize the global variables.
sr = 44100
kr = 4410
ksmps = 10
nchnls = 1

; Initialize the ZAK space.
; Create 1 a-rate variable and 1 k-rate variable.
zakinit 1, 1

; Instrument #1 -- a simple waveform.
instr 1
  ; Linearly vary a k-rate signal from 220 to 1760.
  kline line 220, p3, 1760

  ; Add the linear signal to zk variable #1.
  zkw kline, 1
endin

; Instrument #2 -- generates audio output.
instr 2
  ; Read zk variable #1.
  kfreq zkr 1

  ; Use the value of zk variable #1 to vary 
  ; the frequency of a sine waveform.
  a1 oscil 20000, kfreq, 1

  ; Generate the audio output.
  out a1

  ; Clear the zk variables, get them ready for 
  ; another pass.
  zkcl 0, 1
endin
/* zkcl.orc */
        
\end{lstlisting}
\begin{lstlisting}
/* zkcl.sco */
; Table #1, a sine wave.
f 1 0 16384 10 1

; Play Instrument #1 for three seconds.
i 1 0 3
; Play Instrument #2 for three seconds.
i 2 0 3
e
/* zkcl.sco */
        
\end{lstlisting}
\subsection*{See Also}


 \emph{zacl}
, \emph{zkwm}
, \emph{zkw}
, \emph{zkmod}
, \emph{zkr}

\subsection*{Credits}


 


 


\begin{tabular}{ccc}
Author: Robin Whittle &Australia &May 1997

\end{tabular}



 


 Example written by Kevin Conder.
%\hline 


\begin{comment}
\begin{tabular}{lcr}
Previous &Home &Next \\
ziwm &Up &zkmod

\end{tabular}


\end{document}
\end{comment}
