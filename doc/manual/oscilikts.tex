\begin{comment}
\documentclass[10pt]{article}
\usepackage{fullpage, graphicx, url}
\setlength{\parskip}{1ex}
\setlength{\parindent}{0ex}
\title{oscilikts}
\begin{document}


\begin{tabular}{ccc}
The Alternative Csound Reference Manual & & \\
Previous & &Next

\end{tabular}

%\hline 
\end{comment}
\section{oscilikts}
oscilikts�--� A linearly interpolated oscillator with sync status that allows changing the table number at k-rate. \subsection*{Description}


 \emph{oscilikts}
 is the same as \emph{oscilikt}
. Except it has a sync input that can be used to re-initialize the oscillator to a k-rate phase value. It is slower than \emph{oscilikt}
 and \emph{osciliktp}
. 
\subsection*{Syntax}


 ar \textbf{oscilikts}
 xamp, xcps, kfn, async, kphs [, istor]
\subsection*{Initialization}


 \emph{istor}
 (optional, defaults to 0) -- skip initialization. 
\subsection*{Performance}


 \emph{xamp}
 -- amplitude. 


 \emph{xcps}
 -- frequency in Hz. Zero and negative values are allowed. However, the absolute value must be less than \emph{sr}
 (and recommended to be less than sr/2). 


 \emph{kfn}
 -- function table number. Can be varied at control rate (useful to ``morph'' waveforms, or select from a set of band-limited tables generated by \emph{GEN30}
). 


 \emph{async}
 -- any positive value resets the phase of \emph{oscilikts}
 to \emph{kphs}
. Zero or negative values have no effect. 


 \emph{kphs}
 -- sets the phase, initially and when it is re-initialized with async. 
\subsection*{Examples}


  Here is an example of the oscilikts opcode. It uses the files \emph{oscilikts.orc}
 and \emph{oscilikts.sco}
. 


 \textbf{Example 1. Example of the oscilikts opcode.}

\begin{lstlisting}
/* oscilikts.orc */
; Initialize the global variables.
sr = 44100
kr = 4410
ksmps = 10
nchnls = 1

; Instrument #1: oscilikts example.
instr 1
  ; Frequency envelope.
  kfrq expon 400, p3, 1200
  ; Phase.
  kphs line 0.1, p3, 0.9

  ; Sync 1
  atmp1 phasor 100
  ; Sync 2
  atmp2 phasor 150
  async diff 1 - (atmp1 + atmp2)

  a1 oscilikts 14000, kfrq, 1, async, 0
  a2 oscilikts 14000, kfrq, 1, async, -kphs

  out a1 - a2
endin
/* oscilikts.orc */
        
\end{lstlisting}
\begin{lstlisting}
/* oscilikts.sco */
; Table #1: Sawtooth wave
f 1 0 3 -2 1 0 -1

; Play Instrument #1 for four seconds.
i 1 0 4
e
/* oscilikts.sco */
        
\end{lstlisting}
\subsection*{See Also}


 \emph{oscilikt}
 and \emph{osciliktp}
. 
\subsection*{Credits}


 Author: Istvan Varga


 New in version 4.22
%\hline 


\begin{comment}
\begin{tabular}{lcr}
Previous &Home &Next \\
osciliktp &Up &osciln

\end{tabular}


\end{document}
\end{comment}
