\begin{comment}
\documentclass[10pt]{article}
\usepackage{fullpage, graphicx, url}
\setlength{\parskip}{1ex}
\setlength{\parindent}{0ex}
\title{stix}
\begin{document}


\begin{tabular}{ccc}
The Alternative Csound Reference Manual & & \\
Previous & &Next

\end{tabular}

%\hline 
\end{comment}
\section{stix}
stix�--� Semi-physical model of a stick sound. \subsection*{Description}


 \emph{stix}
 is a semi-physical model of a stick sound. It is one of the PhISEM percussion opcodes. PhISEM (Physically Informed Stochastic Event Modeling) is an algorithmic approach for simulating collisions of multiple independent sound producing objects. 
\subsection*{Syntax}


 ar \textbf{stix}
 iamp, idettack [, inum] [, idamp] [, imaxshake]
\subsection*{Initialization}


 \emph{iamp}
 -- Amplitude of output. Note: As these instruments are stochastic, this is only a approximation. 


 \emph{idettack}
 -- period of time over which all sound is stopped 


 \emph{inum}
 (optional) -- The number of beads, teeth, bells, timbrels, etc. If zero, the default value is 30. 


 \emph{idamp}
 (optional) -- the damping factor, as part of this equation: 


 damping\_amount�=�0.998�+�(idamp�*�0.002)


  The default \emph{damping\_amount}
 is 0.998 which means that the default value of \emph{idamp}
 is 0. The maximum \emph{damping\_amount}
 is 1.0 (no damping). This means the maximum value for \emph{idamp}
 is 1.0. 


  The recommended range for \emph{idamp}
 is usually below 75\% of the maximum value. 


 \emph{imaxshake}
 (optional) -- amount of energy to add back into the system. The value should be in range 0 to 1. 
\subsection*{Examples}


  Here is an example of the stix opcode. It uses the files \emph{stix.orc}
 and \emph{stix.sco}
. 


 \textbf{Example 1. Example of the stix opcode.}

\begin{lstlisting}
/* stix.orc */
;orchestra ---------------

  sr =           44100
  kr =            4410
  ksmps =              10
  nchnls =               1

instr 01                  ;an example of stix
  a1      line 20, p3, 20                           ;preset amplitude increase
  a2      stix p4, 0.01   ;stix needs a little amp help at these settings
  a3      product a1, a2                               ;increase amplitude
          out a3
          endin
/* stix.orc */
        
\end{lstlisting}
\begin{lstlisting}
/* stix.sco */
;score -------------------

   i1 0 1 26000
   e
/* stix.sco */
        
\end{lstlisting}
\subsection*{See Also}


 \emph{cabasa}
, \emph{crunch}
, \emph{sandpaper}
, \emph{sekere}

\subsection*{Credits}


 


 


\begin{tabular}{cccc}
Author: Perry Cook, part of the PhOLIES (Physically-Oriented Library of Imitated Environmental Sounds) &Adapted by John ffitch &University of Bath, Codemist Ltd. &Bath, UK

\end{tabular}



 


 New in Csound version 4.07


 Added notes by Rasmus Ekman on May 2002.
%\hline 


\begin{comment}
\begin{tabular}{lcr}
Previous &Home &Next \\
sr &Up &streson

\end{tabular}


\end{document}
\end{comment}
