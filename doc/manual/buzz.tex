\begin{comment}
\documentclass[10pt]{article}
\usepackage{fullpage, graphicx, url}
\setlength{\parskip}{1ex}
\setlength{\parindent}{0ex}
\title{buzz}
\begin{document}


\begin{tabular}{ccc}
The Alternative Csound Reference Manual & & \\
Previous & &Next

\end{tabular}

%\hline 
\end{comment}
\section{buzz}
buzz�--� Output is a set of harmonically related sine partials. \subsection*{Description}


  Output is a set of harmonically related sine partials. 
\subsection*{Syntax}


 ar \textbf{buzz}
 xamp, xcps, knh, ifn [, iphs]
\subsection*{Initialization}


 \emph{ifn}
 -- table number of a stored function containing a sine wave. A large table of at least 8192 points is recommended. 


 \emph{iphs}
 (optional, default=0) -- initial phase of the fundamental frequency, expressed as a fraction of a cycle (0 to 1). A negative value will cause phase initialization to be skipped. The default value is zero 
\subsection*{Performance}


 \emph{xamp}
 -- amplitude 


 \emph{xcps}
 -- frequency in cycles per second 


  The buzz units generate an additive set of harmonically related cosine partials of fundamental frequency \emph{xcps}
, and whose amplitudes are scaled so their summation peak equals \emph{xamp}
. The selection and strength of partials is determined by the following control parameters: 


 \emph{knh}
 -- total number of harmonics requested. New in Csound version 3.57, \emph{knh}
 defaults to one. If \emph{knh}
 is negative, the absolute value is used. 


 \emph{buzz}
 and \emph{gbuzz}
 are useful as complex sound sources in subtractive synthesis. \emph{buzz}
 is a special case of the more general \emph{gbuzz}
 in which \emph{klh}
 = \emph{kmul}
= 1; it thus produces a set of \emph{knh}
 equal-strength harmonic partials, beginning with the fundamental. (This is a band-limited pulse train; if the partials extend to the Nyquist, i.e. \emph{knh}
 = int (sr / 2 / fundamental freq.), the result is a real pulse train of amplitude \emph{xamp}
.) 


  Although \emph{knh}
 may be varied during performance, its internal value is necessarily integer and may cause ``pops'' due to discontinuities in the output. \emph{buzz}
 can be amplitude- and/or frequency-modulated by either control or audio signals. 


  N.B. This unit has its analog in \emph{GEN11}
, in which the same set of cosines can be stored in a function table for sampling by an oscillator. Although computationally more efficient, the stored pulse train has a fixed spectral content, not a time-varying one as above. 
\subsection*{Examples}


  Here is an example of the buzz opcode. It uses the files \emph{buzz.orc}
 and \emph{buzz.sco}
. 


 \textbf{Example 1. Example of the buzz opcode.}

\begin{lstlisting}
/* buzz.orc */
; Initialize the global variables.
sr = 44100
kr = 4410
ksmps = 10
nchnls = 1

; Instrument #1.
instr 1
  kamp = 20000
  kcps = 440
  knh = 3
  ifn = 1

  a1 buzz kamp, kcps, knh, ifn
  out a1
endin
/* buzz.orc */
        
\end{lstlisting}
\begin{lstlisting}
/* buzz.sco */
; Table #1, a sine wave.
f 1 0 16384 10 1

; Play Instrument #1 for one second.
i 1 0 1
e
/* buzz.sco */
        
\end{lstlisting}
\subsection*{See Also}


 \emph{gbuzz}

\subsection*{Credits}


 September 2003. Thanks to Kanata Motohashi for correcting the mentions of the \emph{kmul}
 parameter.


 Example written by Kevin Conder.
%\hline 


\begin{comment}
\begin{tabular}{lcr}
Previous &Home &Next \\
button &Up &cabasa

\end{tabular}


\end{document}
\end{comment}
