\begin{comment}
\documentclass[10pt]{article}
\usepackage{fullpage, graphicx, url}
\setlength{\parskip}{1ex}
\setlength{\parindent}{0ex}
\title{ftsr}
\begin{document}


\begin{tabular}{ccc}
The Alternative Csound Reference Manual & & \\
Previous & &Next

\end{tabular}

%\hline 
\end{comment}
\section{ftsr}
ftsr�--� Returns the sampling-rate of a stored function table. \subsection*{Description}


  Returns the sampling-rate of a stored function table. 
\subsection*{Syntax}


 \textbf{ftsr}
(x) (init-rate args only)
\subsection*{Performance}


  Returns the sampling-rate of a \emph{GEN01}
 generated table. The sampling-rate is determined from the header of the original file. If the original file has no header or the table was not created by these GEN01, \emph{ftsr}
 returns 0. New in Csound version 3.49. 
\subsection*{Examples}


  Here is an example of the ftsr opcode. It uses the files \emph{ftsr.orc}
, \emph{ftsr.sco}
, and \emph{mary.wav}
. 


 \textbf{Example 1. Example of the ftsr opcode.}

\begin{lstlisting}
/* ftsr.orc */
; Initialize the global variables.
sr = 44100
kr = 4410
ksmps = 10
nchnls = 1

; Instrument #1.
instr 1
  ; Print out the sampling rate of Table #1.
  isr = ftsr(1)
  print isr
endin
/* ftsr.orc */
        
\end{lstlisting}
\begin{lstlisting}
/* ftsr.sco */
; Table #1: Use an audio file.
f 1 0 262144 1 "mary.wav" 0 0 0

; Play Instrument #1 for 1 second.
i 1 0 1
e
/* ftsr.sco */
        
\end{lstlisting}
 Since the audio file ``mary.wav'' uses a 44.1 Khz sampling rate, its output should a line like this: \begin{lstlisting}
instr 1:  isr = 44100.000
      
\end{lstlisting}
\subsection*{See Also}


 \emph{ftchnls}
, \emph{ftlen}
, \emph{ftlptim}
, \emph{nsamp}

\subsection*{Credits}


 


 


\begin{tabular}{ccc}
Author: Gabriel Maldonado &Italy &October 1998

\end{tabular}



 


 Example written by Kevin Conder.
%\hline 


\begin{comment}
\begin{tabular}{lcr}
Previous &Home &Next \\
ftsavek &Up &gain

\end{tabular}


\end{document}
\end{comment}
