\begin{comment}
\documentclass[10pt]{article}
\usepackage{fullpage, graphicx, url}
\setlength{\parskip}{1ex}
\setlength{\parindent}{0ex}
\title{fog}
\begin{document}


\begin{tabular}{ccc}
The Alternative Csound Reference Manual & & \\
Previous & &Next

\end{tabular}

%\hline 
\end{comment}
\section{fog}
fog�--� Audio output is a succession of grains derived from data in a stored function table \subsection*{Description}


  Audio output is a succession of grains derived from data in a stored function table \emph{ifna}
. The local envelope of these grains and their timing is based on the model of \emph{fof}
 synthesis and permits detailed control of the granular synthesis. 
\subsection*{Syntax}


 ar \textbf{fog}
 xamp, xdens, xtrans, aspd, koct, kband, kris, kdur, kdec, iolaps, ifna, ifnb, itotdur [, iphs] [, itmode] [, iskip]
\subsection*{Initialization}


 \emph{iolaps}
 -- number of pre-located spaces needed to hold overlapping grain data. Overlaps are density dependent, and the space required depends on the maximum value of \emph{xdens}
 * \emph{kdur}
. Can be over-estimated at no computation cost. Uses less than 50 bytes of memory per \emph{iolaps}
. 


 \emph{ifna}
, \emph{ifnb}
 -- table numbers of two stored functions. The first is the data used for granulation, usually from a soundfile (\emph{GEN01}
). The second is a rise shape, used forwards and backwards to shape the grain rise and decay; this is normally a sigmoid (\emph{GEN19}
) but may be linear (\emph{GEN05}
). 


 \emph{itotdur}
 -- total time during which this \emph{fog}
 will be active. Normally set to p3. No new grain is created if it cannot complete its \emph{kdur}
 within the remaining \emph{itotdur}
. 


 \emph{iphs}
 (optional) -- initial phase of the fundamental, expressed as a fraction of a cycle (0 to 1). The default value is 0. 


 \emph{itmode}
 (optional) -- transposition mode. If zero, each grain keeps the \emph{xtrans}
 value it was launched with. if non-zero, each is influenced by \emph{xtrans}
 continuously. The default value is 0. 


 \emph{iskip}
 (optional, default=0) -- If non-zero, skip initialization (allows legato use). 
\subsection*{Performance}


 \emph{xamp}
 -- amplitude factor. Amplitude is also dependent on the number of overlapping grains, the interaction of the rise shape (\emph{ifnb}
) and the exponential decay (\emph{kband}
), and the scaling of the grain waveform (\emph{ifna}
). The actual amplitude may therefore exceed \emph{xamp}
. 


 \emph{xdens}
 -- density. The frequency of grains per second. 


 \emph{xtrans}
 -- transposition factor. The rate at which data from the stored function table \emph{ifna}
 is read within each grain. This has the effect of transposing the original material. A value of 1 produces the original pitch. Higher values transpose upwards, lower values downwards. Negative values result in the function table being read backwards. 


 \emph{aspd}
 -- speed. The rate at which successive grains advance through the stored function table \emph{ifna}
. \emph{aspd}
 is in the form of an index (0 to 1) to \emph{ifna}
. This determines the movement of a pointer used as the starting point for reading data within each grain. (\emph{xtrans}
 determines the rate at which data is read starting from this pointer.) 


 \emph{koct}
 -- octaviation index. The operation of this parameter is identical to that in \emph{fof}
. 


 \emph{kband}
, \emph{kris}
, \emph{kdur}
, \emph{kdec}
 -- grain envelope shape. These parameters determine the exponential decay (\emph{kband}
), and the rise (\emph{kris}
), overall duration (\emph{kdur}
,) and decay (\emph{kdec}
 ) times of the grain envelope. Their operation is identical to that of the local envelope parameters in \emph{fof}
. 


  The Csound \emph{fog}
 generator is by Michael Clarke, extending his earlier work based on IRCAM's fof algorithm. 
\subsection*{Examples}


 


 
\begin{lstlisting}
;p4 = transposition factor
;p5 = speed factor
;p6 = function table for grain data
i1  = sr/ftlen(p6) ;scaling to reflect sample rate and table length
a1 \emph{phasor}
 i1*p5 ;index for speed
a2 \emph{fog}
    5000, 100, p4, a1, 0, 0, , .01, .02, .01, 2, p6, 1, p3, 0, 1
        
\end{lstlisting}


 
\subsection*{Credits}


 


 


\begin{tabular}{ccc}
Author: Michael Clark &Huddersfield &May 1997

\end{tabular}



 


 New in version 3.46


  The Csound fog generator is by Michael Clarke, extending his earlier work based on IRCAM's fof algorithm. 


 Added notes by Rasmus Ekman on September 2002.
%\hline 


\begin{comment}
\begin{tabular}{lcr}
Previous &Home &Next \\
fof2 &Up &fold

\end{tabular}


\end{document}
\end{comment}
