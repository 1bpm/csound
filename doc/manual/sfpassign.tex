\begin{comment}
\documentclass[10pt]{article}
\usepackage{fullpage, graphicx, url}
\setlength{\parskip}{1ex}
\setlength{\parindent}{0ex}
\title{sfpassign}
\begin{document}


\begin{tabular}{ccc}
The Alternative Csound Reference Manual & & \\
Previous & &Next

\end{tabular}

%\hline 
\end{comment}
\section{sfpassign}
sfpassign�--� Assigns all presets of a SoundFont2 (SF2) sample file to a sequence of progressive index numbers. \subsection*{Description}


  Assigns all presets of a previously loaded SoundFont2 (SF2) sample file to a sequence of progressive index numbers. These opcodes allow management the sample-structure of SF2 files. In order to understand the usage of these opcodes, the user must have some knowledge of the SF2 format, so a brief description of this format can be found in the \emph{SoundFont2 File Format Appendix}
. 


 \emph{sfpassign}
 should be placed in the header section of a Csound orchestra. 
\subsection*{Syntax}


 \textbf{sfpassign}
 istartindex, ifilhandle
\subsection*{Initialization}


 \emph{istartindex}
 -- starting index preset by the user in bulk preset assignments. 


 \emph{ifilhandle}
 -- unique number generated by \emph{sfload}
 opcode to be used as an identifier for a SF2 file. Several SF2 files can be loaded and activated at the same time. 
\subsection*{Performance}


 \emph{sfpassign}
 assigns all presets of a previously loaded SF2 file to a sequence of progressive index numbers, to be used later with the opcodes \emph{sfplay}
 and \emph{sfplaym}
. \emph{istartindex}
 specifies the starting index number. Any number of \emph{sfpassign}
 instances can be placed in the header section of an orchestra, each one assigning presets belonging to different SF2 files. The user must take care that preset index numbers of different SF2 files do not overlap. 


  These opcodes only support the sample structure of SF2 files. The modulator structure of the SoundFont2 format is not supported in Csound. Any modulation or processing to the sample data is left to the Csound user, bypassing all restrictions forced by the SF2 standard. 
\subsection*{See Also}


 \emph{sfilist}
, \emph{sfinstr}
, \emph{sfinstrm}
, \emph{sfload}
, \emph{sfplay}
, \emph{sfplaym}
, \emph{sfplist}
, \emph{sfpreset}

\subsection*{Credits}


 


 


\begin{tabular}{ccc}
Author: Gabriel Maldonado &Italy &May 2000

\end{tabular}



 


 New in Csound Version 4.07
%\hline 


\begin{comment}
\begin{tabular}{lcr}
Previous &Home &Next \\
sfload &Up &sfplay

\end{tabular}


\end{document}
\end{comment}
