\begin{comment}
\documentclass[10pt]{article}
\usepackage{fullpage, graphicx, url}
\setlength{\parskip}{1ex}
\setlength{\parindent}{0ex}
\title{lpreson}
\begin{document}


\begin{tabular}{ccc}
The Alternative Csound Reference Manual & & \\
Previous & &Next

\end{tabular}

%\hline 
\end{comment}
\section{lpreson}
lpreson�--� Modifies the spectrum of an audio signal with time-varying filter coefficients from a control file. \subsection*{Description}


  Modifies the spectrum of an audio signal with time-varying filter coefficients from a control file. 
\subsection*{Syntax}


 ar \textbf{lpreson}
 asig
\subsection*{Performance}


 \emph{asig}
 -- an audio signal to be modified. 


 \emph{lpread}
 gets its values from the control file according to the input value \emph{ktimpnt}
 (in seconds). If \emph{ktimpnt}
 proceeds at the analysis rate, time-normal synthesis will result; proceeding at a faster, slower, or variable rate will result in time-warped synthesis. At each k-period, \emph{lpread}
 interpolates between adjacent frames to more accurately determine the parameter values (presented as output) and the filter coefficient settings (passed internally to a subsequent \emph{lpreson}
). 
\subsection*{See Also}


 \emph{lpfreson}
, \emph{lpread}

%\hline 


\begin{comment}
\begin{tabular}{lcr}
Previous &Home &Next \\
lpread &Up &lpshold

\end{tabular}


\end{document}
\end{comment}
