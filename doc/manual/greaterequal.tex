\begin{comment}
\documentclass[10pt]{article}
\usepackage{fullpage, graphicx, url}
\setlength{\parskip}{1ex}
\setlength{\parindent}{0ex}
\title{$>$=}
\begin{document}


\begin{tabular}{ccc}
The Alternative Csound Reference Manual & & \\
Previous & &Next

\end{tabular}

%\hline 
\end{comment}
\section{$>$=}
$>$=�--� Determines if one value is greater than or equal to another. \subsection*{Description}


  Determines if one value is greater than or equal to another. 
\subsection*{Syntax}


 (a \textbf{$>$=}
 b \textbf{?}
 v1 \textbf{:}
 v2)


  where \emph{a}
, \emph{b}
, \emph{v1}
 and \emph{v2}
 may be expressions, but \emph{a}
, \emph{b}
 not audio-rate. 
\subsection*{Performance}


  In the above conditionals, \emph{a}
 and \emph{b}
 are first compared. If the indicated relation is true (\emph{a}
 greater than \emph{b}
, \emph{a}
 less than \emph{b}
, \emph{a}
 greater than or equal to \emph{b}
, \emph{a}
 less than or equal to \emph{b}
, \emph{a}
 equal to \emph{b}
, \emph{a}
 not equal to \emph{b}
), then the conditional expression has the value of \emph{v1}
; if the relation is false, the expression has the value of \emph{v2}
. (For convenience, a sole ``\emph{=}
`` will function as ``\emph{= =}
``.) 


  NB.: If \emph{v1}
 or \emph{v2}
 are expressions, these will be evaluated before the conditional is determined. 


  In terms of binding strength, all conditional operators (i.e. the relational operators (\emph{$<$}
, etc.), and \emph{?}
, and \emph{:}
 ) are weaker than the arithmetic and logical operators (\emph{+}
, \emph{-}
, \emph{*}
, \emph{/}
, \emph{\&}
 and \emph{||}
). 


  These are \emph{operators}
 not \emph{opcodes}
. Therefore, they can be used within orchestra statements, but do not form complete statements themselves. 
\subsection*{Examples}


  Here is an example of the $>$= opcode. It uses the files \emph{greaterequal.orc}
 and \emph{greaterequal.sco}
. 


 \textbf{Example 1. Example of the $>$= opcode.}

\begin{lstlisting}
/* greaterequal.orc */
; Initialize the global variables.
sr = 44100
kr = 44100
ksmps = 1
nchnls = 1

; Instrument #1.
instr 1
  ; Get the 4th p-field from the score.
  k1 =  p4

  ; Is it greater than or equal to 3? (1 = true, 0 = false)
  k2 = (p4 >= 3 ? 1 : 0)

  ; Print the values of k1 and k2.
  printks "k1 = %f, k2 = %f\\n", 1, k1, k2
endin
/* greaterequal.orc */
        
\end{lstlisting}
\begin{lstlisting}
/* greaterequal.sco */
; Call Instrument #1 with a p4 = 2.
i 1 0 0.5 2
; Call Instrument #1 with a p4 = 3.
i 1 1 0.5 3
; Call Instrument #1 with a p4 = 4.
i 1 2 0.5 4
e
/* greaterequal.sco */
        
\end{lstlisting}
 Its output should include lines like this: \begin{lstlisting}
k1 = 2.000000, k2 = 0.000000
k1 = 3.000000, k2 = 1.000000
k1 = 4.000000, k2 = 1.000000
      
\end{lstlisting}
\subsection*{See Also}


 \emph{==}
, \emph{$>$}
, \emph{$<$=}
, \emph{$<$}
, \emph{!=}

\subsection*{Credits}


 Example written by Kevin Conder.
%\hline 


\begin{comment}
\begin{tabular}{lcr}
Previous &Home &Next \\
$>$ &Up &$<$

\end{tabular}


\end{document}
\end{comment}
