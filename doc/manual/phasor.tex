\begin{comment}
\documentclass[10pt]{article}
\usepackage{fullpage, graphicx, url}
\setlength{\parskip}{1ex}
\setlength{\parindent}{0ex}
\title{phasor}
\begin{document}


\begin{tabular}{ccc}
The Alternative Csound Reference Manual & & \\
Previous & &Next

\end{tabular}

%\hline 
\end{comment}
\section{phasor}
phasor�--� Produce a normalized moving phase value. \subsection*{Description}


  Produce a normalized moving phase value. 
\subsection*{Syntax}


 ar \textbf{phasor}
 xcps [, iphs]


 kr \textbf{phasor}
 kcps [, iphs]
\subsection*{Initialization}


 \emph{iphs}
 (optional) -- initial phase, expressed as a fraction of a cycle (0 to 1). A negative value will cause phase initialization to be skipped. The default value is zero. 
\subsection*{Performance}


  An internal phase is successively accumulated in accordance with the \emph{kcps}
 or \emph{xcps}
 frequency to produce a moving phase value, normalized to lie in the range 0 $<$= phs $<$ 1. 


  When used as the index to a \emph{table}
 unit, this phase (multiplied by the desired function table length) will cause it to behave like an oscillator. 


  Note that \emph{phasor}
 is a special kind of integrator, accumulating phase increments that represent frequency settings. 
\subsection*{Examples}


  Here is an example of the phasor opcode. It uses the files \emph{phasor.orc}
 and \emph{phasor.sco}
. 


 \textbf{Example 1. Example of the phasor opcode.}

\begin{lstlisting}
/* phasor.orc */
; Initialize the global variables.
sr = 44100
kr = 4410
ksmps = 10
nchnls = 1

; Instrument #1.
instr 1
  ; Create an index that repeats once per second.
  kcps init 1
  kndx phasor kcps

  ; Read Table #1 with our index.
  ifn = 1
  ixmode = 1
  kfreq table kndx, ifn, ixmode

  ; Generate a sine waveform, use our table values 
  ; to vary its frequency.
  a1 oscil 20000, kfreq, 2
  out a1
endin
/* phasor.orc */
        
\end{lstlisting}
\begin{lstlisting}
/* phasor.sco */
; Table #1, a line from 200 to 2,000.
f 1 0 1025 -7 200 1024 2000
; Table #2, a sine wave.
f 2 0 16384 10 1

; Play Instrument #1 for 2 seconds.
i 1 0 2
e
/* phasor.sco */
        
\end{lstlisting}
\subsection*{Credits}


 Example written by Kevin Conder.
%\hline 


\begin{comment}
\begin{tabular}{lcr}
Previous &Home &Next \\
phaser2 &Up &phasorbnk

\end{tabular}


\end{document}
\end{comment}
