\begin{comment}
\documentclass[10pt]{article}
\usepackage{fullpage, graphicx, url}
\setlength{\parskip}{1ex}
\setlength{\parindent}{0ex}
\title{soundout}
\begin{document}


\begin{tabular}{ccc}
The Alternative Csound Reference Manual & & \\
Previous & &Next

\end{tabular}

%\hline 
\end{comment}
\section{soundout}
soundout�--� Writes audio output to a disk file. \subsection*{Description}


  Writes audio output to a disk file. 
\subsection*{Syntax}


 \textbf{soundout}
 asig1, ifilcod [, iformat]
\subsection*{Initialization}


 \emph{ifilcod}
 -- integer or character-string denoting the destination soundfile name. An integer denotes the file soundin.filcod; a character-string (in double quotes, spaces permitted) gives the filename itself, optionally a full pathname. If not a full path, the named file is sought first in the current directory, then in that given by the environment variable SSDIR (if defined) then by SFDIR. See also \emph{GEN01}
. 


 \emph{iformat}
 (optional, default=0) -- specifies the audio data file format: 


 
\begin{itemize}
\item 

 1 = 8-bit signed char (high-order 8 bits of a 16-bit integer)

\item 

 2 = 8-bit A-law bytes

\item 

 3 = 8-bit U-law bytes

\item 

 4 = 16-bit short integers

\item 

 5 = 32-bit long integers

\item 

 6 = 32-bit floats


\end{itemize}


  If \emph{iformat}
 = 0 it is taken from the soundfile header, and if no header from the Csound \emph{-o}
 command-line flag. The default value is 0. 
\subsection*{Performance}


 \emph{soundout}
 writes audio output to a disk file. 
\subsection*{See Also}


 \emph{out}
, \emph{outh}
, \emph{outo}
, \emph{outq}
, \emph{outq1}
, \emph{outq2}
, \emph{outq3}
, \emph{outq4}
, \emph{outs}
, \emph{outs1}
, \emph{outs2}

\subsection*{Credits}


 


 


\begin{tabular}{ccc}
Author: Barry L. Vercoe, Matt Ingalls/Mike Berry &MIT, Mills College &1993-1997

\end{tabular}



 
%\hline 


\begin{comment}
\begin{tabular}{lcr}
Previous &Home &Next \\
soundin &Up &space

\end{tabular}


\end{document}
\end{comment}
