\begin{comment}
\documentclass[10pt]{article}
\usepackage{fullpage, graphicx, url}
\setlength{\parskip}{1ex}
\setlength{\parindent}{0ex}
\title{pset}
\begin{document}


\begin{tabular}{ccc}
The Alternative Csound Reference Manual & & \\
Previous & &Next

\end{tabular}

%\hline 
\end{comment}
\section{pset}
pset�--� Defines and initializes numeric arrays at orchestra load time. \subsection*{Description}


  Defines and initializes numeric arrays at orchestra load time. 
\subsection*{Syntax}


 \textbf{pset}
 icon1 [, icon2] [...]
\subsection*{Initialization}


 \emph{icon1, icon2, ...}
 -- preset values for a MIDI instrument 


 \emph{pset}
 (optional) defines and initializes numeric arrays at orchestra load time. It may be used as an orchestra header statement (i.e. instrument 0) or within an instrument. When defined within an instrument, it is not part of its i-time or performance operation, and only one statement is allowed per instrument. These values are available as i-time defaults. When an instrument is triggered from MIDI it only gets p1 and p2 from the event, and p3, p4, etc. will receive the actual preset values. 
\subsection*{Examples}


  The example below illustrates \emph{pset}
 as used within an instrument. 


 
\begin{lstlisting}
\textbf{instr}
 1
  \emph{pset}
 0,0,3,4,5,6  ; pfield substitutes
  a1 \emph{oscil}
 10000, 440, p6
        
\end{lstlisting}


 
\subsection*{See Also}


 \emph{strset}

%\hline 


\begin{comment}
\begin{tabular}{lcr}
Previous &Home &Next \\
product &Up &pvadd

\end{tabular}


\end{document}
\end{comment}
