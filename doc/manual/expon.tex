\begin{comment}
\documentclass[10pt]{article}
\usepackage{fullpage, graphicx, url}
\setlength{\parskip}{1ex}
\setlength{\parindent}{0ex}
\title{expon}
\begin{document}


\begin{tabular}{ccc}
The Alternative Csound Reference Manual & & \\
Previous & &Next

\end{tabular}

%\hline 
\end{comment}
\section{expon}
expon�--� Trace an exponential curve between specified points. \subsection*{Description}


  Trace an exponential curve between specified points. 
\subsection*{Syntax}


 ar \textbf{expon}
 ia, idur1, ib


 kr \textbf{expon}
 ia, idur1, ib
\subsection*{Initialization}


 \emph{ia}
 -- starting value. Zero is illegal for exponentials. 


 \emph{ib, ic}
, etc. -- value after \emph{dur1}
 seconds, etc. For exponentials, must be non-zero and must agree in sign with \emph{ia}
. 


 \emph{idur1}
 -- duration in seconds of first segment. A zero or negative value will cause all initialization to be skipped. 
\subsection*{Performance}


  These units generate control or audio signals whose values can pass through 2 or more specified points. The sum of \emph{dur}
 values may or may not equal the instrument's performance time: a shorter performance will truncate the specified pattern, while a longer one will cause the last-defined segment to continue on in the same direction. 
\subsection*{Examples}


  Here is an example of the expon opcode. It uses the files \emph{expon.orc}
 and \emph{expon.sco}
. 


 \textbf{Example 1. Example of the expon opcode.}

\begin{lstlisting}
/* expon.orc */
; Initialize the global variables.
sr = 44100
kr = 4410
ksmps = 10
nchnls = 1

; Instrument #1.
instr 1
  ; Define kcps as a frequency value that exponentially declines 
  ; from 880 to 220. It declines over the period set by p3.
  kcps expon 880, p3, 220

  a1 oscil 20000, kcps, 1
  out a1
endin
/* expon.orc */
        
\end{lstlisting}
\begin{lstlisting}
/* expon.sco */
; Table #1, a sine wave.
f 1 0 16384 10 1

; Play Instrument #1 for two seconds.
i 1 0 2
e
/* expon.sco */
        
\end{lstlisting}
\subsection*{See Also}


 \emph{expseg}
, \emph{expsegr}
, \emph{line}
, \emph{linseg}
, \emph{linsegr}

\subsection*{Credits}


 Example written by Kevin Conder.
%\hline 


\begin{comment}
\begin{tabular}{lcr}
Previous &Home &Next \\
exp &Up &exprand

\end{tabular}


\end{document}
\end{comment}
