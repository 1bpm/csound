\begin{comment}
\documentclass[10pt]{article}
\usepackage{fullpage, graphicx, url}
\setlength{\parskip}{1ex}
\setlength{\parindent}{0ex}
\title{table}
\begin{document}


\begin{tabular}{ccc}
The Alternative Csound Reference Manual & & \\
Previous & &Next

\end{tabular}

%\hline 
\end{comment}
\section{table}
table�--� Accesses table values by direct indexing. \subsection*{Description}


  Accesses table values by direct indexing. 
\subsection*{Syntax}


 ar \textbf{table}
 andx, ifn [, ixmode] [, ixoff] [, iwrap]


 ir \textbf{table}
 indx, ifn [, ixmode] [, ixoff] [, iwrap]


 kr \textbf{table}
 kndx, ifn [, ixmode] [, ixoff] [, iwrap]
\subsection*{Initialization}


 \emph{ifn}
 -- function table number. 


 \emph{ixmode}
 (optional) -- index data mode. The default value is 0. 


 
\begin{itemize}
\item 

 0 = raw index

\item 

 1 = normalized (0 to 1)


\end{itemize}


 \emph{ixoff}
 (optional) -- amount by which index is to be offset. For a table with origin at center, use tablesize/2 (raw) or .5 (normalized). The default value is 0. 


 \emph{iwrap}
 (optional) -- wraparound index flag. The default value is 0. 


 
\begin{itemize}
\item 

 0 = nowrap (index $<$ 0 treated as index=0; index tablesize sticks at index=size)

\item 

 1 = wraparound.


\end{itemize}
\subsection*{Performance}


 \emph{table}
 invokes table lookup on behalf of init, control or audio indices. These indices can be raw entry numbers (0,l,2...size - 1) or scaled values (0 to 1-e). Indices are first modified by the offset value then checked for range before table lookup (see \emph{iwrap}
). If index is likely to be full scale, or if interpolation is being used, the table should have an extended guard point. \emph{table}
 indexed by a periodic phasor ( see \emph{phasor}
) will simulate an oscillator. 
\subsection*{Examples}


  Here is an example of the table opcode. It uses the files \emph{table.orc}
 and \emph{table.sco}
. 


 \textbf{Example 1. Example of the table opcode.}

\begin{lstlisting}
/* table.orc */
; Initialize the global variables.
sr = 44100
kr = 4410
ksmps = 10
nchnls = 1

; Instrument #1.
instr 1
  ; Vary our index linearly from 0 to 1.
  kndx line 0, p3, 1

  ; Read Table #1 with our index.
  ifn = 1
  ixmode = 1
  kfreq table kndx, ifn, ixmode

  ; Generate a sine waveform, use our table values 
  ; to vary its frequency.
  a1 oscil 20000, kfreq, 2
  out a1
endin
/* table.orc */
        
\end{lstlisting}
\begin{lstlisting}
/* table.sco */
; Table #1, a line from 200 to 2,000.
f 1 0 1025 -7 200 1024 2000
; Table #2, a sine wave.
f 2 0 16384 10 1

; Play Instrument #1 for 2 seconds.
i 1 0 2
e
/* table.sco */
        
\end{lstlisting}
\subsection*{See Also}


 \emph{tablei}
, \emph{table3}
, \emph{oscil1}
, \emph{oscil1i}
, \emph{osciln}

\subsection*{Credits}


 Example written by Kevin Conder.
%\hline 


\begin{comment}
\begin{tabular}{lcr}
Previous &Home &Next \\
svfilter &Up &table3

\end{tabular}


\end{document}
\end{comment}
