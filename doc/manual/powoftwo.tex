\begin{comment}
\documentclass[10pt]{article}
\usepackage{fullpage, graphicx, url}
\setlength{\parskip}{1ex}
\setlength{\parindent}{0ex}
\title{powoftwo}
\begin{document}


\begin{tabular}{ccc}
The Alternative Csound Reference Manual & & \\
Previous & &Next

\end{tabular}

%\hline 
\end{comment}
\section{powoftwo}
powoftwo�--� Performs a power-of-two calculation. \subsection*{Description}


  Performs a power-of-two calculation. 
\subsection*{Syntax}


 \textbf{powoftwo}
(x) (init-rate or control-rate args only)
\subsection*{Performance}


 \emph{powoftwo}
() function returns 2 \^{} \emph{x}
 and allows positive and negatives numbers as argument. The range of values admitted in \emph{powoftwo}
() is -5 to +5 allowing a precision more fine than one cent in a range of ten octaves. If a greater range of values is required, use the slower opcode \emph{pow}
. 


  These functions are fast, because they read values stored in tables. Also they are very useful when working with tuning ratios. They work at i- and k-rate. 
\subsection*{Examples}


  Here is an example of the powoftwo opcode. It uses the files \emph{powoftwo.orc}
 and \emph{powoftwo.sco}
. 


 \textbf{Example 1. Example of the powoftwo opcode.}

\begin{lstlisting}
/* powoftwo.orc */
; Initialize the global variables.
sr = 44100
kr = 4410
ksmps = 10
nchnls = 1

; Instrument #1.
instr 1
  i1 = powoftwo(12)
  print i1
endin
/* powoftwo.orc */
        
\end{lstlisting}
\begin{lstlisting}
/* powoftwo.sco */
; Play Instrument #1 for one second.
i 1 0 1
e
/* powoftwo.sco */
        
\end{lstlisting}
 Its output should include a line like this: \begin{lstlisting}
instr 1:  i1 = 4096.000
      
\end{lstlisting}
\subsection*{See Also}


 \emph{logbtwo}
, \emph{pow}

\subsection*{Credits}


 


 


\begin{tabular}{ccc}
Author: Gabriel Maldonado &Italy &June 1998

\end{tabular}



 


 


 


\begin{tabular}{cccc}
Author: John ffitch &University of Bath, Codemist, Ltd. &Bath, UK &July 1999

\end{tabular}



 


 Example written by Kevin Conder.


 New in Csound version 3.57
%\hline 


\begin{comment}
\begin{tabular}{lcr}
Previous &Home &Next \\
pow &Up &prealloc

\end{tabular}


\end{document}
\end{comment}
