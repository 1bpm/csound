\begin{comment}
\documentclass[10pt]{article}
\usepackage{fullpage, graphicx, url}
\setlength{\parskip}{1ex}
\setlength{\parindent}{0ex}
\title{readk}
\begin{document}


\begin{tabular}{ccc}
The Alternative Csound Reference Manual & & \\
Previous & &Next

\end{tabular}

%\hline 
\end{comment}
\section{readk}
readk�--� Periodically reads an orchestra control-signal value from an external file. \subsection*{Description}


  Periodically reads an orchestra control-signal value to a named external file in a specific format. 
\subsection*{Syntax}


 kr \textbf{readk}
 ifilname, iformat, ipol [, interp]
\subsection*{Initialization}


 \emph{ifilname}
 -- character string (in double quotes, spaces permitted) denoting the external file name. May either be a full path name with target directory specified or a simple filename to be created within the current directory 


 \emph{iformat}
 -- specifies the output data format: 


 
\begin{itemize}
\item 

 1 = 8-bit signed char(high order 8 bits of a 16-bit integer

\item 

 4 = 16-bit short integers

\item 

 5 = 32-bit long integers

\item 

 6 = 32-bit floats, 7=ASCII long integers

\item 

 8 = ASCII floats (2 decimal places)


\end{itemize}


  Note that A-law and U-law output are not available, and that all formats except the lsat two are binary. The output file contains no header information. 


 \emph{iprd}
 -- the period of \emph{ksig}
 output i seconds, rounded to the nearest orchestra control period. A value of 0 implies one control period (the enforced minimum), which will create an output file sampled at the orchestra control rate. 


 \emph{ipol}
 -- if non-zero, and \emph{iprd}
 implies more than one control period, interpolate the k- signals between the periodic reads from the external file. If the value is 0, repeat each signal between frames. Currently not supported. 
\subsection*{Performance}


 \emph{kr}
 -- a control-rate signal 


  This opcode allows a generated control signal value to be read from a named external file. The file contains no self-defining header information. But it contains a regularly sampled time series, suitable for later input or analysis. There may be any number of \emph{readk}
 opcodes in an instrument or orchestra and they may read from the same or different files. 
\subsection*{Examples}


 


 
\begin{lstlisting}
knum    \emph{=}
         knum+1                                               ; at each k-period
ktemp   \emph{tempest}
   krms, .02, .1, 3, 2, 800, .005, 0, 60, 4, .1, .995   ;estimate the tempo
koct    \emph{specptrk}
  wsig, 6, .9, 0                                       ;and the pitch
        \emph{dumpk3}
    knum, ktemp, cpsoct(koct), "what happened when", 8 0 ;& save them
        
\end{lstlisting}


 
\subsection*{See Also}


 \emph{dumpk}
, \emph{dumpk2}
, \emph{dumpk3}
, \emph{dumpk4}
, \emph{readk2}
, \emph{readk3}
, \emph{readk4}

%\hline 


\begin{comment}
\begin{tabular}{lcr}
Previous &Home &Next \\
readclock &Up &readk2

\end{tabular}


\end{document}
\end{comment}
