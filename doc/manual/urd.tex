\begin{comment}
\documentclass[10pt]{article}
\usepackage{fullpage, graphicx, url}
\setlength{\parskip}{1ex}
\setlength{\parindent}{0ex}
\title{urd}
\begin{document}


\begin{tabular}{ccc}
The Alternative Csound Reference Manual & & \\
Previous & &Next

\end{tabular}

%\hline 
\end{comment}
\section{urd}
urd�--� A discrete user-defined-distribution random generator that can be used as a function. \subsection*{Description}


  A discrete user-defined-distribution random generator that can be used as a function. 
\subsection*{Syntax}


 aout = \textbf{urd}
(ktableNum)


 iout = \textbf{urd}
(itableNum)


 kout = \textbf{urd}
(ktableNum)
\subsection*{Initialization}


 \emph{itableNum}
 -- number of table containing the random-distribution function. Such table is generated by the user. See GEN40, GEN41, and GEN42. The table length does not need to be a power of 2 
\subsection*{Performance}


 \emph{ktableNum}
 -- number of table containing the random-distribution function. Such table is generated by the user. See GEN40, GEN41, and GEN42. The table length does not need to be a power of 2 


 \emph{urd}
 is the same opcode as \emph{duserrnd}
, but can be used in function fashion. 


  For a tutorial about random distribution histograms and functions see: 


 
\begin{itemize}
\item 

  D. Lorrain. ``A panoply of stochastic cannons''. In C. Roads, ed. 1989. Music machine. Cambridge, Massachusetts: MIT press, pp. 351 - 379. 


\end{itemize}
\subsection*{See Also}


 \emph{cuserrnd}
, \emph{duserrnd}

\subsection*{Credits}


 Author: Gabriel Maldonado


 New in Version 4.16
%\hline 


\begin{comment}
\begin{tabular}{lcr}
Previous &Home &Next \\
upsamp &Up &valpass

\end{tabular}


\end{document}
\end{comment}
