\begin{comment}
\documentclass[10pt]{article}
\usepackage{fullpage, graphicx, url}
\setlength{\parskip}{1ex}
\setlength{\parindent}{0ex}
\title{scanhammer}
\begin{document}


\begin{tabular}{ccc}
The Alternative Csound Reference Manual & & \\
Previous & &Next

\end{tabular}

%\hline 
\end{comment}
\section{scanhammer}
scanhammer�--� Copies from one table to another with a gain control. \subsection*{Description}


  This is is a variant of \emph{tablecopy}
, copying from one table to another, starting at \emph{ipos}
, and with a gain control. The number of points copied is determined by the length of the source. Other points are not changed. This opcode can be used to ``hit'' a string in the scanned synthesis code. 
\subsection*{Syntax}


 \textbf{scanhammer}
 isrc, idst, ipos, imult
\subsection*{Initialization}


 \emph{isrc}
 -- source function table. 


 \emph{idst}
 -- destination function table. 


 \emph{ipos}
 -- starting position (in points). 


 \emph{imult}
 -- gain multiplier. A value of 0 will leave values unchanged. 
\subsection*{See Also}


 \emph{scantable}

\subsection*{Credits}


 


 


\begin{tabular}{cc}
Author: Matt Gilliard &April 2002

\end{tabular}



 


 New in version 4.20
%\hline 


\begin{comment}
\begin{tabular}{lcr}
Previous &Home &Next \\
sandpaper &Up &scans

\end{tabular}


\end{document}
\end{comment}
