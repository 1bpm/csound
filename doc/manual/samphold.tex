\begin{comment}
\documentclass[10pt]{article}
\usepackage{fullpage, graphicx, url}
\setlength{\parskip}{1ex}
\setlength{\parindent}{0ex}
\title{samphold}
\begin{document}


\begin{tabular}{ccc}
The Alternative Csound Reference Manual & & \\
Previous & &Next

\end{tabular}

%\hline 
\end{comment}
\section{samphold}
samphold�--� Performs a sample-and-hold operation on its input. \subsection*{Description}


  Performs a sample-and-hold operation on its input. 
\subsection*{Syntax}


 ar \textbf{samphold}
 asig, agate [, ival] [, ivstor]


 kr \textbf{samphold}
 ksig, kgate [, ival] [, ivstor]
\subsection*{Initialization}


 \emph{ival, ivstor}
 (optional) -- controls initial disposition of internal save space. If \emph{ivstor}
 is zero the internal ``hold'' value is set to \emph{ival}
 ; else it retains its previous value. Defaults are 0,0 (i.e. init to zero) 
\subsection*{Performance}


 \emph{kgate, xgate}
 -- controls whether to hold the signal. 


 \emph{samphold}
 performs a sample-and-hold operation on its input according to the value of \emph{gate}
. If \emph{gate !- 0}
, the input samples are passed to the output; If \emph{gate = 0}
, the last output value is repeated. The controlling \emph{gate}
 can be a constant, a control signal, or an audio signal. 
\subsection*{Examples}


 


 
\begin{lstlisting}
asrc  \emph{buzz}
      10000,440,20, 1     ; band-limited pulse train
adif  \emph{diff}
      asrc                ; emphasize the highs
anew  \emph{balance}
   adif, asrc          ;   but retain the power
agate \emph{reson}
     asrc,0,440          ; use a lowpass of the original
asamp \emph{samphold}
  anew, agate         ;   to gate the new audiosig
aout  \emph{tone}
      asamp,100           ; smooth out the rough edges
        
\end{lstlisting}


 
\subsection*{See Also}


 \emph{diff}
, \emph{downsamp}
, \emph{integ}
, \emph{interp}
, \emph{upsamp}

%\hline 


\begin{comment}
\begin{tabular}{lcr}
Previous &Home &Next \\
s32b14 &Up &sandpaper

\end{tabular}


\end{document}
\end{comment}
