\begin{comment}
\documentclass[10pt]{article}
\usepackage{fullpage, graphicx, url}
\setlength{\parskip}{1ex}
\setlength{\parindent}{0ex}
\title{fink}
\begin{document}


\begin{tabular}{ccc}
The Alternative Csound Reference Manual & & \\
Previous & &Next

\end{tabular}

%\hline 
\end{comment}
\section{fink}
fink�--� Read signals from a file at k-rate. \subsection*{Description}


  Read signals from a file at k-rate. 
\subsection*{Syntax}


 \textbf{fink}
 ifilename, iskipframes, iformat, kin1 [, kin2] [, kin3] [,...]
\subsection*{Initialization}


 \emph{ifilename}
 -- input file name (can be a string or a handle number generated by fiopen) 


 \emph{iskipframes}
 -- number of frames to skip at the start (every frame contains a sample of each channel) 


 \emph{iformat}
 -- a number specifying the input file format. 


 
\begin{itemize}
\item 

 0 - 32 bit floating points without header

\item 

 1 - 16 bit integers without header


\end{itemize}
\subsection*{Performance}


 \emph{fink}
 is the same as \emph{fin}
 but operates at k-rate. 
\subsection*{See Also}


 \emph{fin}
, \emph{fini}

\subsection*{Credits}


 


 


\begin{tabular}{ccc}
Author: Gabriel Maldonado &Italy &1999

\end{tabular}



 


 New in Csound version 3.56
%\hline 


\begin{comment}
\begin{tabular}{lcr}
Previous &Home &Next \\
fini &Up &fiopen

\end{tabular}


\end{document}
\end{comment}
