\begin{comment}
\documentclass[10pt]{article}
\usepackage{fullpage, graphicx, url}
\setlength{\parskip}{1ex}
\setlength{\parindent}{0ex}
\title{ftsavek}
\begin{document}


\begin{tabular}{ccc}
The Alternative Csound Reference Manual & & \\
Previous & &Next

\end{tabular}

%\hline 
\end{comment}
\section{ftsavek}
ftsavek�--� Save a set of previously-allocated tables to a file. \subsection*{Description}


  Save a set of previously-allocated tables to a file. 
\subsection*{Syntax}


 \textbf{ftsavek}
 ``filename'', ktrig, iflag, ifn1 [, ifn2] [...]
\subsection*{Initialization}


 \emph{``filename''}
 -- A quoted string containing the name of the file to save. 


 \emph{iflag}
 -- Type of the file to save. (0 = binary file, Non-zero = text file) 


 \emph{ifn1, ifn2, ...}
 -- Numbers of tables to save. 
\subsection*{Performance}


 \emph{ktrig}
 -- The trigger signal. Save the file each time it is non-zero. 


 \emph{ftsavek}
 saves a list of tables to a file. The file's format can be binary or text. Unlike \emph{ftsave}
, the saving operation can be repeated numerous times within the same note by using a trigger signal. 


 


\begin{tabular}{cc}
Warning &\textbf{Warning}
 \\
� &

  The file's format is not compatible with a WAV-file and is not endian-safe. 


\end{tabular}

\subsection*{See Also}


 \emph{ftloadk}
, \emph{ftload}
, \emph{ftsave}

\subsection*{Credits}


 Author: Gabriel Maldonado


 New in version 4.21
%\hline 


\begin{comment}
\begin{tabular}{lcr}
Previous &Home &Next \\
ftsave &Up &ftsr

\end{tabular}


\end{document}
\end{comment}
