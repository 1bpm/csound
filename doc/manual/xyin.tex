\begin{comment}
\documentclass[10pt]{article}
\usepackage{fullpage, graphicx, url}
\setlength{\parskip}{1ex}
\setlength{\parindent}{0ex}
\title{xyin}
\begin{document}


\begin{tabular}{ccc}
The Alternative Csound Reference Manual & & \\
Previous & &Next

\end{tabular}

%\hline 
\end{comment}
\section{xyin}
xyin�--� Sense the cursor position in an output window \subsection*{Description}


  Sense the cursor position in an output window. When \emph{xyin}
 is called the position of the mouse within the output window is used to reply to the request. This simple mechanism does mean that only one \emph{xyin}
 can be used accurately at once. The position of the mouse is reported in the output window. 
\subsection*{Syntax}


 kx, ky \textbf{xyin}
 iprd, ixmin, ixmax, iymin, iymax [, ixinit] [, iyinit]
\subsection*{Initialization}


 \emph{iprd}
 -- period of cursor sensing (in seconds). Typically .1 seconds. 


 \emph{xmin, xmax, ymin, ymax}
 -- edge values for the x-y coordinates of a cursor in the input window. 


 \emph{ixinit, iyinit}
 (optional) -- initial x-y coordinates reported; the default values are 0,0. If these values are not within the given min-max range, they will be coerced into that range. 
\subsection*{Performance}


 \emph{xyin}
 samples the cursor x-y position in an input window every \emph{iprd}
 seconds. Output values are repeated (not interpolated) at the k-rate, and remain fixed until a new change is registered in the window. There may be any number of input windows. This unit is useful for real-time control, but continuous motion should be avoided if \emph{iprd}
 is unusually small. 
\subsection*{Examples}


  Here is an example of the xyin opcode. It uses the files \emph{xyin.orc}
 and \emph{xyin.sco}
. 


 \textbf{Example 1. Example of the xyin opcode.}

\begin{lstlisting}
/* xyin.orc */
; Initialize the global variables.
sr = 44100
kr = 4410
ksmps = 10
nchnls = 1

; Instrument #1.
instr 1
  ; Print and capture values every 0.1 seconds.
  iprd = 0.1
  ; The x values are from 1 to 30.
  ixmin = 1
  ixmax = 30
  ; The y values are from 1 to 30.
  iymin = 1
  iymax = 30
  ; The initial values for X and Y are both 15.
  ixinit = 15
  iyinit = 15

  ; Get the values kx and ky using the xyin opcode.
  kx, ky xyin iprd, ixmin, ixmax, iymin, iymax, ixinit, iyinit

  ; Print out the values of kx and ky.
  printks "kx=%f, ky=%f\\n", iprd, kx, ky

  ; Play an oscillator, use the x values for amplitude and
  ; the y values for frequency.
  kamp = kx * 1000
  kcps = ky * 220
  a1 oscil kamp, kcps, 1

  out a1
endin
/* xyin.orc */
        
\end{lstlisting}
\begin{lstlisting}
/* xyin.sco */
; Table #1, a sine wave.
f 1 0 16384 10 1

; Play Instrument #1 for 30 seconds.
i 1 0 30
e
/* xyin.sco */
        
\end{lstlisting}
 As the values of kx and ky change, they will be printed out like this: \begin{lstlisting}
kx=8.612036, ky=22.677933
kx=10.765685, ky=15.644135
      
\end{lstlisting}
\subsection*{Credits}


 Example written by Kevin Conder.
%\hline 


\begin{comment}
\begin{tabular}{lcr}
Previous &Home &Next \\
xtratim &Up &zacl

\end{tabular}


\end{document}
\end{comment}
