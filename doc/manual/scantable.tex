\begin{comment}
\documentclass[10pt]{article}
\usepackage{fullpage, graphicx, url}
\setlength{\parskip}{1ex}
\setlength{\parindent}{0ex}
\title{scantable}
\begin{document}


\begin{tabular}{ccc}
The Alternative Csound Reference Manual & & \\
Previous & &Next

\end{tabular}

%\hline 
\end{comment}
\section{scantable}
scantable�--� A simpler scanned synthesis implementation. \subsection*{Description}


  A simpler scanned synthesis implementation. This is an implementation of a circular string scanned using external tables. This opcode will allow direct modification and reading of values with the table opcodes. 
\subsection*{Syntax}


 aout \textbf{scantable}
 kamp, kpch, ipos, imass, istiff, idamp, ivel
\subsection*{Initialization}


 \emph{ipos}
 -- table containing position array. 


 \emph{imass}
 -- table containing the mass of the string. 


 \emph{istiff}
 -- table containing the stiffness of the string. 


 \emph{idamp}
 -- table containing the damping factors of the string. 


 \emph{ivel}
 -- table containing the velocities. 
\subsection*{Performance}


 \emph{kamp}
 -- amplitude (gain) of the string. 


 \emph{kpch}
 -- the string's scanned frequency. 
\subsection*{Examples}


  Here is an example of the scantable opcode. It uses the files \emph{scantable.orc}
 and \emph{scantable.sco}
. 


 \textbf{Example 1. Example of the scantable opcode.}

\begin{lstlisting}
/* scantable.orc */
; Initialize the global variables.
sr = 44100
kr = 4410
ksmps = 10
nchnls = 1

; Table #1 - initial position
git1 ftgen 1, 0, 128, 7, 0, 64, 1, 64, 0
; Table #2 - masses
git2 ftgen 2, 0, 128, -7, 1, 128, 1
; Table #3 - stiffness
git3 ftgen 3, 0, 128, -7, 0, 64, 100, 64, 0
; Table #4 - damping
git4 ftgen 4, 0, 128, -7, 1, 128, 1
; Table #5 - initial velocity
git5 ftgen 5, 0, 128, -7, 0, 128, 0

; Instrument #1.
instr 1
  kamp init 20000
  kpch init 220
  ipos = 1
  imass = 2
  istiff = 3
  idamp = 4
  ivel = 5

  a1 scantable kamp, kpch, ipos, imass, istiff, idamp, ivel
  a2 dcblock a1

  out a2
endin
/* scantable.orc */
        
\end{lstlisting}
\begin{lstlisting}
/* scantable.sco */
; Play Instrument #1 for ten seconds.
i 1 0 10
e
/* scantable.sco */
        
\end{lstlisting}
\subsection*{See Also}


 \emph{scanhammer}

\subsection*{Credits}


 


 


\begin{tabular}{cc}
Author: Matt Gilliard &April 2002

\end{tabular}



 


 Example written by Kevin Conder.


 New in version 4.20
%\hline 


\begin{comment}
\begin{tabular}{lcr}
Previous &Home &Next \\
scans &Up &scanu

\end{tabular}


\end{document}
\end{comment}
