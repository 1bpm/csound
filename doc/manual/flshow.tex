\begin{comment}
\documentclass[10pt]{article}
\usepackage{fullpage, graphicx, url}
\setlength{\parskip}{1ex}
\setlength{\parindent}{0ex}
\title{FLshow}
\begin{document}


\begin{tabular}{ccc}
The Alternative Csound Reference Manual & & \\
Previous & &Next

\end{tabular}

%\hline 
\end{comment}
\section{FLshow}
FLshow�--� Restores the visibility of a previously hidden FLTK widget. \subsection*{Description}


 \emph{FLshow}
 restores the visibility of a previously hidden widget. 
\subsection*{Syntax}


 \textbf{FLshow}
 ihandle
\subsection*{Initialization}


 \emph{ihandle}
 -- an integer number (used as unique identifier) taken from the output of a previously located widget opcode (which corresponds to the target widget). It is used to unequivocally identify the widget when modifying its appearance with this class of opcodes. The user must not set the \emph{ihandle}
 value directly, otherwise a Csound crash will occur. 
\subsection*{See Also}


 \emph{FLcolor}
, \emph{FLcolor2}
, \emph{FLhide}
, \emph{FLlabel}
, \emph{FLsetAlign}
, \emph{FLsetBox}
, \emph{FLsetColor}
, \emph{FLsetColor2}
, \emph{FLsetFont}
, \emph{FLsetPosition}
, \emph{FLsetSize}
, \emph{FLsetText}
, \emph{FLsetTextColor}
, \emph{FLsetTextSize}
, \emph{FLsetTextType}
, \emph{FLsetVal\_i}
, \emph{FLsetVal}

\subsection*{Credits}


 Author: Gabriel Maldonado


 New in version 4.22
%\hline 


\begin{comment}
\begin{tabular}{lcr}
Previous &Home &Next \\
FLsetVal &Up &FLslidBnk

\end{tabular}


\end{document}
\end{comment}
