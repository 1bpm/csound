\begin{comment}
\documentclass[10pt]{article}
\usepackage{fullpage, graphicx, url}
\setlength{\parskip}{1ex}
\setlength{\parindent}{0ex}
\title{planet}
\begin{document}


\begin{tabular}{ccc}
The Alternative Csound Reference Manual & & \\
Previous & &Next

\end{tabular}

%\hline 
\end{comment}
\section{planet}
planet�--� Simulates a planet orbiting in a binary star system. \subsection*{Description}


 \emph{planet}
 simulates a planet orbiting in a binary star system. The outputs are the x, y and z coordinates of the orbiting planet. It is possible for the planet to achieve escape velocity by a close encounter with a star. This makes this system somewhat unstable. 
\subsection*{Syntax}


 ax, ay, az \textbf{planet}
 kmass1, kmass2, ksep, ix, iy, iz, ivx, ivy, ivz, idelta [, ifriction]
\subsection*{Initialization}


 \emph{ix, iy, iz}
 -- the initial x, y and z coordinates of the planet 


 \emph{ivx, ivy, ivz}
 -- the initial velocity vector components for the planet. 


 \emph{idelta}
 -- the step size used to approximate the differential equation. 


 \emph{ifriction}
 (optional, default=0) -- a value for friction, which can used to keep the system from blowing up 
\subsection*{Performance}


 \emph{ax, ay, az}
 -- the output x, y, and z coodinates of the planet 


 \emph{kmass1}
 -- the mass of the first star 


 \emph{kmass2}
 -- the mass of the second star 
\subsection*{Examples}


  Here is an example of the planet opcode. It uses the files \emph{planet.orc}
 and \emph{planet.sco}
. 


 \textbf{Example 1. Example of the planet opcode.}

\begin{lstlisting}
/* planet.orc */
; Initialize the global variables.
sr = 44100
kr = 44100
ksmps = 1
nchnls = 2

; Instrument #1 - a planet oribiting in 3D space.
instr 1
  ; Create a basic tone.
  kamp init 5000
  kcps init 440
  ifn = 1
  asnd oscil kamp, kcps, ifn

  ; Figure out its X, Y, Z coordinates.
  km1 init 0.5
  km2 init 0.35
  ksep init 2.2
  ix = 0
  iy = 0.1
  iz = 0
  ivx = 0.5
  ivy = 0
  ivz = 0
  ih = 0.0003
  ifric = -0.1
  ax1, ay1, az1 planet km1, km2, ksep, ix, iy, iz, \
                       ivx, ivy, ivz, ih, ifric

  ; Place the basic tone within 3D space.
  kx downsamp ax1
  ky downsamp ay1
  kz downsamp az1
  idist = 1
  ift = 0
  imode = 1
  imdel = 1.018853416
  iovr = 2
  aw2, ax2, ay2, az2 spat3d asnd, kx, ky, kz, idist, \
                            ift, imode, imdel, iovr

  ; Convert the 3D sound to stereo.
  aleft = aw2 + ay2
  aright = aw2 - ay2

  outs aleft, aright
endin
/* planet.orc */
        
\end{lstlisting}
\begin{lstlisting}
/* planet.sco */
; Table #1 a sine wave.
f 1 0 16384 10 1

; Play Instrument #1 for 10 seconds.
i 1 0 10
e
/* planet.sco */
        
\end{lstlisting}
\subsection*{Credits}


 


 


\begin{tabular}{cc}
Author: Hans Mikelson &December 1998

\end{tabular}



 


 New in Csound version 3.50
%\hline 


\begin{comment}
\begin{tabular}{lcr}
Previous &Home &Next \\
pitchamdf &Up &pluck

\end{tabular}


\end{document}
\end{comment}
