\begin{comment}
\documentclass[10pt]{article}
\usepackage{fullpage, graphicx, url}
\setlength{\parskip}{1ex}
\setlength{\parindent}{0ex}
\title{specdisp}
\begin{document}


\begin{tabular}{ccc}
The Alternative Csound Reference Manual & & \\
Previous & &Next

\end{tabular}

%\hline 
\end{comment}
\section{specdisp}
specdisp�--� Displays the magnitude values of the spectrum. \subsection*{Description}


  Displays the magnitude values of the spectrum. 
\subsection*{Syntax}


 \textbf{specdisp}
 wsig, iprd [, iwtflg]
\subsection*{Initialization}


 \emph{iprd}
 -- the period, in seconds, of each new display. 


 \emph{iwtflg}
 (optional, default=0) -- wait flag. If non-zero, hold each display until released by the user. The default value is 0 (no wait). 
\subsection*{Performance}


 \emph{wsig}
 -- the input spectrum. 


  Displays the magnitude values of spectrum \emph{wsig}
 every \emph{iprd}
 seconds (rounded to some integral number of \emph{wsig}
's originating \emph{iprd}
). 
\subsection*{Examples}


 


 
\begin{lstlisting}
  ksum     \emph{specsum}
   wsig,  1                    ; sum the spec bins, and ksmooth
           \emph{if}
        ksum < 2000   \emph{kgoto}
  zero   ; if sufficient amplitude
  koct     \emph{specptrk}
  wsig                        ;    pitch-track the signal
           \emph{kgoto}
      contin
zero:  
  koct    =     0                                ; else output zero
contin:
        
\end{lstlisting}


 
\subsection*{See Also}


 \emph{specsum}

%\hline 


\begin{comment}
\begin{tabular}{lcr}
Previous &Home &Next \\
specdiff &Up &specfilt

\end{tabular}


\end{document}
\end{comment}
