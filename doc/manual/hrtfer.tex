\begin{comment}
\documentclass[10pt]{article}
\usepackage{fullpage, graphicx, url}
\setlength{\parskip}{1ex}
\setlength{\parindent}{0ex}
\title{hrtfer}
\begin{document}


\begin{tabular}{ccc}
The Alternative Csound Reference Manual & & \\
Previous & &Next

\end{tabular}

%\hline 
\end{comment}
\section{hrtfer}
hrtfer�--� Creates 3D audio for two speakers. \subsection*{Description}


  Output is binaural (headphone) 3D audio. 
\subsection*{Syntax}


 aleft, aright \textbf{hrtfer}
 asig, kaz, kelev, ``HRTFcompact''
\subsection*{Initialization}


 \emph{kAz}
 -- azimuth value in degrees. Positive values represent position on the right, negative values are positions on the left. 


 \emph{kElev}
 -- elevation value in degrees. Positive values represent position above horizontal, negative values are positions above horizontal. 


  At present, the only file which can be used with \emph{hrtfer}
 is \emph{HRTFcompact}
. It must be passed to the opcode as the last argument within quotes as shown above. 


  HRTFcompact may also be obtained via anonymous ftp from: \emph{\url{ftp://ftp.cs.bath.ac.uk/pub/dream/utilities/Analysis/HRTFcompact}}

\subsection*{Performance}


  These unit generators place a mono input signal in a virtual 3D space around the listener by convolving the input with the appropriate HRTF data specified by the opcode's azimuth and elevation values. \emph{hrtfer}
 allows these values to be k-values, allowing for dynamic spatialization. \emph{hrtfer}
 can only place the input at the requested position because the HRTF is loaded in at i-time (remember that currently, CSound has a limit of 20 files it can hold in memory, otherwise it causes a segmentation fault). The output will need to be scaled either by using balance or by multiplying the output by some scaling constant. 


 


\begin{tabular}{cc}
\textbf{Note}
 \\
� &

  The sampling rate of the orchestra must be 44.1kHz. This is because 44.1kHz is the sampling rate at which the HRTFs were measured. In order to be used at a different rate, the HRTFs would need to be re-sampled at the desired rate. 


\end{tabular}

\subsection*{Examples}


  Here is an example of the hrtfer opcode. It uses the files \emph{hrtfer.orc}
, \emph{hrtfer.sco}
, \emph{HRTFcompact}
, and \emph{beats.wav}
. 


 \textbf{Example 1. Example of the hrtfer opcode.}

\begin{lstlisting}
/* hrtfer.orc */
; Initialize the global variables.
sr = 44100
kr = 4410
ksmps = 10
nchnls = 2

instr 1
  kaz          linseg 0, p3, -360  ; move the sound in circle
  kel          linseg -40, p3, 45  ; around the listener, changing
                                    ; elevation as its turning
  asrc         soundin "beats.wav"
  aleft,aright hrtfer asrc, kaz, kel, "HRTFcompact"
  aleftscale   = aleft * 200
  arightscale  = aright * 200

  outs         aleftscale, arightscale
endin        
/* hrtfer.orc */
        
\end{lstlisting}
\begin{lstlisting}
/* hrtfer.sco */
i 1 0 2
e
/* hrtfer.sco */
        
\end{lstlisting}
\subsection*{Credits}


 


 


\begin{tabular}{ccc}
Authors: Eli Breder and David MacIntyre &Montreal &1996

\end{tabular}



 


 Fixed the example thanks to a message from Istvan Varga.
%\hline 


\begin{comment}
\begin{tabular}{lcr}
Previous &Home &Next \\
hilbert &Up &hsboscil

\end{tabular}


\end{document}
\end{comment}
