\begin{comment}
\documentclass[10pt]{article}
\usepackage{fullpage, graphicx, url}
\setlength{\parskip}{1ex}
\setlength{\parindent}{0ex}
\title{noise}
\begin{document}


\begin{tabular}{ccc}
The Alternative Csound Reference Manual & & \\
Previous & &Next

\end{tabular}

%\hline 
\end{comment}
\section{noise}
noise�--� A white noise generator with an IIR lowpass filter. \subsection*{Description}


  A white noise generator with an IIR lowpass filter. 
\subsection*{Syntax}


 ar \textbf{noise}
 xamp, kbeta
\subsection*{Initialization}


 \emph{ioffset}
 -- the delay before the first impulse. If it is negative, the value is taken as the number of samples, otherwise it is in seconds. Default is zero. 
\subsection*{Performance}


 \emph{xamp}
 -- amplitude of final output 


 \emph{kbeta}
 -- beta of the lowpass filter. Should be in the range of 0 to 1. 


  The filter equation is: 


 y\_n�=�sqrt(1-beta\^{}2)�*�x\_n�+�beta�Y\_(n-1)\\ 
 ������
 where \emph{x\_n}
 is white noise. \subsection*{Examples}


  Here is an example of the noise opcode. It uses the files \emph{noise.orc}
 and \emph{noise.sco}
. 


 \textbf{Example 1. Example of the noise opcode.}

\begin{lstlisting}
/* noise.orc */
; Initialize the global variables.
sr = 44100
kr = 4410
ksmps = 10
nchnls = 1

; Instrument #1.
instr 1
  kamp = 30000

  ; Change the beta value linearly from 0 to 1.
  kbeta line 0, p3, 1

  a1 noise kamp, kbeta
  out a1
endin
/* noise.orc */
        
\end{lstlisting}
\begin{lstlisting}
/* noise.sco */
; Play Instrument #1 for one second.
i 1 0 1
e
/* noise.sco */
        
\end{lstlisting}
\subsection*{Credits}


 


 


\begin{tabular}{cccc}
Author: John ffitch &University of Bath, Codemist. Ltd. &Bath, UK &December 2000

\end{tabular}



 


 Example written by Kevin Conder.


 New in Csound version 4.10
%\hline 


\begin{comment}
\begin{tabular}{lcr}
Previous &Home &Next \\
nlfilt &Up &noteoff

\end{tabular}


\end{document}
\end{comment}
