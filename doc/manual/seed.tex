\begin{comment}
\documentclass[10pt]{article}
\usepackage{fullpage, graphicx, url}
\setlength{\parskip}{1ex}
\setlength{\parindent}{0ex}
\title{seed}
\begin{document}


\begin{tabular}{ccc}
The Alternative Csound Reference Manual & & \\
Previous & &Next

\end{tabular}

%\hline 
\end{comment}
\section{seed}
seed�--� Sets the global seed value. \subsection*{Description}


  Sets the global seed value for all \emph{x-class noise generators}
, as well as other opcodes that use a random call, such as \emph{grain}
. \emph{rand}
, \emph{randi}
, \emph{randh}
, \emph{rnd}
(x), and \emph{birnd}
(x) are not affected by seed. 
\subsection*{Syntax}


 \textbf{seed}
 ival
\subsection*{Performance}


  Use of \emph{seed}
 will provide predictable results from an orchestra using with random generators, when required from multiple performances. 


  When specifying a seed value, \emph{ival}
 should be an integer between 0 and 232. If \emph{ival}
 = 0, the value of \emph{ival}
 will be derived from the system clock. 
%\hline 


\begin{comment}
\begin{tabular}{lcr}
Previous &Home &Next \\
schedwhen &Up &sekere

\end{tabular}


\end{document}
\end{comment}
