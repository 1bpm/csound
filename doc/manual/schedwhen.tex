\begin{comment}
\documentclass[10pt]{article}
\usepackage{fullpage, graphicx, url}
\setlength{\parskip}{1ex}
\setlength{\parindent}{0ex}
\title{schedwhen}
\begin{document}


\begin{tabular}{ccc}
The Alternative Csound Reference Manual & & \\
Previous & &Next

\end{tabular}

%\hline 
\end{comment}
\section{schedwhen}
schedwhen�--� Adds a new score event. \subsection*{Description}


  Adds a new score event. 
\subsection*{Syntax}


 \textbf{schedwhen}
 ktrigger, kinsnum, kwhen, kdur [, ip4] [, ip5] [...]


 \textbf{schedwhen}
 ktrigger, ``insname'', kwhen, kdur [, ip4] [, ip5] [...]
\subsection*{Initialization}


 \emph{ip4, ip5, ...}
 -- Equivalent to p4, p5, etc., in a score \emph{i statement}
. 
\subsection*{Performance}


 \emph{kinsnum}
 -- instrument number. Equivalent to p1 in a score \emph{i statement}
. 


 \emph{``insname''}
 -- A string (in double-quotes) representing a named instrument. 


 \emph{ktrigger}
 -- trigger value for new event 


 \emph{kwhen}
 -- start time of the new event. Equivalent to p2 in a score \emph{i statement}
. 


 \emph{kdur}
 -- duration of event. Equivalent to p3 in a score \emph{i statement}
. 


 \emph{schedwhen}
 adds a new score event. The event is only scheduled when the k-rate value \emph{ktrigger}
 is first non-zero. The arguments, including options, are the same as in a score. The \emph{iwhen}
 time (p2) is measured from the time of this event. 


  If the duration is zero or negative the new event is of MIDI type, and inherits the release sub-event from the scheduling instruction. 


 


\begin{tabular}{cc}
Warning &\textbf{Warning}
 \\
� &

 Support for named instruments is broken in version 4.23


\end{tabular}

\subsection*{Examples}


  Here is an example of the schedwhen opcode. It uses the files \emph{schedwhen.orc}
 and \emph{schedwhen.sco}
. 


 \textbf{Example 1. Example of the schedwhen opcode.}

\begin{lstlisting}
/* schedwhen.orc */
; Initialize the global variables.
sr = 44100
kr = 44100
ksmps = 1
nchnls = 1

; Instrument #1 - oscillator with a high note.
instr 1
  ; Use the fourth p-field as the trigger.
  ktrigger = p4
  kinsnum = 2
  kwhen = 0
  kdur = p3

  ; Play Instrument #2 at the same time, if the trigger is set.
  schedwhen ktrigger, kinsnum, kwhen, kdur

  ; Play a high note.
  a1 oscils 10000, 880, 1
  out a1
endin

; Instrument #2 - oscillator with a low note.
instr 2
  ; Play a low note.
  a1 oscils 10000, 220, 1
  out a1
endin
/* schedwhen.orc */
        
\end{lstlisting}
\begin{lstlisting}
/* schedwhen.sco */
; Table #1, a sine wave.
f 1 0 16384 10 1

; p4 = trigger for Instrument #2 (when p4 > 0).
; Play Instrument #1 for half a second, trigger Instrument #2.
i 1 0 0.5 1
; Play Instrument #1 for half a second, no trigger.
i 1 1 0.5 0
e
/* schedwhen.sco */
        
\end{lstlisting}
\subsection*{See Also}


 \emph{schedule}

\subsection*{Credits}


 


 


\begin{tabular}{cccc}
Author: John ffitch &University of Bath/Codemist Ltd. &Bath, UK &November 1998

\end{tabular}



 


 Example written by Kevin Conder.


 New in Csound version 3.491


 Based on work by Gabriel Maldonado
%\hline 


\begin{comment}
\begin{tabular}{lcr}
Previous &Home &Next \\
schedule &Up &seed

\end{tabular}


\end{document}
\end{comment}
