\begin{comment}
\documentclass[10pt]{article}
\usepackage{fullpage, graphicx, url}
\setlength{\parskip}{1ex}
\setlength{\parindent}{0ex}
\title{elseif}
\begin{document}


\begin{tabular}{ccc}
The Alternative Csound Reference Manual & & \\
Previous & &Next

\end{tabular}

%\hline 
\end{comment}
\section{elseif}
elseif�--� Defines another ``if...then'' condition when a ``if...then'' condition is false. \subsection*{Description}


  Defines another ``if...then'' condition when a ``if...then'' condition is false. 
\subsection*{Syntax}


 \textbf{elseif}
 xa R xb \textbf{then}



  where \emph{label}
 is in the same instrument block and is not an expression, and where \emph{R}
 is one of the Relational operators (\emph{$<$}
, \emph{=}
, \emph{$<$=}
, \emph{==}
, \emph{!=}
) (and \emph{=}
 for convenience, see also under \emph{Conditional Values}
). 
\subsection*{Performance}


 \emph{elseif}
 is used inside of a block of code between the \emph{``if...then''}
 and \emph{endif}
 opcodes. When a ``if...then'' condition is false, it defines another ``if...then'' condition to be met. Any number of \emph{elseif}
 statements are allowed. 
\subsection*{Examples}


  See the example for the \emph{if}
 opcode. 
\subsection*{See Also}


 \emph{else}
, \emph{endif}
, \emph{goto}
, \emph{if}
, \emph{igoto}
, \emph{kgoto}
, \emph{tigoto}
, \emph{timout}

\subsection*{Credits}


 New in version 4.21
%\hline 


\begin{comment}
\begin{tabular}{lcr}
Previous &Home &Next \\
else &Up &endif

\end{tabular}


\end{document}
\end{comment}
