\begin{comment}
\documentclass[10pt]{article}
\usepackage{fullpage, graphicx, url}
\setlength{\parskip}{1ex}
\setlength{\parindent}{0ex}
\title{random}
\begin{document}


\begin{tabular}{ccc}
The Alternative Csound Reference Manual & & \\
Previous & &Next

\end{tabular}

%\hline 
\end{comment}
\section{random}
random�--� Generates is a controlled pseudo-random number series between min and max values. \subsection*{Description}


  Generates is a controlled pseudo-random number series between min and max values. 
\subsection*{Syntax}


 ar \textbf{random}
 kmin, kmax


 ir \textbf{random}
 imin, imax


 kr \textbf{random}
 kmin, kmax
\subsection*{Initialization}


 \emph{imin}
 -- minimum range limit 


 \emph{imax}
 -- maximum range limit 
\subsection*{Performance}


 \emph{kmin}
 -- minimum range limit 


 \emph{kmax}
 -- maximum range limit 


  The \emph{random}
 opcode is similar to \emph{linrand}
 and \emph{trirand}
 but sometimes I [Gabriel Maldonado] find it more convenient because allows the user to set arbitrary minimum and maximum values. 
\subsection*{Examples}


  Here is an example of the random opcode. It uses the files \emph{random.orc}
 and \emph{random.sco}
. 


 \textbf{Example 1. Example of the random opcode.}

\begin{lstlisting}
/* random.orc */
; Initialize the global variables.
sr = 44100
kr = 4410
ksmps = 10
nchnls = 1

; Instrument #1.
instr 1
  ; Generate a random number between 220 and 440.
  kmin init 220
  kmax init 440
  k1 random kmin, kmax

  printks "k1 = %f\\n", 0.1, k1
endin
/* random.orc */
        
\end{lstlisting}
\begin{lstlisting}
/* random.sco */
; Play Instrument #1 for one second.
i 1 0 1
e
/* random.sco */
        
\end{lstlisting}
 Its output should include lines like: \begin{lstlisting}
k1 = 414.232056
k1 = 419.393402
k1 = 275.376373
      
\end{lstlisting}
\subsection*{See Also}


 \emph{linrand}
, \emph{randomh}
, \emph{randomi}
, \emph{trirand}

\subsection*{Credits}


 Author: Gabriel Maldonado


 Example written by Kevin Conder.
%\hline 


\begin{comment}
\begin{tabular}{lcr}
Previous &Home &Next \\
randi &Up &randomh

\end{tabular}


\end{document}
\end{comment}
