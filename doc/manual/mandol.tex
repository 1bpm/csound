\begin{comment}
\documentclass[10pt]{article}
\usepackage{fullpage, graphicx, url}
\setlength{\parskip}{1ex}
\setlength{\parindent}{0ex}
\title{mandol}
\begin{document}


\begin{tabular}{ccc}
The Alternative Csound Reference Manual & & \\
Previous & &Next

\end{tabular}

%\hline 
\end{comment}
\section{mandol}
mandol�--� An emulation of a mandolin. \subsection*{Description}


  An emulation of a mandolin. 
\subsection*{Syntax}


 ar \textbf{mandol}
 kamp, kfreq, kpluck, kdetune, kgain, ksize, ifn [, iminfreq]
\subsection*{Initialization}


 \emph{ifn}
 -- table number containing the pluck wave form. The file \emph{mandpluk.aiff}
 is suitable for this. It is also available at \emph{\url{ftp://ftp.cs.bath.ac.uk/pub/dream/documentation/sounds/modelling/}}
. 


 \emph{iminfreq}
 (optional, default=0) -- Lowest frequency to be played on the note. If it is omitted it is taken to be the same as the initial \emph{kfreq}
. 
\subsection*{Performance}


 \emph{kamp}
 -- Amplitude of note. 


 \emph{kfreq}
 -- Frequency of note played. 


 \emph{kpluck}
 -- The pluck position, in range 0 to 1. Suggest 0.4. 


 \emph{kdetune }
 -- The proportional detuning between the two strings. Suggested range 0.9 to 1. 


 \emph{kgain}
 -- the loop gain of the model, in the range 0.97 to 1. 


 \emph{ksize}
 -- The size of the body of the mandolin. Range 0 to 2. 
\subsection*{Examples}


  Here is an example of the mandol opcode. It uses the files \emph{mandol.orc}
, \emph{mandol.sco}
, and \emph{mandpluk.aiff}
. 


 \textbf{Example 1. Example of the mandol opcode.}

\begin{lstlisting}
/* mandol.orc */
; Initialize the global variables.
sr = 22050
kr = 2205
ksmps = 10
nchnls = 1

; Instrument #1.
instr 1
  ; kamp = 30000
  ; kfreq = 880
  ; kpluck = 0.4
  ; kdetune = 0.99
  ; kgain = 0.99
  ; ksize = 2
  ; ifn = 1
  ; ifreq = 220

  a1 mandol 30000, 880, 0.4, 0.99, 0.99, 2, 1, 220

  out a1
endin
/* mandol.orc */
        
\end{lstlisting}
\begin{lstlisting}
/* mandol.sco */
; Table #1: the "mandpluk.aiff" audio file
f 1 0 8192 1 "mandpluk.aiff" 0 0 0

; Play Instrument #1 for one second.
i 1 0 1
e
/* mandol.sco */
        
\end{lstlisting}
\subsection*{Credits}


 


 


\begin{tabular}{ccc}
Author: John ffitch (after Perry Cook) &University of Bath, Codemist Ltd. &Bath, UK

\end{tabular}



 


 Example written by Kevin Conder.


 New in Csound version 3.47
%\hline 


\begin{comment}
\begin{tabular}{lcr}
Previous &Home &Next \\
madsr &Up &marimba

\end{tabular}


\end{document}
\end{comment}
