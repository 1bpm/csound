\begin{comment}
\documentclass[10pt]{article}
\usepackage{fullpage, graphicx, url}
\setlength{\parskip}{1ex}
\setlength{\parindent}{0ex}
\title{ftlen}
\begin{document}


\begin{tabular}{ccc}
The Alternative Csound Reference Manual & & \\
Previous & &Next

\end{tabular}

%\hline 
\end{comment}
\section{ftlen}
ftlen�--� Returns the size of a stored function table. \subsection*{Description}


  Returns the size of a stored function table. 
\subsection*{Syntax}


 \textbf{ftlen}
(x) (init-rate args only)
\subsection*{Performance}


  Returns the size (number of points, excluding guard point) of stored function table, number \emph{x}
. While most units referencing a stored table will automatically take its size into account (so tables can be of arbitrary length), this function reports the actual size if that is needed. Note that \emph{ftlen}
 will always return a power-of-2 value, i.e. the function table guard point (see \emph{f Statement}
) is not included.As of Csound version 3.53, \emph{ftlen}
 works with deferred function tables (see \emph{GEN01}
). 
\subsection*{Examples}


  Here is an example of the ftlen opcode. It uses the files \emph{ftlen.orc}
, \emph{ftlen.sco}
, and \emph{mary.wav}
. 


 \textbf{Example 1. Example of the ftlen opcode.}

\begin{lstlisting}
/* ftlen.orc */
; Initialize the global variables.
sr = 44100
kr = 4410
ksmps = 10
nchnls = 1

; Instrument #1.
instr 1
  ; Print out the size of Table #1.
  ; The size will be the number of points excluding the guard point.
  ilen = ftlen(1)
  print ilen
endin
/* ftlen.orc */
        
\end{lstlisting}
\begin{lstlisting}
/* ftlen.sco */
; Table #1: Use an audio file, Csound will determine its size.
f 1 0 0 1 "mary.wav" 0 0 0

; Play Instrument #1 for 1 second.
i 1 0 1
e
/* ftlen.sco */
        
\end{lstlisting}
 The audio file ``mary.wav'' is 154390 samples long. The ftlen opcode reports it as 154389 samples long because it reserves 1 point for the guard point. Its output should include a line like this: \begin{lstlisting}
instr 1:  ilen = 154389.000
      
\end{lstlisting}
\subsection*{See Also}


 \emph{ftchnls}
, \emph{ftlptim}
, \emph{ftsr}
, \emph{nsamp}

\subsection*{Credits}


 


 


\begin{tabular}{cccc}
Author: Barry L. Vercoe &MIT &Cambridge, Massachussetts &1997

\end{tabular}



 


 Example written by Kevin Conder.
%\hline 


\begin{comment}
\begin{tabular}{lcr}
Previous &Home &Next \\
ftgen &Up &ftload

\end{tabular}


\end{document}
\end{comment}
