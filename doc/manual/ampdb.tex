\begin{comment}
\documentclass[10pt]{article}
\usepackage{fullpage, graphicx, url}
\setlength{\parskip}{1ex}
\setlength{\parindent}{0ex}
\title{ampdb}
\begin{document}


\begin{tabular}{ccc}
The Alternative Csound Reference Manual & & \\
Previous & &Next

\end{tabular}

%\hline 
\end{comment}
\section{ampdb}
ampdb�--� Returns the amplitude equivalent of the decibel value x. \subsection*{Description}


  Returns the amplitude equivalent of the decibel value x. Thus: 


 
\begin{itemize}
\item 

 60 dB = 1000

\item 

 66 dB = 1995.262

\item 

 72 dB = 3891.07

\item 

 78 dB = 7943.279

\item 

 84 dB = 15848.926

\item 

 90 dB = 31622.764


\end{itemize}
\subsection*{Syntax}


 \textbf{ampdb}
(x) (no rate restriction)
\subsection*{Examples}


  Here is an example of the ampdb opcode. It uses the files \emph{ampdb.orc}
 and \emph{ampdb.sco}
. 


 \textbf{Example 1. Example of the ampdb opcode.}

\begin{lstlisting}
/* ampdb.orc */
; Initialize the global variables.
sr = 44100
kr = 4410
ksmps = 10
nchnls = 1

; Instrument #1.
instr 1
  idb = 90
  iamp = ampdb(idb)

  print iamp
endin
/* ampdb.orc */
        
\end{lstlisting}
\begin{lstlisting}
/* ampdb.sco */
; Play Instrument #1 for one second.
i 1 0 1
e
/* ampdb.sco */
        
\end{lstlisting}
 Its output should include lines like: \begin{lstlisting}
instr 1:  iamp = 31622.764
      
\end{lstlisting}
\subsection*{See Also}


 \emph{ampdbfs}
, \emph{db}
, \emph{dbamp}
, \emph{dbfsamp}

\subsection*{Credits}


 Example written by Kevin Conder.
%\hline 


\begin{comment}
\begin{tabular}{lcr}
Previous &Home &Next \\
alpass &Up &ampdbfs

\end{tabular}


\end{document}
\end{comment}
