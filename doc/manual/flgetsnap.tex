\begin{comment}
\documentclass[10pt]{article}
\usepackage{fullpage, graphicx, url}
\setlength{\parskip}{1ex}
\setlength{\parindent}{0ex}
\title{FLgetsnap}
\begin{document}


\begin{tabular}{ccc}
The Alternative Csound Reference Manual & & \\
Previous & &Next

\end{tabular}

%\hline 
\end{comment}
\section{FLgetsnap}
FLgetsnap�--� Retrieves a previously stored FLTK snapshot. \subsection*{Description}


  Retrieves a previously stored snapshot (in memory), i.e. sets all valuator to the corresponding values stored in that snaphot. 
\subsection*{Syntax}


 inumsnap \textbf{FLgetsnap}
 index
\subsection*{Initialization}


 \emph{inumsnap}
 -- current number of snapshots. 


 \emph{index}
 -- a number referring unequivocally to a snapshot. Several snapshots can be stored in the same bank. 
\subsection*{Performance}


 \emph{FLgetsnap}
 retrieves a previously stored snapshot (in memory), i.e. sets all valuator to the corresponding values stored in that snapshot. The \emph{index}
 argument unequivocally must refer to an already existing snapshot. If the \emph{index}
 argument refers to an empty snapshot or to a snapshot that doesn't exist, no action is done. \emph{FLsetsnap}
 outputs the current number of snapshots (\emph{inumsnap}
 argument). 
\subsection*{See Also}


 \emph{FLloadsnap}
, \emph{FLrun}
, \emph{FLsavesnap}
, \emph{FLsetsnap}
, \emph{FLupdate}

\subsection*{Credits}


 Author: Gabriel Maldonado


 New in version 4.22
%\hline 


\begin{comment}
\begin{tabular}{lcr}
Previous &Home &Next \\
FLcount &Up &FLgroup

\end{tabular}


\end{document}
\end{comment}
