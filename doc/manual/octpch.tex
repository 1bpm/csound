\begin{comment}
\documentclass[10pt]{article}
\usepackage{fullpage, graphicx, url}
\setlength{\parskip}{1ex}
\setlength{\parindent}{0ex}
\title{octpch}
\begin{document}


\begin{tabular}{ccc}
The Alternative Csound Reference Manual & & \\
Previous & &Next

\end{tabular}

%\hline 
\end{comment}
\section{octpch}
octpch�--� Converts a pitch-class value to octave-point-decimal. \subsection*{Description}


  Converts a pitch-class value to octave-point-decimal. 
\subsection*{Syntax}


 \textbf{octpch}
 (pch) (init- or control-rate args only)


  where the argument within the parentheses may be a further expression. 
\subsection*{Performance}


  These are really \emph{value converters}
 with a special function of manipulating pitch data. 


  Data concerning pitch and frequency can exist in any of the following forms: 


 \textbf{Table 1. Pitch and Frequency Values}



\begin{tabular}{ccc}
NameAbbreviation & & \\
octave point pitch-class (8ve.pc)pch &octave point decimaloct &cycles per secondcps

\end{tabular}



  The first two forms consist of a whole number, representing octave registration, followed by a specially interpreted fractional part. For \emph{pch}
, the fraction is read as two decimal digits representing the 12 equal-tempered pitch classes from .00 for C to.11 for B. For \emph{oct}
, the fraction is interpreted as a true decimal fractional part of an octave. The two fractional forms are thus related by the factor 100/12. In both forms, the fraction is preceded by a whole number octave index such that 8.00 represents Middle C, 9.00 the C above, etc. Thus A440 can be represented alternatively by 440 (\emph{cps}
),8.09 (\emph{pch}
), or 8.75 (\emph{oct}
). Microtonal divisions of the \emph{pch}
 semitone can be encoded by using more than two decimal places. 


  The mnemonics of the pitch conversion units are derived from morphemes of the forms involved, the second morpheme describing the source and the first morpheme the object (result). Thus \emph{cpspch}
(8.09) will convert the pitch argument 8.09 to its \emph{cps}
 (or Hertz) equivalent, giving the value of 440. Since the argument is constant over the duration of the note, this conversion will take place at i-time, before any samples for the current note are produced. 


  By contrast, the conversion \emph{cpsoct}
(8.75 + k1) which gives the value of A440 transposed by the octave interval \emph{k1}
. The calculation will be repeated every k-period since that is the rate at which \emph{k1}
 varies. 


 


\begin{tabular}{cc}
\textbf{Note}
 \\
� &

  The conversion from \emph{pch}
 or \emph{oct}
 into \emph{cps}
 is not a linear operation but involves an exponential process that could be time-consuming when executed repeatedly. Csound now uses a built-in table lookup to do this efficiently, even at audio rates. 


\end{tabular}

\subsection*{Examples}


  Here is an example of the octpch opcode. It uses the files \emph{octpch.orc}
 and \emph{octpch.sco}
. 


 \textbf{Example 1. Example of the octpch opcode.}

\begin{lstlisting}
/* octpch.orc */
; Initialize the global variables.
sr = 44100
kr = 4410
ksmps = 10
nchnls = 1

; Instrument #1.
instr 1
  ; Convert a pitch-class value into an 
  ; octave-point-decimal value.
  ipch = 8.09
  ioct = octpch(ipch)

  print ioct
endin
/* octpch.orc */
        
\end{lstlisting}
\begin{lstlisting}
/* octpch.sco */
; Play Instrument #1 for one second.
i 1 0 1
e
/* octpch.sco */
        
\end{lstlisting}
 Its output should include a line like this: \begin{lstlisting}
instr 1:  ioct = 8.750
      
\end{lstlisting}
\subsection*{See Also}


 \emph{cpsoct}
, \emph{cpspch}
, \emph{octcps}
, \emph{pchoct}

\subsection*{Credits}


 Example written by Kevin Conder.
%\hline 


\begin{comment}
\begin{tabular}{lcr}
Previous &Home &Next \\
octmidib &Up &opcode

\end{tabular}


\end{document}
\end{comment}
