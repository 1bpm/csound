\begin{comment}
\documentclass[10pt]{article}
\usepackage{fullpage, graphicx, url}
\setlength{\parskip}{1ex}
\setlength{\parindent}{0ex}
\title{vbap8}
\begin{document}


\begin{tabular}{ccc}
The Alternative Csound Reference Manual & & \\
Previous & &Next

\end{tabular}

%\hline 
\end{comment}
\section{vbap8}
vbap8�--� Distributes an audio signal among 8 channels. \subsection*{Description}


  Distributes an audio signal among 8 channels. 
\subsection*{Syntax}


 ar1, ..., ar8 \textbf{vbap8}
 asig, iazim [, ielev] [, ispread]
\subsection*{Initialization}


 \emph{iazim}
 -- azimuth angle of the virtual source 


 \emph{ielev}
 (optional) -- elevation angle of the virtual source 


 \emph{ispread}
 (optional) -- spreading of the virtual source (range 0 - 100). If value is zero, conventional amplitude panning is used. When \emph{ispread}
 is increased, the number of loudspeakers used in panning increases. If value is 100, the sound is applied to all loudspeakers. 
\subsection*{Performance}


 \emph{asig}
 -- audio signal to be panned 


 \emph{vbap8}
 takes an input signal, \emph{asig}
, and distributes it among 8 outputs, according to the controls \emph{iazim}
 and \emph{ielev}
, and the configured loudspeaker placement. If \emph{idim}
 = 2, \emph{ielev}
 is set to zero. The distribution is performed using Vector Base Amplitude Panning (VBAP - See reference). VBAP distributes the signal using loudspeaker data configured with \emph{vbaplsinit}
. The signal is applied to, at most, two loudspeakers in 2-D loudspeaker configurations, and three loudspeakers in 3-D loudspeaker configurations. If the virtual source is panned outside the region spanned by loudspeakers, the nearest loudspeakers are used in panning. 
\subsection*{Examples}


 


 \textbf{Example 1. 2-D panning example with stationary virtual sources}

\begin{lstlisting}
  \emph{sr}
      =          4100
  \emph{kr}
      =           441
  \emph{ksmps}
   =           100
  \emph{nchnls}
  =             4
  \emph{vbaplsinit}
         2, 6,  0, 45, 90, 135, 200, 245, 290, 315 

          \emph{instr}
 1	           
  asig    \emph{oscil}
      20000, 440, 1                    
  a1,a2,a3,a4,a5,a6,a7,a8   \emph{vbap8}
  asig, p4, 0, 20 ;p4 = azimuth
	
  ;render twice with alternate \emph{outq}
 statements
  ;  to obtain two 4 channel .wav files:

          \emph{outq}
       a1,a2,a3,a4
  ;       \emph{outq}
       a5,a6,a7,a8
          \emph{endin}

        
\end{lstlisting}
\subsection*{Reference}


  Ville Pulkki: ``Virtual Sound Source Positioning Using Vector Base Amplitude Panning'' \emph{Journal of the Audio Engineering Society}
, 1997 June, Vol. 45/6, p. 456. 
\subsection*{See Also}


 \emph{vbap16}
, \emph{vbap16move}
, \emph{vbap4}
, \emph{vbap4move}
, \emph{vbap8move}
, \emph{vbaplsinit}
, \emph{vbapz}
, \emph{vbapzmove}

\subsection*{Credits}


 


 


\begin{tabular}{cccccc}
Author: Ville Pulkki &Sibelius Academy Computer Music Studio &Laboratory of Acoustics and Audio Signal Processing &Helsinki University of Technology &Helsinki, Finland &May 2000

\end{tabular}



 


 New in Csound Version 4.07
%\hline 


\begin{comment}
\begin{tabular}{lcr}
Previous &Home &Next \\
vbap4move &Up &vbap8move

\end{tabular}


\end{document}
\end{comment}
