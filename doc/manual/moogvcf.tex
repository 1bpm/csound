\begin{comment}
\documentclass[10pt]{article}
\usepackage{fullpage, graphicx, url}
\setlength{\parskip}{1ex}
\setlength{\parindent}{0ex}
\title{moogvcf}
\begin{document}


\begin{tabular}{ccc}
The Alternative Csound Reference Manual & & \\
Previous & &Next

\end{tabular}

%\hline 
\end{comment}
\section{moogvcf}
moogvcf�--� A digital emulation of the Moog diode ladder filter configuration. \subsection*{Description}


  A digital emulation of the Moog diode ladder filter configuration. 
\subsection*{Syntax}


 ar \textbf{moogvcf}
 asig, xfco, xres [, iscale]
\subsection*{Initialization}


 \emph{iscale}
 (optional, default=1) -- internal scaling factor. Use if \emph{asig}
 is not in the range +/-1. Input is first divided by \emph{iscale}
, then output is mutliplied \emph{iscale}
. Default value is 1. (New in Csound version 3.50) 
\subsection*{Performance}


 \emph{asig}
 -- input signal 


 \emph{xfco}
 -- filter cut-off frequency in Hz. As of version 3.50, may i-,k-, or a-rate. 


 \emph{xres}
 -- amount of resonance. Self-oscillation occurs when \emph{xres}
 is approximately one. As of version 3.50, may a-rate, i-rate, or k-rate. 


 \emph{moogvcf}
 is a digital emulation of the Moog diode ladder filter configuration. This emulation is based loosely on the paper ``Analyzing the Moog VCF with Considerations for Digital Implemnetation'' by Stilson and Smith (CCRMA). This version was originally coded in Csound by Josep Comajuncosas. Some modifications and conversion to C were done by Hans Mikelson 


 \emph{Note}
: This filter requires that the input signal be normalized to one. 
\subsection*{Examples}


  Here is an example of the moogvcf opcode. It uses the files \emph{moogvcf.orc}
 and \emph{moogvcf.sco}
. 


 \textbf{Example 1. Example of the moogvcf opcode.}

\begin{lstlisting}
/* moogvcf.orc */
; Initialize the global variables.
sr = 44100
kr = 4410
ksmps = 10
nchnls = 1

; Instrument #1.
instr 1
  ; Use a nice sawtooth waveform.
  asig vco 32000, 220, 1

  ; Vary the filter-cutoff frequency from .2 to 2 KHz.
  kfco line 200, p3, 2000

  ; Set the resonance amount to one.
  krez init 1

  ; Scale the amplitude to 32768.
  iscale = 32768

  a1 moogvcf asig, kfco, krez, iscale

  out a1
endin
/* moogvcf.orc */
        
\end{lstlisting}
\begin{lstlisting}
/* moogvcf.sco */
; Table #1, a sine wave for the vco opcode.
f 1 0 16384 10 1

; Play Instrument #1 for three seconds.
i 1 0 3
e
/* moogvcf.sco */
        
\end{lstlisting}
\subsection*{See Also}


 \emph{biquad}
, \emph{rezzy}

\subsection*{Credits}


 


 


\begin{tabular}{cc}
Author: Hans Mikelson &October 1998

\end{tabular}



 


 Example written by Kevin Conder.


 New in Csound version 3.49
%\hline 


\begin{comment}
\begin{tabular}{lcr}
Previous &Home &Next \\
moog &Up &moscil

\end{tabular}


\end{document}
\end{comment}
