\begin{comment}
\documentclass[10pt]{article}
\usepackage{fullpage, graphicx, url}
\setlength{\parskip}{1ex}
\setlength{\parindent}{0ex}
\title{downsamp}
\begin{document}


\begin{tabular}{ccc}
The Alternative Csound Reference Manual & & \\
Previous & &Next

\end{tabular}

%\hline 
\end{comment}
\section{downsamp}
downsamp�--� Modify a signal by down-sampling. \subsection*{Description}


  Modify a signal by down-sampling. 
\subsection*{Syntax}


 kr \textbf{downsamp}
 asig [, iwlen]
\subsection*{Initialization}


 \emph{iwlen}
 (optional) -- window length in samples over which the audio signal is averaged to determine a downsampled value. Maximum length is \emph{ksmps}
; 0 and 1 imply no window averaging. The default value is 0. 
\subsection*{Performance}


 \emph{downsamp}
 converts an audio signal to a control signal by downsampling. It produces one kval for each audio control period. The optional window invokes a simple averaging process to suppress foldover. 
\subsection*{Examples}


  Here is an example of the downsamp opcode. It uses the files \emph{downsamp.orc}
 and \emph{downsamp.sco}
. 


 \textbf{Example 1. Example of the downsamp opcode.}

\begin{lstlisting}
/* downsamp.orc */
; Initialize the global variables.
sr = 44100
kr = 4410
ksmps = 10
nchnls = 1

; Instrument #1.
instr 1
  ; Create a noise signal at a-rate.
  anoise noise 20000, 0.2

  ; Downsample the noise signal to k-rate.
  knoise downsamp anoise

  ; Use the noise signal at k-rate.
  a1 oscil 30000, knoise, 1
  out anoise
endin
/* downsamp.orc */
        
\end{lstlisting}
\begin{lstlisting}
/* downsamp.sco */
; Table #1, a sine wave.
f 1 0 16384 10 1

; Play Instrument #1 for one second.
i 1 0 1
e
/* downsamp.sco */
        
\end{lstlisting}
\subsection*{See Also}


 \emph{diff}
, \emph{integ}
, \emph{interp}
, \emph{samphold}
, \emph{upsamp}

\subsection*{Credits}


 Example written by Kevin Conder.
%\hline 


\begin{comment}
\begin{tabular}{lcr}
Previous &Home &Next \\
divz &Up &dripwater

\end{tabular}


\end{document}
\end{comment}
