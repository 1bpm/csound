\begin{comment}
\documentclass[10pt]{article}
\usepackage{fullpage, graphicx, url}
\setlength{\parskip}{1ex}
\setlength{\parindent}{0ex}
\title{endin}
\begin{document}


\begin{tabular}{ccc}
The Alternative Csound Reference Manual & & \\
Previous & &Next

\end{tabular}

%\hline 
\end{comment}
\section{endin}
endin�--� Ends the current instrument block. \subsection*{Description}


  Ends the current instrument block. 
\subsection*{Syntax}


 \textbf{endin}

\subsection*{Initialization}


  Ends the current instrument block. 


  Instruments can be defined in any order (but they will always be both initialized and performed in ascending instrument number order). Instrument blocks cannot be nested (i.e. one block cannot contain another). 


 


\begin{tabular}{cc}
\textbf{Note}
 \\
� &

  There may be any number of instrument blocks in an orchestra. 


\end{tabular}

\subsection*{Examples}


  Here is an example of the endin opcode. It uses the files \emph{endin.orc}
 and \emph{endin.sco}
. 


 \textbf{Example 1. Example of the endin opcode.}

\begin{lstlisting}
/* endin.orc */
; Initialize the global variables.
sr = 44100
kr = 4410
ksmps = 10
nchnls = 1

; Instrument #1.
instr 1
  iamp = 10000
  icps = 440
  iphs = 0

  a1 oscils iamp, icps, iphs
  out a1
endin
/* endin.orc */
        
\end{lstlisting}
\begin{lstlisting}
/* endin.sco */
; Play Instrument #1 for 2 seconds.
i 1 0 2
e
/* endin.sco */
        
\end{lstlisting}
\subsection*{See Also}


 \emph{instr}

\subsection*{Credits}


 Example written by Kevin Conder.
%\hline 


\begin{comment}
\begin{tabular}{lcr}
Previous &Home &Next \\
endif &Up &endop

\end{tabular}


\end{document}
\end{comment}
