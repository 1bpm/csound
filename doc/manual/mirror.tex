\begin{comment}
\documentclass[10pt]{article}
\usepackage{fullpage, graphicx, url}
\setlength{\parskip}{1ex}
\setlength{\parindent}{0ex}
\title{mirror}
\begin{document}


\begin{tabular}{ccc}
The Alternative Csound Reference Manual & & \\
Previous & &Next

\end{tabular}

%\hline 
\end{comment}
\section{mirror}
mirror�--� Reflects the signal that exceeds the low and high thresholds. \subsection*{Description}


  Reflects the signal that exceeds the low and high thresholds. 
\subsection*{Syntax}


 ar \textbf{mirror}
 asig, klow, khigh


 ir \textbf{mirror}
 isig, ilow, ihigh


 kr \textbf{mirror}
 ksig, klow, khigh
\subsection*{Initialization}


 \emph{isig}
 -- input signal 


 \emph{ilow}
 -- low threshold 


 \emph{ihigh}
 -- high threshold 
\subsection*{Performance}


 \emph{xsig}
 -- input signal 


 \emph{klow}
 -- low threshold 


 \emph{khigh}
 -- high threshold 


 \emph{mirror}
 ``reflects'' the signal that exceeds the low and high thresholds. 


  This opcode is useful in several situations, such as table indexing or for clipping and modeling a-rate, i-rate or k-rate signals. 
\subsection*{See Also}


 \emph{limit}
, \emph{wrap}

\subsection*{Credits}


 


 


\begin{tabular}{cc}
Author: Gabriel Maldonado &Italy

\end{tabular}



 


 New in Csound version 3.49
%\hline 


\begin{comment}
\begin{tabular}{lcr}
Previous &Home &Next \\
midiprogramchange &Up &moog

\end{tabular}


\end{document}
\end{comment}
