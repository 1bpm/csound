\begin{comment}
\documentclass[10pt]{article}
\usepackage{fullpage, graphicx, url}
\setlength{\parskip}{1ex}
\setlength{\parindent}{0ex}
\title{goto}
\begin{document}


\begin{tabular}{ccc}
The Alternative Csound Reference Manual & & \\
Previous & &Next

\end{tabular}

%\hline 
\end{comment}
\section{goto}
goto�--� Transfer control on every pass. \subsection*{Description}


  Transfer control to \emph{label}
 on every pass. (Combination of \emph{igoto}
 and \emph{kgoto}
) 
\subsection*{Syntax}


 \textbf{goto}
 label


  where \emph{label}
 is in the same instrument block and is not an expression, and where \emph{R}
 is one of the Relational operators (\emph{$<$}
,\emph{ =}
, \emph{$<$=}
, \emph{==}
, \emph{!=}
) (and \emph{=}
 for convenience, see also under \emph{Conditional Values}
). 
\subsection*{Examples}


  Here is an example of the goto opcode. It uses the files \emph{goto.orc}
 and \emph{goto.sco}
. 


 \textbf{Example 1. Example of the goto opcode.}

\begin{lstlisting}
/* goto.orc */
; Initialize the global variables.
sr = 44100
kr = 4410
ksmps = 10
nchnls = 1

; Instrument #1.
instr 1
  a1 oscil 10000, 440, 1
  goto playit

  ; The goto will go to the playit label.
  ; It will skip any code in between like this comment.

playit:
  out a1
endin
/* goto.orc */
        
\end{lstlisting}
\begin{lstlisting}
/* goto.sco */
; Table #1: a simple sine wave.
f 1 0 32768 10 1

; Play Instrument #1 for one second.
i 1 0 1
e
/* goto.sco */
        
\end{lstlisting}
\subsection*{See Also}


 \emph{cggoto}
, \emph{cigoto}
, \emph{ckgoto}
, \emph{if}
, \emph{igoto}
, \emph{kgoto}
, \emph{tigoto}
, \emph{timout}

\subsection*{Credits}


 Example written by Kevin Conder.


 Added a note by Jim Aikin.
%\hline 


\begin{comment}
\begin{tabular}{lcr}
Previous &Home &Next \\
gogobel &Up &grain

\end{tabular}


\end{document}
\end{comment}
