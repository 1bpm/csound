\begin{comment}
\documentclass[10pt]{article}
\usepackage{fullpage, graphicx, url}
\setlength{\parskip}{1ex}
\setlength{\parindent}{0ex}
\title{grain}
\begin{document}


\begin{tabular}{ccc}
The Alternative Csound Reference Manual & & \\
Previous & &Next

\end{tabular}

%\hline 
\end{comment}
\section{grain}
grain�--� Generates granular synthesis textures. \subsection*{Description}


  Generates granular synthesis textures. 
\subsection*{Syntax}


 ar \textbf{grain}
 xamp, xpitch, xdens, kampoff, kpitchoff, kgdur, igfn, iwfn, imgdur [, igrnd]
\subsection*{Initialization}


 \emph{igfn}
 -- The ftable number of the grain waveform. This can be just a sine wave or a sampled sound. 


 \emph{iwfn}
 -- Ftable number of the amplitude envelope used for the grains (see also \emph{GEN20}
). 


 \emph{imgdur}
 -- Maximum grain duration in seconds. This the biggest value to be assigned to \emph{kgdur}
. 


 \emph{igrnd}
 (optional) -- if non-zero, turns off grain offset randomness. This means that all grains will begin reading from the beginning of the \emph{igfn}
 table. If zero (the default), grains will start reading from random \emph{igfn}
 table positions. 
\subsection*{Performance}


 \emph{xamp}
 -- Amplitude of each grain. 


 \emph{xpitch}
 -- Grain pitch. To use the original frequency of the input sound, use the formula: 


 ���sndsr�/�ftlen(\emph{igfn}
)\\ 
 ������


  where sndsr is the original sample rate of the \emph{igfn}
 sound. 


 \emph{xdens}
 -- Density of grains measured in grains per second. If this is constant then the output is synchronous granular synthesis, very similar to \emph{fof}
. If \emph{xdens}
 has a random element (like added noise), then the result is more like asynchronous granular synthesis. 


 \emph{kampoff}
 -- Maximum amplitude deviation from \emph{kamp}
. This means that the maximum amplitude a grain can have is \emph{kamp}
 + \emph{kampoff}
 and the minimum is \emph{kamp}
. If \emph{kampoff}
 is set to zero then there is no random amplitude for each grain. 


 \emph{kpitchoff}
 -- Maximum pitch deviation from \emph{kpitch}
 in Hz. Similar to \emph{kampoff}
. 


 \emph{kgdur}
 -- Grain duration in seconds. The maximum value for this should be declared in \emph{imgdur}
. If \emph{kgdur}
 at any point becomes greater than \emph{imgdur}
, it will be truncated to \emph{imgdur}
. 


  The grain generator is based primarily on work and writings of Barry Truax and Curtis Roads. 
\subsection*{Examples}


  This example generates a texture with gradually shorter grains and wider amp and pitch spread. It uses the files \emph{grain.orc}
, \emph{grain.sco}
, and \emph{mary.wav}
. 


 \textbf{Example 1. Example of the grain opcode.}

\begin{lstlisting}
/* grain.orc */
; Initialize the global variables.
sr = 44100
kr = 44100
ksmps = 1
nchnls = 1

instr 1
    insnd = 10
    ibasfrq = 44100 / ftlen(insnd)   ; Use original sample rate of insnd file

    kamp   expseg 220, p3/2, 600, p3/2, 220
    kpitch line ibasfrq, p3, ibasfrq * .8
    kdens  line 600, p3, 200
    kaoff  line 0, p3, 5000
    kpoff  line 0, p3, ibasfrq * .5
    kgdur  line .4, p3, .1
    imaxgdur =  .5

    ar  grain kamp, kpitch, kdens, kaoff, kpoff, kgdur, insnd, 5, imaxgdur, 0.0
    out ar
endin
/* grain.orc */
        
\end{lstlisting}
\begin{lstlisting}
/* grain.sco */
f5  0 512  20 2                  ; Hanning window
f10 0 262144 1  "mary.wav" 0 0 0
i1 0 6
e
/* grain.sco */
        
\end{lstlisting}
\subsection*{Credits}


 


 


\begin{tabular}{ccc}
Author: Paris Smaragdis &MIT &May 1997

\end{tabular}



 
%\hline 


\begin{comment}
\begin{tabular}{lcr}
Previous &Home &Next \\
goto &Up &grain2

\end{tabular}


\end{document}
\end{comment}
