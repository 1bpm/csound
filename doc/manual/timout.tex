\begin{comment}
\documentclass[10pt]{article}
\usepackage{fullpage, graphicx, url}
\setlength{\parskip}{1ex}
\setlength{\parindent}{0ex}
\title{timout}
\begin{document}


\begin{tabular}{ccc}
The Alternative Csound Reference Manual & & \\
Previous & &Next

\end{tabular}

%\hline 
\end{comment}
\section{timout}
timout�--� Conditional branch during p-time depending on elapsed note time. \subsection*{Description}


  Conditional branch during p-time depending on elapsed note time. \emph{istrt}
 and \emph{idur}
 specify time in seconds. The branch to \emph{label}
 will become effective at time \emph{istrt}
, and will remain so for just \emph{idur}
 seconds. Note that \emph{timout}
 can be reinitialized for multiple activation within a single note (see example under \emph{reinit}
). 
\subsection*{Syntax}


 \textbf{timout}
 istrt, idur, label


  where \emph{label}
 is in the same instrument block and is not an expression, and where \emph{R}
 is one of the Relational operators (\emph{$<$}
,\emph{ =}
, \emph{$<$=}
, \emph{==}
, \emph{!=}
) (and \emph{=}
 for convenience, see also under \emph{Conditional Values}
). 
\subsection*{See Also}


 \emph{goto}
, \emph{if}
, \emph{igoto}
, \emph{kgoto}
, \emph{tigoto}

\subsection*{Credits}


 Added a note by Jim Aikin.
%\hline 


\begin{comment}
\begin{tabular}{lcr}
Previous &Home &Next \\
times &Up &tival

\end{tabular}


\end{document}
\end{comment}
