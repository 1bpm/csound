\begin{comment}
\documentclass[10pt]{article}
\usepackage{fullpage, graphicx, url}
\setlength{\parskip}{1ex}
\setlength{\parindent}{0ex}
\title{resony}
\begin{document}


\begin{tabular}{ccc}
The Alternative Csound Reference Manual & & \\
Previous & &Next

\end{tabular}

%\hline 
\end{comment}
\section{resony}
resony�--� A bank of second-order bandpass filters, connected in parallel. \subsection*{Description}


  A bank of second-order bandpass filters, connected in parallel. 
\subsection*{Syntax}


 ar \textbf{resony}
 asig, kbf, kbw, inum, ksep [, isepmode] [, iscl] [, iskip]
\subsection*{Initialization}


 \emph{inum}
 -- number of filters 


 \emph{isepmode}
 (optional, default=0) -- if \emph{isepmode}
 = 0, the separation of center frequencies of each filter is generated logarithmically (using octave as unit of measure). If \emph{isepmode}
 not equal to 0, the separation of center frequencies of each filter is generated linearly (using Hertz). Default value is 0. 


 \emph{iscl}
 (optional, default=0) -- coded scaling factor for resonators. A value of 1 signifies a peak response factor of 1, i.e. all frequencies other than \emph{kcf}
 are attenuated in accordance with the (normalized) response curve. A value of 2 raises the response factor so that its overall RMS value equals 1. (This intended equalization of input and output power assumes all frequencies are physically present; hence it is most applicable to white noise.) A zero value signifies no scaling of the signal, leaving that to some later adjustment (e.g. \emph{balance}
). The default value is 0. 


 \emph{iskip}
 (optional, default=0) -- initial disposition of internal data space. Since filtering incorporates a feedback loop of previous output, the initial status of the storage space used is significant. A zero value will clear the space; a non-zero value will allow previous information to remain. The default value is 0. 
\subsection*{Performance}


 \emph{asig}
 -- audio input signal 


 \emph{kbf}
 -- base frequency, i.e. center frequency of lowest filter in Hz 


 \emph{kbw}
 -- bandwidth in Hz 


 \emph{ksep}
 -- separation of the center frequency of filters in octaves 


 \emph{resony}
 is a bank of second-order bandpass filters, with k-rate variant frequency separation, base frequency and bandwidth, connected in parallel (i.e. the resulting signal is a mix of the output of each filter). The center frequency of each filter depends of \emph{kbf}
 and \emph{ksep}
 variables. The maximum number of filters is set to 100. 
\subsection*{Examples}


  Here is an example of the resony opcode. It uses the files \emph{resony.orc}
, \emph{resony.sco}
, and \emph{beats.wav}
. 


 \textbf{Example 1. Example of the resony opcode.}

\begin{lstlisting}
/* resony.orc */
; Initialize the global variables.
sr = 44100
kr = 4410
ksmps = 10
nchnls = 1

; Instrument #1.
instr 1
  ; Use a nice sawtooth waveform.
  asig vco 32000, 220, 1

  ; Vary the base frequency from 60 to 600 Hz.
  kbf line 60, p3, 600
  kbw = 50
  inum = 2
  ksep = 1
  isepmode = 0
  iscl = 1

  a1 resony asig, kbf, kbw, inum, ksep, isepmode, iscl

  out a1
endin
/* resony.orc */
        
\end{lstlisting}
\begin{lstlisting}
/* resony.sco */
; Table #1, a sine wave for the vco opcode.
f 1 0 16384 10 1

; Play Instrument #1 for two seconds.
i 1 0 2
e
/* resony.sco */
        
\end{lstlisting}
\subsection*{Credits}


 


 


\begin{tabular}{ccc}
Author: Gabriel Maldonado &Italy &1999

\end{tabular}



 


 Example written by Kevin Conder.


 New in Csound version 3.56
%\hline 


\begin{comment}
\begin{tabular}{lcr}
Previous &Home &Next \\
resonx &Up &resonz

\end{tabular}


\end{document}
\end{comment}
