\begin{comment}
\documentclass[10pt]{article}
\usepackage{fullpage, graphicx, url}
\setlength{\parskip}{1ex}
\setlength{\parindent}{0ex}
\title{guiro}
\begin{document}


\begin{tabular}{ccc}
The Alternative Csound Reference Manual & & \\
Previous & &Next

\end{tabular}

%\hline 
\end{comment}
\section{guiro}
guiro�--� Semi-physical model of a guiro sound. \subsection*{Description}


 \emph{guiro}
 is a semi-physical model of a guiro sound. It is one of the PhISEM percussion opcodes. PhISEM (Physically Informed Stochastic Event Modeling) is an algorithmic approach for simulating collisions of multiple independent sound producing objects. 
\subsection*{Syntax}


 ar \textbf{guiro}
 kamp, idettack [, inum] [, idamp] [, imaxshake] [, ifreq] [, ifreq1]
\subsection*{Initialization}


 \emph{idettack}
 -- period of time over which all sound is stopped 


 \emph{inum}
 (optional) -- The number of beads, teeth, bells, timbrels, etc. If zero, the default value is 128. 


 \emph{idamp}
 (optional) -- the damping factor of the instrument. \emph{Not used.}



 \emph{imaxshake}
 (optional, default=0) -- amount of energy to add back into the system. The value should be in range 0 to 1. 


 \emph{ifreq}
 (optional) -- the main resonant frequency. The default value is 2500. 


 \emph{ifreq1}
 (optional) -- the first resonant frequency. 
\subsection*{Performance}


 \emph{kamp}
 -- Amplitude of output. Note: As these instruments are stochastic, this is only an approximation. 
\subsection*{Examples}


  Here is an example of the guiro opcode. It uses the files \emph{guiro.orc}
 and \emph{guiro.sco}
. 


 \textbf{Example 1. Example of the guiro opcode.}

\begin{lstlisting}
/* guiro.orc */
sr = 44100
kr = 4410
ksmps = 10
nchnls = 1

    instr 01  ;example of a guiro
a1  guiro p4, 0.01
    out a1
    endin
/* guiro.orc */
        
\end{lstlisting}
\begin{lstlisting}
/* guiro.sco */
i1 0 1 20000
e
/* guiro.sco */
        
\end{lstlisting}
\subsection*{See Also}


 \emph{bamboo}
, \emph{dripwater}
, \emph{sleighbells}
, \emph{tambourine}

\subsection*{Credits}


 


 


\begin{tabular}{cccc}
Author: Perry Cook, part of the PhISEM (Physically Informed Stochastic Event Modeling) &Adapted by John ffitch &University of Bath, Codemist Ltd. &Bath, UK

\end{tabular}



 


 New in Csound version 4.07


 Added notes by Rasmus Ekman on May 2002.
%\hline 


\begin{comment}
\begin{tabular}{lcr}
Previous &Home &Next \\
granule &Up &harmon

\end{tabular}


\end{document}
\end{comment}
