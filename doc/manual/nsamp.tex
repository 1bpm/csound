\begin{comment}
\documentclass[10pt]{article}
\usepackage{fullpage, graphicx, url}
\setlength{\parskip}{1ex}
\setlength{\parindent}{0ex}
\title{nsamp}
\begin{document}


\begin{tabular}{ccc}
The Alternative Csound Reference Manual & & \\
Previous & &Next

\end{tabular}

%\hline 
\end{comment}
\section{nsamp}
nsamp�--� Returns the number of samples loaded into a stored function table number. \subsection*{Description}


  Returns the number of samples loaded into a stored function table number. 
\subsection*{Syntax}


 \textbf{nsamp}
(x) (init-rate args only)
\subsection*{Performance}


  Returns the number of samples loaded into stored function table number \emph{x}
 by \emph{GEN01}
. This is useful when a sample is shorter than the power-of-two function table that holds it. New in Csound version 3.49. 
\subsection*{Examples}


  Here is an example of the nsamp opcode. It uses the files \emph{nsamp.orc}
, \emph{nsamp.sco}
, and \emph{mary.wav}
. 


 \textbf{Example 1. Example of the nsamp opcode.}

\begin{lstlisting}
/* nsamp.orc */
; Initialize the global variables.
sr = 44100
kr = 4410
ksmps = 10
nchnls = 1

; Instrument #1.
instr 1
  ; Print out the size (in samples) of Table #1.
  isz = nsamp(1)
  print isz
endin
/* nsamp.orc */
        
\end{lstlisting}
\begin{lstlisting}
/* nsamp.sco */
; Table #1: Use an audio file.
f 1 0 262144 1 "mary.wav" 0 0 0

; Play Instrument #1 for 1 second.
i 1 0 1
e
/* nsamp.sco */
        
\end{lstlisting}
 Since the audio file ``mary.wav'' has 154390 samples, its output should include a line like this: \begin{lstlisting}
instr 1:  isz = 154390.000
      
\end{lstlisting}
\subsection*{See Also}


 \emph{ftchnls}
, \emph{ftlen}
, \emph{ftlptim}
, \emph{ftsr}

\subsection*{Credits}


 


 


\begin{tabular}{ccc}
Author: Gabriel Maldonado &Italy &October 1998

\end{tabular}



 


 Example written by Kevin Conder.
%\hline 


\begin{comment}
\begin{tabular}{lcr}
Previous &Home &Next \\
nrpn &Up &nstrnum

\end{tabular}


\end{document}
\end{comment}
