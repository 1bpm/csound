\begin{comment}
\documentclass[10pt]{article}
\usepackage{fullpage, graphicx, url}
\setlength{\parskip}{1ex}
\setlength{\parindent}{0ex}
\title{randomh}
\begin{document}


\begin{tabular}{ccc}
The Alternative Csound Reference Manual & & \\
Previous & &Next

\end{tabular}

%\hline 
\end{comment}
\section{randomh}
randomh�--� Generates random numbers with a user-defined limit and holds them for a period of time. \subsection*{Description}


  Generates random numbers with a user-defined limit and holds them for a period of time. 
\subsection*{Syntax}


 ar \textbf{randomh}
 kmin, kmax, acps


 kr \textbf{randomh}
 kmin, kmax, kcps
\subsection*{Performance}


 \emph{kmin}
 -- minimum range limit 


 \emph{kmax}
 -- maximum range limit 


 \emph{kcps, acps}
 -- rate of random break-point generation 


  The \emph{randomh}
 opcode is similar to \emph{randh}
 but allows the user to set arbitrary minimum and maximum values. 
\subsection*{Examples}


  Here is an example of the randomh opcode. It uses the files \emph{randomh.orc}
 and \emph{randomh.sco}
. 


 \textbf{Example 1. Example of the randomh opcode.}

\begin{lstlisting}
/* randomh.orc */
; Initialize the global variables.
sr = 44100
kr = 4410
ksmps = 10
nchnls = 1

; Instrument #1.
instr 1
  ; Choose a random frequency between 220 and 440 Hz.
  ; Generate new random numbers at 10 Hz.
  kmin = 220
  kmax = 440
  kcps = 10

  k1 randomh kmin, kmax, kcps

  printks "k1 = %f\\n", 0.1, k1
endin
/* randomh.orc */
        
\end{lstlisting}
\begin{lstlisting}
/* randh.sco */
; Table #1, a sine wave.
f 1 0 16384 10 1

; Play Instrument #1 for one second.
i 1 0 1
e
/* randh.sco */
        
\end{lstlisting}
 Its output should include lines like: \begin{lstlisting}
k1 = 220.000000
k1 = 414.232056
k1 = 284.095184
      
\end{lstlisting}
\subsection*{See Also}


 \emph{randh}
, \emph{random}
, \emph{randomi}

\subsection*{Credits}


 Author: Gabriel Maldonado


 Example written by Kevin Conder.
%\hline 


\begin{comment}
\begin{tabular}{lcr}
Previous &Home &Next \\
random &Up &randomi

\end{tabular}


\end{document}
\end{comment}
