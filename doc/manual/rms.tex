\begin{comment}
\documentclass[10pt]{article}
\usepackage{fullpage, graphicx, url}
\setlength{\parskip}{1ex}
\setlength{\parindent}{0ex}
\title{rms}
\begin{document}


\begin{tabular}{ccc}
The Alternative Csound Reference Manual & & \\
Previous & &Next

\end{tabular}

%\hline 
\end{comment}
\section{rms}
rms�--� Determines the root-mean-square amplitude of an audio signal. \subsection*{Description}


  Determines the root-mean-square amplitude of an audio signal. 
\subsection*{Syntax}


 kr \textbf{rms}
 asig [, ihp] [, iskip]
\subsection*{Initialization}


 \emph{ihp}
 (optional, default=10) -- half-power point (in Hz) of a special internal low-pass filter. The default value is 10. 


 \emph{iskip}
 (optional, default=0) -- initial disposition of internal data space (see \emph{reson}
). The default value is 0. 
\subsection*{Performance}


 \emph{asig}
 -- input audio signal 


 \emph{rms}
 output values \emph{kr}
 will trace the root-mean-square value of the audio input \emph{asig}
. This unit is not a signal modifier, but functions rather as a signal power-gauge. 
\subsection*{Examples}


 


 
\begin{lstlisting}
asrc \emph{buzz}
    10000,440, sr/440, 1 ; band-limited pulse train
a1   \emph{reson}
   asrc, 1000,100       ; sent through
a2   \emph{reson}
   a1,3000,500          ; 2 filters
afin \emph{balance}
 a2, asrc             ; then balanced with source
        
\end{lstlisting}


 
\subsection*{See Also}


 \emph{balance}
, \emph{gain}

%\hline 


\begin{comment}
\begin{tabular}{lcr}
Previous &Home &Next \\
rireturn &Up &rnd

\end{tabular}


\end{document}
\end{comment}
