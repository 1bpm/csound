\begin{comment}
\documentclass[10pt]{article}
\usepackage{fullpage, graphicx, url}
\setlength{\parskip}{1ex}
\setlength{\parindent}{0ex}
\title{ftsave}
\begin{document}


\begin{tabular}{ccc}
The Alternative Csound Reference Manual & & \\
Previous & &Next

\end{tabular}

%\hline 
\end{comment}
\section{ftsave}
ftsave�--� Save a set of previously-allocated tables to a file. \subsection*{Description}


  Save a set of previously-allocated tables to a file. 
\subsection*{Syntax}


 \textbf{ftsave}
 ``filename'', iflag, ifn1 [, ifn2] [...]
\subsection*{Initialization}


 \emph{``filename''}
 -- A quoted string containing the name of the file to save. 


 \emph{iflag}
 -- Type of the file to save. (0 = binary file, Non-zero = text file) 


 \emph{ifn1, ifn2, ...}
 -- Numbers of tables to save. 
\subsection*{Performance}


 \emph{ftsave}
 saves a list of tables to a file. The file's format can be binary or text. 


 


\begin{tabular}{cc}
Warning &\textbf{Warning}
 \\
� &

  The file's format is not compatible with a WAV-file and is not endian-safe. 


\end{tabular}

\subsection*{Examples}


  Here is an example of the ftsave opcode. It uses the files \emph{ftsave.orc}
 and \emph{ftsave.sco}
. 


 \textbf{Example 1. Example of the ftsave opcode.}

\begin{lstlisting}
/* ftsave.orc */
; Initialize the global variables.
sr = 44100
kr = 4410
ksmps = 10
nchnls = 1

; Table #1, make a sine wave using the GEN10 routine.
gitmp1 ftgen 1, 0, 32768, 10, 1
; Table #2, create an empty table.
gitmp2 ftgen 2, 0, 32768, 7, 0, 32768, 0

; Instrument #1 - a basic oscillator.
instr 1
  kamp = 20000
  kcps = 440
  ; Use Table #1.
  ifn = 1

  a1 oscil kamp, kcps, ifn
  out a1
endin


; Instrument #2 - Load Table #1 into Table #2.
instr 2
  ; Save Table #1 to a file called "table1.ftsave".
  ftsave "table1.ftsave", 0, 1

  ; Load the "table1.ftsave" file into Table #2.
  ftload "table1.ftsave", 0, 2

  kamp = 20000
  kcps = 440
  ; Use Table #2, it should contain Table #1's sine wave now.
  ifn = 2

  a1 oscil kamp, kcps, ifn
  out a1
endin
/* ftsave.orc */
        
\end{lstlisting}
\begin{lstlisting}
/* ftsave.sco */
; Play Instrument #1 for 1 second.
i 1 0 1
; Play Instrument #2 for 1 second.
i 2 2 1
e
/* ftsave.sco */
        
\end{lstlisting}
\subsection*{See Also}


 \emph{ftloadk}
, \emph{ftload}
, \emph{ftsavek}

\subsection*{Credits}


 Author: Gabriel Maldonado


 Example written by Kevin Conder.


 New in version 4.21
%\hline 


\begin{comment}
\begin{tabular}{lcr}
Previous &Home &Next \\
ftmorf &Up &ftsavek

\end{tabular}


\end{document}
\end{comment}
