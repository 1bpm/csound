\begin{comment}
\documentclass[10pt]{article}
\usepackage{fullpage, graphicx, url}
\setlength{\parskip}{1ex}
\setlength{\parindent}{0ex}
\title{clip}
\begin{document}


\begin{tabular}{ccc}
The Alternative Csound Reference Manual & & \\
Previous & &Next

\end{tabular}

%\hline 
\end{comment}
\section{clip}
clip�--� Clips a signal to a predefined limit. \subsection*{Description}


  Clips an a-rate signal to a predefined limit, in a ``soft'' manner, using one of three methods. 
\subsection*{Syntax}


 ar \textbf{clip}
 asig, imeth, ilimit [, iarg]
\subsection*{Initialization}


 \emph{imeth}
 -- selects the clipping method. The default is 0. The methods are: 


 
\begin{itemize}
\item 

 0 = Bram de Jong method (default)

\item 

 1 = sine clipping

\item 

 2 = tanh clipping


\end{itemize}


 \emph{ilimit}
 -- limiting value 


 \emph{iarg}
 (optional, default=0.5) -- when \emph{imeth}
 = 0, indicates the point at which clipping starts, in the range 0 - 1. Not used when \emph{imeth}
 = 1 or \emph{imeth}
 = 2. Default is 0.5. 
\subsection*{Performance}


 \emph{asig}
 -- a-rate input signal 


  The Bram de Jong method (\emph{imeth}
 = 0) applies the algorithm: 


 


 
\begin{lstlisting}
|\emph{x}
| > \emph{a}
:     f(\emph{x}
) = sin(\emph{x}
) * (a+(\emph{x-a}
)/(1+((\emph{x-a}
)/(1-\emph{a}
))2 |\emph{x}
| > 1: f(\emph{x}
) = sin(\emph{x}
) * (\emph{a}
+1)/2
        
\end{lstlisting}


 


  This method requires that \emph{asig}
 be normalized to 1. 


  The second method (\emph{imeth}
 = 1) is the sine clip: 


 


 
\begin{lstlisting}
|\emph{x}
| < \emph{limit}
: f(\emph{x}
) = \emph{limit}
 * sin(&#960;*\emph{x}
/(2*\emph{limit}
)) f(\emph{x}
) = \emph{limit}
 * sin(\emph{x}
)
        
\end{lstlisting}


 


  The third method (imeth = 0) is the tanh clip: 


 


 
\begin{lstlisting}
|\emph{x}
| < \emph{limit}
: f(\emph{x}
) = \emph{limit}
 * tanh(\emph{x/limit}
)/tanh(1) f(\emph{x}
) = \emph{limit}
 * sin(\emph{x}
)
        
\end{lstlisting}


 


 


\begin{tabular}{cc}
\textbf{Note}
 \\
� &

  Method 1 appears to be non-functional at release of Csound version 4.07. 


\end{tabular}

\subsection*{Examples}


  Here is an example of the clip opcode. It uses the files \emph{clip.orc}
 and \emph{clip.sco}
. 


 \textbf{Example 1. Example of the clip opcode.}

\begin{lstlisting}
/* clip.orc */
; Initialize the global variables.
sr = 44100
kr = 4410
ksmps = 10
nchnls = 1

; Instrument #1.
instr 1
  ; Generate a noisy waveform.
  arnd rand 44100
  ; Clip the noisy waveform's amplitude to 20,000
  a1 clip arnd, 2, 20000

  out a1
endin
/* clip.orc */
        
\end{lstlisting}
\begin{lstlisting}
/* clip.sco */
; Play Instrument #1 for one second.
i 1 0 1
e
/* clip.sco */
        
\end{lstlisting}
\subsection*{Credits}


 


 


\begin{tabular}{cccc}
Author: John ffitch &University of Bath, Codemist Ltd. &Bath, UK &August, 2000

\end{tabular}



 


 New in Csound version 4.07
%\hline 


\begin{comment}
\begin{tabular}{lcr}
Previous &Home &Next \\
clfilt &Up &clock

\end{tabular}


\end{document}
\end{comment}
