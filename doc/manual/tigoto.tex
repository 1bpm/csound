\begin{comment}
\documentclass[10pt]{article}
\usepackage{fullpage, graphicx, url}
\setlength{\parskip}{1ex}
\setlength{\parindent}{0ex}
\title{tigoto}
\begin{document}


\begin{tabular}{ccc}
The Alternative Csound Reference Manual & & \\
Previous & &Next

\end{tabular}

%\hline 
\end{comment}
\section{tigoto}
tigoto�--� Transfer control at i-time when a new note is being tied onto a previously held note \subsection*{Description}


  Similar to \emph{igoto}
 but effective only during an i-time pass at which a new note is being ``tied'' onto a previously held note. (See \emph{i Statement}
) It does not work when a tie has not taken place. Allows an instrument to skip initialization of units according to whether a proposed tie was in fact successful. (See also \emph{tival}
, \emph{delay}
). 
\subsection*{Syntax}


 \textbf{tigoto}
 label


  where \emph{label}
 is in the same instrument block and is not an expression, and where \emph{R}
 is one of the Relational operators (\emph{$<$}
,\emph{ =}
, \emph{$<$=}
, \emph{==}
, \emph{!=}
) (and \emph{=}
 for convenience, see also under \emph{Conditional Values}
). 
\subsection*{See Also}


 \emph{cigoto}
, \emph{goto}
, \emph{if}
, \emph{igoto}
, \emph{kgoto}
, \emph{timout}

\subsection*{Credits}


 Added a note by Jim Aikin.
%\hline 


\begin{comment}
\begin{tabular}{lcr}
Previous &Home &Next \\
tempoval &Up &timeinstk

\end{tabular}


\end{document}
\end{comment}
