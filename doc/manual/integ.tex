\begin{comment}
\documentclass[10pt]{article}
\usepackage{fullpage, graphicx, url}
\setlength{\parskip}{1ex}
\setlength{\parindent}{0ex}
\title{integ}
\begin{document}


\begin{tabular}{ccc}
The Alternative Csound Reference Manual & & \\
Previous & &Next

\end{tabular}

%\hline 
\end{comment}
\section{integ}
integ�--� Modify a signal by integration. \subsection*{Description}


  Modify a signal by integration. 
\subsection*{Syntax}


 ar \textbf{integ}
 asig [, iskip]


 kr \textbf{integ}
 ksig [, iskip]
\subsection*{Initialization}


 \emph{iskip}
 (optional) -- initial disposition of internal save space (see \emph{reson}
). The default value is 0. 
\subsection*{Performance}


 \emph{integ}
 and \emph{diff}
 perform integration and differentiation on an input control signal or audio signal. Each is the converse of the other, and applying both will reconstruct the original signal. Since these units are special cases of low-pass and high-pass filters, they produce a scaled (and phase shifted) output that is frequency-dependent. Thus \emph{diff}
 of a sine produces a cosine, with amplitude \emph{2 * sin(pi * Hz / \emph{sr}
)}
 that of the original (for each component partial); \emph{integ}
 will inversely affect the magnitudes of its component inputs. With this understanding, these units can provide useful signal modification. 
\subsection*{Examples}


  Here is an example of the integ opcode. It uses the files \emph{integ.orc}
 and \emph{integ.sco}
. 


 \textbf{Example 1. Example of the integ opcode.}

\begin{lstlisting}
/* integ.orc */
; Initialize the global variables.
sr = 44100
kr = 4410
ksmps = 10
nchnls = 1

; Instrument #1 -- a differentiated signal.
instr 1
  ; Generate a band-limited pulse train.
  asrc buzz 20000, 440, 20,  1

  ; Differentiate the signal.
  adiff diff asrc

  out adiff
endin

; Instrument #2 -- a re-integrated signal.
instr 2
  ; Generate a band-limited pulse train.
  asrc buzz 20000, 440, 20,  1

  ; Differentiate the signal.
  adiff diff asrc

  ; Re-integrate the previously differentiated signal.
  a1 integ adiff

  out a1
endin
/* integ.orc */
        
\end{lstlisting}
\begin{lstlisting}
/* integ.sco */
; Table #1, a sine wave.
f 1 0 16384 10 1

; Play Instrument #1 for one second.
i 1 0 1
; Play Instrument #2 for one second.
i 2 1 1
e
/* integ.sco */
        
\end{lstlisting}
\subsection*{See Also}


 \emph{diff}
, \emph{downsamp}
, \emph{interp}
, \emph{samphold}
, \emph{upsamp}

\subsection*{Credits}


 Example written by Kevin Conder.
%\hline 


\begin{comment}
\begin{tabular}{lcr}
Previous &Home &Next \\
int &Up &interp

\end{tabular}


\end{document}
\end{comment}
