\begin{comment}
\documentclass[10pt]{article}
\usepackage{fullpage, graphicx, url}
\setlength{\parskip}{1ex}
\setlength{\parindent}{0ex}
\title{table3}
\begin{document}


\begin{tabular}{ccc}
The Alternative Csound Reference Manual & & \\
Previous & &Next

\end{tabular}

%\hline 
\end{comment}
\section{table3}
table3�--� Accesses table values by direct indexing with cubic interpolation. \subsection*{Description}


  Accesses table values by direct indexing with cubic interpolation. 
\subsection*{Syntax}


 ar \textbf{table3}
 andx, ifn [, ixmode] [, ixoff] [, iwrap]


 ir \textbf{table3}
 indx, ifn [, ixmode] [, ixoff] [, iwrap]


 kr \textbf{table3}
 kndx, ifn [, ixmode] [, ixoff] [, iwrap]
\subsection*{Initialization}


 \emph{ifn}
 -- function table number. 


 \emph{ixmode}
 (optional) -- index data mode. The default value is 0. 


 
\begin{itemize}
\item 

 0 = raw index

\item 

 1 = normalized (0 to 1)


\end{itemize}


 \emph{ixoff}
 (optional) -- amount by which index is to be offset. For a table with origin at center, use tablesize/2 (raw) or .5 (normalized). The default value is 0. 


 \emph{iwrap}
 (optional) -- wraparound index flag. The default value is 0. 


 
\begin{itemize}
\item 

 0 = nowrap (index $<$ 0 treated as index=0; index tablesize sticks at index=size)

\item 

 1 = wraparound.


\end{itemize}
\subsection*{Performance}


 \emph{table3}
 is experimental, and is identical to \emph{tablei}
, except that it uses cubic interpolation. (New in Csound version 3.50.) 
\subsection*{See Also}


 \emph{table}
, \emph{tablei}
, \emph{oscil1}
, \emph{oscil1i}
, \emph{osciln}

%\hline 


\begin{comment}
\begin{tabular}{lcr}
Previous &Home &Next \\
table &Up &tablecopy

\end{tabular}


\end{document}
\end{comment}
