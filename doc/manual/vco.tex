\begin{comment}
\documentclass[10pt]{article}
\usepackage{fullpage, graphicx, url}
\setlength{\parskip}{1ex}
\setlength{\parindent}{0ex}
\title{vco}
\begin{document}


\begin{tabular}{ccc}
The Alternative Csound Reference Manual & & \\
Previous & &Next

\end{tabular}

%\hline 
\end{comment}
\section{vco}
vco�--� Implementation of a band limited, analog modeled oscillator. \subsection*{Description}


  Implementation of a band limited, analog modeled oscillator, based on integration of band limited impulses. \emph{vco}
 can be used to simulate a variety of analog wave forms. 
\subsection*{Syntax}


 ar \textbf{vco}
 xamp, xcps, iwave, kpw [, ifn] [, imaxd] [, ileak] [, inyx] [, iphs]
\subsection*{Initialization}


 \emph{iwave}
 -- determines the waveform: 


 
\begin{itemize}
\item 

 \emph{iwave}
 = 1 - sawtooth

\item 

 \emph{iwave}
 = 2 - Square/PWM

\item 

 \emph{iwave}
 = 3 - triangle/Saw/Ramp


\end{itemize}


 \emph{ifn}
 (optional, default = 1) -- should be the table number of a of a stored sine wave. 


 \emph{imaxd}
 (optional, default = 1) -- is the maximum delay time. A time of 1/ifqc may be required for the pwm and triangle waveform. To bend the pitch down this value must be as large as 1/(minimum frequency). 


 \emph{ileak}
 (optional, default = 0) -- If ileak is between zero and one (0 $<$ ileak $<$ 1) then ileak is used as the leaky integrator value. Otherwise a leaky integrator value of .999 is used for the saw and square waves and .995 is used for the triangle wave. This can be used to ``flatten'' the square wave or ``straighten'' the saw wave at low frequencies by setting ileak to .99999 or a similar value. This should give a hollower sounding square wave. 


 \emph{inyx}
 (optional, default = .5) -- This is used to determine the number of harmonics in the band limited pulse. All overtones up to sr * inyx will be used. The default gives sr * .5 (sr / 2). For sr / 4 use inyx = .25. This can generate a ``fatter'' sound in some cases. 


 \emph{iphs}
 (optional, default = 0) -- This is a phase value. There is an artifact (bug-like feature) in \emph{vco}
 which occurs during the first half cycle of the square wave which causes the waveform to be greater in magnitude than all others. The value of \emph{iphs}
 has an effect on this artifact. In particular setting \emph{iphs}
 to .5 will cause the first half cycle of the square wave to resemble a small triangle wave. This may be more desirable than the large wave artifact which is the current default. 
\subsection*{Performance}


 \emph{kpw}
 -- determines either the pulse width (if \emph{iwave}
 is 2) or the saw/ramp character (if \emph{iwave}
 is 3) The value of \emph{kpw}
 should be greater than 0 and less than 2. A value of 1 will generate either a square wave (if \emph{iwave}
 is 2) or a triangle wave (if \emph{iwave}
 is 3). 


 \emph{xamp}
 -- determines the amplitude 


 \emph{xcps}
 -- is the frequency of the wave in cycles per second. 
\subsection*{Examples}


  Here is an example of the vco opcode. It uses the files \emph{vco.orc}
 and \emph{vco.sco}
. 


 \textbf{Example 1. Example of the vco opcode.}

\begin{lstlisting}
/* vco.orc */
; Initialize the global variables.
sr = 44100
kr = 44100
ksmps = 1
nchnls = 1

; Instrument #1
instr 1
  ; Set the amplitude.
  kamp = p4 

  ; Set the frequency.
  kcps = cpspch(p5) 

  ; Select the wave form.
  iwave = p6

  ; Set the pulse-width/saw-ramp character.
  kpw init 0.5

  ; Use Table #1.
  ifn = 1
  
  ; Generate the waveform.
  asig vco kamp, kcps, iwave, kpw, ifn

  ; Output and amplification.
  out asig
endin
/* vco.orc */
        
\end{lstlisting}
\begin{lstlisting}
/* vco.sco */
; Table #1, a sine wave.
f 1 0 65536 10 1

; Define the score.
; p4 = raw amplitude (0-32767)
; p5 = frequency, in pitch-class notation.
; p6 = the waveform (1=Saw, 2=Square/PWM, 3=Tri/Saw-Ramp-Mod)
i 1 00 02 20000 05.00 1
i 1 02 02 20000 05.00 2
i 1 04 02 20000 05.00 3

i 1 06 02 20000 07.00 1
i 1 08 02 20000 07.00 2
i 1 10 02 20000 07.00 3

i 1 12 02 20000 09.00 1
i 1 14 02 20000 09.00 2
i 1 16 02 20000 09.00 3

i 1 18 02 20000 11.00 1
i 1 20 02 20000 11.00 2
i 1 22 02 20000 11.00 3
e
/* vco.sco */
        
\end{lstlisting}
\subsection*{See Also}


 \emph{vco2}

\subsection*{Credits}


 


 


\begin{tabular}{cc}
Author: Hans Mikelson &December 1998

\end{tabular}



 


 New in Csound version 3.50


 November 2002. Corrected the documentation for the \emph{kpw}
 parameter thanks to Luis Jure and Hans Mikelson.
%\hline 


\begin{comment}
\begin{tabular}{lcr}
Previous &Home &Next \\
vbapzmove &Up &vco2

\end{tabular}


\end{document}
\end{comment}
