\begin{comment}
\documentclass[10pt]{article}
\usepackage{fullpage, graphicx, url}
\setlength{\parskip}{1ex}
\setlength{\parindent}{0ex}
\title{divz}
\begin{document}


\begin{tabular}{ccc}
The Alternative Csound Reference Manual & & \\
Previous & &Next

\end{tabular}

%\hline 
\end{comment}
\section{divz}
divz�--� Safely divides two numbers. \subsection*{Syntax}


 ar \textbf{divz}
 xa, xb, ksubst


 ir \textbf{divz}
 ia, ib, isubst


 kr \textbf{divz}
 ka, kb, ksubst
\subsection*{Description}


  Safely divides two numbers. 
\subsection*{Initialization}


  Whenever \emph{b}
 is not zero, set the result to the value \emph{a / b}
; when \emph{b}
 is zero, set it to the value of \emph{subst}
 instead. 
\subsection*{Examples}


  Here is an example of the divz opcode. It uses the files \emph{divz.orc}
 and \emph{divz.sco}
. 


 \textbf{Example 1. Example of the divz opcode.}

\begin{lstlisting}
/* divz.orc */
; Initialize the global variables.
sr = 44100
kr = 4410
ksmps = 10
nchnls = 1

; Instrument #1.
instr 1
  ; Define the numbers to be divided.
  ka init 200
  ; Linearly change the value of kb from 200 to 0.
  kb line 0, p3, 200
  ; If a "divide by zero" error occurs, substitute -1.
  ksubst init -1
  
  ; Safely divide the numbers.
  kresults divz ka, kb, ksubst

  ; Print out the results.
  printks "%f / %f = %f\\n", 0.1, ka, kb, kresults
endin
/* divz.orc */
        
\end{lstlisting}
\begin{lstlisting}
/* divz.sco */
; Play Instrument #1 for one second.
i 1 0 1
e
/* divz.sco */
        
\end{lstlisting}
 Its output should include lines like: \begin{lstlisting}
200.000000 / 0.000000 = -1.000000
200.000000 / 19.999887 = 10.000056
200.000000 / 40.000027 = 4.999997
      
\end{lstlisting}
\subsection*{See Also}


 \emph{=}
, \emph{init}
, \emph{tival}

\subsection*{Credits}


 Example written by Kevin Conder.
%\hline 


\begin{comment}
\begin{tabular}{lcr}
Previous &Home &Next \\
distort1 &Up &downsamp

\end{tabular}


\end{document}
\end{comment}
