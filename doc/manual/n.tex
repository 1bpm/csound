\begin{comment}
\documentclass[10pt]{article}
\usepackage{fullpage, graphicx, url}
\setlength{\parskip}{1ex}
\setlength{\parindent}{0ex}
\title{n Statement}
\begin{document}


\begin{tabular}{ccc}
The Alternative Csound Reference Manual & & \\
Previous & &Next

\end{tabular}

%\hline 
\end{comment}
\section{n Statement}
n�--� Repeats a section. \subsection*{Description}


  Repeats a section from the referenced \emph{m statement}
. 
\subsection*{Syntax}


 \textbf{n}
 p1
\subsection*{Initialization}


 \emph{p1}
 -- Name of mark to repeat. 
\subsection*{Performance}


  This can be helpful in setting a up verse and chorus structure in the score. Names may contain letters and numerals. 
\subsection*{Credits}


 


 


\begin{tabular}{cccc}
Author: John ffitch &University of Bath/Codemist Ltd. &Bath, UK &April 1998

\end{tabular}



 


 New in Csound version 3.48
%\hline 


\begin{comment}
\begin{tabular}{lcr}
Previous &Home &Next \\
m Statement (Mark Statement) &Up &q Statement

\end{tabular}


\end{document}
\end{comment}
