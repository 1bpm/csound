\begin{comment}
\documentclass[10pt]{article}
\usepackage{fullpage, graphicx, url}
\setlength{\parskip}{1ex}
\setlength{\parindent}{0ex}
\title{mute}
\begin{document}


\begin{tabular}{ccc}
The Alternative Csound Reference Manual & & \\
Previous & &Next

\end{tabular}

%\hline 
\end{comment}
\section{mute}
mute�--� Mutes/unmutes new instances of a given instrument. \subsection*{Description}


  Mutes/unmutes new instances of a given instrument. 
\subsection*{Syntax}


 \textbf{mute}
 insnum [, iswitch]


 \textbf{mute}
 ``insname'' [, iswitch]
\subsection*{Initialization}


 \emph{insnum}
 -- instrument number. Equivalent to \emph{p1}
 in a score \emph{i statement}
. 


 \emph{``insname''}
 -- A string (in double-quotes) representing a named instrument. 


 \emph{iswitch}
 (optional, default=0) -- represents a switch to mute/unmute an instrument. A value of 0 will mute new instances of an instrument, other values will unmute them. The default value is 0. 
\subsection*{Performance}


  All new instances of instrument inst will me muted (iswitch = 0) or unmuted (iswitch not equal to 0). There is no difficulty with muting muted instruments or unmuting unmuted instruments. The mechanism is the same as used by the score \emph{q statement}
. For example, it is possible to mute in the score and unmute in some instrument. 


  Muting/Unmuting is indicated by a message (depending on message level). 
\subsection*{Examples}


  Here is an example of the mute opcode. It uses the files \emph{mute.orc}
 and \emph{mute.sco}
. 


 \textbf{Example 1. Example of the mute opcode.}

\begin{lstlisting}
/* mute.orc */
; Initialize the global variables.
sr = 44100
kr = 4410
ksmps = 10
nchnls = 1

; Mute Instrument #2.
mute 2

; Instrument #1.
instr 1
  a1 oscils 10000, 440, 0
  out a1
endin

; Instrument #2.
instr 2
  a1 oscils 10000, 880, 0
  out a1
endin
/* mute.orc */
        
\end{lstlisting}
\begin{lstlisting}
/* mute.sco */
; Play Instrument #1 for one second.
i 1 0 1
; Play Instrument #2 for one second.
i 2 0 1
e
/* mute.sco */
        
\end{lstlisting}
\subsection*{Credits}


 Example written by Kevin Conder.


 New in version 4.22
%\hline 


\begin{comment}
\begin{tabular}{lcr}
Previous &Home &Next \\
multitap &Up &mxadsr

\end{tabular}


\end{document}
\end{comment}
