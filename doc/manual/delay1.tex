\begin{comment}
\documentclass[10pt]{article}
\usepackage{fullpage, graphicx, url}
\setlength{\parskip}{1ex}
\setlength{\parindent}{0ex}
\title{delay1}
\begin{document}


\begin{tabular}{ccc}
The Alternative Csound Reference Manual & & \\
Previous & &Next

\end{tabular}

%\hline 
\end{comment}
\section{delay1}
delay1�--� Delays an input signal by one sample. \subsection*{Description}


  Delays an input signal by one sample. 
\subsection*{Syntax}


 ar \textbf{delay1}
 asig [, iskip]
\subsection*{Initialization}


 \emph{iskip}
 (optional, default=0) -- initial disposition of delay-loop data space (see \emph{reson}
). The default value is 0. 
\subsection*{Performance}


 \emph{delay1}
 is a special form of delay that serves to delay the audio signal \emph{asig}
 by just one sample. It is thus functionally equivalent to the \emph{delay}
 opcode but is more efficient in both time and space. This unit is particularly useful in the fabrication of generalized non-recursive filters. 
\subsection*{See Also}


 \emph{delay}
, \emph{delayr}
, \emph{delayw}

%\hline 


\begin{comment}
\begin{tabular}{lcr}
Previous &Home &Next \\
delay &Up &delayr

\end{tabular}


\end{document}
\end{comment}
