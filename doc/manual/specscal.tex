\begin{comment}
\documentclass[10pt]{article}
\usepackage{fullpage, graphicx, url}
\setlength{\parskip}{1ex}
\setlength{\parindent}{0ex}
\title{specscal}
\begin{document}


\begin{tabular}{ccc}
The Alternative Csound Reference Manual & & \\
Previous & &Next

\end{tabular}

%\hline 
\end{comment}
\section{specscal}
specscal�--� Scales an input spectral datablock with spectral envelopes. \subsection*{Description}


  Scales an input spectral datablock with spectral envelopes. 
\subsection*{Syntax}


 wsig \textbf{specscal}
 wsigin, ifscale, ifthresh
\subsection*{Initialization}


 \emph{ifscale}
 -- scale function table. A function table containing values by which a value's magnitude is rescaled. 


 \emph{ifthresh}
 -- threshold function table. If \emph{ifthresh}
 is non-zero, each magnitude is reduced by its corresponding table-value (to not less than zero) 
\subsection*{Performance}


 \emph{wsig}
 -- the output spectrum 


 \emph{wsigin}
 -- the input spectra 


  Scales an input spectral datablock with spectral envelopes. Function tables \emph{ifthresh}
 and \emph{ifscale}
 are initially sampled across the (logarithmic) frequency space of the input spectrum; then each time a new input spectrum is sensed the sampled values are used to scale each of its magnitude channels as follows: if \emph{ifthresh}
 is non-zero, each magnitude is reduced by its corresponding table-value (to not less than zero); then each magnitude is rescaled by the corresponding \emph{ifscale}
 value, and the resulting spectrum written to \emph{wsig}
. 
\subsection*{Examples}


 


 
\begin{lstlisting}
  wsig2    \emph{specdiff}
         wsig1               ; sense onsets 
  wsig3    \emph{specfilt}
         wsig2, 2            ; absorb slowly 
           \emph{specdisp}
         wsig2, .1           ; & display both spectra 
           \emph{specdisp}
         wsig3, .1
        
\end{lstlisting}


 
\subsection*{See Also}


 \emph{specaddm}
, \emph{specdiff}
, \emph{specfilt}
, \emph{spechist}

%\hline 


\begin{comment}
\begin{tabular}{lcr}
Previous &Home &Next \\
specptrk &Up &specsum

\end{tabular}


\end{document}
\end{comment}
