\begin{comment}
\documentclass[10pt]{article}
\usepackage{fullpage, graphicx, url}
\setlength{\parskip}{1ex}
\setlength{\parindent}{0ex}
\title{noteon}
\begin{document}


\begin{tabular}{ccc}
The Alternative Csound Reference Manual & & \\
Previous & &Next

\end{tabular}

%\hline 
\end{comment}
\section{noteon}
noteon�--� Send a noteon message to the MIDI OUT port. \subsection*{Description}


  Send a noteon message to the MIDI OUT port. 
\subsection*{Syntax}


 \textbf{noteon}
 ichn, inum, ivel
\subsection*{Initialization}


 \emph{ichn}
 -- MIDI channel number (1-16) 


 \emph{inum}
 -- note number (0-127) 


 \emph{ivel}
 -- velocity (0-127) 
\subsection*{Performance}


 \emph{noteon}
 (i-rate note on) and \emph{noteoff}
 (i-rate note off) are the simplest MIDI OUT opcodes. \emph{noteon}
 sends a MIDI noteon message to MIDI OUT port, and \emph{noteoff}
 sends a noteoff message. A \emph{noteon}
 opcode must always be followed by an \emph{noteoff}
 with the same channel and number inside the same instrument, otherwise the note will play endlessly. 


  These \emph{noteon}
 and \emph{noteoff}
 opcodes are useful only when introducing a \emph{timout}
 statement to play a non-zero duration MIDI note. For most purposes, it is better to use \emph{noteondur}
 and \emph{noteondur2}
. 
\subsection*{See Also}


 \emph{noteoff}
, \emph{noteondur}
, \emph{noteondur2}

\subsection*{Credits}


 


 


\begin{tabular}{cc}
Author: Gabriel Maldonado &Italy

\end{tabular}



 


 New in Csound version 3.47


 Thanks goes to Rasmus Ekman for pointing out the correct MIDI channel and controller number ranges.
%\hline 


\begin{comment}
\begin{tabular}{lcr}
Previous &Home &Next \\
noteoff &Up &noteondur

\end{tabular}


\end{document}
\end{comment}
