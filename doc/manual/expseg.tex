\begin{comment}
\documentclass[10pt]{article}
\usepackage{fullpage, graphicx, url}
\setlength{\parskip}{1ex}
\setlength{\parindent}{0ex}
\title{expseg}
\begin{document}


\begin{tabular}{ccc}
The Alternative Csound Reference Manual & & \\
Previous & &Next

\end{tabular}

%\hline 
\end{comment}
\section{expseg}
expseg�--� Trace a series of exponential segments between specified points. \subsection*{Description}


  Trace a series of exponential segments between specified points. 
\subsection*{Syntax}


 ar \textbf{expseg}
 ia, idur1, ib [, idur2] [, ic] [...]


 kr \textbf{expseg}
 ia, idur1, ib [, idur2] [, ic] [...]
\subsection*{Initialization}


 \emph{ia}
 -- starting value. Zero is illegal for exponentials. 


 \emph{ib, ic}
, etc. -- value after \emph{dur1}
 seconds, etc. For exponentials, must be non-zero and must agree in sign with \emph{ia}
. 


 \emph{idur1}
 -- duration in seconds of first segment. A zero or negative value will cause all initialization to be skipped. 


 \emph{idur2, idur3}
, etc. -- duration in seconds of subsequent segments. A zero or negative value will terminate the initialization process with the preceding point, permitting the last-defined line or curve to be continued indefinitely in performance. The default is zero. 
\subsection*{Performance}


  These units generate control or audio signals whose values can pass through 2 or more specified points. The sum of \emph{dur}
 values may or may not equal the instrument's performance time: a shorter performance will truncate the specified pattern, while a longer one will cause the last-defined segment to continue on in the same direction. 


  Note that the \emph{expseg}
 opcode does not operate correctly at audio rate when segments are shorter than a k-period. Try the \emph{expsega}
 opcode instead. 
\subsection*{Examples}


  Here is an example of the expseg opcode. It uses the files \emph{expseg.orc}
 and \emph{expseg.sco}
. 


 \textbf{Example 1. Example of the expseg opcode.}

\begin{lstlisting}
/* expseg.orc */
; Initialize the global variables.
sr = 44100
kr = 4410
ksmps = 10
nchnls = 1

; Instrument #1.
instr 1
  ; p4 = frequency in pitch-class notation.
  kcps = cpspch(p4)

  ; Create an amplitude envelope.
  kenv expseg 0.01, p3*0.25, 1, p3*0.75, 0.01
  kamp = kenv * 30000

  a1 oscil kamp, kcps, 1
  out a1
endin
/* expseg.orc */
        
\end{lstlisting}
\begin{lstlisting}
/* expseg.sco */
; Table #1, a sine wave.
f 1 0 16384 10 1

; Play Instrument #1 for a half-second, p4=8.00
i 1 0 0.5 8.00
; Play Instrument #1 for a half-second, p4=8.01
i 1 1 0.5 8.01
; Play Instrument #1 for a half-second, p4=8.02
i 1 2 0.5 8.02
; Play Instrument #1 for a half-second, p4=8.03
i 1 3 0.5 8.03
e
/* expseg.sco */
        
\end{lstlisting}
\subsection*{See Also}


 \emph{expon}
, \emph{expsega}
, \emph{expsegr}
, \emph{line}
, \emph{linseg}
, \emph{linsegr}

\subsection*{Credits}


 Author: Gabriel Maldonado


 Example written by Kevin Conder.


 New in Csound 3.57
%\hline 


\begin{comment}
\begin{tabular}{lcr}
Previous &Home &Next \\
exprand &Up &expsega

\end{tabular}


\end{document}
\end{comment}
