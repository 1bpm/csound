\begin{comment}
\documentclass[10pt]{article}
\usepackage{fullpage, graphicx, url}
\setlength{\parskip}{1ex}
\setlength{\parindent}{0ex}
\title{oscili}
\begin{document}


\begin{tabular}{ccc}
The Alternative Csound Reference Manual & & \\
Previous & &Next

\end{tabular}

%\hline 
\end{comment}
\section{oscili}
oscili�--� A simple oscillator with linear interpolation. \subsection*{Description}


  Table \emph{ifn}
 is incrementally sampled modulo the table length and the value obtained is multiplied by \emph{amp}
. 
\subsection*{Syntax}


 ar \textbf{oscili}
 xamp, xcps, ifn [, iphs]


 kr \textbf{oscili}
 kamp, kcps, ifn [, iphs]
\subsection*{Initialization}


 \emph{ifn}
 -- function table number. Requires a wrap-around guard point. 


 \emph{iphs}
 (optional) -- initial phase of sampling, expressed as a fraction of a cycle (0 to 1). A negative value will cause phase initialization to be skipped. The default value is 0. 
\subsection*{Performance}


 \emph{kamp, xamp}
 -- amplitude 


 \emph{kcps, xcps}
 -- frequency in cycles per second. 


 \emph{oscili}
 differs from \emph{oscil}
 in that the standard procedure of using a truncated phase as a sampling index is here replaced by a process that interpolates between two successive lookups. Interpolating generators will produce a noticeably cleaner output signal, but they may take as much as twice as long to run. Adequate accuracy can also be gained without the time cost of interpolation by using large stored function tables of 2K, 4K or 8K points if the space is available. 
\subsection*{Examples}


  Here is an example of the oscili opcode. It uses the files \emph{oscili.orc}
 and \emph{oscili.sco}
. 


 \textbf{Example 1. Example of the oscili opcode.}

\begin{lstlisting}
/* oscili.orc */
; Initialize the global variables.
sr = 44100
kr = 4410
ksmps = 10
nchnls = 1

; Instrument #1 - a basic oscillator.
instr 1
  kamp = 10000
  kcps = 220
  ifn = 1

  a1 oscil kamp, kcps, ifn
  out a1
endin

; Instrument #2 - the basic oscillator with extra interpolation.
instr 2
  kamp = 10000
  kcps = 220
  ifn = 1

  a1 oscili kamp, kcps, ifn
  out a1
endin
/* oscili.orc */
        
\end{lstlisting}
\begin{lstlisting}
/* oscili.sco */
; Table #1, a sine wave table with a small amount of data.
f 1 0 32 10 0 1

; Play Instrument #1, the basic oscillator, for 
; two seconds. This should sound relatively rough.
i 1 0 2

; Play Instrument #2, the interpolated oscillator, for
; two seconds. This should sound relatively smooth.
i 2 2 2
e
/* oscili.sco */
        
\end{lstlisting}
\subsection*{See Also}


 \emph{oscil}
, \emph{oscil3}

\subsection*{Credits}


 Example written by Kevin Conder.
%\hline 


\begin{comment}
\begin{tabular}{lcr}
Previous &Home &Next \\
oscil3 &Up &oscilikt

\end{tabular}


\end{document}
\end{comment}
