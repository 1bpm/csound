\begin{comment}
\documentclass[10pt]{article}
\usepackage{fullpage, graphicx, url}
\setlength{\parskip}{1ex}
\setlength{\parindent}{0ex}
\title{srconv}
\begin{document}


\begin{tabular}{ccc}
The Alternative Csound Reference Manual & & \\
Previous & &Next

\end{tabular}

%\hline 
\end{comment}
\section{srconv}
srconv�--� Converts the sample rate of an audio file. \subsection*{Description}


  Converts the sample rate of an audio file at sample rate Rin to a sample rate of Rout. Optionally the ratio (Rin / Rout) may be linearly time-varying according to a set of (time, ratio) pairs in an auxiliary file. 
\subsection*{Syntax}


 \textbf{srconv}
 [flags] infile
\subsection*{Initialization}


  Flags: 


 
\begin{itemize}
\item 

 -\emph{P num}
 = pitch transposition ratio (srate / r) [don't specify both P and r]

\item 

 -\emph{P num}
 = pitch transposition ratio (srate / r) [don't specify both P and r]

\item 

 -\emph{Q num}
 =quality factor (1, 2, 3, or 4: default = 2)

\item 

 -\emph{i filnam}
 = break file

\item 

 -\emph{r num}
 = output sample rate (must be specified)

\item 

 -\emph{o fnam}
 = sound output filename

\item 

 -\emph{A}
 = create an AIFF format output soundfile

\item 

 -\emph{J}
 = create an IRCAM format output soundfile

\item 

 -\emph{W}
 = create a WAV format output soundfile

\item 

 -\emph{h}
 = no header on output soundfile

\item 

 -\emph{c}
 = 8-bit signed\_char sound samples

\item 

 -\emph{a}
 = alaw sound samples

\item 

 -\emph{8}
 = 8-bit unsigned\_char sound samples

\item 

 -\emph{u}
 = ulaw sound samples

\item 

 -\emph{s}
 = short\_int sound samples

\item 

 -\emph{l}
 = long\_int sound samples

\item 

 -\emph{f}
 = float sound samples

\item 

 -\emph{r N}
 = orchestra srate override

\item 

 -\emph{K}
 = Do not generate PEAK chunks

\item 

 -\emph{R}
 = continually rewrite header while writing soundfile (WAV/AIFF)

\item 

 -\emph{H\#}
 = print a heartbeat style 1, 2 or 3 at each soundfile write

\item 

 -\emph{N}
 = notify (ring the bell) when score or miditrack is done

\item 

 -\emph{- fnam}
 = log output to file


\end{itemize}


  This program performs arbitrary sample-rate conversion with high fidelity. The method is to step through the input at the desired sampling increment, and to compute the output points as appropriately weighted averages of the surrounding input points. There are two cases to consider: 


 
\begin{enumerate}
\item 

 sample rates are in a small-integer ratio - weights are obtained from table.

\item 

 sample rates are in a large-integer ratio - weights are linearly interpolated from table.


\end{enumerate}


  Calculate increment: if decimating, then window is impulse response of low-pass filter with cutoff frequency at half of output sample rate; if interpolating, then window is impulse response of lowpass filter with cutoff frequency at half of input sample rate. 
\subsection*{Credits}


 Author: Mark Dolson


 August 26, 1989


 Author: John ffitch


 December 30, 2000
%\hline 


\begin{comment}
\begin{tabular}{lcr}
Previous &Home &Next \\
sdif2ad &Up &Cscore

\end{tabular}


\end{document}
\end{comment}
