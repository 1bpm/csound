\begin{comment}
\documentclass[10pt]{article}
\usepackage{fullpage, graphicx, url}
\setlength{\parskip}{1ex}
\setlength{\parindent}{0ex}
\title{hsboscil}
\begin{document}


\begin{tabular}{ccc}
The Alternative Csound Reference Manual & & \\
Previous & &Next

\end{tabular}

%\hline 
\end{comment}
\section{hsboscil}
hsboscil�--� An oscillator which takes tonality and brightness as arguments. \subsection*{Description}


  An oscillator which takes tonality and brightness as arguments, relative to a base frequency. 
\subsection*{Syntax}


 ar \textbf{hsboscil}
 kamp, ktone, kbrite, ibasfreq, iwfn, ioctfn [, ioctcnt] [, iphs]
\subsection*{Initialization}


 \emph{ibasfreq}
 -- base frequency to which tonality and brighness are relative 


 \emph{iwfn}
 -- function table of the waveform, usually a sine 


 \emph{ioctfn}
 -- function table used for weighting the octaves, usually something like: 


 f1�0��1024��-19��1��0.5��270��0.5\\ 
 ������


 \emph{ioctcnt}
 (optional) -- number of octaves used for brightness blending. Must be in the range 2 to 10. Default is 3. 


 \emph{iphs}
 (optional, default=0) -- initial phase of the oscillator. If \emph{iphs}
 = -1, initialization is skipped. 
\subsection*{Performance}


 \emph{kamp}
 -- amplitude of note 


 \emph{ktone}
 -- cyclic tonality parameter relative to \emph{ibasfreq}
 in logarithmic octave, range 0 to 1, values $>$ 1 can be used, and are internally reduced to \emph{frac}
(\emph{ktone}
). 


 \emph{kbrite}
 -- brightness parameter relative to \emph{ibasfreq}
, achieved by weighting \emph{ioctcnt}
 octaves. It is scaled in such a way, that a value of 0 corresponds to the orignal value of \emph{ibasfreq}
, 1 corresponds to one octave above \emph{ibasfreq}
, -2 corresponds to two octaves below \emph{ibasfreq}
, etc. \emph{kbrite}
 may be fractional. 


 \emph{hsboscil}
 takes tonality and brightness as arguments, relative to a base frequency (\emph{ibasfreq}
). Tonality is a cyclic parameter in the logarithmic octave, brightness is realized by mixing multiple weighted octaves. It is useful when tone space is understood in a concept of polar coordinates. 


  Making \emph{ktone}
 a line, and \emph{kbrite}
 a constant, produces Risset's glissando. 


  Oscillator table \emph{iwfn}
 is always read interpolated. Performance time requires about \emph{ioctcnt}
 * \emph{oscili}
. 
\subsection*{Examples}


  Here is an example of the hsboscil opcode. It uses the files \emph{hsboscil.orc}
 and \emph{hsboscil.sco}
. 


 \textbf{Example 1. Example of the hsboscil opcode.}

\begin{lstlisting}
/* hsboscil.orc */
; Initialize the global variables.
sr = 44100
kr = 4410
ksmps = 10
nchnls = 1

; synth waveform
giwave ftgen 1, 0, 1024, 10, 1, 1, 1, 1
; blending window
giblend ftgen 2, 0, 1024, -19, 1, 0.5, 270, 0.5

; Instrument #1 - produces Risset's glissando.
instr 1
  kamp = 10000
  kbrite = 0.5
  ibasfreq = 200
  ioctcnt = 5

  ; Change ktone linearly from 0 to 1, 
  ; over the period defined by p3.
  ktone line 0, p3, 1

  a1 hsboscil kamp, ktone, kbrite, ibasfreq, giwave, giblend, ioctcnt
  out a1
endin
/* hsboscil.orc */
        
\end{lstlisting}
\begin{lstlisting}
/* hsboscil.sco */
; Play Instrument #1 for ten seconds.
i 1 0 10
e
/* hsboscil.sco */
        
\end{lstlisting}


  Here is an example of the hsboscil opcode in a MIDI instrument. It uses the files \emph{hsboscil\_midi.orc}
 and \emph{hsboscil\_midi.sco}
. 


 \textbf{Example 2. Example of the hsboscil opcode in a MIDI instrument.}

\begin{lstlisting}
/* hsboscil_midi.orc */
; Initialize the global variables.
sr = 44100
kr = 4410
ksmps = 10
nchnls = 1

; synth waveform
giwave ftgen 1, 0, 1024, 10, 1, 1, 1, 1
; blending window
giblend ftgen 2, 0, 1024, -19, 1, 0.5, 270, 0.5

; Instrument #1 - use hsboscil in a MIDI instrument.
instr 1
  ibase = cpsoct(6)
  ioctcnt = 5

  ; all octaves sound alike.
  itona octmidi
  ; velocity is mapped to brightness
  ibrite ampmidi 3

  ; Map an exponential envelope for the amplitude.
  kenv expon 20000, 1, 100

  asig hsboscil kenv, itona, ibrite, ibase, giwave, giblend, ioctcnt
  out asig
endin
/* hsboscil_midi.orc */
        
\end{lstlisting}
\begin{lstlisting}
/* hsboscil_midi.sco */
; Play Instrument #1 for ten minutes
i 1 0 6000
e
/* hsboscil_midi.sco */
        
\end{lstlisting}
\subsection*{Credits}


 


 


\begin{tabular}{ccc}
Author: Peter Neub\"acker &Munich, Germany &August, 1999

\end{tabular}



 


 New in Csound version 3.58
%\hline 


\begin{comment}
\begin{tabular}{lcr}
Previous &Home &Next \\
hrtfer &Up &i

\end{tabular}


\end{document}
\end{comment}
