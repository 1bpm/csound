\begin{comment}
\documentclass[10pt]{article}
\usepackage{fullpage, graphicx, url}
\setlength{\parskip}{1ex}
\setlength{\parindent}{0ex}
\title{Soundfile Formats.}
\begin{document}


\begin{tabular}{ccc}
The Alternative Csound Reference Manual & & \\
Previous &The Utility Programs &Next

\end{tabular}

%\hline 
\end{comment}
\section{Soundfile Formats.}


  Csound can read and write audio files in a variety of formats. Write formats are described by Csound command flags. On reading, the format is determined from the soundfile header, and the data automatically converted to floating-point during internal processing. When Csound is installed on a host with local soundfile conventions (SUN, NeXT, Macintosh) it may conditionally include local packaging code which creates soundfiles not portable to other hosts. However, Csound on any host can always generate and read AIFF files, which is thus a portable format. Sampled sound libraries are typically AIFF, and the variable SSDIR usually points to a directory of such sounds. If defined, the SSDIR directory is in the search path during soundfile access. Note that some AIFF sampled sounds have an audio looping feature for sustained performance; the analysis programs will traverse any loop segment once only. 


  For soundfiles without headers, an SR value may be supplied by the \emph{-R flag}
 (or its default). If both the \emph{SR header}
 and the command-line flag are present, the flag value will override the header. 


  When sound is accessed by the audio Analysis programs , only a single channel is read. For stereo or quad files, the default is channel one; alternate channels may be obtained on request. 
%\hline 


\begin{comment}
\begin{tabular}{lcr}
Previous &Home &Next \\
The Utility Programs &Up &Credits

\end{tabular}


\end{document}
\end{comment}
