\begin{comment}
\documentclass[10pt]{article}
\usepackage{fullpage, graphicx, url}
\setlength{\parskip}{1ex}
\setlength{\parindent}{0ex}
\title{rspline}
\begin{document}


\begin{tabular}{ccc}
The Alternative Csound Reference Manual & & \\
Previous & &Next

\end{tabular}

%\hline 
\end{comment}
\section{rspline}
rspline�--� Generate random spline curves. \subsection*{Description}


  Generate random spline curves. 
\subsection*{Syntax}


 ar \textbf{rspline}
 xrangeMin, xrangeMax, kcpsMin, kcpsMax


 kr \textbf{rspline}
 krangeMin, krangeMax, kcpsMin, kcpsMax
\subsection*{Performance}


 \emph{kr, ar}
 -- Output signal 


 \emph{xrangeMin, xrangeMax}
 -- Range of values of random-generated points 


 \emph{kcpsMin, kcpsMax}
 -- Range of point-generation rate. Min and max limits are expressed in cps. 


 \emph{xamp}
 -- Amplitude factor 


 \emph{rspline}
 (random-spline-curve generator) is similar to \emph{jspline}
 but output range is defined by means of two limit values. Also in this case, real output range could be a bit greater of range values, because of interpolating curves beetween each pair of random-points. 


  At present time generated curves are quite smooth when cpsMin is not too different from cpsMax. When cpsMin-cpsMax interval is big, some little discontinuity could occurr, but it should not be a problem, in most cases. Maybe the algorithm will be improved in next versions. 


  These opcodes are often better than \emph{jitter}
 when user wants to ``naturalize'' or ``analogize'' digital sounds. They could be used also in algorithmic composition, to generate smooth random melodic lines when used together with \emph{samphold}
 opcode. 


  Note that the result is quite different from the one obtained by filtering white noise, and they allow the user to obtain a much more precise control. 
\subsection*{Credits}


 Author: Gabriel Maldonado


 New in version 4.15
%\hline 


\begin{comment}
\begin{tabular}{lcr}
Previous &Home &Next \\
rnd31 &Up &rtclock

\end{tabular}


\end{document}
\end{comment}
