\begin{comment}
\documentclass[10pt]{article}
\usepackage{fullpage, graphicx, url}
\setlength{\parskip}{1ex}
\setlength{\parindent}{0ex}
\title{trigger}
\begin{document}


\begin{tabular}{ccc}
The Alternative Csound Reference Manual & & \\
Previous & &Next

\end{tabular}

%\hline 
\end{comment}
\section{trigger}
trigger�--� Informs when a krate signal crosses a threshold. \subsection*{Description}


  Informs when a krate signal crosses a threshold. 
\subsection*{Syntax}


 kout \textbf{trigger}
 ksig, kthreshold, kmode
\subsection*{Performance}


 \emph{ksig}
 -- input signal 


 \emph{kthreshold}
 -- trigger threshold 


 \emph{kmode}
 -- can be 0 , 1 or 2 


  Normally \emph{trigger}
 outputs zeroes: only each time \emph{ksig}
 crosses \emph{kthreshold}
 \emph{trigger}
 outputs a 1. There are three modes of using \emph{ktrig}
: 


 
\begin{itemize}
\item 

 \emph{kmode}
 = 0 - (down-up) \emph{ktrig}
 outputs a 1 when current value of \emph{ksig}
 is higher than \emph{kthreshold,}
 while old value of \emph{ksig}
 was equal to or lower than \emph{kthreshold}
.

\item 

 \emph{kmode}
 = 1 - (up-down) \emph{ktrig}
 outputs a 1 when current value of \emph{ksig}
 is lower than \emph{kthreshold}
 while old value of \emph{ksig}
 was equal or higher than \emph{kthreshold}
.

\item 

 \emph{kmode}
 = 2 - (both) \emph{ktrig}
 outputs a 1 in both the two previous cases.


\end{itemize}
\subsection*{Examples}


  Here is an example of the trigger opcode. It uses the files \emph{trigger.orc}
 and \emph{trigger.sco}
. 


 \textbf{Example 1. Example of the trigger opcode.}

\begin{lstlisting}
/* trigger.orc */
; Initialize the global variables.
sr = 44100
kr = 4410
ksmps = 10
nchnls = 1

; Instrument #1.
instr 1
  ; Use a square-wave low frequency oscillator as the trigger.
  klf lfo 1, 10, 3
  ktr trigger klf, 1, 2

  ; When the value of the trigger isn't equal to 0, print it out.
  if (ktr == 0) kgoto contin
    ; Print the value of the trigger and the time it occurred.
    ktm times
    printks "time = %f seconds, trigger = %f\\n", 0, ktm, ktr

contin:
  ; Continue with processing.
endin
/* trigger.orc */
        
\end{lstlisting}
\begin{lstlisting}
/* trigger.sco */
; Play Instrument #1 for one second.
i 1 0 1
e
/* trigger.sco */
        
\end{lstlisting}
 Its output should include lines like this: \begin{lstlisting}
time = 0.050340 seconds, trigger = 1.000000
time = 0.150340 seconds, trigger = 1.000000
time = 0.250340 seconds, trigger = 1.000000
time = 0.350340 seconds, trigger = 1.000000
time = 0.450340 seconds, trigger = 1.000000
time = 0.550340 seconds, trigger = 1.000000
time = 0.650340 seconds, trigger = 1.000000
time = 0.750340 seconds, trigger = 1.000000
time = 0.850340 seconds, trigger = 1.000000
time = 0.950340 seconds, trigger = 1.000000
      
\end{lstlisting}
\subsection*{Credits}


 


 


\begin{tabular}{cc}
Author: Gabriel Maldonado &Italy

\end{tabular}



 


 Example written by Kevin Conder.


 New in Csound version 3.49
%\hline 


\begin{comment}
\begin{tabular}{lcr}
Previous &Home &Next \\
transeg &Up &trigseq

\end{tabular}


\end{document}
\end{comment}
