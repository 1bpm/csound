\begin{comment}
\documentclass[10pt]{article}
\usepackage{fullpage, graphicx, url}
\setlength{\parskip}{1ex}
\setlength{\parindent}{0ex}
\title{schedkwhen}
\begin{document}


\begin{tabular}{ccc}
The Alternative Csound Reference Manual & & \\
Previous & &Next

\end{tabular}

%\hline 
\end{comment}
\section{schedkwhen}
schedkwhen�--� Adds a new score event generated by a k-rate trigger. \subsection*{Description}


  Adds a new score event generated by a k-rate trigger. 
\subsection*{Syntax}


 \textbf{schedkwhen}
 ktrigger, kmintim, kmaxnum, kinsnum, kwhen, kdur [, ip4] [, ip5] [...]


 \textbf{schedkwhen}
 ktrigger, kmintim, kmaxnum, ``insname'', kwhen, kdur [, ip4] [, ip5] [...]
\subsection*{Initialization}


 \emph{``insname''}
 -- A string (in double-quotes) representing a named instrument. 


 \emph{ip4, ip5, ...}
 -- Equivalent to p4, p5, etc., in a score \emph{i statement}

\subsection*{Performance}


 \emph{ktrigger}
 -- triggers a new score event. If \emph{ktrigger}
 = 0, no new event is triggered. 


 \emph{kmintim}
 -- minimum time between generated events, in seconds. If \emph{kmintim}
 $<$= 0, no time limit exists. If the \emph{kinsnum}
 is negative (to turn off an instrument), this test is bypassed. 


 \emph{kmaxnum}
 -- maximum number of simultaneous instances of instrument \emph{kinsnum}
 allowed. If the number of extant instances of \emph{kinsnum}
 is $>$= \emph{kmaxnum}
, no new event is generated. If \emph{kmaxnum}
 is $<$= 0, it is not used to limit event generation. If the \emph{kinsnum}
 is negative (to turn off an instrument), this test is bypassed. 


 \emph{kinsnum}
 -- instrument number. Equivalent to p1 in a score \emph{i statement}
. 


 \emph{kwhen}
 -- start time of the new event. Equivalent to p2 in a score \emph{i statement}
. Measured from the time of the triggering event. \emph{kwhen}
 must be $>$= 0. If \emph{kwhen}
 $>$ 0, the instrument will not be initialized until the actual time when it should start performing. 


 \emph{kdur}
 -- duration of event. Equivalent to p3 in a score \emph{i statement}
. If \emph{kdur}
 = 0, the instrument will only do an initialization pass, with no performance. If \emph{kdur}
 is negative, a held note is initiated. (See \emph{ihold}
 and \emph{i statement}
.) 


 \emph{Note}
: While waiting for events to be triggered by \emph{schedkwhen}
, the performance must be kept going, or Csound may quit if no score events are expected. To guarantee continued performance, an \emph{f0 statement}
 may be used in the score. 
\subsection*{Examples}


  Here is an example of the schedkwhen opcode. It uses the files \emph{schedkwhen.orc}
 and \emph{schedkwhen.sco}
. 


 \textbf{Example 1. Example of the schedkwhen opcode.}

\begin{lstlisting}
/* schedkwhen.orc */
; Initialize the global variables.
sr = 44100
kr = 44100
ksmps = 1
nchnls = 1

; Instrument #1 - oscillator with a high note.
instr 1
  ; Use the fourth p-field as the trigger.
  ktrigger = p4
  kmintim = 0
  kmaxnum = 2
  kinsnum = 2
  kwhen = 0
  kdur = 0.5

  ; Play Instrument #2 at the same time, if the trigger is set.
  schedkwhen ktrigger, kmintim, kmaxnum, kinsnum, kwhen, kdur

  ; Play a high note.
  a1 oscils 10000, 880, 1
  out a1
endin

; Instrument #2 - oscillator with a low note.
instr 2
  ; Play a low note.
  a1 oscils 10000, 220, 1
  out a1
endin
/* schedkwhen.orc */
        
\end{lstlisting}
\begin{lstlisting}
/* schedkwhen.sco */
; Table #1, a sine wave.
f 1 0 16384 10 1

; p4 = trigger for Instrument #2 (when p4 > 0).
; Play Instrument #1 for half a second, no trigger.
i 1 0 0.5 0
; Play Instrument #1 for half a second, trigger Instrument #2.
i 1 1 0.5 1
e
/* schedkwhen.sco */
        
\end{lstlisting}
\subsection*{Credits}


 


 


\begin{tabular}{cc}
Author: Rasmus Ekman &EMS, Stockholm, Sweden

\end{tabular}



 


 Example written by Kevin Conder.


 New in Csound version 3.59
%\hline 


\begin{comment}
\begin{tabular}{lcr}
Previous &Home &Next \\
scanu &Up &schedkwhennamed

\end{tabular}


\end{document}
\end{comment}
