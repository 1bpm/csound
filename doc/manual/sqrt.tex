\begin{comment}
\documentclass[10pt]{article}
\usepackage{fullpage, graphicx, url}
\setlength{\parskip}{1ex}
\setlength{\parindent}{0ex}
\title{sqrt}
\begin{document}


\begin{tabular}{ccc}
The Alternative Csound Reference Manual & & \\
Previous & &Next

\end{tabular}

%\hline 
\end{comment}
\section{sqrt}
sqrt�--� Returns a square root value. \subsection*{Description}


  Returns the square root of \emph{x}
 (\emph{x}
 non-negative). 


  The argument value is restricted for \emph{log}
, \emph{log10}
, and \emph{sqrt}
. 
\subsection*{Syntax}


 \textbf{sqrt}
(x) (no rate restriction)


  where the argument within the parentheses may be an expression. Value converters perform arithmetic translation from units of one kind to units of another. The result can then be a term in a further expression. 
\subsection*{Examples}


  Here is an example of the sqrt opcode. It uses the files \emph{sqrt.orc}
 and \emph{sqrt.sco}
. 


 \textbf{Example 1. Example of the sqrt opcode.}

\begin{lstlisting}
/* sqrt.orc */
; Initialize the global variables.
sr = 44100
kr = 4410
ksmps = 10
nchnls = 1

; Instrument #1.
instr 1
  i1 = sqrt(64)
  print i1
endin
/* sqrt.orc */
        
\end{lstlisting}
\begin{lstlisting}
/* sqrt.sco */
; Play Instrument #1 for one second.
i 1 0 1
e
/* sqrt.sco */
        
\end{lstlisting}
 Its output should include lines like this: \begin{lstlisting}
instr 1:  i1 = 8.000
      
\end{lstlisting}
\subsection*{See Also}


 \emph{abs}
, \emph{exp}
, \emph{frac}
, \emph{int}
, \emph{log}
, \emph{log10}
, \emph{i}

\subsection*{Credits}


 Example written by Kevin Conder.
%\hline 


\begin{comment}
\begin{tabular}{lcr}
Previous &Home &Next \\
spsend &Up &sr

\end{tabular}


\end{document}
\end{comment}
