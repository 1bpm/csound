\begin{comment}
\documentclass[10pt]{article}
\usepackage{fullpage, graphicx, url}
\setlength{\parskip}{1ex}
\setlength{\parindent}{0ex}
\title{vdelay3}
\begin{document}


\begin{tabular}{ccc}
The Alternative Csound Reference Manual & & \\
Previous & &Next

\end{tabular}

%\hline 
\end{comment}
\section{vdelay3}
vdelay3�--� An variable time delay with cubic interpolation. \subsection*{Description}


 \emph{vdelay3}
 is experimental. It is the same as \emph{vdelay}
 except that it uses cubic interpolation. (New in Version 3.50.) 
\subsection*{Syntax}


 ar \textbf{vdelay3}
 asig, adel, imaxdel [, iskip]
\subsection*{Initialization}


 \emph{imaxdel}
 -- Maximum value of delay in milliseconds. If \emph{adel}
 gains a value greater than \emph{imaxdel}
 it is folded around \emph{imaxdel}
. This should not happen. 


 \emph{iskip}
 (optional) -- Skip initialization if present and non-zero. 
\subsection*{Performance}


  With this unit generator it is possible to do Doppler effects or chorusing and flanging. 


 \emph{asig}
 -- Input signal. 


 \emph{adel}
 -- Current value of delay in milliseconds. Note that linear functions have no pitch change effects. Fast changing values of \emph{adel}
 will cause discontinuities in the waveform resulting noise. 
\subsection*{Examples}


 


 
\begin{lstlisting}
  f1 0 8192 10 1
  ims     =          100             ; Maximum delay time in msec
  a1      \emph{oscil}
      10000, 1737, 1  ; Make a signal
  a2      \emph{oscil}
      ims/2, 1/p3, 1  ; Make an LFO
  a2      =          a2 + ims/2      ; Offset the LFO so that it is positive
  a3      \emph{vdelay}
     a1, a2, ims     ; Use the LFO to control delay time
          \emph{out}
        a3
        
\end{lstlisting}


 


  Two important points here. First, the delay time must be always positive. And second, even though the delay time can be controlled in k-rate, it is not advised to do so, since sudden time changes will create clicks. 
\subsection*{See Also}


 \emph{vdelay}

\subsection*{Credits}


 


 


\begin{tabular}{ccc}
Author: Paris Smaragdis &MIT, Cambridge &1995

\end{tabular}



 
%\hline 


\begin{comment}
\begin{tabular}{lcr}
Previous &Home &Next \\
vdelay &Up &vdelayx

\end{tabular}


\end{document}
\end{comment}
