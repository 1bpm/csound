\begin{comment}
\documentclass[10pt]{article}
\usepackage{fullpage, graphicx, url}
\setlength{\parskip}{1ex}
\setlength{\parindent}{0ex}
\title{FLsetColor}
\begin{document}


\begin{tabular}{ccc}
The Alternative Csound Reference Manual & & \\
Previous & &Next

\end{tabular}

%\hline 
\end{comment}
\section{FLsetColor}
FLsetColor�--� Sets the primary color of a FLTK widget. \subsection*{Description}


 \emph{FLsetColor}
 sets the primary color of the target widget. 
\subsection*{Syntax}


 \textbf{FLsetColor}
 ired, igreen, iblue, ihandle
\subsection*{Initialization}


 \emph{ired}
 -- The red color of the target widget. The range for each RGB component is 0-255 


 \emph{igreen}
 -- The green color of the target widget. The range for each RGB component is 0-255 


 \emph{iblue}
 -- The blue color of the target widget. The range for each RGB component is 0-255 


 \emph{ihandle}
 -- an integer number (used as unique identifier) taken from the output of a previously located widget opcode (which corresponds to the target widget). It is used to unequivocally identify the widget when modifying its appearance with this class of opcodes. The user must not set the \emph{ihandle}
 value directly, otherwise a Csound crash will occur. 
\subsection*{Examples}


  Here is an example of the flsetcolor opcode. It uses the files \emph{flsetcolor.orc}
 and \emph{flsetcolor.sco}
. 


 \textbf{Example 1. Example of the flsetcolor opcode.}

\begin{lstlisting}
/* flsetcolor.orc */
; Using the opcode flsetcolor to change from the
; default colours for widgets
sr = 44100
kr = 441
ksmps = 100
nchnls = 1

FLpanel "Coloured Sliders", 900, 360, 50, 50
    gkfreq, ihandle FLslider "A Red Slider", 200, 5000, -1, 5, -1, 750, 30, 85, 50
    ired1 = 255
    igreen1 = 0
    iblue1 = 0
    FLsetColor ired1, igreen1, iblue1, ihandle

    gkfreq, ihandle FLslider "A Green Slider", 200, 5000, -1, 5, -1, 750, 30, 85, 150
    ired1 = 0
    igreen1 = 255
    iblue1 = 0
    FLsetColor ired1, igreen1, iblue1, ihandle

    gkfreq, ihandle FLslider "A Blue Slider", 200, 5000, -1, 5, -1, 750, 30, 85, 250
    ired1 = 0
    igreen1 = 0
    iblue1 = 255
    FLsetColor ired1, igreen1, iblue1, ihandle
; End of panel contents
FLpanelEnd
; Run the widget thread!
FLrun

instr 1
endin
/* flsetcolor.orc */
        
\end{lstlisting}
\begin{lstlisting}
/* flsetcolor.sco */
; 'Dummy' score event for 1 hour.
f 0 3600
e
/* flsetcolor.sco */
        
\end{lstlisting}
\subsection*{See Also}


 \emph{FLcolor}
, \emph{FLcolor2}
, \emph{FLhide}
, \emph{FLlabel}
, \emph{FLsetAlign}
, \emph{FLsetBox}
, \emph{FLsetColor2}
, \emph{FLsetFont}
, \emph{FLsetPosition}
, \emph{FLsetSize}
, \emph{FLsetText}
, \emph{FLsetTextColor}
, \emph{FLsetTextSize}
, \emph{FLsetTextType}
, \emph{FLsetVal\_i}
, \emph{FLsetVal}
, \emph{FLshow}

\subsection*{Credits}


 Author: Gabriel Maldonado


 New in version 4.22


 Example written by Iain McCurdy, edited by Kevin Conder.
%\hline 


\begin{comment}
\begin{tabular}{lcr}
Previous &Home &Next \\
FLsetBox &Up &FLsetColor2

\end{tabular}


\end{document}
\end{comment}
