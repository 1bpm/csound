\begin{comment}
\documentclass[10pt]{article}
\usepackage{fullpage, graphicx, url}
\setlength{\parskip}{1ex}
\setlength{\parindent}{0ex}
\title{a}
\begin{document}


\begin{tabular}{ccc}
The Alternative Csound Reference Manual & & \\
Previous & &Next

\end{tabular}

%\hline 
\end{comment}
\section{a}
a�--� Converts a k-rate parameter to an a-rate value with interpolation. \subsection*{Description}


  Converts a k-rate parameter to an a-rate value with interpolation. 
\subsection*{Syntax}


 \textbf{a}
(x) (control-rate args only)


  where the argument within the parentheses may be an expression. Value converters perform arithmetic translation from units of one kind to units of another. The result can then be a term in a further expression. 
\subsection*{Examples}


  Here is an example of the a opcode. It uses the files \emph{a.orc}
 and \emph{a.sco}
. 


 \textbf{Example 1. Example of the a opcode.}

\begin{lstlisting}
/* a.orc */
; Initialize the global variables.
sr = 44100
kr = 4410
ksmps = 10
nchnls = 1

; Instrument #1.
instr 1
  ; Create a sine wave at k-rate.
  kwave oscil 20000, 440, 1

  ; Convert the k-rate sine wave to the audio-rate.
  awave = a(kwave)

  ; Output the audio-rate version of sine wave.
  out awave
endin
/* a.orc */
        
\end{lstlisting}
\begin{lstlisting}
/* a.sco */
; Table #1, a sine wave.
f 1 0 16384 10 1

; Play Instrument #1 for one second.
i 1 0 1
e
/* a.sco */
        
\end{lstlisting}
\subsection*{See Also}


 \emph{i}

\subsection*{Credits}


 Author: Gabriel Maldonado


 Example written by Kevin Conder.


 New in version 4.21
%\hline 


\begin{comment}
\begin{tabular}{lcr}
Previous &Home &Next \\
0dbfs &Up &abetarand

\end{tabular}


\end{document}
\end{comment}
