\begin{comment}
\documentclass[10pt]{article}
\usepackage{fullpage, graphicx, url}
\setlength{\parskip}{1ex}
\setlength{\parindent}{0ex}
\title{FLsetFont}
\begin{document}


\begin{tabular}{ccc}
The Alternative Csound Reference Manual & & \\
Previous & &Next

\end{tabular}

%\hline 
\end{comment}
\section{FLsetFont}
FLsetFont�--� Sets the font type of a FLTK widget. \subsection*{Description}


 \emph{FLsetFont}
 sets the font type of the target widget. 
\subsection*{Syntax}


 \textbf{FLsetFont}
 ifont, ihandle
\subsection*{Initialization}


 \emph{ifont}
 -- sets the the font type of the label of a widget. 


  Legal values for ifont argument are: 


 
\begin{itemize}
\item 

 1 - Helvetica (same as Arial under Windows)

\item 

 2 - Helvetica Bold

\item 

 3 - Helvetica Italic

\item 

 4 - Helvetica Bold Italic

\item 

 5 - Courier

\item 

 6 - Courier Bold

\item 

 7 - Courier Italic

\item 

 8 - Courier Bold Italic

\item 

 9 - Times

\item 

 10 - Times Bold

\item 

 11 - Times Italic

\item 

 12 - Times Bold Italic

\item 

 13 - Symbol

\item 

 14 - Screen

\item 

 15 - Screen Bold

\item 

 16 - Dingbats


\end{itemize}


 \emph{ihandle}
 -- an integer number (used as unique identifier) taken from the output of a previously located widget opcode (which corresponds to the target widget). It is used to unequivocally identify the widget when modifying its appearance with this class of opcodes. The user must not set the \emph{ihandle}
 value directly, otherwise a Csound crash will occur. 
\subsection*{See Also}


 \emph{FLcolor}
, \emph{FLcolor2}
, \emph{FLhide}
, \emph{FLlabel}
, \emph{FLsetAlign}
, \emph{FLsetBox}
, \emph{FLsetColor}
, \emph{FLsetColor2}
, \emph{FLsetPosition}
, \emph{FLsetSize}
, \emph{FLsetText}
, \emph{FLsetTextColor}
, \emph{FLsetTextSize}
, \emph{FLsetTextType}
, \emph{FLsetVal\_i}
, \emph{FLsetVal}
, \emph{FLshow}

\subsection*{Credits}


 Author: Gabriel Maldonado


 New in version 4.22
%\hline 


\begin{comment}
\begin{tabular}{lcr}
Previous &Home &Next \\
FLsetColor2 &Up &FLsetPosition

\end{tabular}


\end{document}
\end{comment}
