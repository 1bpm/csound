\begin{comment}
\documentclass[10pt]{article}
\usepackage{fullpage, graphicx, url}
\setlength{\parskip}{1ex}
\setlength{\parindent}{0ex}
\title{lposcil3}
\begin{document}


\begin{tabular}{ccc}
The Alternative Csound Reference Manual & & \\
Previous & &Next

\end{tabular}

%\hline 
\end{comment}
\section{lposcil3}
lposcil3�--� Read sampled sound from a table with high precision and cubic interpolation. \subsection*{Description}


  Read sampled sound (mono or stereo) from a table, with optional sustain and release looping, and high precision. \emph{lposcil3}
 uses cubic interpolation. 
\subsection*{Syntax}


 ar \textbf{lposcil3}
 kamp, kfreqratio, kloop, kend, ifn [, iphs]
\subsection*{Initialization}


 \emph{ifn}
 -- function table number 
\subsection*{Performance}


 \emph{kamp}
 -- amplitude 


 \emph{kfreqratio}
 -- multiply factor of table frequency (for example: 1 = original frequency, 1.5 = a fifth up , .5 = an octave down) 


 \emph{kloop}
 -- loop point (in samples) 


 \emph{kend}
 -- end loop point (in samples) 


 \emph{lposcil}
 (looping precise oscillator) allows varying at k-rate, the starting and ending point of a sample contained in a table (\emph{GEN01}
). This can be useful when reading a sampled loop of a wavetable, where repeat speed can be varied during the performance. 
\subsection*{See Also}


 \emph{lposcil}

\subsection*{Credits}


 


 


\begin{tabular}{cc}
Author: Gabriel Maldonado &Italy

\end{tabular}



 


 New in Csound version 3.52
%\hline 


\begin{comment}
\begin{tabular}{lcr}
Previous &Home &Next \\
lposcil &Up &lpread

\end{tabular}


\end{document}
\end{comment}
