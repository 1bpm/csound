\begin{comment}
\documentclass[10pt]{article}
\usepackage{fullpage, graphicx, url}
\setlength{\parskip}{1ex}
\setlength{\parindent}{0ex}
\title{ftlptim}
\begin{document}


\begin{tabular}{ccc}
The Alternative Csound Reference Manual & & \\
Previous & &Next

\end{tabular}

%\hline 
\end{comment}
\section{ftlptim}
ftlptim�--� Returns the loop segment start-time of a stored function table number. \subsection*{Description}


  Returns the loop segment start-time of a stored function table number. 
\subsection*{Syntax}


 \textbf{ftlptim}
(x) (init-rate args only)
\subsection*{Performance}


  Returns the loop segment start-time (in seconds) of stored function table number \emph{x}
. This reports the duration of the direct recorded attack and decay parts of a sound sample, prior to its looped segment. Returns zero (and a warning message) if the sample does not contain loop points. 
\subsection*{Examples}


  Here is an example of the ftlptim opcode. It uses the files \emph{ftlptim.orc}
, \emph{ftlptim.sco}
, and \emph{mary.wav}
. 


 \textbf{Example 1. Example of the ftlptim opcode.}

\begin{lstlisting}
/* ftlptim.orc */
; Initialize the global variables.
sr = 44100
kr = 4410
ksmps = 10
nchnls = 1

; Instrument #1.
instr 1
  ; Print out the loop-segment start time in Table #1.
  itim = ftlptim(1)
  print itim
endin
/* ftlptim.orc */
        
\end{lstlisting}
\begin{lstlisting}
/* ftlptim.sco */
; Table #1: Use an audio file, Csound will determine its size.
f 1 0 0 1 "mary.wav" 0 0 0

; Play Instrument #1 for 1 second.
i 1 0 1
e
/* ftlptim.sco */
        
\end{lstlisting}
 Since the audio file ``mary.wav'' is non-looping, its output should include lines like this: \begin{lstlisting}
WARNING: non-looping sample
instr 1:  itim = 0.000
      
\end{lstlisting}
\subsection*{See Also}


 \emph{ftchnls}
, \emph{ftlen}
, \emph{ftsr}
, \emph{nsamp}

\subsection*{Credits}


 


 


\begin{tabular}{cccc}
Author: Barry L. Vercoe &MIT &Cambridge, Massachussetts &1997

\end{tabular}



 


 Example written by Kevin Conder.
%\hline 


\begin{comment}
\begin{tabular}{lcr}
Previous &Home &Next \\
ftloadk &Up &ftmorf

\end{tabular}


\end{document}
\end{comment}
