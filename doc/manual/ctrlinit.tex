\begin{comment}
\documentclass[10pt]{article}
\usepackage{fullpage, graphicx, url}
\setlength{\parskip}{1ex}
\setlength{\parindent}{0ex}
\title{ctrlinit}
\begin{document}


\begin{tabular}{ccc}
The Alternative Csound Reference Manual & & \\
Previous & &Next

\end{tabular}

%\hline 
\end{comment}
\section{ctrlinit}
ctrlinit�--� Sets the initial values for a set of MIDI controllers. \subsection*{Description}


  Sets the initial values for a set of MIDI controllers. 
\subsection*{Syntax}


 \textbf{ctrlinit}
 ichnl, ictlno1, ival1 [, ictlno2] [, ival2] [, ictlno3] [, ival3] [,...ival32]
\subsection*{Initialization}


 \emph{ichnl}
 -- MIDI channel number (1-16) 


 \emph{ictlno1}
, \emph{ictlno1}
, etc. -- MIDI controller numbers (0-127) 


 \emph{ival1}
, \emph{ival2}
, etc. -- initial value for corresponding MIDI controller number 
\subsection*{Performance}


  Sets the initial values for a set of MIDI controllers. 
\subsection*{See Also}


 \emph{massign}

\subsection*{Credits}


 


 


\begin{tabular}{cc}
Author: Barry L. Vercoe - Mike Berry &MIT, Cambridge, Mass.

\end{tabular}



 


 New in Csound version 3.47


 Thanks goes to Rasmus Ekman for pointing out the correct MIDI channel and controller number ranges.
%\hline 


\begin{comment}
\begin{tabular}{lcr}
Previous &Home &Next \\
ctrl7 &Up &cuserrnd

\end{tabular}


\end{document}
\end{comment}
