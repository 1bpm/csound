\begin{comment}
\documentclass[10pt]{article}
\usepackage{fullpage, graphicx, url}
\setlength{\parskip}{1ex}
\setlength{\parindent}{0ex}
\title{multitap}
\begin{document}


\begin{tabular}{ccc}
The Alternative Csound Reference Manual & & \\
Previous & &Next

\end{tabular}

%\hline 
\end{comment}
\section{multitap}
multitap�--� Multitap delay line implementation. \subsection*{Description}


  Multitap delay line implementation. 
\subsection*{Syntax}


 ar \textbf{multitap}
 asig [, itime1] [, igain1] [, itime2] [, igain2] [...]
\subsection*{Initialization}


  The arguments \emph{itime}
 and \emph{igain}
 set the position and gain of each tap. 


  The delay line is fed by \emph{asig}
. 
\subsection*{Examples}


 


 
\begin{lstlisting}
  a1      \emph{oscil}
      1000, 100, 1
  a2      \emph{multitap}
   a1, 1.2, .5, 1.4, .2
          \emph{out}
        a2
        
\end{lstlisting}


 


  This results in two delays, one with length of 1.2 and gain of .5, and one with length of 1.4 and gain of .2. 
\subsection*{Credits}


 


 


\begin{tabular}{ccc}
Author: Paris Smaragdis &MIT, Cambridge &1996

\end{tabular}



 
%\hline 


\begin{comment}
\begin{tabular}{lcr}
Previous &Home &Next \\
mrtmsg &Up &mute

\end{tabular}


\end{document}
\end{comment}
