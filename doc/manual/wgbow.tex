\begin{comment}
\documentclass[10pt]{article}
\usepackage{fullpage, graphicx, url}
\setlength{\parskip}{1ex}
\setlength{\parindent}{0ex}
\title{wgbow}
\begin{document}


\begin{tabular}{ccc}
The Alternative Csound Reference Manual & & \\
Previous & &Next

\end{tabular}

%\hline 
\end{comment}
\section{wgbow}
wgbow�--� Creates a tone similar to a bowed string. \subsection*{Description}


  Audio output is a tone similar to a bowed string, using a physical model developed from Perry Cook, but re-coded for Csound. 
\subsection*{Syntax}


 ar \textbf{wgbow}
 kamp, kfreq, kpres, krat, kvibf, kvamp, ifn [, iminfreq]
\subsection*{Initialization}


 \emph{ifn}
 -- table of shape of vibrato, usually a sine table, created by a function 


 \emph{iminfreq}
 (optional) -- lowest frequency at which the instrument will play. If it is omitted it is taken to be the same as the initial \emph{kfreq}
. If \emph{iminfreq}
 is negative, initialization will be skipped. 
\subsection*{Performance}


  A note is played on a string-like instrument, with the arguments as below. 


 \emph{kamp}
 -- amplitude of note. 


 \emph{kfreq}
 -- frequency of note played. 


 \emph{kpres}
 -- a parameter controlling the pressure of the bow on the string. Values should be about 3. The useful range is approximately 1 to 5. 


 \emph{krat}
 -- the position of the bow along the string. Usual playing is about 0.127236. The suggested range is 0.025 to 0.23. 


 \emph{kvibf}
 -- frequency of vibrato in Hertz. Suggested range is 0 to 12 


 \emph{kvamp}
 -- amplitude of the vibrato 
\subsection*{Examples}


  Here is an example of the wgbow opcode. It uses the files \emph{wgbow.orc}
 and \emph{wgbow.sco}
. 


 \textbf{Example 1. Example of the wgbow opcode.}

\begin{lstlisting}
/* wgbow.orc */
; Initialize the global variables.
sr = 44100
kr = 4410
ksmps = 10
nchnls = 1

; Instrument #1.
instr 1
  kamp = 31129.60
  kfreq = 440
  kpres = 3.0
  krat = 0.127236
  kvibf = 6.12723
  ifn = 1

  ; Create an amplitude envelope for the vibrato.
  kv linseg 0, 0.5, 0, 1, 1, p3-0.5, 1
  kvamp = kv * 0.01

  a1 wgbow kamp, kfreq, kpres, krat, kvibf, kvamp, ifn
  out a1
endin
/* wgbow.orc */
        
\end{lstlisting}
\begin{lstlisting}
/* wgbow.sco */
; Table #1, a sine wave.
f 1 0 128 10 1

; Play Instrument #1 for two seconds.
i 1 0 2
e
/* wgbow.sco */
        
\end{lstlisting}
\subsection*{Credits}


 


 


\begin{tabular}{ccc}
Author: John ffitch (after Perry Cook) &University of Bath, Codemist Ltd. &Bath, UK

\end{tabular}



 


 New in Csound version 3.47
%\hline 


\begin{comment}
\begin{tabular}{lcr}
Previous &Home &Next \\
weibull &Up &wgbowedbar

\end{tabular}


\end{document}
\end{comment}
