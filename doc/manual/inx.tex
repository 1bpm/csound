\begin{comment}
\documentclass[10pt]{article}
\usepackage{fullpage, graphicx, url}
\setlength{\parskip}{1ex}
\setlength{\parindent}{0ex}
\title{inx}
\begin{document}


\begin{tabular}{ccc}
The Alternative Csound Reference Manual & & \\
Previous & &Next

\end{tabular}

%\hline 
\end{comment}
\section{inx}
inx�--� Reads a 16-channel audio signal from an external device or stream. \subsection*{Description}


  Reads a 16-channel audio signal from an external device or stream. 
\subsection*{Syntax}


 ar1, ar2, ar3, ar4, ar5, ar6, ar7, ar8, ar9, ar10, ar11, ar12, ar13, ar14, ar15, ar16 \textbf{inx}

\subsection*{Performance}


 \emph{inx}
 reads a 16-channel audio signal from an external device or stream. If the command-line \emph{-i}
 flag is set, sound is read continuously from the audio input stream (e.g. \emph{stdin}
 or a soundfile) into an internal buffer. 
\subsection*{Credits}


 \emph{in32}
, \emph{inch}
, \emph{inz}

\subsection*{Credits}


 


 


\begin{tabular}{cccc}
Author: John ffitch &University of Bath/Codemist Ltd. &Bath, UK &May 2000

\end{tabular}



 


 New in Csound Version 4.07
%\hline 


\begin{comment}
\begin{tabular}{lcr}
Previous &Home &Next \\
invalue &Up &inz

\end{tabular}


\end{document}
\end{comment}
