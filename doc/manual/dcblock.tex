\begin{comment}
\documentclass[10pt]{article}
\usepackage{fullpage, graphicx, url}
\setlength{\parskip}{1ex}
\setlength{\parindent}{0ex}
\title{dcblock}
\begin{document}


\begin{tabular}{ccc}
The Alternative Csound Reference Manual & & \\
Previous & &Next

\end{tabular}

%\hline 
\end{comment}
\section{dcblock}
dcblock�--� A DC blocking filter. \subsection*{Description}


  Implements the DC blocking filter 


 Y[i]�=�X[i]�-�X[i-1]�+�(igain�*�Y[i-1])\\ 
 ������


  Based on work by Perry Cook. 
\subsection*{Syntax}


 ar \textbf{dcblock}
 ain [, igain]
\subsection*{Initialization}


 \emph{igain}
 -- the gain of the filter, which defaults to 0.99 
\subsection*{Performance}


 \emph{ain}
 -- audio signal input 
\subsection*{Examples}


  Here is an example of the dcblock opcode. It uses the files \emph{dcblock.orc}
, \emph{dcblock.sco}
, and \emph{beats.wav}
. 


 \textbf{Example 1. Example of the dcblock opcode.}

\begin{lstlisting}
/* dcblock.orc */
; Initialize the global variables.
sr = 44100
kr = 4410
ksmps = 10
nchnls = 1

; Instrument #1 -- normal audio signal.
instr 1
  asig soundin "beats.wav"
  out asig
endin

; Instrument #2 -- dcblock-ed audio signal.
instr 2
  asig soundin "beats.wav"

  igain = 0.75
  a1 dcblock asig, igain

  out a1
endin
/* dcblock.orc */
        
\end{lstlisting}
\begin{lstlisting}
/* dcblock.sco */
; Play Instrument #1 for 2 seconds.
i 1 0 2
; Play Instrument #2 for 2 seconds.
i 2 2 2
e
/* dcblock.sco */
        
\end{lstlisting}
\subsection*{Credits}


 


 


\begin{tabular}{ccc}
Author: John ffitch &University of Bath, Codemist Ltd. &Bath, UK

\end{tabular}



 


 Example written by Kevin Conder.


 New in Csound version 3.49


 February 2003: Thanks to a note from Anders Andersson, corrected the formula.
%\hline 


\begin{comment}
\begin{tabular}{lcr}
Previous &Home &Next \\
dbfsamp &Up &dconv

\end{tabular}


\end{document}
\end{comment}
