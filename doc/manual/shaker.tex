\begin{comment}
\documentclass[10pt]{article}
\usepackage{fullpage, graphicx, url}
\setlength{\parskip}{1ex}
\setlength{\parindent}{0ex}
\title{shaker}
\begin{document}


\begin{tabular}{ccc}
The Alternative Csound Reference Manual & & \\
Previous & &Next

\end{tabular}

%\hline 
\end{comment}
\section{shaker}
shaker�--� Sounds like the shaking of a maraca or similar gourd instrument. \subsection*{Description}


  Audio output is a tone related to the shaking of a maraca or similar gourd instrument. The method is a physically inspired model developed from Perry Cook, but re-coded for Csound. 
\subsection*{Syntax}


 ar \textbf{shaker}
 kamp, kfreq, kbeans, kdamp, ktimes [, idecay]
\subsection*{Initialization}


 \emph{idecay}
 -- If present indicates for how long at the end of the note the shaker is to be damped. The default value is zero. 
\subsection*{Performance}


  A note is played on a maraca-like instrument, with the arguments as below. 


 \emph{kamp}
 -- Amplitude of note. 


 \emph{kfreq}
 -- Frequency of note played. 


 \emph{kbeans}
 -- The number of beans in the gourd. A value of 8 seems suitable, 


 \emph{kdamp}
 -- The damping value of the shaker. Values of 0.98 to 1 seems suitable, with 0.99 a reasonable default. 


 \emph{ktimes}
 -- Number of times shaken. 


 


\begin{tabular}{cc}
\textbf{Note}
 \\
� &

  The argument \emph{knum}
 was redundant, so it was removed in version 3.49. 


\end{tabular}

\subsection*{Examples}


  Here is an example of the shaker opcode. It uses the files \emph{shaker.orc}
 and \emph{shaker.sco}
. 


 \textbf{Example 1. Example of the shaker opcode.}

\begin{lstlisting}
/* shaker.orc */
; Initialize the global variables.
sr = 22050
kr = 2205
ksmps = 10
nchnls = 1

; Instrument #1
instr 1
   a1 shaker 10000, 440, 8, 0.999, 100, 0
   out a1
endin
/* shaker.orc */
        
\end{lstlisting}
\begin{lstlisting}
/* shaker.sco */
i 1 0 1
e
/* shaker.sco */
        
\end{lstlisting}
\subsection*{Credits}


 


 


\begin{tabular}{ccc}
Author: John ffitch (after Perry Cook) &University of Bath, Codemist Ltd. &Bath, UK

\end{tabular}



 


 New in Csound version 3.47


 Fixed the example thanks to a message from Istvan Varga.
%\hline 


\begin{comment}
\begin{tabular}{lcr}
Previous &Home &Next \\
sfpreset &Up &sin

\end{tabular}


\end{document}
\end{comment}
