\begin{comment}
\documentclass[10pt]{article}
\usepackage{fullpage, graphicx, url}
\setlength{\parskip}{1ex}
\setlength{\parindent}{0ex}
\title{poscil}
\begin{document}


\begin{tabular}{ccc}
The Alternative Csound Reference Manual & & \\
Previous & &Next

\end{tabular}

%\hline 
\end{comment}
\section{poscil}
poscil�--� High precision oscillator. \subsection*{Description}


  High precision oscillator. 
\subsection*{Syntax}


 ar \textbf{poscil}
 aamp, acps, ifn [, iphs]


 ar \textbf{poscil}
 aamp, kcps, ifn [, iphs]


 ar \textbf{poscil}
 kamp, acps, ifn [, iphs]


 ar \textbf{poscil}
 kamp, kcps, ifn [, iphs]


 ir \textbf{poscil}
 kamp, kcps, ifn [, iphs]


 kr \textbf{poscil}
 kamp, kcps, ifn [, iphs]
\subsection*{Initialization}


 \emph{ifn}
 -- function table number 


 \emph{iphs}
 (optional, default=0) -- initial phase (in samples) 
\subsection*{Performance}


 \emph{ar}
 -- output signal 


 \emph{kamp}
, \emph{aamp}
 -- the amplitude of the output signal. 


 \emph{kcps}
, \emph{acps}
 -- the frequency of the output signal in cycles per second. 


 \emph{poscil}
 (precise oscillator) is the same as \emph{oscili}
, but allows much more precise frequency control, especially when using long tables and low frequency values. It uses floating-point table indexing, instead of integer math, like \emph{oscil}
 and \emph{oscili}
. It is only a bit slower than \emph{oscili}
. 


  Since Csound 4.22, \emph{poscil}
 can accept also negative frequency values and use a-rate values both for amplitude and frequency. So both AM and FM are allowed using this opcode. 
\subsection*{Examples}


  Here is an example of the poscil opcode. It uses the files \emph{poscil.orc}
 and \emph{poscil.sco}
. 


 \textbf{Example 1. Example of the poscil opcode.}

\begin{lstlisting}
/* poscil.orc */
; Initialize the global variables.
sr = 44100
kr = 4410
ksmps = 10
nchnls = 1

; Instrument #1 - a basic oscillator.
instr 1
  kamp = 10000
  kcps = 440
  ifn = 1

  a1 poscil kamp, kcps, ifn
  out a1
endin
/* poscil.orc */
        
\end{lstlisting}
\begin{lstlisting}
/* poscil.sco */
; Table #1, a sine wave.
f 1 0 16384 10 1

; Play Instrument #1 for 2 seconds.
i 1 0 2
e
/* poscil.sco */
        
\end{lstlisting}
\subsection*{See Also}


 \emph{poscil3}

\subsection*{Credits}


 


 


\begin{tabular}{ccc}
Author: Gabriel Maldonado &Italy &1998

\end{tabular}



 


 Example written by Kevin Conder.


 November 2002. Added a note about the changes to Csound version 4.22, thanks to Rasmus Ekman.


 New in Csound version 3.52
%\hline 


\begin{comment}
\begin{tabular}{lcr}
Previous &Home &Next \\
portk &Up &poscil3

\end{tabular}


\end{document}
\end{comment}
