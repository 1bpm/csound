\begin{comment}
\documentclass[10pt]{article}
\usepackage{fullpage, graphicx, url}
\setlength{\parskip}{1ex}
\setlength{\parindent}{0ex}
\title{cuserrnd}
\begin{document}


\begin{tabular}{ccc}
The Alternative Csound Reference Manual & & \\
Previous & &Next

\end{tabular}

%\hline 
\end{comment}
\section{cuserrnd}
cuserrnd�--� Continuous USER-defined-distribution RaNDom generator. \subsection*{Description}


  Continuous USER-defined-distribution RaNDom generator. 
\subsection*{Syntax}


 aout \textbf{cuserrnd}
 kmin, kmax, ktableNum


 iout \textbf{cuserrnd}
 imin, imax, itableNum


 kout \textbf{cuserrnd}
 kmin, kmax, ktableNum
\subsection*{Initialization}


 \emph{imin}
 -- minimum range limit 


 \emph{imax}
 -- maximum range limit 


 \emph{itableNum}
 -- number of table containing the random-distribution function. Such table is generated by the user. See GEN40, GEN41, and GEN42. The table length does not need to be a power of 2 
\subsection*{Performance}


 \emph{ktableNum}
 -- number of table containing the random-distribution function. Such table is generated by the user. See GEN40, GEN41, and GEN42. The table length does not need to be a power of 2 


 \emph{kmin}
 -- minimum range limit 


 \emph{kmax}
 -- maximum range limit 


 \emph{cuserrnd}
 (continuous user-defined-distribution random generator) generates random values according to a continuous random distribution created by the user. In this case the shape of the distribution histogram can be drawn or generated by any GEN routine. The table containing the shape of such histogram must then be translated to a distribution function by means of GEN40 (see GEN40 for more details). Then such function must be assigned to the XtableNum argument of cuserrnd. The output range can then be rescaled according to the Xmin and Xmax arguments. cuserrnd linearly interpolates between table elements, so it is not recommended for discrete distributions (GEN41 and GEN42). 


  For a tutorial about random distribution histograms and functions see: 


 
\begin{itemize}
\item 

  D. Lorrain. ``A panoply of stochastic cannons''. In C. Roads, ed. 1989. Music machine. Cambridge, Massachusetts: MIT press, pp. 351 - 379. 


\end{itemize}
\subsection*{See Also}


 \emph{duserrnd}
, \emph{urd}

\subsection*{Credits}


 Author: Gabriel Maldonado


 New in Version 4.16
%\hline 


\begin{comment}
\begin{tabular}{lcr}
Previous &Home &Next \\
ctrlinit &Up &dam

\end{tabular}


\end{document}
\end{comment}
