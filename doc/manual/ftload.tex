\begin{comment}
\documentclass[10pt]{article}
\usepackage{fullpage, graphicx, url}
\setlength{\parskip}{1ex}
\setlength{\parindent}{0ex}
\title{ftload}
\begin{document}


\begin{tabular}{ccc}
The Alternative Csound Reference Manual & & \\
Previous & &Next

\end{tabular}

%\hline 
\end{comment}
\section{ftload}
ftload�--� Load a set of previously-allocated tables from a file. \subsection*{Description}


  Load a set of previously-allocated tables from a file. 
\subsection*{Syntax}


 \textbf{ftload}
 ``filename'', iflag, ifn1 [, ifn2] [...]
\subsection*{Initialization}


 \emph{``filename''}
 -- A quoted string containing the name of the file to load. 


 \emph{iflag}
 -- Type of the file to load/save. (0 = binary file, Non-zero = text file) 


 \emph{ifn1, ifn2, ...}
 -- Numbers of tables to load. 
\subsection*{Performance}


 \emph{ftload}
 loads a list of tables from a file. (The tables have to be already allocated though.) The file's format can be binary or text. 


 


\begin{tabular}{cc}
Warning &\textbf{Warning}
 \\
� &

  The file's format is not compatible with a WAV-file and is not endian-safe. 


\end{tabular}

\subsection*{Examples}


  See the example for \emph{ftsave}
. 
\subsection*{See Also}


 \emph{ftloadk}
, \emph{ftsavek}
, \emph{ftsave}

\subsection*{Credits}


 Author: Gabriel Maldonado


 New in version 4.21
%\hline 


\begin{comment}
\begin{tabular}{lcr}
Previous &Home &Next \\
ftlen &Up &ftloadk

\end{tabular}


\end{document}
\end{comment}
