\begin{comment}
\documentclass[10pt]{article}
\usepackage{fullpage, graphicx, url}
\setlength{\parskip}{1ex}
\setlength{\parindent}{0ex}
\title{dispfft}
\begin{document}


\begin{tabular}{ccc}
The Alternative Csound Reference Manual & & \\
Previous & &Next

\end{tabular}

%\hline 
\end{comment}
\section{dispfft}
displayfft�--� Displays the Fourier Transform of an audio or control signal. \subsection*{Description}


  These units will print orchestra init-values, or produce graphic display of orchestra control signals and audio signals. Uses X11 windows if enabled, else (or if \emph{-g}
 flag is set) displays are approximated in ASCII characters. 
\subsection*{Syntax}


 \textbf{dispfft}
 xsig, iprd, iwsiz [, iwtyp] [, idbout] [, iwtflg]
\subsection*{Initialization}


 \emph{iprd}
 -- the period of display in seconds. 


 \emph{iwsiz}
 -- size of the input window in samples. A window of \emph{iwsiz}
 points will produce a Fourier transform of \emph{iwsiz}
/2 points, spread linearly in frequency from 0 to sr/2. \emph{iwsiz}
 must be a power of 2, with a minimum of 16 and a maximum of 4096. The windows are permitted to overlap. 


 \emph{iwtyp}
 (optional, default=0) -- window type. 0 = rectangular, 1 = Hanning. The default value is 0 (rectangular). 


 \emph{idbout}
 (optional, default=0) -- units of output for the Fourier coefficients. 0 = magnitude, 1 = decibels. The default is 0 (magnitude). 


 \emph{iwtflg}
 (optional, default=0) -- wait flag. If non-zero, each display is held until released by the user. The default value is 0 (no wait). 
\subsection*{Performance}


 \emph{dispfft}
 -- displays the Fourier Transform of an audio or control signal (\emph{asig}
 or \emph{ksig}
) every \emph{iprd}
 seconds using the Fast Fourier Transform method. 
\subsection*{Examples}


  Here is an example of the dispfft opcode. It uses the files \emph{dispfft.orc}
, \emph{dispfft.sco}
 and \emph{beats.wav}
. 


 \textbf{Example 1. Example of the dispfft opcode.}

\begin{lstlisting}
/* dispfft.orc */
; Initialize the global variables.
sr = 44100
kr = 4410
ksmps = 10
nchnls = 1

; Instrument #1.
instr 1
  asig soundin "beats.wav"
  dispfft asig, 1, 512
  out asig
endin
/* dispfft.orc */
        
\end{lstlisting}
\begin{lstlisting}
/* dispfft.sco */
; Play Instrument #1 for three seconds.
i 1 0 3
e
/* dispfft.sco */
        
\end{lstlisting}
\subsection*{See Also}


 \emph{display}
, \emph{print}

\subsection*{Credits}


 Comments about the \emph{inprds}
 parameter contributed by Rasmus Ekman.


 Example written by Kevin Conder.
%\hline 


\begin{comment}
\begin{tabular}{lcr}
Previous &Home &Next \\
diskin &Up &display

\end{tabular}


\end{document}
\end{comment}
