\begin{comment}
\documentclass[10pt]{article}
\usepackage{fullpage, graphicx, url}
\setlength{\parskip}{1ex}
\setlength{\parindent}{0ex}
\title{flanger}
\begin{document}


\begin{tabular}{ccc}
The Alternative Csound Reference Manual & & \\
Previous & &Next

\end{tabular}

%\hline 
\end{comment}
\section{flanger}
flanger�--� A user controlled flanger. \subsection*{Description}


  A user controlled flanger. 
\subsection*{Syntax}


 ar \textbf{flanger}
 asig, adel, kfeedback [, imaxd]
\subsection*{Initialization}


 \emph{imaxd}
(optional) -- maximum delay in seconds (needed for inital memory allocation) 
\subsection*{Performance}


 \emph{asig}
 -- input signal 


 \emph{adel}
 -- delay in seconds 


 \emph{kfeedback}
 -- feedback amount (in normal tasks this should not exceed 1, even if bigger values are allowed) 


  This unit is useful for generating choruses and flangers. The delay must be varied at a-rate connecting \emph{adel}
 to an oscillator output. Also the feedback can vary at k-rate. This opcode is implemented to allow \emph{kr}
 different than \emph{sr}
 (else delay could not be lower than \emph{ksmps}
) enhancing realtime performance. This unit is very similar to \emph{wguide1}
, the only difference is \emph{flanger}
 does not have the lowpass filter. 
\subsection*{Examples}


  Here is an example of the flanger opcode. It uses the files \emph{flanger.orc}
, \emph{flanger.sco}
, and \emph{beats.wav}
. 


 \textbf{Example 1. Example of the flanger opcode.}

\begin{lstlisting}
/* flanger.orc */
; Initialize the global variables.
sr = 44100
kr = 4410
ksmps = 10
nchnls = 1

; Instrument #1.
instr 1
  ; Use the "beat.wav" audio file.
  asig soundin "beats.wav"

  ; Vary the delay amount from 0 to 0.01 seconds.
  adel line 0, p3, 0.01
  kfeedback = 0.7

  ; Apply flange to the input signal.
  aflang flanger asig, adel, kfeedback

  ; It can get loud, so clip its amplitude to 30,000.
  a1 clip aflang, 1, 30000
  out a1
endin
/* flanger.orc */
        
\end{lstlisting}
\begin{lstlisting}
/* flanger.sco */
; Play Instrument #1 for two seconds.
i 1 0 2
e
/* flanger.sco */
        
\end{lstlisting}
\subsection*{Credits}


 


 


\begin{tabular}{cc}
Author: Gabriel Maldonado &Italy

\end{tabular}



 


 Example written by Kevin Conder.


 New in Csound version 3.49
%\hline 


\begin{comment}
\begin{tabular}{lcr}
Previous &Home &Next \\
fiopen &Up &flashtxt

\end{tabular}


\end{document}
\end{comment}
