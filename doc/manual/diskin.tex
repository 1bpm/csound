\begin{comment}
\documentclass[10pt]{article}
\usepackage{fullpage, graphicx, url}
\setlength{\parskip}{1ex}
\setlength{\parindent}{0ex}
\title{diskin}
\begin{document}


\begin{tabular}{ccc}
The Alternative Csound Reference Manual & & \\
Previous & &Next

\end{tabular}

%\hline 
\end{comment}
\section{diskin}
diskin�--� Reads audio data from an external device or stream and can alter its pitch. \subsection*{Description}


  Reads audio data from an external device or stream and can alter its pitch. 
\subsection*{Syntax}


 ar1 [,ar2] [, ar3] [, ar4] \textbf{diskin}
 ifilcod, kpitch [, iskiptim] [, iwraparound] [, iformat]
\subsection*{Initialization}


 \emph{ifilcod}
 -- integer or character-string denoting the source soundfile name. An integer denotes the file soundin.filcod ; a character-string (in double quotes, spaces permitted) gives the filename itself, optionally a full pathname. If not a full path, the named file is sought first in the current directory, then in that given by the environment variable SSDIR (if defined) then by SFDIR. See also \emph{GEN01}
. 


 \emph{iskptim}
 (optional) -- time in seconds of input sound to be skipped. The default value is 0. 


 \emph{iformat}
 (optional) -- specifies the audio data file format: 


 
\begin{itemize}
\item 

 1 = 8-bit signed char (high-order 8 bits of a 16-bit integer)

\item 

 2 = 8-bit A-law bytes

\item 

 3 = 8-bit U-law bytes

\item 

 4 = 16-bit short integers

\item 

 5 = 32-bit long integers

\item 

 6 = 32-bit floats


\end{itemize}


 \emph{iwraparound}
 -- 1 = on, 0 = off (wraps around to end of file either direction) 


  If \emph{iformat}
 = 0 it is taken from the soundfile header, and if no header from the Csound \emph{-o}
 command-line flag. The default value is 0. 
\subsection*{Performance}


 \emph{kpitch}
 -- can be any real number. a negative number signifies backwards playback. The given number is a pitch ratio, where: 


 
\begin{itemize}
\item 

 \emph{1}
 = normal pitch

\item 

 \emph{2}
 = 1 octave higher

\item 

 \emph{3}
 = 12th higher, etc.

\item 

 \emph{.5}
 = 1 octave lower

\item 

 \emph{.25}
 = 2 octaves lower, etc.

\item 

 \emph{-1}
 = normal pitch backwards

\item 

 \emph{-2}
 = 1 octave higher backwards, etc.


\end{itemize}


 \emph{diskin}
 is identical to \emph{soundin}
 except that it can alter the pitch of the sound that is being read. 


 


\begin{tabular}{cc}
Caution &\textbf{Note to Windows users}
 \\
� &

  Windows users typically use back-slashes, ``$\backslash$'', when specifying the paths of their files. As an example, a Windows user might use the path ``c:$\backslash$music$\backslash$samples$\backslash$loop001.wav''. This is problematic because back-slashes are normally used to specify special characters. 


  To correctly specify this path in Csound, one may alternately: 


 
\begin{itemize}
\item 

 \emph{Use forward slashes}
: c:/music/samples/loop001.wav

\item 

 \emph{Use back-slash special characters, ``$\backslash$$\backslash$''}
: c:$\backslash$$\backslash$music$\backslash$$\backslash$samples$\backslash$$\backslash$loop001.wav


\end{itemize}


\end{tabular}

\subsection*{Examples}


  Here is an example of the diskin opcode. It uses the files \emph{diskin.orc}
, \emph{diskin.sco}
, \emph{beats.wav}
. 


 \textbf{Example 1. Example of the diskin opcode.}

\begin{lstlisting}
/* diskin.orc */
; Initialize the global variables.
sr = 44100
kr = 44100
ksmps = 1
nchnls = 1

; Instrument #1 - play an audio file.
instr 1
  ; Play the audio file backwards.
  asig diskin "beats.wav", -1
  out asig
endin
/* diskin.orc */
        
\end{lstlisting}
\begin{lstlisting}
/* diskin.sco */
; Play Instrument #1, the audio file, for three seconds.
i 1 0 3
e
/* diskin.sco */
        
\end{lstlisting}
\subsection*{See Also}


 \emph{in}
, \emph{inh}
, \emph{ino}
, \emph{inq}
, \emph{ins}
, \emph{soundin}

\subsection*{Credits}


 


 


\begin{tabular}{ccc}
Authors: Barry L. Vercoe, Matt Ingalls/Mike Berry &MIT, Mills College &1993-1997

\end{tabular}



 


 Example written by Kevin Conder.


 Warning to Windows users added by Kevin Conder, April 2002
%\hline 


\begin{comment}
\begin{tabular}{lcr}
Previous &Home &Next \\
diff &Up &dispfft

\end{tabular}


\end{document}
\end{comment}
