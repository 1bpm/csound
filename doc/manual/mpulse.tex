\begin{comment}
\documentclass[10pt]{article}
\usepackage{fullpage, graphicx, url}
\setlength{\parskip}{1ex}
\setlength{\parindent}{0ex}
\title{mpulse}
\begin{document}


\begin{tabular}{ccc}
The Alternative Csound Reference Manual & & \\
Previous & &Next

\end{tabular}

%\hline 
\end{comment}
\section{mpulse}
mpulse�--� Generates a set of impulses. \subsection*{Description}


  Generates a set of impulses of amplitude \emph{kamp}
 at frequency \emph{kfreq}
. The first impulse is after a delay of \emph{ioffset}
 seconds. The value of \emph{kfreq}
 is read only after an impulse, so it is the interval to the next impulse at the time of an impulse. 
\subsection*{Syntax}


 ar \textbf{mpulse}
 kamp, kfreq [, ioffset]
\subsection*{Initialization}


 \emph{ioffset}
 (optional, default=0) -- the delay before the first impulse. If it is negative, the value is taken as the number of samples, otherwise it is in seconds. Default is zero. 
\subsection*{Performance}


 \emph{kamp}
 -- amplitude of the impulses generated 


 \emph{kfreq}
 -- frequency of the impulse train 


  After the initial delay, an impulse of \emph{kamp}
 amplitude is generated as a single sample. Immediately after generating the impulse, the time of the next one is calculated. If \emph{kfreq}
 is zero, there is an infinite wait to the next impulse. If \emph{kfreq}
 is negative, the frequency is counted in samples rather than seconds. 
\subsection*{Examples}


  Here is an example of the mpulse opcode. It uses the files \emph{mpulse.orc}
 and \emph{mpulse.sco}
. 


 \textbf{Example 1. Example of the mpulse opcode.}

\begin{lstlisting}
/* mpulse.orc */
; Initialize the global variables.
sr = 44100
kr = 4410
ksmps = 10
nchnls = 1

; Instrument #1.
instr 1
  ; Generate an impulse every 1/10th of a second.
  kamp = 30000
  kfreq = 0.1

  a1 mpulse kamp, kfreq
  out a1
endin
/* mpulse.orc */
        
\end{lstlisting}
\begin{lstlisting}
/* mpulse.sco */
; Play Instrument #1 for one second.
i 1 0 1
e
/* mpulse.sco */
        
\end{lstlisting}
\subsection*{Credits}


 Example written by Kevin Conder.
%\hline 


\begin{comment}
\begin{tabular}{lcr}
Previous &Home &Next \\
moscil &Up &mrtmsg

\end{tabular}


\end{document}
\end{comment}
