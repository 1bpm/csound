\begin{comment}
\documentclass[10pt]{article}
\usepackage{fullpage, graphicx, url}
\setlength{\parskip}{1ex}
\setlength{\parindent}{0ex}
\title{wterrain}
\begin{document}


\begin{tabular}{ccc}
The Alternative Csound Reference Manual & & \\
Previous & &Next

\end{tabular}

%\hline 
\end{comment}
\section{wterrain}
wterrain�--� A simple wave-terrain synthesis opcode. \subsection*{Description}


  A simple wave-terrain synthesis opcode. 
\subsection*{Syntax}


 aout \textbf{wterrain}
 kamp, kpch, k\_xcenter, k\_ycenter, k\_xradius, k\_yradius, itabx, itaby
\subsection*{Initialization}


 \emph{itabx, itaby}
 -- The two tables that define the terrain. 
\subsection*{Performance}


  The output is the result of drawing an ellipse with axes \emph{k\_xradius}
 and \emph{k\_yradius}
 centered at (\emph{k\_xcenter}
, \emph{k\_ycenter}
), and traversing it at frequency \emph{kpch}
. 
\subsection*{Examples}


  Here is an example of the wterrain opcode. It uses the files \emph{wterrain.orc}
 and \emph{wterrain.sco}
. 


 \textbf{Example 1. Example of the wterrain opcode.}

\begin{lstlisting}
/* wterrain.orc */
; Initialize the global variables.
sr = 44100
kr = 4410
ksmps = 10
nchnls = 1

instr 1
kdclk   linseg  0, 0.01, 1, p3-0.02, 1, 0.01, 0
kcx     line    0.1, p3, 1.9
krx     linseg  0.1, p3/2, 0.5, p3/2, 0.1
kpch    line    cpspch(p4), p3, p5 * cpspch(p4)
a1      wterrain    10000, kpch, kcx, kcx, -krx, krx, p6, p7
a1      dcblock a1
        out     a1*kdclk
endin
/* wterrain.orc */
        
\end{lstlisting}
\begin{lstlisting}
/* wterrain.sco */
f1      0       8192    10      1 0 0.33 0 0.2 0 0.14 0 0.11
f2      0       4096    10      1

i1      0       4       7.00 1 1 1
i1      4       4       6.07 1 1 2
i1      8       8       6.00 1 2 2
e
/* wterrain.sco */
        
\end{lstlisting}
\subsection*{Credits}


 


 


\begin{tabular}{cc}
Author: Matthew Gillard &New in version 4.19

\end{tabular}



 
%\hline 


\begin{comment}
\begin{tabular}{lcr}
Previous &Home &Next \\
wrap &Up &xadsr

\end{tabular}


\end{document}
\end{comment}
