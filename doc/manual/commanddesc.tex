\begin{comment}
\documentclass[10pt]{article}
\usepackage{fullpage, graphicx, url}
\setlength{\parskip}{1ex}
\setlength{\parindent}{0ex}
\title{Description}
\begin{document}


\begin{tabular}{ccc}
The Alternative Csound Reference Manual & & \\
Previous &The Csound Command &Next

\end{tabular}

%\hline 
\end{comment}
\section{Description}


  Flags may appear anywhere in the command line, either separately or bundled together. A flag taking a Name or Number will find it in that argument, or in the immediately subsequent one. The following are thus equivalent commands: 


 
\begin{lstlisting}
\textbf{csound}
 -nm3 orchname -Sxxfilename scorename
\textbf{csound}
 -n -m 3 orchname -x xfilename -S scorename
       
\end{lstlisting}


 


  All flags and names are optional. The default values are: 


 
\begin{lstlisting}
csound -s -otest -b1024 -B1024 -m7 -P128 orchname scorename
       
\end{lstlisting}


 


  where \emph{orchname}
 is a file containing Csound orchestra code, and scorename is a file of score data in standard numeric score format, optionally presorted and time-warped. If \emph{scorename}
 is omitted, there are two default options: 


 
\begin{enumerate}
\item 

 if real-time input is expected (-L, -M or -F), a dummy score file is substituted consisting of the single statement 'f 0 3600' (i.e. listen for RT input for one hour)

\item 

 else CSound uses the previously processed \emph{score.srt}
 in the current directory.


\end{enumerate}


  Csound reports on the various stages of score and orchestra processing as it goes, doing various syntax and error checks along the way. Once the actual performance has begun, any error messages will derive from either the instrument loader or the unit generators themselves. A CSound command may include any rational combination of flag arguments. 
%\hline 


\begin{comment}
\begin{tabular}{lcr}
Previous &Home &Next \\
The Csound Command &Up &Command-line Flags

\end{tabular}


\end{document}
\end{comment}
