\begin{comment}
\documentclass[10pt]{article}
\usepackage{fullpage, graphicx, url}
\setlength{\parskip}{1ex}
\setlength{\parindent}{0ex}
\title{How to Install Csound}
\begin{document}


\begin{tabular}{ccc}
The Alternative Csound Reference Manual & & \\
Previous &Introduction &Next

\end{tabular}

%\hline 
\end{comment}
\section{How to Install Csound}
\subsection*{Linux}


  Detailed instructions for installing and configuring Csound on a Linux system may be obtained from: 


  \emph{\url{http://www.csounds.com/secondprinting/cdroms/installing/linux/}}
 
\subsection*{Macintosh}


  Detailed instructions for installing and configuring Csound on Macintosh systems may be obtained from: 


  \emph{\url{http://www.csounds.com/installing/howtomacintosh/index.html}}
 
\subsection*{MS-DOS and Windows 95/NT}


  Detailed instructions for installing and configuring Csound on a MS-DOS or Windows 95/NT system may be obtained from: 


  \emph{\url{http://hem.passagen.se/rasmuse/PCinstal.htm}}
 
\subsection*{Windows 95/98/2000}


  Detailed instructions for installing and configuring Csound on a Windows 95, Windows 98, or Windows 2000 system may be obtained from: 


  \emph{\url{http://www.csounds.com/installing/howtowindows/index.html}}
 
\subsection*{Other Platforms}


  For information on availability of Csound for other platforms, see The Csound FrontPage: 


  \emph{\url{http://mitpress.mit.edu/e-books/csound/frontpage.html}}
 
%\hline 


\begin{comment}
\begin{tabular}{lcr}
Previous &Home &Next \\
Introduction &Up &The Csound Mailing List

\end{tabular}


\end{document}
\end{comment}
