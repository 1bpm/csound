\begin{comment}
\documentclass[10pt]{article}
\usepackage{fullpage, graphicx, url}
\setlength{\parskip}{1ex}
\setlength{\parindent}{0ex}
\title{butterbr}
\begin{document}


\begin{tabular}{ccc}
The Alternative Csound Reference Manual & & \\
Previous & &Next

\end{tabular}

%\hline 
\end{comment}
\section{butterbr}
butterbr�--� A band-reject Butterworth filter. \subsection*{Description}


  Implementation of a second-order band-reject Butterworth filter. This opcode can also be written as \emph{butbr}
. 
\subsection*{Syntax}


 ar \textbf{butterbr}
 asig, kfreq, kband [, iskip]
\subsection*{Initialization}


 \emph{iskip}
 (optional, default=0) -- Skip initialization if present and non-zero. 
\subsection*{Performance}


  These filters are Butterworth second-order IIR filters. They are slightly slower than the original filters in Csound, but they offer an almost flat passband and very good precision and stopband attenuation. 


 \emph{asig}
 -- Input signal to be filtered. 


 \emph{kfreq}
 -- Cutoff or center frequency for each of the filters. 


 \emph{kband}
 -- Bandwidth of the bandpass and bandreject filters. 
\subsection*{Examples}


  Here is an example of the butterbr opcode. It uses the files \emph{butterbr.orc}
 and \emph{butterbr.sco}
. 


 \textbf{Example 1. Example of the butterbr opcode.}

\begin{lstlisting}
/* butterbr.orc */
; Initialize the global variables.
sr = 22050
kr = 2205
ksmps = 10
nchnls = 1

; Instrument #1 - an unfiltered noise waveform.
instr 1
  ; White noise signal
  asig rand 22050

  out asig
endin


; Instrument #2 - a filtered noise waveform.
instr 2
  ; White noise signal
  asig rand 22050

  ; Filter it, cutting 2000 to 6000 Hz.
  abr butterbr asig, 4000, 2000

  out abr
endin
/* butterbr.orc */
        
\end{lstlisting}
\begin{lstlisting}
/* butterbr.sco */
; Play Instrument #1 for two seconds.
i 1 0 2
; Play Instrument #2 for two seconds.
i 2 2 2
e
/* butterbr.sco */
        
\end{lstlisting}
\subsection*{See Also}


 \emph{butterbp}
, \emph{butterhp}
, \emph{butterlp}

\subsection*{Credits}


 


 


\begin{tabular}{ccc}
Author: Paris Smaragdis &MIT, Cambridge &1995

\end{tabular}



 
%\hline 


\begin{comment}
\begin{tabular}{lcr}
Previous &Home &Next \\
butterbp &Up &butterhp

\end{tabular}


\end{document}
\end{comment}
