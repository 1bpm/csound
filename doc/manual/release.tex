\begin{comment}
\documentclass[10pt]{article}
\usepackage{fullpage, graphicx, url}
\setlength{\parskip}{1ex}
\setlength{\parindent}{0ex}
\title{release}
\begin{document}


\begin{tabular}{ccc}
The Alternative Csound Reference Manual & & \\
Previous & &Next

\end{tabular}

%\hline 
\end{comment}
\section{release}
release�--� Indicates whether a note is in its ``release'' stage. \subsection*{Description}


  Indicates whether a note is in its ``release'' stage. 
\subsection*{Syntax}


 kflag \textbf{release}

\subsection*{Performance}


 \emph{kflag}
 -- indicates whether the note is in its ``release'' stage. 


 \emph{release}
 outputs current note state. If current note is in the ``release'' stage (i.e. if its duration has been extended with \emph{xtratim}
 opcode and if it has only just deactivated), then the \emph{kflag}
 output argument is set to 1. Otherwise (in sustain stage of current note), \emph{kflag}
 is set to 0. 


  This opcode is useful for implementing complex release-oriented envelopes. 
\subsection*{Examples}


 


 
\begin{lstlisting}
 \emph{instr}
 1 ;allows complex ADSR envelope with MIDI events
  inum \emph{notnum}

  icps \emph{cpsmidi}

  iamp \emph{ampmid}
i 4000
 ;
 ;------- complex envelope block ------
  \emph{xtratim}
 1 ;extra-time, i.e. release dur
  krel \emph{init}
 0
  krel \emph{release}
 ;outputs release-stage flag (0 or 1 values)
  if (krel  .5) \emph{kgoto}
 rel ;if in release-stage goto release section
 ;
 ;************ attack and sustain section ***********
  kmp1 \emph{linseg}
 0, .03, 1, .05, 1, .07, 0, .08, .5, 4, 1, 50, 1
  kmp = kmp1*iamp
   \emph{kgoto}
 done
 ;
 ;--------- release section --------
   rel:
  kmp2 \emph{linseg}
 1, .3, .2, .7, 0
  kmp = kmp1*kmp2*iamp
  done:
 ;------
  a1 \emph{oscili}
 kmp, icps, 1
  \emph{out}
 a1
 \emph{endin}

        
\end{lstlisting}


 
\subsection*{See Also}


 \emph{xtratim}

\subsection*{Credits}


 


 


\begin{tabular}{cc}
Author: Gabriel Maldonado &Italy

\end{tabular}



 


 New in Csound version 3.47
%\hline 


\begin{comment}
\begin{tabular}{lcr}
Previous &Home &Next \\
reinit &Up &repluck

\end{tabular}


\end{document}
\end{comment}
