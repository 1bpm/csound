\begin{comment}
\documentclass[10pt]{article}
\usepackage{fullpage, graphicx, url}
\setlength{\parskip}{1ex}
\setlength{\parindent}{0ex}
\title{atone}
\begin{document}


\begin{tabular}{ccc}
The Alternative Csound Reference Manual & & \\
Previous & &Next

\end{tabular}

%\hline 
\end{comment}
\section{atone}
atone�--� A notch filter whose transfer functions are the complements of the tone opcode. \subsection*{Description}


  A notch filter whose transfer functions are the complements of the tone opcode. 
\subsection*{Syntax}


 ar \textbf{atone}
 asig, khp [, iskip]
\subsection*{Initialization}


 \emph{iskip}
 (optional, default=0) -- initial disposition of internal data space. Since filtering incorporates a feedback loop of previous output, the initial status of the storage space used is significant. A zero value will clear the space; a non-zero value will allow previous information to remain. The default value is 0. 
\subsection*{Performance}


 \emph{ar}
 -- the output signal at audio rate. 


 \emph{asig}
 -- the input signal at audio rate. 


 \emph{khp}
 -- the response curve's half-power point, in Hertz. Half power is defined as peak power / root 2. 


 \emph{atone}
 is a filter whose transfer functions is the complement of \emph{tone}
. \emph{atone}
 is thus a form of high-pass filter whose transfer functions represent the ``filtered out'' aspects of their complements. However, power scaling is not normalized in \emph{atone}
 but remains the true complement of the corresponding unit. Thus an audio signal, filtered by parallel matching \emph{tone}
 and \emph{atone}
 units, would under addition simply reconstruct the original spectrum. 


  This property is particularly useful for controlled mixing of different sources (see \emph{lpreson}
). Complex response curves such as those with multiple peaks can be obtained by using a bank of suitable filters in series. (The resultant response is the product of the component responses.) In such cases, the combined attenuation may result in a serious loss of signal power, but this can be regained by the use of \emph{balance}
. 
\subsection*{Examples}


  Here is an example of the atone opcode. It uses the files \emph{atone.orc}
 and \emph{atone.sco}
. 


 \textbf{Example 1. Example of the atone opcode.}

\begin{lstlisting}
/* atone.orc */
; Initialize the global variables.
sr = 22050
kr = 2205
ksmps = 10
nchnls = 1

; Instrument #1 - an unfiltered noise waveform.
instr 1
  ; Generate a white noise signal.
  asig rand 20000

  out asig
endin


; Instrument #2 - a filtered noise waveform.
instr 2
  ; Generate a white noise signal.
  asig rand 20000

  ; Filter it using the atone opcode.
  khp init 2000
  afilt atone asig, khp

  ; Clip the filtered signal's amplitude to 85 dB.
  a1 clip afilt, 2, ampdb(85)
  out a1
endin
/* atone.orc */
        
\end{lstlisting}
\begin{lstlisting}
/* atone.sco */
; Play Instrument #1 for two seconds.
i 1 0 2
; Play Instrument #2 for two seconds.
i 2 2 2
e
/* atone.sco */
        
\end{lstlisting}
\subsection*{See Also}


 \emph{areson}
, \emph{aresonk}
, \emph{atonek}
, \emph{port}
, \emph{portk}
, \emph{reson}
, \emph{resonk}
, \emph{tone}
, \emph{tonek}

%\hline 


\begin{comment}
\begin{tabular}{lcr}
Previous &Home &Next \\
aresonk &Up &atonek

\end{tabular}


\end{document}
\end{comment}
