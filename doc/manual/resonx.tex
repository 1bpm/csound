\begin{comment}
\documentclass[10pt]{article}
\usepackage{fullpage, graphicx, url}
\setlength{\parskip}{1ex}
\setlength{\parindent}{0ex}
\title{resonx}
\begin{document}


\begin{tabular}{ccc}
The Alternative Csound Reference Manual & & \\
Previous & &Next

\end{tabular}

%\hline 
\end{comment}
\section{resonx}
resonx�--� Emulates a stack of filters using the reson opcode. \subsection*{Description}


 \emph{resonx}
 is equivalent to a filters consisting of more layers of \emph{reson}
 with the same arguments, serially connected. Using a stack of a larger number of filters allows a sharper cutoff. They are faster than using a larger number instances in a Csound orchestra of the old opcodes, because only one initialization and k- cycle are needed at time and the audio loop falls entirely inside the cache memory of processor. 
\subsection*{Syntax}


 ar \textbf{resonx}
 asig, kcf, kbw [, inumlayer] [, iscl] [, iskip]
\subsection*{Initialization}


 \emph{inumlayer}
 (optional) -- number of elements in the filter stack. Default value is 4. 


 \emph{iscl}
 (optional, default=0) -- coded scaling factor for resonators. A value of 1 signifies a peak response factor of 1, i.e. all frequencies other than kcf are attenuated in accordance with the (normalized) response curve. A value of 2 raises the response factor so that its overall RMS value equals 1. (This intended equalization of input and output power assumes all frequencies are physically present; hence it is most applicable to white noise.) A zero value signifies no scaling of the signal, leaving that to some later adjustment (see \emph{balance}
). The default value is 0. 


 \emph{iskip}
 (optional, default=0) -- initial disposition of internal data space. Since filtering incorporates a feedback loop of previous output, the initial status of the storage space used is significant. A zero value will clear the space; a non-zero value will allow previous information to remain. The default value is 0. 
\subsection*{Performance}


 \emph{asig}
 -- input signal 


 \emph{kcf}
 -- the center frequency of the filter, or frequency position of the peak response. 


 \emph{kbw}
 -- bandwidth of the filter (the Hz difference between the upper and lower half-power points) 
\subsection*{See Also}


 \emph{atonex}
, \emph{tonex}

\subsection*{Credits}


 


 


\begin{tabular}{cc}
Author: Gabriel Maldonado (adapted by John ffitch) &Italy

\end{tabular}



 


 New in Csound version 3.49
%\hline 


\begin{comment}
\begin{tabular}{lcr}
Previous &Home &Next \\
resonr &Up &resony

\end{tabular}


\end{document}
\end{comment}
