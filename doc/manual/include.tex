\begin{comment}
\documentclass[10pt]{article}
\usepackage{fullpage, graphicx, url}
\setlength{\parskip}{1ex}
\setlength{\parindent}{0ex}
\title{\#include}
\begin{document}


\begin{tabular}{ccc}
The Alternative Csound Reference Manual & & \\
Previous & &Next

\end{tabular}

%\hline 
\end{comment}
\section{\#include}
\#include�--� Includes an external file for processing. \subsection*{Description}


  Includes an external file for processing. 
\subsection*{Syntax}


 \textbf{\#include}
 ``filename''
\subsection*{Performance}


  It is sometimes convenient to have the orchestra arranged in a number of files, for example with each instrument in a separate file. This style is supported by the \emph{\#include}
 facility which is part of the macro system. A line containing the text 


 
\begin{lstlisting}
#include "filename"
        
\end{lstlisting}


 
 where the character `` can be replaced by any suitable character. For most uses the double quote symbol will probably be the most convenient. The file name can include a full path. 

  This takes input from the named file until it ends, when input reverts to the previous input. There is currently a limit of 20 on the depth of included files and macros. 


  Another suggested use of \emph{\#include}
 would be to define a set of macros which are part of the composer's style. 


  An extreme form would be to have each instrument defines as a macro, with the instrument number as a parameter. Then an entire orchestra could be constructed from a number of \emph{\#include}
 statements followed by macro calls. 


 
\begin{lstlisting}
  \emph{#include}
 "clarinet"
  \emph{#include}
 "flute"
  \emph{#include}
 "bassoon"
  $CLARINET(1)
  $FLUTE(2)
  $BASSOON(3)
        
\end{lstlisting}


 
 It must be stressed that these changes are at the textual level and so take no cognizance of any meaning. \subsection*{Examples}


  Here is an example of the include opcode. It uses the files \emph{include.orc}
, \emph{include.sco}
, and \emph{table1.inc}
. 


 \textbf{Example 1. Example of the include opcode.}

\begin{lstlisting}
/* include.orc */
sr = 44100
kr = 4410
ksmps = 10
nchnls = 1

; Instrument #1 - a basic oscillator.
instr 1
  kamp = 10000
  kcps = 440
  ifn = 1

  a1 oscil kamp, kcps, ifn
  out a1
endin
/* include.orc */
        
\end{lstlisting}
\begin{lstlisting}
/* table1.inc */
; Table #1, a sine wave.
f 1 0 16384 10 1
/* table1.inc */
        
\end{lstlisting}
\begin{lstlisting}
/* include.sco */
; Include the file for Table #1.
#include "table1.inc"

; Play Instrument #1 for 2 seconds.
i 1 0 2
e
/* include.sco */
        
\end{lstlisting}
\subsection*{Credits}


 


 


\begin{tabular}{cccc}
Author: John ffitch &University of Bath/Codemist Ltd. &Bath, UK &April 1998

\end{tabular}



 


 Example written by Kevin Conder.


 New in Csound version 3.48
%\hline 


\begin{comment}
\begin{tabular}{lcr}
Previous &Home &Next \\
\#define &Up &\#undef

\end{tabular}


\end{document}
\end{comment}
