\begin{comment}
\documentclass[10pt]{article}
\usepackage{fullpage, graphicx, url}
\setlength{\parskip}{1ex}
\setlength{\parindent}{0ex}
\title{Multiple File Score}
\begin{document}


\begin{tabular}{ccc}
The Alternative Csound Reference Manual & & \\
Previous &The Standard Numeric Score &Next

\end{tabular}

%\hline 
\end{comment}
\section{Multiple File Score}
\subsection*{Description}


  Using the score in more than one file. 
\subsection*{Syntax}


 \textbf{\#include}
 ``filename''
\subsection*{Performance}


  It is sometimes convenient to have the score in more than one file. This use is supported by the \emph{\#include}
 facility which is part of the macro system. A line containing the text 


 
\begin{lstlisting}
\emph{#include}
 "filename"
        
\end{lstlisting}


 
 where the character \emph{``}
 can be replaced by any suitable character. For most uses the double quote symbol will probably be the most convenient. The file name can include a full path. 

  This takes input from the named file until it ends, when input reverts to the previous input. There is currently a limit of 20 on the depth of included files and macros. 


  A suggested use of \emph{\#include}
 would be to define a set of macros which are part of the composer's style. It could also be used to provide repeated sections. 


 
\begin{lstlisting}
s
#include :section1:
;; Repeat that
s
#include :section1:
        
\end{lstlisting}


 


  Alternative methods of doing repeats, use the \emph{r statement}
, \emph{m statement}
, and \emph{n statement}
. 
\subsection*{Credits}


 Author: John ffitch


 University of Bath/Codemist Ltd.


 Bath, UK


 April, 1998 (New in Csound version 3.48)


 Thanks to Luis Jure for pointing out the incorrect syntax in multiple file include statement.
%\hline 


\begin{comment}
\begin{tabular}{lcr}
Previous &Home &Next \\
Score Macros &Up &Evaluation of Expressions

\end{tabular}


\end{document}
\end{comment}
