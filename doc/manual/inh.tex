\begin{comment}
\documentclass[10pt]{article}
\usepackage{fullpage, graphicx, url}
\setlength{\parskip}{1ex}
\setlength{\parindent}{0ex}
\title{inh}
\begin{document}


\begin{tabular}{ccc}
The Alternative Csound Reference Manual & & \\
Previous & &Next

\end{tabular}

%\hline 
\end{comment}
\section{inh}
inh�--� Reads six-channel audio data from an external device or stream. \subsection*{Description}


  Reads six-channel audio data from an external device or stream. 
\subsection*{Syntax}


 ar1, ar2, ar3, ar4, ar5, ar6 \textbf{inh}

\subsection*{Performance}


  Reads six-channel audio data from an external device or stream. If the command-line \emph{-i}
 flag is set, sound is read continuously from the audio input stream (e.g. \emph{stdin}
 or a soundfile) into an internal buffer. Any number of these opcodes can read freely from this buffer. 
\subsection*{See Also}


 \emph{diskin}
, \emph{in}
, \emph{ino}
, \emph{inq}
, \emph{ins}
, \emph{soundin}

\subsection*{Credits}


 


 


\begin{tabular}{ccc}
Authors: Barry L. Vercoe, Matt Ingalls/Mike Berry &MIT, Mills College &1993-1997

\end{tabular}



 
%\hline 


\begin{comment}
\begin{tabular}{lcr}
Previous &Home &Next \\
inch &Up &init

\end{tabular}


\end{document}
\end{comment}
