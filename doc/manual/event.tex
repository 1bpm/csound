\begin{comment}
\documentclass[10pt]{article}
\usepackage{fullpage, graphicx, url}
\setlength{\parskip}{1ex}
\setlength{\parindent}{0ex}
\title{event}
\begin{document}


\begin{tabular}{ccc}
The Alternative Csound Reference Manual & & \\
Previous & &Next

\end{tabular}

%\hline 
\end{comment}
\section{event}
event�--� Generates a score event from an instrument. \subsection*{Description}


  Generates a score event from an instrument. 
\subsection*{Syntax}


 \textbf{event}
 ``scorechar'', kinsnum, kdelay, kdur, [, kp4] [, kp5] [, ...]


 \textbf{event}
 ``scorechar'', ``insname'', kdelay, kdur, [, kp4] [, kp5] [, ...]
\subsection*{Initialization}


 \emph{``scorechar''}
 -- A string (in double-quotes) representing the first p-field in a score statement. This is usually \emph{``e''}
, \emph{``f''}
, or \emph{``i''}
. 


 \emph{``insname''}
 -- A string (in double-quotes) representing a named instrument. 
\subsection*{Performance}


 \emph{kinsnum}
 -- The instrument to use for the event. This corresponds to the first p-field, p1, in a score statement. 


 \emph{kdelay}
 -- When (in seconds) the event will occur from the current performance time. This corresponds to the second p-field, p2, in a score statement. 


 \emph{kdur}
 -- How long (in seconds) the event will happen. This corresponds to the third p-field, p3, in a score statement. 


 \emph{kp4, kp5, ...}
 (optional) -- Parameters representing additional p-field in a score statement. It starts with the fourth p-field, p4. 
\subsection*{Examples}


  Here is an example of the event opcode. It uses the files \emph{event.orc}
 and \emph{event.sco}
. 


 \textbf{Example 1. Example of the event opcode.}

\begin{lstlisting}
/* event.orc */
; Initialize the global variables.
sr = 44100
kr = 4410
ksmps = 10
nchnls = 1

; Instrument #1 - an oscillator with a high note.
instr 1
  ; Create a trigger and set its initial value to 1.
  ktrigger init 1

  ; If the trigger is equal to 0, continue playing.
  ; If not, schedule another event.
  if (ktrigger == 0) goto contin
    ; kscoreop="i", an i-statement.
    ; kinsnum=2, play Instrument #2.
    ; kwhen=1, start at 1 second.
    ; kdur=0.5, play for a half-second.
    event "i", 2, 1, 0.5

    ; Make sure the event isn't triggered again.
    ktrigger = 0

contin:
  a1 oscils 10000, 440, 1
  out a1
endin

; Instrument #2 - an oscillator with a low note.
instr 2
  a1 oscils 10000, 220, 1
  out a1
endin
/* event.orc */
        
\end{lstlisting}
\begin{lstlisting}
/* event.sco */
; Make sure the score plays for two seconds.
f 0 2

; Play Instrument #1 for a half-second.
i 1 0 0.5
e
/* event.sco */
        
\end{lstlisting}


  Here is an example of the event opcode using a named instrument. It uses the files \emph{event\_named.orc}
 and \emph{event\_named.sco}
. 


 \textbf{Example 2. Example of the event opcode using a named instrument.}

\begin{lstlisting}
/* event_named.orc */
; Initialize the global variables.
sr = 44100
kr = 4410
ksmps = 10
nchnls = 1

; Instrument #1 - an oscillator with a high note.
instr 1
  ; Create a trigger and set its initial value to 1.
  ktrigger init 1

  ; If the trigger is equal to 0, continue playing.
  ; If not, schedule another event.
  if (ktrigger == 0) goto contin
    ; kscoreop="i", an i-statement.
    ; kinsnum="low_note", instrument named "low_note".
    ; kwhen=1, start at 1 second.
    ; kdur=0.5, play for a half-second.
    event "i", "low_note", 1, 0.5

    ; Make sure the event isn't triggered again.
    ktrigger = 0

contin:
  a1 oscils 10000, 440, 1
  out a1
endin

; Instrument "low_note" - an oscillator with a low note.
instr low_note
  a1 oscils 10000, 220, 1
  out a1
endin
/* event_named.orc */
        
\end{lstlisting}
\begin{lstlisting}
/* event_named.sco */
; Make sure the score plays for two seconds.
f 0 2

; Play Instrument #1 for a half-second.
i 1 0 0.5
e
/* event_named.sco */
        
\end{lstlisting}
\subsection*{Credits}


 Examples written by Kevin Conder.


 New in version 4.17


 Thanks goes to Matt Ingalls for helping to fix the example.


 Thanks goes to Matt Ingalls for helping clarify the kwhen/kdelay parameter.
%\hline 


\begin{comment}
\begin{tabular}{lcr}
Previous &Home &Next \\
envlpxr &Up &exp

\end{tabular}


\end{document}
\end{comment}
