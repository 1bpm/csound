\begin{comment}
\documentclass[10pt]{article}
\usepackage{fullpage, graphicx, url}
\setlength{\parskip}{1ex}
\setlength{\parindent}{0ex}
\title{pvsfread}
\begin{document}


\begin{tabular}{ccc}
The Alternative Csound Reference Manual & & \\
Previous & &Next

\end{tabular}

%\hline 
\end{comment}
\section{pvsfread}
pvsfread�--� Read a selected channel from a PVOC-EX analysis file. \subsection*{Description}


  Create an fsig stream by reading a selected channel from a PVOC-EX analysis file loaded into memory, with frame interpolation. Only format 0 files (amplitude+frequency) are currently supported. The operation of this opcode mirrors that of pvoc, but outputs an fsig instead of a resynthesized signal. 
\subsection*{Syntax}


 fsig \textbf{pvsfread}
 ktimpt, ifn [, ichan]
\subsection*{Initialization}


 \emph{ifn}
 -- Name of the analysis file. This must have the .pvx file extension. 


  A multi-channel PVOC-EX file can be generated using the extended \emph{pvanal utility}
. 


 \emph{ichan}
 -- (optional) The channel to read (counting from 0). Default is 0. 
\subsection*{Performance}


 \emph{ktimpt}
 -- Time pointer into analysis file, in seconds. See the description of the same parameter of \emph{pvoc}
 for usage. 


  Note that analysis files can be very large, especially if multi-channel. Reading such files into memory will very likely incur breaks in the audio during real-time performance. As the file is read only once, and is then available to all other interested opcodes, it can be expedient to arrange for a dedicated instrument to preload all such analysis files at startup. 
\subsection*{Examples}


 


 
\begin{lstlisting}
idur  filelen   "test.pvx"         ; find dur of (stereo) analysis file
kpos  line      0,p3,idur          ; to ensure we process whole file
fsigr pvsfread  kpos,"test.pvx",1  ; create fsig from R channel
        
\end{lstlisting}


 
 (NB: as this example shows, the filelen opcode has been extended to accept both old and new analysis file formats). \subsection*{Credits}


 


 


\begin{tabular}{cc}
Author: Richard Dobson &August 2001 

\end{tabular}



 


 New in version 4.13
%\hline 


\begin{comment}
\begin{tabular}{lcr}
Previous &Home &Next \\
pvscross &Up &pvsftr

\end{tabular}


\end{document}
\end{comment}
