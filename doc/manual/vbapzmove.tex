\begin{comment}
\documentclass[10pt]{article}
\usepackage{fullpage, graphicx, url}
\setlength{\parskip}{1ex}
\setlength{\parindent}{0ex}
\title{vbapzmove}
\begin{document}


\begin{tabular}{ccc}
The Alternative Csound Reference Manual & & \\
Previous & &Next

\end{tabular}

%\hline 
\end{comment}
\section{vbapzmove}
vbapzmove�--� Writes a multi-channel audio signal to a ZAK array with moving virtual sources. \subsection*{Description}


  Writes a multi-channel audio signal to a ZAK array with moving virtual sources. 
\subsection*{Syntax}


 \textbf{vbapzmove}
 inumchnls, istartndx, asig, idur, ispread, ifldnum, ifld1, ifld2, [...]
\subsection*{Initialization}


 \emph{inumchnls}
 -- number of channels to write to the ZA array. Must be in the range 2 - 256. 


 \emph{istartndx}
 -- first index or position in the ZA array to use 


 \emph{ispread}
 -- spreading of the virtual source (range 0 - 100). If value is zero, conventional amplitude panning is used. When \emph{ispread}
 is increased, the number of loudspeakers used in panning increases. If value is 100, the sound is applied to all loudspeakers. 


 \emph{ifldnum}
 -- number of fields (absolute value must be 2 or larger). If \emph{ifldnum}
 is positive, the virtual source movement is a polyline specified by given directions. Each transition is performed in an equal time interval. If \emph{ifldnum}
 is negative, specified angular velocities are applied to the virtual source during specified relative time intervals (see below). 


 \emph{ifld1, ifld2, ...}
 -- azimuth angles or angular velocities, and relative durations of movement phases (see below). 
\subsection*{Performance}


 \emph{asig}
 -- audio signal to be panned 


  The opcode \emph{vbapzmove}
 is the multiple channel analog of the opcodes like \emph{vbap4move}
, working on \emph{inumchnls}
 and using a ZAK array for output. 
\subsection*{Examples}


 


 \textbf{Example 1. 2-D panning example with stationary virtual sources}

\begin{lstlisting}
  \emph{sr}
      =          4100
  \emph{kr}
      =           441
  \emph{ksmps}
   =           100
  \emph{nchnls}
  =             4
  \emph{vbaplsinit}
         2, 6,  0, 45, 90, 135, 200, 245, 290, 315 

          \emph{instr}
 1	           
  asig    \emph{oscil}
      20000, 440, 1                    
  a1,a2,a3,a4,a5,a6,a7,a8   \emph{vbap8}
  asig, p4, 0, 20 ;p4 = azimuth
	
  ;render twice with alternate \emph{outq}
 statements
  ;  to obtain two 4 channel .wav files:

          \emph{outq}
       a1,a2,a3,a4
  ;       \emph{outq}
       a5,a6,a7,a8
          \emph{endin}

        
\end{lstlisting}
\subsection*{Reference}


  Ville Pulkki: ``Virtual Sound Source Positioning Using Vector Base Amplitude Panning'' \emph{Journal of the Audio Engineering Society}
, 1997 June, Vol. 45/6, p. 456. 
\subsection*{See Also}


 \emph{vbap16}
, \emph{vbap16move}
, \emph{vbap4}
, \emph{vbap4move}
, \emph{vbap8}
, \emph{vbap8move}
, \emph{vbaplsinit}
, \emph{vbapz}

\subsection*{Credits}


 


 


\begin{tabular}{cccc}
John ffitch &University of Bath/Codemist Ltd. &Bath, UK &May 2000

\end{tabular}



 


 New in Csound Version 4.07
%\hline 


\begin{comment}
\begin{tabular}{lcr}
Previous &Home &Next \\
vbapz &Up &vco

\end{tabular}


\end{document}
\end{comment}
