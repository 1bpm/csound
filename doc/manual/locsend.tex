\begin{comment}
\documentclass[10pt]{article}
\usepackage{fullpage, graphicx, url}
\setlength{\parskip}{1ex}
\setlength{\parindent}{0ex}
\title{locsend}
\begin{document}


\begin{tabular}{ccc}
The Alternative Csound Reference Manual & & \\
Previous & &Next

\end{tabular}

%\hline 
\end{comment}
\section{locsend}
locsend�--� Distributes the audio signals of a previous \emph{locsig}
 opcode. \subsection*{Description}


 \emph{locsend}
 depends upon the existence of a previously defined \emph{locsig}
. The number of output signals must match the number in the previous \emph{locsig}
. The output signals from \emph{locsend}
 are derived from the values given for distance and reverb in the \emph{locsig}
 and are ready to be sent to local or global reverb units (see example below). The reverb amount and the balance between the 2 or 4 channels are calculated in the same way as described in the Dodge book (an essential text!). 
\subsection*{Syntax}


 a1, a2 \textbf{locsend}



 a1, a2, a3, a4 \textbf{locsend}

\subsection*{Examples}


 


 
\begin{lstlisting}
  asig some audio signal
  kdegree            \emph{line}
    0, p3, 360
  kdistance          \emph{line}
    1, p3, 10
  a1, a2, a3, a4     \emph{locsig}
  asig, kdegree, kdistance, .1
  ar1, ar2, ar3, ar4 \emph{locsend}

  ga1 = ga1+ar1
  ga2 = ga2+ar2
  ga3 = ga3+ar3
  ga4 = ga4+ar4
                     \emph{outq}
    a1, a2, a3, a4
\emph{endin}


\emph{instr}
 99 ; reverb instrument
  a1                 \emph{reverb2}
 ga1, 2.5, .5
  a2                 \emph{reverb2}
 ga2, 2.5, .5
  a3                 \emph{reverb2}
 ga3, 2.5, .5
  a4                 \emph{reverb2}
 ga4, 2.5, .5
                     \emph{outq}
    a1, a2, a3, a4
  ga1=0
  ga2=0
  ga3=0
  ga4=0
        
\end{lstlisting}


 


  In the above example, the signal, \emph{asig}
, is sent around a complete circle once during the duration of a note while at the same time it becomes more and more ``distant'' from the listeners' location. \emph{locsig}
 sends the appropriate amount of the signal internally to \emph{locsend}
. The outputs of the \emph{locsend}
 are added to global accumulators in a common Csound style and the global signals are used as inputs to the reverb units in a separate instrument. 


 \emph{locsig}
 is useful for quad and stereo panning as well as fixed placed of sounds anywhere between two loudspeakers. Below is an example of the fixed placement of sounds in a stereo field. 


 


 
\begin{lstlisting}
\emph{instr}
 1
  a1, a2             \emph{locsig}
  asig, p4, p5, .1
  ar1, ar2           \emph{locsend}

  ga1=ga1+ar1
  ga2=ga2+ar2
                     \emph{outs}
 a1, a
\emph{endin
instr}
 99 
  ; reverb....
\emph{endin}

        
\end{lstlisting}


 


  A few notes: 


 
\begin{lstlisting}
  ;place the sound in the left speaker and near:
  i1 0 1 0 1
  
  ;place the sound in the right speaker and far:
  i1 1 1 90 25
  
  ;place the sound equally between left and right and in the middle ground distance:
  i1 2 1 45 12
  e
        
\end{lstlisting}


 


  The next example shows a simple intuitive use of the distance value to simulate Doppler shift. The same value is used to scale the frequency as is used as the distance input to \emph{locsig}
. 


 
\begin{lstlisting}
  kdistance          \emph{line}
    1, p3, 10
  kfreq = (ifreq * 340) / (340 + kdistance)
  asig               \emph{oscili}
  iamp, kfreq, 1
  kdegree            \emph{line}
    0, p3, 360
  a1, a2, a3, a4     \emph{locsig}
  asig, kdegree, kdistance, .1
  ar1, ar2, ar3, ar4 \emph{locsend}

        
\end{lstlisting}


 
\subsection*{See Also}


 \emph{locsig}

\subsection*{Credits}


 


 


\begin{tabular}{ccc}
Author: Richard Karpen &Seattle, WA USA &1998

\end{tabular}



 


 New in Csound version 3.48
%\hline 


\begin{comment}
\begin{tabular}{lcr}
Previous &Home &Next \\
linsegr &Up &locsig

\end{tabular}


\end{document}
\end{comment}
