\begin{comment}
\documentclass[10pt]{article}
\usepackage{fullpage, graphicx, url}
\setlength{\parskip}{1ex}
\setlength{\parindent}{0ex}
\title{upsamp}
\begin{document}


\begin{tabular}{ccc}
The Alternative Csound Reference Manual & & \\
Previous & &Next

\end{tabular}

%\hline 
\end{comment}
\section{upsamp}
upsamp�--� Modify a signal by up-sampling. \subsection*{Description}


  Modify a signal by up-sampling. 
\subsection*{Syntax}


 ar \textbf{upsamp}
 ksig
\subsection*{Performance}


 \emph{upsamp}
 converts a control signal to an audio signal. It does it by simple repetition of the kval. \emph{upsamp}
 is a slightly more efficient form of the assignment, \emph{asig}
 = \emph{ksig}
. 
\subsection*{Examples}


 


 
\begin{lstlisting}
asrc  \emph{buzz}
      10000,440,20, 1     ; band-limited pulse train
adif  \emph{diff}
      asrc                ; emphasize the highs
anew  \emph{balance}
   adif, asrc          ;   but retain the power
agate \emph{reson}
     asrc,0,440          ; use a lowpass of the original
asamp \emph{samphold}
  anew, agate         ;   to gate the new audiosig
aout  \emph{tone}
      asamp,100           ; smooth out the rough edges
        
\end{lstlisting}


 
\subsection*{See Also}


 \emph{diff}
, \emph{downsamp}
, \emph{integ}
, \emph{interp}
, \emph{samphold}

%\hline 


\begin{comment}
\begin{tabular}{lcr}
Previous &Home &Next \\
unirand &Up &urd

\end{tabular}


\end{document}
\end{comment}
