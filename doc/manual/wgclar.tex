\begin{comment}
\documentclass[10pt]{article}
\usepackage{fullpage, graphicx, url}
\setlength{\parskip}{1ex}
\setlength{\parindent}{0ex}
\title{wgclar}
\begin{document}


\begin{tabular}{ccc}
The Alternative Csound Reference Manual & & \\
Previous & &Next

\end{tabular}

%\hline 
\end{comment}
\section{wgclar}
wgclar�--� Creates a tone similar to a clarinet. \subsection*{Description}


  Audio output is a tone similar to a clarinet, using a physical model developed from Perry Cook, but re-coded for Csound. 
\subsection*{Syntax}


 ar \textbf{wgclar}
 kamp, kfreq, kstiff, iatt, idetk, kngain, kvibf, kvamp, ifn [, iminfreq]
\subsection*{Initialization}


 \emph{iatt}
 -- time in seconds to reach full blowing pressure. 0.1 seems to correspond to reasonable playing. A longer time gives a definite initial wind sound. 


 \emph{idetk}
 -- time in seconds taken to stop blowing. 0.1 is a smooth ending 


 \emph{ifn}
 -- table of shape of vibrato, usually a sine table, created by a function 


 \emph{iminfreq}
 (optional) -- lowest frequency at which the instrument will play. If it is omitted it is taken to be the same as the initial \emph{kfreq}
. If \emph{iminfreq}
 is negative, initialization will be skipped. 
\subsection*{Performance}


  A note is played on a clarinet-like instrument, with the arguments as below. 


 \emph{kamp}
 -- Amplitude of note. 


 \emph{kfreq}
 -- Frequency of note played. 


 \emph{kstiff}
 -- a stiffness parameter for the reed. Values should be negative, and about -0.3. The useful range is approximately -0.44 to -0.18. 


 \emph{kngain}
 -- amplitude of the noise component, about 0 to 0.5 


 \emph{kvibf}
 -- frequency of vibrato in Hertz. Suggested range is 0 to 12 


 \emph{kvamp}
 -- amplitude of the vibrato 
\subsection*{Examples}


  Here is an example of the wgclar opcode. It uses the files \emph{wgclar.orc}
 and \emph{wgclar.sco}
. 


 \textbf{Example 1. Example of the wgclar opcode.}

\begin{lstlisting}
/* wgclar.orc */
; Initialize the global variables.
sr = 44100
kr = 4410
ksmps = 10
nchnls = 1

; Instrument #1.
instr 1
  kamp init 31129.60
  kfreq = 440
  kstiff = -0.3
  iatt = 0.1
  idetk = 0.1
  kngain = 0.2
  kvibf = 5.735
  kvamp = 0.1
  ifn = 1

  a1 wgclar kamp, kfreq, kstiff, iatt, idetk, kngain, kvibf, kvamp, ifn

  out a1
endin
/* wgclar.orc */
        
\end{lstlisting}
\begin{lstlisting}
/* wgclar.sco */
; Table #1, a sine wave.
f 1 0 16384 10 1

; Play Instrument #1 for one second.
i 1 0 1
e
/* wgclar.sco */
        
\end{lstlisting}
\subsection*{Credits}


 


 


\begin{tabular}{ccc}
Author: John ffitch (after Perry Cook) &University of Bath, Codemist Ltd. &Bath, UK

\end{tabular}



 


 New in Csound version 3.47
%\hline 


\begin{comment}
\begin{tabular}{lcr}
Previous &Home &Next \\
wgbrass &Up &wgflute

\end{tabular}


\end{document}
\end{comment}
