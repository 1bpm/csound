\begin{comment}
\documentclass[10pt]{article}
\usepackage{fullpage, graphicx, url}
\setlength{\parskip}{1ex}
\setlength{\parindent}{0ex}
\title{trigseq}
\begin{document}


\begin{tabular}{ccc}
The Alternative Csound Reference Manual & & \\
Previous & &Next

\end{tabular}

%\hline 
\end{comment}
\section{trigseq}
trigseq�--� Accepts a trigger signal as input and outputs a group of values. \subsection*{Description}


  Accepts a trigger signal as input and outputs a group of values. 
\subsection*{Syntax}


 \textbf{trigseq}
 ktrig\_in, kstart, kloop, kinitndx, kfn\_values, kout1 [, kout2] [...]
\subsection*{Performance}


 \emph{ktrig\_in}
 -- input trigger signal 


 \emph{kstart}
 -- start index of looped section 


 \emph{kloop}
 -- end index of looped section 


 \emph{kinitndx}
 -- initial index 


 


\begin{tabular}{cc}
\textbf{Note}
 \\
� &

  Although \emph{kinitndx}
 is listed as k-rate, it is in fact accessed only at init-time. So if you are using a k-rate argument, it must be assigned with \emph{init}
. 


\end{tabular}



 \emph{kfn\_values}
 -- numer of a table containing a sequence of groups of values 


 \emph{kout1}
 -- output values 


 \emph{kout2, ...}
 (optional) -- more output values 


  This opcode handles timed-sequences of groups of values stored into a table. 


 \emph{trigseq}
 accepts a trigger signal (\emph{ktrig\_in}
) as input and outputs group of values (contained in the \emph{kfn\_values}
 table) each time \emph{ktrig\_in}
 assumes a non-zero value. Each time a group of values is triggered, table pointer is advanced of a number of positions corresponding to the number of group-elements, in order to point to the next group of values. The number of elements of groups is determined by the number of \emph{koutX}
 arguments. 


  It is possible to start the sequence from a value different than the first, by assigning to initndx an index different than zero (which corresponds to the first value of the table). Normally the sequence is looped, and the start and end of loop can be adjusted by modifying \emph{kstart}
 and \emph{kloop}
 arguments. User must be sure that values of these arguments (as well as \emph{kinitndx}
) correspond to valid table numbers, otherwise Csound will crash because no range-checking is implemented. 


  It is possible to disable loop (one-shot mode) by assigning the same value both to \emph{kstart}
 and \emph{kloop}
 arguments. In this case, the last read element will be the one corresponding to the value of such arguments. Table can be read backward by assigning a negative \emph{kloop}
 value. 


 \emph{trigseq}
 is designed to be used together with \emph{seqtime}
 or \emph{trigger}
 opcodes. 
\subsection*{See Also}


 \emph{seqtime}
, \emph{trigger}

\subsection*{Credits}


 Author: Gabriel Maldonado


 November 2002. Added a note about the \emph{kinitndx}
 parameter, thanks to Rasmus Ekman.


 January 2003. Thanks to a note from Oeyvind Brandtsegg, I corrected the credits.


 New in version 4.06
%\hline 


\begin{comment}
\begin{tabular}{lcr}
Previous &Home &Next \\
trigger &Up &trirand

\end{tabular}


\end{document}
\end{comment}
