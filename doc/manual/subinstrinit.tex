\begin{comment}
\documentclass[10pt]{article}
\usepackage{fullpage, graphicx, url}
\setlength{\parskip}{1ex}
\setlength{\parindent}{0ex}
\title{subinstrinit}
\begin{document}


\begin{tabular}{ccc}
The Alternative Csound Reference Manual & & \\
Previous & &Next

\end{tabular}

%\hline 
\end{comment}
\section{subinstrinit}
subinstrinit�--� Creates and runs a numbered instrument instance at init-time. \subsection*{Description}


  Same as \emph{subinstr}
, but init-time only and has no output arguments. 
\subsection*{Syntax}


 \textbf{subinstrinit}
 instrnum [, p4] [, p5] [...]


 \textbf{subinstrinit}
 ``insname'' [, p4] [, p5] [...]
\subsection*{Initialization}


 \emph{instrnum}
 -- Number of the instrument to be called. 


 \emph{``insname''}
 -- A string (in double-quotes) representing a named instrument. 


  For more information about specifying input and output interfaces, see \emph{Calling an Instrument within an Instrument}
. 
\subsection*{Performance}


 \emph{p4, p5, ...}
 -- Additional input values the are mapped to the called instrument p-fields, starting with p4. 


  The called instrument's p2 and p3 values will be identical to the host instrument's values. While the host instrument can \emph{control its own duration}
, any such attempts inside the called instrument will most likely have no effect. 
\subsection*{See Also}


 \emph{Calling an Instrument within an Instrument}
, \emph{event}
, \emph{schedule}
, \emph{subinstr}

\subsection*{Credits}


 New in version 4.23
%\hline 


\begin{comment}
\begin{tabular}{lcr}
Previous &Home &Next \\
subinstr &Up &sum

\end{tabular}


\end{document}
\end{comment}
