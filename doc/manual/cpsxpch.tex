\begin{comment}
\documentclass[10pt]{article}
\usepackage{fullpage, graphicx, url}
\setlength{\parskip}{1ex}
\setlength{\parindent}{0ex}
\title{cpsxpch}
\begin{document}


\begin{tabular}{ccc}
The Alternative Csound Reference Manual & & \\
Previous & &Next

\end{tabular}

%\hline 
\end{comment}
\section{cpsxpch}
cpsxpch�--� Converts a pitch-class value into cycles-per-second (Hz) for equal divisions of any interval. \subsection*{Description}


  Converts a pitch-class value into cycles-per-second (Hz) for equal divisions of any interval. There is a restriction of no more than 100 equal divisions. 
\subsection*{Syntax}


 icps \textbf{cpsxpch}
 ipch, iequal, irepeat, ibase
\subsection*{Initialization}


 \emph{ipch}
 -- Input number of the form 8ve.pc, indicating an 'octave' and which note in the octave. 


 \emph{iequal}
 -- if positive, the number of equal intervals into which the 'octave' is divided. Must be less than or equal to 100. If negative, is the number of a table of frequency multipliers. 


 \emph{irepeat}
 -- Number indicating the interval which is the 'octave.' The integer 2 corresponds to octave divisions, 3 to a twelfth, 4 is two octaves, and so on. This need not be an integer, but must be positive. 


 \emph{ibase}
 -- The frequency which corresponds to pitch 0.0 


 


\begin{tabular}{cc}
\textbf{Note}
 \\
� &

 


 
\begin{enumerate}
\item 

 The following are essentially the same 


 ia��=��\emph{cpspch}
(8.02)\\ 
 ib�����\emph{cps2pch}
��8.02,�12\\ 
 ic�����\emph{cpsxpch}
��8.02,�12,�2,�1.02197503906\\ 
 ������������

\item 

  These are opcodes not functions 

\item 

  Negative values of \emph{ipch}
 are allowed, but not negative \emph{irepeat}
, \emph{iequal}
 or \emph{ibase}
. 


\end{enumerate}


\end{tabular}

\subsection*{Examples}


  Here is an example of the cpsxpch opcode. It uses the files \emph{cpsxpch.orc}
 and \emph{cpsxpch.sco}
. 


 \textbf{Example 1. Example of the cpsxpch opcode.}

\begin{lstlisting}
/* cpsxpch.orc */
; Initialize the global variables.
sr = 44100
kr = 4410
ksmps = 10
nchnls = 1

; Instrument #1.
instr 1
  ; Use a normal twelve-tone scale.
  ipch = 8.02
  iequal = 12
  irepeat = 2
  ibase = 1.02197503906

  icps cpsxpch ipch, iequal, irepeat, ibase

  print icps
endin
/* cpsxpch.orc */
        
\end{lstlisting}
\begin{lstlisting}
/* cpsxpch.sco */
; Play Instrument #1 for one second.
i 1 0 1
e
/* cpsxpch.sco */
        
\end{lstlisting}
 Its output should include lines like this: \begin{lstlisting}
instr 1:  icps = 293.666
      
\end{lstlisting}


  Here is an example of the cpsxpch opcode using a 10.5 ET scale. It uses the files \emph{cpsxpch\_105et.orc}
 and \emph{cpsxpch\_105et.sco}
. 


 \textbf{Example 2. Example of the cpsxpch opcode using a 10.5 ET scale.}

\begin{lstlisting}
/* cpsxpch_105et.orc */
; Initialize the global variables.
sr = 44100
kr = 4410
ksmps = 10
nchnls = 1

; Instrument #1.
instr 1
  ; Use a 10.5ET scale.
  ipch = 4.02
  iequal = 21
  irepeat = 4
  ibase = 16.35160062496

  icps cpsxpch ipch, iequal, irepeat, ibase

  print icps
endin
/* cpsxpch_105et.orc */
        
\end{lstlisting}
\begin{lstlisting}
/* cpsxpch_105et.sco */
; Play Instrument #1 for one second.
i 1 0 1
e
/* cpsxpch_105et.sco */
        
\end{lstlisting}
 Its output should include lines like this: \begin{lstlisting}
instr 1:  icps = 4776.824
      
\end{lstlisting}


  Here is an example of the cpsxpch opcode using a Pierce scale centered on middle A. It uses the files \emph{cpsxpch\_pierce.orc}
 and \emph{cpsxpch\_pierce.sco}
. 


 \textbf{Example 3. Example of the cpsxpch opcode using a Pierce scale centered on middle A.}

\begin{lstlisting}
/* cpsxpch_pierce.orc */
; Initialize the global variables.
sr = 44100
kr = 4410
ksmps = 10
nchnls = 1

; Instrument #1.
instr 1
  ; Use a Pierce scale centered on middle A.
  ipch = 2.02
  iequal = 12
  irepeat = 3
  ibase = 261.62561

  icps cpsxpch ipch, iequal, irepeat, ibase

  print icps
endin
/* cpsxpch_pierce.orc */
        
\end{lstlisting}
\begin{lstlisting}
/* cpsxpch_pierce.sco */
; Play Instrument #1 for one second.
i 1 0 1
e
/* cpsxpch_pierce.sco */
        
\end{lstlisting}
 Its output should include lines like this: \begin{lstlisting}
instr 1:  icps = 2827.762
      
\end{lstlisting}
\subsection*{See Also}


 \emph{cpspch}
, \emph{cps2pch}

\subsection*{Credits}


 


 


\begin{tabular}{cccc}
Author: John ffitch &University of Bath/Codemist Ltd. &Bath, UK &1997

\end{tabular}



 


 


 


\begin{tabular}{ccc}
Author: Gabriel Maldonado &Italy &1998

\end{tabular}



 


 New in Csound version 3.492
%\hline 


\begin{comment}
\begin{tabular}{lcr}
Previous &Home &Next \\
cpstuni &Up &cpuprc

\end{tabular}


\end{document}
\end{comment}
