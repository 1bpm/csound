\begin{comment}
\documentclass[10pt]{article}
\usepackage{fullpage, graphicx, url}
\setlength{\parskip}{1ex}
\setlength{\parindent}{0ex}
\title{delayr}
\begin{document}


\begin{tabular}{ccc}
The Alternative Csound Reference Manual & & \\
Previous & &Next

\end{tabular}

%\hline 
\end{comment}
\section{delayr}
delayr�--� Reads from an automatically established digital delay line. \subsection*{Description}


  Reads from an automatically established digital delay line. 
\subsection*{Syntax}


 ar \textbf{delayr}
 idlt [, iskip]
\subsection*{Initialization}


 \emph{idlt}
 -- requested delay time in seconds. This can be as large as available memory will permit. The space required for n seconds of delay is 4n * \emph{sr}
 bytes. It is allocated at the time the instrument is first initialized, and returned to the pool at the end of a score section. 


 \emph{iskip}
 (optional, default=0) -- initial disposition of delay-loop data space (see \emph{reson}
). The default value is 0. 
\subsection*{Performance}


 \emph{delayr}
 reads from an automatically established digital delay line, in which the signal retrieved has been resident for \emph{idlt}
 seconds. This unit must be paired with and precede an accompanying \emph{delayw}
 unit. Any other Csound statements can intervene. 
\subsection*{Examples}


  See the example for \emph{delayw}
. 
\subsection*{See Also}


 \emph{delay}
, \emph{delay1}
, \emph{delayw}

%\hline 


\begin{comment}
\begin{tabular}{lcr}
Previous &Home &Next \\
delay1 &Up &delayw

\end{tabular}


\end{document}
\end{comment}
