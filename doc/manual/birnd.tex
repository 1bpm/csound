\begin{comment}
\documentclass[10pt]{article}
\usepackage{fullpage, graphicx, url}
\setlength{\parskip}{1ex}
\setlength{\parindent}{0ex}
\title{birnd}
\begin{document}


\begin{tabular}{ccc}
The Alternative Csound Reference Manual & & \\
Previous & &Next

\end{tabular}

%\hline 
\end{comment}
\section{birnd}
birnd�--� Returns a random number in a bi-polar range. \subsection*{Description}


  Returns a random number in a bi-polar range. 
\subsection*{Syntax}


 \textbf{birnd}
(x) (init- or control-rate only)


  Where the argument within the parentheses may be an expression. These value converters sample a global random sequence, but do not reference \emph{seed}
. The result can be a term in a further expression. 
\subsection*{Performance}


  Returns a random number in the bipolar range -\emph{x}
 to \emph{x}
. \emph{rnd}
 and \emph{birnd}
 obtain values from a global pseudo-random number generator, then scale them into the requested range. The single global generator will thus distribute its sequence to these units throughout the performance, in whatever order the requests arrive. 
\subsection*{Examples}


  Here is an example of the birnd opcode. It uses the files \emph{birnd.orc}
 and \emph{birnd.sco}
. 


 \textbf{Example 1. Example of the birnd opcode.}

\begin{lstlisting}
/* birnd.orc */
; Initialize the global variables.
sr = 44100
kr = 4410
ksmps = 10
nchnls = 1

; Instrument #1.
instr 1
  ; Generate a random number from -1 to 1.
  i1 = birnd(1)
  print i1
endin
/* birnd.orc */
        
\end{lstlisting}
\begin{lstlisting}
/* birnd.sco */
; Play Instrument #1 for one second.
i 1 0 1
; Play Instrument #1 for one second.
i 1 1 1
e
/* birnd.sco */
        
\end{lstlisting}
 Its output should include lines like: \begin{lstlisting}
instr 1:  i1 = 0.947
instr 1:  i1 = -0.721
      
\end{lstlisting}
\subsection*{See Also}


 \emph{rnd}

\subsection*{Credits}


 


 


\begin{tabular}{cccc}
Author: Barry L. Vercoe &MIT &Cambridge, Massachussetts &1997

\end{tabular}



 


 Example written by Kevin Conder.
%\hline 


\begin{comment}
\begin{tabular}{lcr}
Previous &Home &Next \\
biquada &Up &bqrez

\end{tabular}


\end{document}
\end{comment}
