\begin{comment}
\documentclass[10pt]{article}
\usepackage{fullpage, graphicx, url}
\setlength{\parskip}{1ex}
\setlength{\parindent}{0ex}
\title{setctrl}
\begin{document}


\begin{tabular}{ccc}
The Alternative Csound Reference Manual & & \\
Previous & &Next

\end{tabular}

%\hline 
\end{comment}
\section{setctrl}
setctrl�--� Configurable slider controls for realtime user input. \subsection*{Description}


  Configurable slider controls for realtime user input. Requires Winsound or TCL/TK. \emph{setctrl}
 sets a slider to a specific value, or sets a minimum or maximum range. 
\subsection*{Syntax}


 \textbf{setctrl}
 inum, ival, itype
\subsection*{Initialization}


 \emph{inum}
 -- number of the slider to set 


 \emph{ival}
 -- value to be sent to the slider 


 \emph{itype}
 -- type of value sent to the slider as follows: 


 
\begin{itemize}
\item 

 1 -- set the current value. Initial value is 0.

\item 

 2 -- set the minimum value. Default is 0.

\item 

 3 -- set the maximum value. Default is 127.

\item 

 4 -- set the label. (New in Csound version 4.09)


\end{itemize}
\subsection*{Performance}


  Calling \emph{setctrl}
 will create a new slider on the screen. There is no theoretical limit to the number of sliders. Windows and TCL/TK use only integers for slider values, so the values may need rescaling. GUIs usually pass values at a fairly slow rate, so it may be advisable to pass the output of control through \emph{port}
. 
\subsection*{Examples}


  Here is an example of the setctrl opcode. It uses the files \emph{setctrl.orc}
 and \emph{setctrl.sco}
. 


 \textbf{Example 1. Example of the setctrl opcode.}

\begin{lstlisting}
/* setctrl.orc */
; Initialize the global variables.
sr = 44100
kr = 4410
ksmps = 10
nchnls = 1

; Instrument #1.
instr 1
  ; Display the label "Volume" on Slider #1.
  setctrl 1, "Volume", 4
  ; Set Slider #1's initial value to 20.
  setctrl 1, 20, 1
  
  ; Capture and display the values for Slider #1.
  k1 control 1
  printk2 k1

  ; Play a simple oscillator.
  ; Use the values from Slider #1 for amplitude.
  kamp = k1 * 128
  a1 oscil kamp, 440, 1
  out a1
endin
/* setctrl.orc */
        
\end{lstlisting}
\begin{lstlisting}
/* setsctrl.sco */
; Table #1, a sine wave.
f 1 0 16384 10 1

; Play Instrument #1 for thirty seconds.
i 1 0 30
e
/* setsctrl.sco */
        
\end{lstlisting}
 Its output should include lines like this: \begin{lstlisting}
 i1    38.00000
 i1    40.00000
 i1    43.00000
      
\end{lstlisting}
\subsection*{See Also}


 \emph{control}

\subsection*{Credits}


 


 


\begin{tabular}{cccc}
Author: John ffitch &University of Bath, Codemist. Ltd. &Bath, UK &May 2000

\end{tabular}



 


 Example written by Kevin Conder.


 New in Csound version 4.06
%\hline 


\begin{comment}
\begin{tabular}{lcr}
Previous &Home &Next \\
seqtime &Up &setksmps

\end{tabular}


\end{document}
\end{comment}
