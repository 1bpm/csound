\begin{comment}
\documentclass[10pt]{article}
\usepackage{fullpage, graphicx, url}
\setlength{\parskip}{1ex}
\setlength{\parindent}{0ex}
\title{active}
\begin{document}


\begin{tabular}{ccc}
The Alternative Csound Reference Manual & & \\
Previous & &Next

\end{tabular}

%\hline 
\end{comment}
\section{active}
active�--� Returns the number of active instances of an instrument. \subsection*{Description}


  Returns the number of active instances of an instrument. 
\subsection*{Syntax}


 ir \textbf{active}
 insnum


 kr \textbf{active}
 kinsnum
\subsection*{Initialization}


 \emph{insnum}
 -- number of the instrument to be reported 
\subsection*{Performance}


 \emph{kinsnum}
 -- number of the instrument to be reported 


 \emph{active}
 returns the number of active instances of instrument number insnum/kinsnum. As of Csound4.17 the output is updated at k-rate (if input arg is k-rate), to allow running count of instr instances. 
\subsection*{Examples}


  Here is a simple example of the active opcode. It uses the files \emph{active.orc}
 and \emph{active.sco}
. 


 \textbf{Example 1. Simple example of the active opcode.}

\begin{lstlisting}
/* active.orc */
; Initialize the global variables.
sr = 44100
kr = 4410
ksmps = 10
nchnls = 1

; Instrument #1 - a noisy waveform.
instr 1
  ; Generate a really noisy waveform.
  anoisy rand 44100
  ; Turn down its amplitude.
  aoutput gain anoisy, 2500
  ; Send it to the output.
  out aoutput
endin

; Instrument #2 - counts active instruments.
instr 2
  ; Count the active instances of Instrument #1.
  icount active 1
  ; Print the number of active instances.
  print icount
endin
/* active.orc */
        
\end{lstlisting}
\begin{lstlisting}
/* active.sco */
; Start the first instance of Instrument #1 at 0:00 seconds.
i 1 0.0 3.0

; Start the second instance of Instrument #1 at 0:015 seconds.
i 1 1.5 1.5

; Play Instrument #2 at 0:01 seconds, when we have only 
; one active instance of Instrument #1.
i 2 1.0 0.1

; Play Instrument #2 at 0:02 seconds, when we have 
; two active instances of Instrument #1.
i 2 2.0 0.1
e
/* active.sco */
        
\end{lstlisting}
 Its output should include lines like this: \begin{lstlisting}
instr 2:  icount = 1.000
instr 2:  icount = 2.000
      
\end{lstlisting}


  Here is a more advanced example of the active opcode. It displays the results of the active opcode at k-rate instead of i-rate. It uses the files \emph{active\_k.orc}
 and \emph{active\_k.sco}
. 


 \textbf{Example 2. Example of the active opcode at k-rate.}

\begin{lstlisting}
/* active_k.orc */
; Initialize the global variables.
sr = 44100
kr = 4410
ksmps = 10
nchnls = 1

; Instrument #1 - a noisy waveform.
instr 1
  ; Generate a really noisy waveform.
  anoisy rand 44100
  ; Turn down its amplitude.
  aoutput gain anoisy, 2500
  ; Send it to the output.
  out aoutput
endin

; Instrument #2 - counts active instruments at k-rate.
instr 2
  ; Count the active instances of Instrument #1.
  kcount active 1
  ; Print the number of active instances.
  printk2 kcount
endin
/* active_k.orc */
        
\end{lstlisting}
\begin{lstlisting}
/* active_k.sco */
; Start the first instance of Instrument #1 at 0:00 seconds.
i 1 0.0 3.0

; Start the second instance of Instrument #1 at 0:015 seconds.
i 1 1.5 1.5

; Play Instrument #2 at 0:01 seconds, when we have only 
; one active instance of Instrument #1.
i 2 1.0 0.1

; Play Instrument #2 at 0:02 seconds, when we have 
; two active instances of Instrument #1.
i 2 2.0 0.1
e
/* active_k.sco */
        
\end{lstlisting}
 Its output should include lines like: \begin{lstlisting}
 i2     1.00000
 i2     2.00000
      
\end{lstlisting}
\subsection*{Credits}


 


 


\begin{tabular}{cccc}
Author: John ffitch &University of Bath/Codemist Ltd. &Bath, UK &July, 1999

\end{tabular}



 


 Examples written by Kevin Conder.


 New in Csound version 3.57
%\hline 


\begin{comment}
\begin{tabular}{lcr}
Previous &Home &Next \\
acauchy &Up &adsr

\end{tabular}


\end{document}
\end{comment}
