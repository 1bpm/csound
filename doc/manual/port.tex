\begin{comment}
\documentclass[10pt]{article}
\usepackage{fullpage, graphicx, url}
\setlength{\parskip}{1ex}
\setlength{\parindent}{0ex}
\title{port}
\begin{document}


\begin{tabular}{ccc}
The Alternative Csound Reference Manual & & \\
Previous & &Next

\end{tabular}

%\hline 
\end{comment}
\section{port}
port�--� Applies portamento to a step-valued control signal. \subsection*{Description}


  Applies portamento to a step-valued control signal. 
\subsection*{Syntax}


 kr \textbf{port}
 ksig, ihtim [, isig]
\subsection*{Initialization}


 \emph{ihtim}
 -- half-time of the function, in seconds. 


 \emph{isig}
 (optional, default=0) -- initial (i.e. previous) value for internal feedback. The default value is 0. 
\subsection*{Performance}


 \emph{kr}
 -- the output signal at control-rate. 


 \emph{ksig}
 -- the input signal at control-rate. 


 \emph{port}
 applies portamento to a step-valued control signal. At each new step value, \emph{ksig}
 is low-pass filtered to move towards that value at a rate determined by \emph{ihtim}
. \emph{ihtim}
 is the ``half-time'' of the function (in seconds), during which the curve will traverse half the distance towards the new value, then half as much again, etc., theoretically never reaching its asymptote. With \emph{portk}
, the half-time can be varied at the control rate. 
\subsection*{See Also}


 \emph{areson}
, \emph{aresonk}
, \emph{atone}
, \emph{atonek}
, \emph{portk}
, \emph{reson}
, \emph{resonk}
, \emph{tone}
, \emph{tonek}

%\hline 


\begin{comment}
\begin{tabular}{lcr}
Previous &Home &Next \\
polyaft &Up &portk

\end{tabular}


\end{document}
\end{comment}
