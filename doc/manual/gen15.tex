\begin{comment}
\documentclass[10pt]{article}
\usepackage{fullpage, graphicx, url}
\setlength{\parskip}{1ex}
\setlength{\parindent}{0ex}
\title{GEN15}
\begin{document}


\begin{tabular}{ccc}
The Alternative Csound Reference Manual & & \\
Previous & &Next

\end{tabular}

%\hline 
\end{comment}
\section{GEN15}
GEN15�--� Creates two tables of stored polynomial functions. \subsection*{Description}


  This subroutine creates two tables of stored polynomial functions, suitable for use in phase quadrature operations. 
\subsection*{Syntax}


 \textbf{f}
 \# time size 15 xint xamp h0 phs0 h1 phs1 h2 phs2 ...
\subsection*{Initialization}


 \emph{size}
 -- number of points in the table. Must be a power of 2 or a power-of-2 plus 1 (see \emph{f statement}
). The normal value is power-of-2 plus 1. 


 \emph{xint}
 -- provides the left and right values [\emph{-xint}
, \emph{+xint}
] of the \emph{x}
 interval over which the polynomial is to be drawn. This subroutine will eventually call \emph{GEN03}
 to draw both functions; this p5 value is therefor expanded to a negative-positive p5, p6 pair before \emph{GEN03}
 is actually called. The normal value is 1. 


 \emph{xamp }
 -- amplitude scaling factor of the sinusoid input that is expected to produce the following spectrum. 


 \emph{h0, h1, h2, ... hn}
 -- relative strength of partials 0 (DC), 1 (fundamental), 2 ... that will result when a sinusoid of amplitude 


 xamp�*�int(size/2)/xint\\ 
 ������
 is waveshaped using this function table. These values thus describe a frequency spectrum associated with a particular factor \emph{xamp}
 of the input signal. 

 \emph{phs0, phs1, ... }
 -- phase in degrees of desired harmonics \emph{h0, h1, ...}
 when the two functions of \emph{GEN15}
 are used with phase quadrature. 


 


\begin{tabular}{cc}
\textbf{Note}
 \\
� &

 \emph{GEN15}
 creates two tables of equal size, labeled \emph{f }
\# and \emph{f}
 \# + 1. Table \# will contain a Chebyshev function of the first kind, drawn using \emph{GEN03}
 with partial strengths \emph{h0cos(phs0), h1cos(phs1), ...}
 Table \#+1 will contain a Chebyshev function of the 2nd kind by calling \emph{GEN14}
 with partials \emph{h1sin(phs1), h2sin(phs2),...}
 (note the harmonic displacement). The two tables can be used in conjunction in a waveshaping network that exploits phase quadrature. 


\end{tabular}

\subsection*{See Also}


 \emph{GEN03}
, \emph{GEN13}
, and \emph{GEN14}
. 
%\hline 


\begin{comment}
\begin{tabular}{lcr}
Previous &Home &Next \\
GEN14 &Up &GEN16

\end{tabular}


\end{document}
\end{comment}
