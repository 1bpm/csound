\begin{comment}
\documentclass[10pt]{article}
\usepackage{fullpage, graphicx, url}
\setlength{\parskip}{1ex}
\setlength{\parindent}{0ex}
\title{fini}
\begin{document}


\begin{tabular}{ccc}
The Alternative Csound Reference Manual & & \\
Previous & &Next

\end{tabular}

%\hline 
\end{comment}
\section{fini}
fini�--� Read signals from a file at i-rate. \subsection*{Description}


  Read signals from a file at i-rate. 
\subsection*{Syntax}


 \textbf{fini}
 ifilename, iskipframes, iformat, in1 [, in2] [, in3] [, ...]
\subsection*{Initialization}


 \emph{ifilename}
 -- input file name (can be a string or a handle number generated by \emph{fiopen}
) 


 \emph{iskipframes}
 -- number of frames to skip at the start (every frame contains a sample of each channel) 


 \emph{iformat}
 -- a number specifying the input file format. 


 
\begin{itemize}
\item 

 0 - floating points in text format (loop; see below)

\item 

 1 - floating points in text format (no loop; see below)

\item 

 2 - 32 bit floating points in binary format (no loop)


\end{itemize}
\subsection*{Performance}


 \emph{fini}
 is the complement of \emph{fouti}
 and \emph{foutir}
. It reads the values each time the corresponding instrument note is activated. When \emph{iformat}
 is set to 0 and the end of file is reached, the file pointer is zeroed. This restarts the scan from the beginning. When \emph{iformat}
 is set to 1 or 2, no looping is enabled and at the end of file the corresponding variables will be filled with zeroes. 
\subsection*{See Also}


 \emph{fin}
, \emph{fink}

\subsection*{Credits}


 


 


\begin{tabular}{ccc}
Author: Gabriel Maldonado &Italy &1999

\end{tabular}



 


 New in Csound version 3.56
%\hline 


\begin{comment}
\begin{tabular}{lcr}
Previous &Home &Next \\
fin &Up &fink

\end{tabular}


\end{document}
\end{comment}
