\begin{comment}
\documentclass[10pt]{article}
\usepackage{fullpage, graphicx, url}
\setlength{\parskip}{1ex}
\setlength{\parindent}{0ex}
\title{midiout}
\begin{document}


\begin{tabular}{ccc}
The Alternative Csound Reference Manual & & \\
Previous & &Next

\end{tabular}

%\hline 
\end{comment}
\section{midiout}
midiout�--� Sends a generic MIDI message to the MIDI OUT port. \subsection*{Description}


  Sends a generic MIDI message to the MIDI OUT port. 
\subsection*{Syntax}


 \textbf{midiout}
 kstatus, kchan, kdata1, kdata2
\subsection*{Performance}


 \emph{kstatus}
 -- the type of MIDI message. Can be: 


 
\begin{itemize}
\item 

 128 (note off)

\item 

 144 (note on)

\item 

 160 (polyphonic aftertouch)

\item 

 176 (control change)

\item 

 192 (program change)

\item 

 208 (channel aftertouch)

\item 

 224 (pitch bend)

\item 

 0 when no MIDI messages must be sent to the MIDI OUT port


\end{itemize}


 \emph{kchan}
 -- MIDI channel (1-16) 


 \emph{kdata1, kdata2}
 -- message-dependent data values 


 \emph{midiout}
 has no output arguments, because it sends a message to the MIDI OUT port implicitly. It works at k-rate. It sends a MIDI message only when \emph{kstatus}
 is non-zero. 


 


\begin{tabular}{cc}
Warning &

 \emph{Warning:}
 Normally \emph{kstatus}
 should be set to 0. Only when the user intends to send a MIDI message, can it be set to the corresponding message type number. 


\end{tabular}

\subsection*{Credits}


 


 


\begin{tabular}{ccc}
Author: Gabriel Maldonado &Italy &1998

\end{tabular}



 


 New in Csound version 3.492
%\hline 


\begin{comment}
\begin{tabular}{lcr}
Previous &Home &Next \\
midion2 &Up &midipitchbend

\end{tabular}


\end{document}
\end{comment}
