\begin{comment}
\documentclass[10pt]{article}
\usepackage{fullpage, graphicx, url}
\setlength{\parskip}{1ex}
\setlength{\parindent}{0ex}
\title{else}
\begin{document}


\begin{tabular}{ccc}
The Alternative Csound Reference Manual & & \\
Previous & &Next

\end{tabular}

%\hline 
\end{comment}
\section{else}
else�--� Executes a block of code when an ``if...then'' condition is false. \subsection*{Description}


  Executes a block of code when an ``if...then'' condition is false. 
\subsection*{Syntax}


 \textbf{else}

\subsection*{Performance}


 \emph{else}
 is used inside of a block of code between the \emph{``if...then''}
 and \emph{endif}
 opcodes. It defines which statements are executed when a ``if...then'' condition is false. Only one \emph{else}
 statement may occur and it must be the last conditional statement before the \emph{endif}
 opcode. 
\subsection*{Examples}


  See the example for the \emph{if}
 opcode. 
\subsection*{See Also}


 \emph{elseif}
, \emph{endif}
, \emph{goto}
, \emph{if}
, \emph{igoto}
, \emph{kgoto}
, \emph{tigoto}
, \emph{timout}

\subsection*{Credits}


 New in version 4.21
%\hline 


\begin{comment}
\begin{tabular}{lcr}
Previous &Home &Next \\
duserrnd &Up &elseif

\end{tabular}


\end{document}
\end{comment}
