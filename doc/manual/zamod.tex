\begin{comment}
\documentclass[10pt]{article}
\usepackage{fullpage, graphicx, url}
\setlength{\parskip}{1ex}
\setlength{\parindent}{0ex}
\title{zamod}
\begin{document}


\begin{tabular}{ccc}
The Alternative Csound Reference Manual & & \\
Previous & &Next

\end{tabular}

%\hline 
\end{comment}
\section{zamod}
zamod�--� Modulates one a-rate signal by a second one. \subsection*{Description}


  Modulates one a-rate signal by a second one. 
\subsection*{Syntax}


 ar \textbf{zamod}
 asig, kzamod
\subsection*{Performance}


 \emph{asig}
 -- the input signal 


 \emph{kzamod}
 -- controls which za variable is used for modulation. A positive value means additive modulation, a negative value means multiplicative modulation. A value of 0 means no change to \emph{asig}
. 


 \emph{zamod}
 modulates one a-rate signal by a second one, which comes from a za variable. The location of the modulating variable is controlled by the i-rate or k-rate variable \emph{kzamod}
. This is the a-rate version of \emph{zkmod}
. 
\subsection*{Examples}


  Here is an example of the zamod opcode. It uses the files \emph{zamod.orc}
 and \emph{zamod.sco}
. 


 \textbf{Example 1. Example of the zamod opcode.}

\begin{lstlisting}
/* zamod.orc */
; Initialize the global variables.
sr = 44100
kr = 4410
ksmps = 10
nchnls = 1

; Initialize the ZAK space.
; Create 2 a-rate variables and 2 k-rate variables.
zakinit 2, 2

; Instrument #1 -- a simple waveform.
instr 1
  ; Vary an a-rate signal linearly from 20,000 to 0.
  asig line 20000, p3, 0

  ; Send the signal to za variable #1.
  zaw asig, 1
endin

; Instrument #2 -- generates audio output.
instr 2
  ; Generate a simple sine wave.
  asin oscil 1, 440, 1
  
  ; Modify the sine wave, multiply its amplitude by 
  ; za variable #1.
  a1 zamod asin, -1

  ; Generate the audio output.
  out a1

  ; Clear the za variables, prepare them for 
  ; another pass.
  zacl 0, 2
endin
/* zamod.orc */
        
\end{lstlisting}
\begin{lstlisting}
/* zamod.sco */
; Table #1, a sine wave.
f 1 0 16384 10 1

; Play Instrument #1 for 2 seconds.
i 1 0 2
; Play Instrument #2 for 2 seconds.
i 2 0 2
e
/* zamod.sco */
        
\end{lstlisting}
\subsection*{See Also}


 \emph{zacl}
, \emph{ziw}
, \emph{ziwm}

\subsection*{Credits}


 


 


\begin{tabular}{ccc}
Author: Robin Whittle &Australia &May 1997

\end{tabular}



 


 Example written by Kevin Conder.
%\hline 


\begin{comment}
\begin{tabular}{lcr}
Previous &Home &Next \\
zakinit &Up &zar

\end{tabular}


\end{document}
\end{comment}
