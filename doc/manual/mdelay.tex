\begin{comment}
\documentclass[10pt]{article}
\usepackage{fullpage, graphicx, url}
\setlength{\parskip}{1ex}
\setlength{\parindent}{0ex}
\title{mdelay}
\begin{document}


\begin{tabular}{ccc}
The Alternative Csound Reference Manual & & \\
Previous & &Next

\end{tabular}

%\hline 
\end{comment}
\section{mdelay}
mdelay�--� A MIDI delay opcode. \subsection*{Description}


  A MIDI delay opcode. 
\subsection*{Syntax}


 \textbf{mdelay}
 kstatus, kchan, kd1, kd2, kdelay
\subsection*{Performance}


 \emph{kstatus}
 -- status byte of MIDI message to be delayed 


 \emph{kchan}
 -- MIDI channel (1-16) 


 \emph{kd1}
 -- first MIDI data byte 


 \emph{kd2}
 -- second MIDI data byte 


 \emph{kdelay}
 -- delay time in seconds 


  Each time that \emph{kstatus}
 is other than zero, \emph{mdelay}
 outputs a MIDI message to the MIDI out port after \emph{kdelay}
 seconds. This opcode is useful in implementing MIDI delays. Several instances of \emph{mdelay}
 can be present in the same instrument with different argument values, so complex and colorful MIDI echoes can be implemented. Further, the delay time can be changed at k-rate. 
\subsection*{Credits}


 


 


\begin{tabular}{ccc}
Author: Gabriel Maldonado &Italy &November 1998

\end{tabular}



 


 New in Csound version 3.492
%\hline 


\begin{comment}
\begin{tabular}{lcr}
Previous &Home &Next \\
mclock &Up &midic14

\end{tabular}


\end{document}
\end{comment}
