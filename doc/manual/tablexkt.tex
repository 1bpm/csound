\begin{comment}
\documentclass[10pt]{article}
\usepackage{fullpage, graphicx, url}
\setlength{\parskip}{1ex}
\setlength{\parindent}{0ex}
\title{tablexkt}
\begin{document}


\begin{tabular}{ccc}
The Alternative Csound Reference Manual & & \\
Previous & &Next

\end{tabular}

%\hline 
\end{comment}
\section{tablexkt}
tablexkt�--� Reads function tables with linear, cubic, or sinc interpolation. \subsection*{Description}


  Reads function tables with linear, cubic, or sinc interpolation. 
\subsection*{Syntax}


 ar \textbf{tablexkt}
 xndx, kfn, kwarp, iwsize [, ixmode] [, ixoff] [, iwrap]
\subsection*{Initialization}


 \emph{iwsize}
 -- This parameter controls the type of interpolation to be used: 


 
\begin{itemize}
\item 

 \emph{2:}
 Use linear interpolation. This is the lowest quality, but also the fastest mode.

\item 

 \emph{4:}
 Cubic interpolation. Slightly better quality than iwsize = 2, at the expense of being somewhat slower.

\item 

 \emph{8 and above (up to 1024):}
 sinc interpolation with window size set to iwsize (should be an integer multiply of 4). Better quality than linear or cubic interpolation, but very slow. When transposing up, a kwarp value above 1 can be used for anti-aliasing (this is even slower).


\end{itemize}


 \emph{ixmode1}
 (optional) -- index data mode. The default value is 0. 


 
\begin{itemize}
\item 

 \emph{0:}
 raw index

\item 

 \emph{any non-zero value:}
 normalized (0 to 1)


\end{itemize}


 


\begin{tabular}{cc}
\textbf{Notes}
 \\
� &

  if \emph{tablexkt}
 is used to play back samples with looping (e.g. table index is generated by lphasor), there must be at least iwsize / 2 extra samples after the loop end point for interpolation, otherwise audible clicking may occur (also, at least iwsize / 2 samples should be before the loop start point). 


\end{tabular}



 \emph{ixoff}
 (optional) -- amount by which index is to be offset. For a table with origin at center, use tablesize / 2 (raw) or 0.5 (normalized). The default value is 0. 


 \emph{iwrap}
 (optional) -- wraparound index flag. The default value is 0. 


 
\begin{itemize}
\item 

 \emph{0:}
 Nowrap (index $<$ 0 treated as index = 0; index $>$= tablesize (or 1.0 in normalized mode) sticks at the guard point).

\item 

 \emph{any non-zero value:}
 Index is wrapped to the allowed range (not including the guard point in this case).


\end{itemize}


 


\begin{tabular}{cc}
\textbf{Note}
 \\
� &

 \emph{iwrap}
 also applies to extra samples for interpolation. 


\end{tabular}

\subsection*{Performance}


 \emph{ar}
 -- audio output 


 \emph{xndx}
 -- table index 


 \emph{kfn}
 -- function table number 


 \emph{kwarp}
 -- if greater than 1, use sin (x / kwarp) / x function for sinc interpolation, instead of the default sin (x) / x. This is useful to avoid aliasing when transposing up (\emph{kwarp}
 should be set to the transpose factor in this case, e.g. 2.0 for one octave), however it makes rendering up to twice as slow. Also, \emph{iwsize}
 should be at least kwarp * 8. This feature is experimental, and may be optimized both in terms of speed and quality in new versions. 


 


\begin{tabular}{cc}
\textbf{Note}
 \\
� &

 \emph{kwarp}
 has no effect if it is less than, or equal to 1, or linear or cubic interpolation is used. 


\end{tabular}

\subsection*{Credits}


 


 


\begin{tabular}{cc}
Author: Istvan Varga &January 2002

\end{tabular}



 


 New in version 4.18
%\hline 


\begin{comment}
\begin{tabular}{lcr}
Previous &Home &Next \\
tablewkt &Up &tablexseg

\end{tabular}


\end{document}
\end{comment}
