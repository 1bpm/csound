\begin{comment}
\documentclass[10pt]{article}
\usepackage{fullpage, graphicx, url}
\setlength{\parskip}{1ex}
\setlength{\parindent}{0ex}
\title{GEN04}
\begin{document}


\begin{tabular}{ccc}
The Alternative Csound Reference Manual & & \\
Previous & &Next

\end{tabular}

%\hline 
\end{comment}
\section{GEN04}
GEN04�--� Generates a normalizing function. \subsection*{Description}


  This subroutine generates a normalizing function by examining the contents of an existing table. 
\subsection*{Syntax}


 \textbf{f}
 \# time size 4 source\# sourcemode
\subsection*{Initialization}


 \emph{size}
 -- number of points in the table. Should be power-of-2 plus 1. Must not exceed (except by 1) the size of the source table being examined; limited to just half that size if the sourcemode is of type offset (see below). 


 \emph{source \#}
 -- table number of stored function to be examined. 


 \emph{sourcemode}
 -- a coded value, specifying how the source table is to be scanned to obtain the normalizing function. Zero indicates that the source is to be scanned from left to right. Non-zero indicates that the source has a bipolar structure; scanning will begin at the mid-point and progress outwards, looking at pairs of points equidistant from the center. 


 


\begin{tabular}{cc}
\textbf{Note}
 \\
� &

 


 
\begin{itemize}
\item 

  The normalizing function derives from the progressive absolute maxima of the source table being scanned. The new table is created left-to-right, with stored values equal to 1/(absolute maximum so far scanned). Stored values will thus begin with 1/(first value scanned), then get progressively smaller as new maxima are encountered. For a source table which is normalized (values $<$= 1), the derived values will range from 1/(first value scanned) down to 1. If the first value scanned is zero, that inverse will be set to 1. 

\item 

  The normalizing function from \emph{GEN04}
 is not itself normalized. 

\item 

 \emph{GEN04}
 is useful for scaling a table-derived signal so that it has a consistent peak amplitude. A particular application occurs in waveshaping when the carrier (or indexing) signal is less than full amplitude. 


\end{itemize}


\end{tabular}

\subsection*{Examples}


 


 
\begin{lstlisting}
\emph{f}
   2   0   512   4    1   1   
        
\end{lstlisting}


 
 This creates a normalizing function for use in connection with the \emph{GEN03}
 table 1 example. Midpoint bipolar offset is specified. %\hline 


\begin{comment}
\begin{tabular}{lcr}
Previous &Home &Next \\
GEN03 &Up &GEN05

\end{tabular}


\end{document}
\end{comment}
