\begin{comment}
\documentclass[10pt]{article}
\usepackage{fullpage, graphicx, url}
\setlength{\parskip}{1ex}
\setlength{\parindent}{0ex}
\title{=}
\begin{document}


\begin{tabular}{ccc}
The Alternative Csound Reference Manual & & \\
Previous & &Next

\end{tabular}

%\hline 
\end{comment}
\section{=}
=�--� Performs a simple assignment. \subsection*{Syntax}


 ar \textbf{=}
 xarg


 ir \textbf{=}
 iarg


 kr \textbf{=}
 karg
\subsection*{Description}


  Performs a simple assignment. 
\subsection*{Initialization}


 \emph{=}
 (simple assignment) - Put the value of the expression \emph{iarg}
 (\emph{karg, xarg}
) into the named result. This provides a means of saving an evaluated result for later use. 
\subsection*{Examples}


  Here is an example of the assign opcode. It uses the files \emph{assign.orc}
 and \emph{assign.sco}
. 


 \textbf{Example 1. Example of the assign opcode.}

\begin{lstlisting}
/* assign.orc */
; Initialize the global variables.
sr = 44100
kr = 4410
ksmps = 10
nchnls = 1

; Instrument #1.
instr 1
  ; Assign a value to the variable i1.
  i1 = 1234

  ; Print the value of the i1 variable.
  print i1
endin
/* assign.orc */
        
\end{lstlisting}
\begin{lstlisting}
/* assign.sco */
; Play Instrument #1 for one second.
i 1 0 1
e
/* assign.sco */
        
\end{lstlisting}
 Its output should include a line like this: \begin{lstlisting}
instr 1:  i1 = 1234.000
      
\end{lstlisting}
\subsection*{See Also}


 \emph{divz}
, \emph{init}
, \emph{tival}

\subsection*{Credits}


 Example written by Kevin Conder.
%\hline 


\begin{comment}
\begin{tabular}{lcr}
Previous &Home &Next \\
/ &Up &==

\end{tabular}


\end{document}
\end{comment}
