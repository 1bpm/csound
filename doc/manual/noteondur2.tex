\begin{comment}
\documentclass[10pt]{article}
\usepackage{fullpage, graphicx, url}
\setlength{\parskip}{1ex}
\setlength{\parindent}{0ex}
\title{noteondur2}
\begin{document}


\begin{tabular}{ccc}
The Alternative Csound Reference Manual & & \\
Previous & &Next

\end{tabular}

%\hline 
\end{comment}
\section{noteondur2}
noteondur2�--� Sends a noteon and a noteoff MIDI message both with the same channel, number and velocity. \subsection*{Description}


  Sends a noteon and a noteoff MIDI message both with the same channel, number and velocity. 
\subsection*{Syntax}


 \textbf{noteondur2}
 ichn, inum, ivel, idur
\subsection*{Initialization}


 \emph{ichn}
 -- MIDI channel number (1-16) 


 \emph{inum}
 -- note number (0-127) 


 \emph{ivel}
 -- velocity (0-127) 


 \emph{idur}
 -- how long, in seconds, this note should last. 
\subsection*{Performance}


 \emph{noteondur2}
 (i-rate note on with duration) sends a noteon and a noteoff MIDI message both with the same channel, number and velocity. Noteoff message is sent after \emph{idur}
 seconds are elapsed by the time \emph{noteondur2}
 was active. 


 \emph{noteondur}
 differs from \emph{noteondur2}
 in that \emph{noteondur}
 truncates note duration when current instrument is deactivated by score or by real-time playing, while \emph{noteondur2}
 will extend performance time of current instrument until \emph{idur}
 seconds have elapsed. In real-time playing, it is suggested to use \emph{noteondur}
 also for undefined durations, giving a large \emph{idur}
 value. 


  Any number of \emph{noteondur2}
 opcodes can appear in the same Csound instrument, allowing chords to be played by a single instrument. 
\subsection*{See Also}


 \emph{noteoff}
, \emph{noteon}
, \emph{noteondur}

\subsection*{Credits}


 


 


\begin{tabular}{cc}
Author: Gabriel Maldonado &Italy

\end{tabular}



 


 New in Csound version 3.47


 Thanks goes to Rasmus Ekman for pointing out the correct MIDI channel and controller number ranges.
%\hline 


\begin{comment}
\begin{tabular}{lcr}
Previous &Home &Next \\
noteondur &Up &notnum

\end{tabular}


\end{document}
\end{comment}
