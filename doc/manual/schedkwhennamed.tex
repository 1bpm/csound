\begin{comment}
\documentclass[10pt]{article}
\usepackage{fullpage, graphicx, url}
\setlength{\parskip}{1ex}
\setlength{\parindent}{0ex}
\title{schedkwhennamed}
\begin{document}


\begin{tabular}{ccc}
The Alternative Csound Reference Manual & & \\
Previous & &Next

\end{tabular}

%\hline 
\end{comment}
\section{schedkwhennamed}
schedkwhennamed�--� Similar to schedkwhen but uses a named instrument at init-time. \subsection*{Description}


  Similar to \emph{schedkwhen}
 but uses a named instrument at init-time. 
\subsection*{Syntax}


 \textbf{schedkwhennamed}
 ktrigger, kmintim, kmaxnum, ``name'', kwhen, kdur [, ip4] [, ip5] [...]
\subsection*{Initialization}


 \emph{ip4, ip5, ...}
 -- Equivalent to p4, p5, etc., in a score \emph{i statement}

\subsection*{Performance}


 \emph{ktrigger}
 -- triggers a new score event. If \emph{ktrigger}
 is 0, no new event is triggered. 


 \emph{kmintim}
 -- minimum time between generated events, in seconds. If \emph{kmintim}
 is less than or equal to 0, no time limit exists. 


 \emph{kmaxnum}
 -- maximum number of simultaneous instances of named instrument allowed. If the number of extant instances of the named instrument is greater than or equal to \emph{kmaxnum}
, no new event is generated. If \emph{kmaxnum}
 is less than or equal to 0, it is not used to limit event generation. 


 \emph{``name''}
 -- the named instrument's name. 


 \emph{kwhen}
 -- start time of the new event. Equivalent to p2 in a score \emph{i statement}
. Measured from the time of the triggering event. \emph{kwhen}
 must be greater than or equal to 0. If \emph{kwhen}
 greater than 0, the instrument will not be initialized until the actual time when it should start performing. 


 \emph{kdur}
 -- duration of event. Equivalent to p3 in a score \emph{i statement}
. If \emph{kdur}
 is 0, the instrument will only do an initialization pass, with no performance. If \emph{kdur}
 is negative, a held note is initiated. (See \emph{ihold}
 and \emph{i statement}
.) 


 \emph{Note}
: While waiting for events to be triggered by \emph{schedkwhennamed}
, the performance must be kept going, or Csound may quit if no score events are expected. To guarantee continued performance, an \emph{f0 statement}
 may be used in the score. 
\subsection*{See Also}


 \emph{schedkwhen}

\subsection*{Credits}


 


 


\begin{tabular}{cc}
Author: Rasmus Ekman &EMS, Stockholm, Sweden

\end{tabular}



 


 New in Csound version 4.23
%\hline 


\begin{comment}
\begin{tabular}{lcr}
Previous &Home &Next \\
schedkwhen &Up &schedule

\end{tabular}


\end{document}
\end{comment}
