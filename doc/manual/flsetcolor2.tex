\begin{comment}
\documentclass[10pt]{article}
\usepackage{fullpage, graphicx, url}
\setlength{\parskip}{1ex}
\setlength{\parindent}{0ex}
\title{FLsetColor2}
\begin{document}


\begin{tabular}{ccc}
The Alternative Csound Reference Manual & & \\
Previous & &Next

\end{tabular}

%\hline 
\end{comment}
\section{FLsetColor2}
FLsetColor2�--� Sets the secondary (or selection) color of a FLTK widget. \subsection*{Description}


 \emph{FLsetColor2}
 sets the secondary (or selection) color of the target widget. 
\subsection*{Syntax}


 \textbf{FLsetColor2}
 ired, igreen, iblue, ihandle
\subsection*{Initialization}


 \emph{ired}
 -- The red color of the target widget. The range for each RGB component is 0-255 


 \emph{igreen}
 -- The green color of the target widget. The range for each RGB component is 0-255 


 \emph{iblue}
 -- The blue color of the target widget. The range for each RGB component is 0-255 


 \emph{ihandle}
 -- an integer number (used as unique identifier) taken from the output of a previously located widget opcode (which corresponds to the target widget). It is used to unequivocally identify the widget when modifying its appearance with this class of opcodes. The user must not set the \emph{ihandle}
 value directly, otherwise a Csound crash will occur. 
\subsection*{See Also}


 \emph{FLcolor}
, \emph{FLcolor2}
, \emph{FLhide}
, \emph{FLlabel}
, \emph{FLsetAlign}
, \emph{FLsetBox}
, \emph{FLsetColor}
, \emph{FLsetFont}
, \emph{FLsetPosition}
, \emph{FLsetSize}
, \emph{FLsetText}
, \emph{FLsetTextColor}
, \emph{FLsetTextSize}
, \emph{FLsetTextType}
, \emph{FLsetVal\_i}
, \emph{FLsetVal}
, \emph{FLshow}

\subsection*{Credits}


 Author: Gabriel Maldonado


 New in version 4.22
%\hline 


\begin{comment}
\begin{tabular}{lcr}
Previous &Home &Next \\
FLsetColor &Up &FLsetFont

\end{tabular}


\end{document}
\end{comment}
