\begin{comment}
\documentclass[10pt]{article}
\usepackage{fullpage, graphicx, url}
\setlength{\parskip}{1ex}
\setlength{\parindent}{0ex}
\title{vbap4move}
\begin{document}


\begin{tabular}{ccc}
The Alternative Csound Reference Manual & & \\
Previous & &Next

\end{tabular}

%\hline 
\end{comment}
\section{vbap4move}
vbap4move�--� Distributes an audio signal among 4 channels with moving virtual sources. \subsection*{Description}


  Distributes an audio signal among 4 channels with moving virtual sources. 
\subsection*{Syntax}


 ar1, ar2, ar3, ar4 \textbf{vbap4move}
 asig, ispread, ifldnum, ifld1 [, ifld2] [...]
\subsection*{Initialization}


 \emph{ispread}
 -- spreading of the virtual source (range 0 - 100). If value is zero, conventional amplitude panning is used. When \emph{ispread}
 is increased, the number of loudspeakers used in panning increases. If value is 100, the sound is applied to all loudspeakers. 


 \emph{ifldnum}
 -- number of fields (absolute value must be 2 or larger). If \emph{ifldnum}
 is positive, the virtual source movement is a polyline specified by given directions. Each transition is performed in an equal time interval. If \emph{ifldnum}
 is negative, specified angular velocities are applied to the virtual source during specified relative time intervals (see below). 


 \emph{ifld1, ifld2, ...}
 -- azimuth angles or angular velocities, and relative durations of movement phases (see below). 
\subsection*{Performance}


 \emph{asig}
 -- audio signal to be panned 


 \emph{vbap4move}
 allows the use of moving virtual sources. If \emph{ifldnum}
 is positive, the fields represent directions of virtual sources and equal times, \emph{iazi1}
, [\emph{iele1}
,] \emph{iazi2}
, [\emph{iele2}
,], etc. The position of the virtual source is interpolated between directions starting from the first direction and ending at the last. Each interval is interpolated in time that is fraction total\_time / number\_of\_intervals of the duration of the sound event. 


  If \emph{ifldnum}
 is negative, the fields represent angular velocities and equal times. The first field is, however, the starting direction, \emph{iazi1}
, [\emph{iele1}
,] \emph{iazi\_vel1}
, [\emph{iele\_vel1}
,] \emph{iazi\_vel2}
, [\emph{iele\_vel2}
,] .... Each velocity is applied to the note that is fraction total\_time / number\_of\_velocities of the duration of the sound event. If the elevation of the virtual source becomes greater than 90 degrees or less than 0 degrees, the polarity of angular velocity is changed. Thus the elevational angular velocity produces a virtual source that moves up and down between 0 and 90 degrees. 
\subsection*{Examples}


 


 \textbf{Example 1. 2-D panning example with stationary virtual sources}

\begin{lstlisting}
  \emph{sr}
      =          4100
  \emph{kr}
      =           441
  \emph{ksmps}
   =           100
  \emph{nchnls}
  =             4
  \emph{vbaplsinit}
         2, 6,  0, 45, 90, 135, 200, 245, 290, 315 

          \emph{instr}
 1	           
  asig    \emph{oscil}
      20000, 440, 1                    
  a1,a2,a3,a4,a5,a6,a7,a8   \emph{vbap8}
  asig, p4, 0, 20 ;p4 = azimuth
	
  ;render twice with alternate \emph{outq}
 statements
  ;  to obtain two 4 channel .wav files:

          \emph{outq}
       a1,a2,a3,a4
  ;       \emph{outq}
       a5,a6,a7,a8
          \emph{endin}

        
\end{lstlisting}
\subsection*{Reference}


  Ville Pulkki: ``Virtual Sound Source Positioning Using Vector Base Amplitude Panning'' \emph{Journal of the Audio Engineering Society}
, 1997 June, Vol. 45/6, p. 456. 
\subsection*{See Also}


 \emph{vbap16}
, \emph{vbap16move}
, \emph{vbap4}
, \emph{vbap8}
, \emph{vbap8move}
, \emph{vbaplsinit}
, \emph{vbapz}
, \emph{vbapzmove}

\subsection*{Credits}


 


 


\begin{tabular}{cccccc}
Author: Ville Pulkki &Sibelius Academy Computer Music Studio &Laboratory of Acoustics and Audio Signal Processing &Helsinki University of Technology &Helsinki, Finland &May 2000

\end{tabular}



 


 New in Csound Version 4.07
%\hline 


\begin{comment}
\begin{tabular}{lcr}
Previous &Home &Next \\
vbap4 &Up &vbap8

\end{tabular}


\end{document}
\end{comment}
