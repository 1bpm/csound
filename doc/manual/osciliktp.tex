\begin{comment}
\documentclass[10pt]{article}
\usepackage{fullpage, graphicx, url}
\setlength{\parskip}{1ex}
\setlength{\parindent}{0ex}
\title{osciliktp}
\begin{document}


\begin{tabular}{ccc}
The Alternative Csound Reference Manual & & \\
Previous & &Next

\end{tabular}

%\hline 
\end{comment}
\section{osciliktp}
osciliktp�--� A linearly interpolated oscillator that allows allows phase modulation. \subsection*{Description}


 \emph{osciliktp}
 allows phase modulation (which is actually implemented as k-rate frequency modulation, by differentiating phase input). The disadvantage is that there is no amplitude control, and frequency can be varied only at the control-rate. This opcode can be faster or slower than \emph{oscilikt}
, depending on the control-rate. 
\subsection*{Syntax}


 ar \textbf{osciliktp}
 kcps, kfn, kphs [, istor]
\subsection*{Initialization}


 \emph{istor}
 (optional, defaults to 0) -- Skips initialization. 
\subsection*{Performance}


 \emph{ar}
 -- audio-rate ouptut signal. 


 \emph{kcps}
 -- frequency in Hz. Zero and negative values are allowed. However, the absolute value must be less than \emph{sr}
 (and recommended to be less than sr/2). 


 \emph{kfn}
 -- function table number. Can be varied at control rate (useful to ``morph'' waveforms, or select from a set of band-limited tables generated by \emph{GEN30}
). 


 \emph{kphs}
 -- phase (k-rate), the expected range is 0 to 1. The absolute value of the difference of the current and previous value of \emph{kphs}
 must be less than \emph{ksmps}
. 
\subsection*{Examples}


  Here is an example of the osciliktp opcode. It uses the files \emph{osciliktp.orc}
 and \emph{osciliktp.sco}
. 


 \textbf{Example 1. Example of the osciliktp opcode.}

\begin{lstlisting}
/* osciliktp.orc */
; Initialize the global variables.
sr = 44100
kr = 4410
ksmps = 10
nchnls = 1

; Instrument #1: osciliktp example
instr 1
  kphs line 0, p3, 4

  a1x osciliktp 220.5, 1, 0
  a1y osciliktp 220.5, 1, -kphs
  a1 =  a1x - a1y

  out a1 * 14000
endin
/* osciliktp.orc */
        
\end{lstlisting}
\begin{lstlisting}
/* osciliktp.sco */
; Table #1: Sawtooth wave
f 1 0 3 -2 1 0 -1

; Play Instrument #1 for four seconds.
i 1 0 4
e
/* osciliktp.sco */
        
\end{lstlisting}
\subsection*{See Also}


 \emph{oscilikt}
 and \emph{oscilikts}
. 
\subsection*{Credits}


 Author: Istvan Varga


 New in version 4.22
%\hline 


\begin{comment}
\begin{tabular}{lcr}
Previous &Home &Next \\
oscilikt &Up &oscilikts

\end{tabular}


\end{document}
\end{comment}
