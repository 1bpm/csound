\begin{comment}
\documentclass[10pt]{article}
\usepackage{fullpage, graphicx, url}
\setlength{\parskip}{1ex}
\setlength{\parindent}{0ex}
\title{pvsmaska}
\begin{document}


\begin{tabular}{ccc}
The Alternative Csound Reference Manual & & \\
Previous & &Next

\end{tabular}

%\hline 
\end{comment}
\section{pvsmaska}
pvsmaska�--� Modify amplitudes using a function table, with dynamic scaling. \subsection*{Description}


  Modify amplitudes of fsrc using function table, with dynamic scaling. 
\subsection*{Syntax}


 fsig \textbf{pvsmaska}
 fsrc, ifn, kdepth
\subsection*{Initialization}


 \emph{ifn}
 -- The f-table to use. Given fsrc has N analysis bins, table ifn must be of size N or larger. The table need not be normalized, but values should lie within the range 0 to 1. It can be supplied from the score in the usual way, or from within the orchestra by using \emph{pvsinfo}
 to find the size of fsrc, (returned by pvsinfo in inbins), which can then be passed to ftgen to create the f-table. 
\subsection*{Performance}


 \emph{kdepth}
 -- Controls the degree of modification applied to fsrc, using simple linear scaling. 0 leaves amplitudes unchanged, 1 applies the full profile of ifn. 


  Note that power-of-two FFT sizes are particularly convenient when using table-based processing, as the number of analysis bins (inbins) is then a power-of-two plus one, for which an exactly matching f-table can be created. In this case it is important that the f-table be created with a size of inbins, rather than as a power of two, as the latter will copy the first table value to the guard point, which is inappropriate for this opcode. 
\subsection*{Examples}


 


 \textbf{Example 1. Example (using score-supplied f-table, assuming fsig fftsize = 1024)}

\begin{lstlisting}
; score f-table using cubic spline to define shaped peaks
f1 0 513 8 0 2 1 3 0 4 1 6 0 10 1 12 0 16 1 32 0 1 0 436 0
 
asig  buzz     20000,199,50,3        ; pulsewave source
fsig  pvsanal  asig,1024,256,1024,0  ; create fsig
kmod  linseg   0,p3/2,1,p3/2,0       ; simple control sig
 
fsig  pvsmaska fsig,2,kmod           ; apply weird eq to fsig
aout  pvsynth  fsig                  ; resynthesize,
      dispfft  aout,0.1,1024         ; and view the effect
        
\end{lstlisting}
 This also illustrates that the usual Csound behaviour applies to fsigs; the same name can be used for both input and output. \subsection*{Credits}


 


 


\begin{tabular}{cc}
Author: Richard Dobson &August 2001 

\end{tabular}



 


 New in version 4.13
%\hline 


\begin{comment}
\begin{tabular}{lcr}
Previous &Home &Next \\
pvsinfo &Up &pvsynth

\end{tabular}


\end{document}
\end{comment}
