\begin{comment}
\documentclass[10pt]{article}
\usepackage{fullpage, graphicx, url}
\setlength{\parskip}{1ex}
\setlength{\parindent}{0ex}
\title{deltapx}
\begin{document}


\begin{tabular}{ccc}
The Alternative Csound Reference Manual & & \\
Previous & &Next

\end{tabular}

%\hline 
\end{comment}
\section{deltapx}
deltapx�--� Read to or write from a delay line with interpolation. \subsection*{Description}


 \emph{deltapx}
 is similar to \emph{deltapi}
 or \emph{deltap3}
. However, it allows higher quality interpolation. This opcode can read from and write to a delayr/delayw delay line with interpolation. 
\subsection*{Syntax}


 aout \textbf{deltapx}
 adel, iwsize
\subsection*{Initialization}


 \emph{iwsize}
 -- interpolation window size in samples. Allowed values are integer multiplies of 4 in the range 4 to 1024. iwsize = 4 uses cubic interpolation. Increasing iwsize improves sound quality at the expense of CPU usage, and minimum delay time. 
\subsection*{Performance}


 \emph{aout}
 -- Output signal 


 \emph{adel}
 -- Delay time in seconds. 


 


 
\begin{lstlisting}
a1      delayr idlr
        deltapxw a2, adl1, iws1
a3      deltapx adl2, iws2
        deltapxw a4, adl3, iws3
        delayw a5
        
\end{lstlisting}


 


  Minimum and maximum delay times: 


 
\begin{lstlisting}
idlr >= 1/kr                                Delay line length
 
adl1 >= (iws1/2)/sr                         Write before read
adl1 <= idlr - (1 + iws1/2)/sr              (allows shorter delays)
 
adl2 >= 1/kr + (iws2/2)/sr                  Read time
adl2 <= idlr - (1 + iws2/2)/sr
adl2 >= adl1 + (iws1 + iws2) / (2*sr)
adl2 >= 1/kr + adl3 + (iws2 + iws3) / (2*sr)
 
adl3 >= (iws3/2)/sr                         Write after read
adl3 <= idlr - (1 + iws3/2)/sr              (allows feedback)
        
\end{lstlisting}


 


 


\begin{tabular}{cc}
\textbf{Note}
 \\
� &

  Window sizes for opcodes other than deltapx are: deltap, deltapn: 1, deltapi: 2 (linear), deltap3: 4 (cubic) 


\end{tabular}

\subsection*{Examples}


 


 
\begin{lstlisting}
a1      phasor 300.0
a1      =  a1 - 0.5
a_      delayr 1.0
adel    phasor 4.0
adel    =  sin (2.0 * 3.14159265 * adel) * 0.01 + 0.2
        deltapxw a1, adel, 32
adel    phasor 2.0
adel    =  sin (2.0 * 3.14159265 * adel) * 0.01 + 0.2
        deltapxw a1, adel, 32
adel    =  0.3
a2      deltapx adel, 32
a1      =  0
        delayw a1
 
        out a2 * 20000.0
        
\end{lstlisting}


 
\subsection*{See Also}


 \emph{deltapxw}

\subsection*{Credits}


 


 


\begin{tabular}{cc}
Author: Istvan Varga &August 2001

\end{tabular}



 


 New in version 4.13
%\hline 


\begin{comment}
\begin{tabular}{lcr}
Previous &Home &Next \\
deltapn &Up &deltapxw

\end{tabular}


\end{document}
\end{comment}
