\begin{comment}
\documentclass[10pt]{article}
\usepackage{fullpage, graphicx, url}
\setlength{\parskip}{1ex}
\setlength{\parindent}{0ex}
\title{init}
\begin{document}


\begin{tabular}{ccc}
The Alternative Csound Reference Manual & & \\
Previous & &Next

\end{tabular}

%\hline 
\end{comment}
\section{init}
init�--� Puts the value of the i-time expression into a k- or a-rate variable. \subsection*{Syntax}


 ar \textbf{init}
 iarg


 ir \textbf{init}
 iarg


 kr \textbf{init}
 iarg
\subsection*{Description}


  Put the value of the i-time expression into a k- or a-rate variable. 
\subsection*{Initialization}


  Puts the value of the i-time expression \emph{iarg}
 into a k- or a-rate variable, i.e., initialize the result. Note that \emph{init}
 provides the only case of an init-time statement being permitted to write into a perf-time (k- or a-rate) result cell; the statement has no effect at perf-time. 
\subsection*{See Also}


 \emph{=}
, \emph{divz}
, \emph{tival}

%\hline 


\begin{comment}
\begin{tabular}{lcr}
Previous &Home &Next \\
inh &Up &initc14

\end{tabular}


\end{document}
\end{comment}
