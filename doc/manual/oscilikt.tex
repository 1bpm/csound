\begin{comment}
\documentclass[10pt]{article}
\usepackage{fullpage, graphicx, url}
\setlength{\parskip}{1ex}
\setlength{\parindent}{0ex}
\title{oscilikt}
\begin{document}


\begin{tabular}{ccc}
The Alternative Csound Reference Manual & & \\
Previous & &Next

\end{tabular}

%\hline 
\end{comment}
\section{oscilikt}
oscilikt�--� A linearly interpolated oscillator that allows changing the table number at k-rate. \subsection*{Description}


 \emph{oscilikt}
 is very similar to \emph{oscili}
, but allows changing the table number at k-rate. It is slightly slower than \emph{oscili}
 (especially with high control rate), although also more accurate as it uses a 31-bit phase accumulator, as opposed to the 24-bit one used by oscili. 
\subsection*{Syntax}


 ar \textbf{oscilikt}
 xamp, xcps, kfn [, iphs] [, istor]


 kr \textbf{oscilikt}
 kamp, kcps, kfn [, iphs] [, istor]
\subsection*{Initialization}


 \emph{iphs}
 (optional, defaults to 0) -- initial phase in the range 0 to 1. Other values are wrapped to the allowed range. 


 \emph{istor}
 (optional, defaults to 0) -- skip initialization. 
\subsection*{Performance}


 \emph{kamp}
, \emph{xamp}
 -- amplitude. 


 \emph{kcps}
, \emph{xcps}
 -- frequency in Hz. Zero and negative values are allowed. However, the absolute value must be less than \emph{sr}
 (and recommended to be less than sr/2). 


 \emph{kfn}
 -- function table number. Can be varied at control rate (useful to ``morph'' waveforms, or select from a set of band-limited tables generated by \emph{GEN30}
). 
\subsection*{Examples}


  Here is an example of the oscilikt opcode. It uses the files \emph{oscilikt.orc}
 and \emph{oscilikt.sco}
. 


 \textbf{Example 1. Example of the oscilikt opcode.}

\begin{lstlisting}
/* oscilikt.orc */
; Initialize the global variables.
sr = 44100
kr = 4410
ksmps = 10
nchnls = 1

; Instrument #1.
instr 1
  ; Generate a uni-polar (0-1) square wave.
  kamp1 init 1 
  kcps1 init 2
  itype = 3
  ksquare lfo kamp1, kcps1, itype

  ; Use the square wave to switch between Tables #1 and #2.
  kamp2 init 20000
  kcps2 init 220
  kfn = ksquare + 1

  a1 oscilikt kamp2, kcps2, kfn
  out a1
endin
/* oscilikt.orc */
        
\end{lstlisting}
\begin{lstlisting}
/* oscilikt.sco */
; Table #1, a sine waveform.
f 1 0 4096 10 0 1
; Table #2: a sawtooth wave
f 2 0 3 -2 1 0 -1

; Play Instrument #1 for two seconds.
i 1 0 2
/* oscilikt.sco */
        
\end{lstlisting}
\subsection*{See Also}


 \emph{osciliktp}
 and \emph{oscilikts}
. 
\subsection*{Credits}


 Author: Istvan Varga


 Example written by Kevin Conder.


 New in version 4.22
%\hline 


\begin{comment}
\begin{tabular}{lcr}
Previous &Home &Next \\
oscili &Up &osciliktp

\end{tabular}


\end{document}
\end{comment}
