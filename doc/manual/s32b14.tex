\begin{comment}
\documentclass[10pt]{article}
\usepackage{fullpage, graphicx, url}
\setlength{\parskip}{1ex}
\setlength{\parindent}{0ex}
\title{s32b14}
\begin{document}


\begin{tabular}{ccc}
The Alternative Csound Reference Manual & & \\
Previous & &Next

\end{tabular}

%\hline 
\end{comment}
\section{s32b14}
s32b14�--� Creates a bank of 32 different 14-bit MIDI control message numbers. \subsection*{Description}


  Creates a bank of 32 different 14-bit MIDI control message numbers. 
\subsection*{Syntax}


 i1,...,i32 \textbf{s32b14}
 ichan, ictlno\_msb1, ictlno\_lsb1, imin1, imax1, initvalue1, ifn1,..., ictlno\_msb32, ictlno\_lsb32, imin32, imax32, initvalue32, ifn32


 k1,...,k32 \textbf{s32b14}
 ichan, ictlno\_msb1, ictlno\_lsb1, imin1, imax1, initvalue1, ifn1,..., ictlno\_msb32, ictlno\_lsb32, imin32, imax32, initvalue32, ifn32
\subsection*{Initialization}


 \emph{i1 ... i64}
 -- output values 


 \emph{ichan}
 -- MIDI channel (1-16) 


 \emph{ictlno\_msb1 .... ictlno\_msb32}
 -- MIDI control number, most significant byte (0-127) 


 \emph{ictlno\_lsb1 .... ictlno\_lsb32}
 -- MIDI control number, least significant byte (0-127) 


 \emph{imin1 ... imin64}
 -- minimum values for each controller 


 \emph{imax1 ... imax64}
 -- maximum values for each controller 


 \emph{init1 ... init64}
 -- initial value for each controller 


 \emph{ifn1 ... ifn64}
 -- function table for conversion for each controller 


 \emph{icutoff1 ... icutoff64}
 -- low-pass filter cutoff frequency for each controller 
\subsection*{Performance}


 \emph{k1 ... k64}
 -- output values 


 \emph{s32b14}
 is a bank of MIDI controllers, useful when using MIDI mixer such as Kawai MM-16 or others for changing whatever sound parameter in real-time. The raw MIDI control messages at the input port are converted to agree with \emph{iminN}
 and \emph{imaxN}
, and an initial value can be set. Also, an optional non-interpolated function table with a custom translation curve is allowed, useful for enabling exponential response curves. 


  When no function table translation is required, set the \emph{ifnN}
 value to 0, else set \emph{ifnN}
 to a valid function table number. When table translation is enabled (i.e. setting \emph{ifnN}
 value to a non-zero number referring to an already allocated function table), \emph{initN}
 value should be set equal to \emph{iminN}
 or \emph{imaxN}
 value, else the initial output value will not be the same as specified in \emph{initN}
 argument. 


 \emph{s32b14}
 allows a bank of 32 different MIDI control message numbers. It uses 14-bit values instead of MIDI's normal 7-bit values. 


  As the input and output arguments are many, you can split the line using '$\backslash$' (backslash) character (new in 3.47 version) to improve the readability. Using these opcodes is considerably more efficient than using the separate ones (\emph{ctrl7}
 and \emph{tonek}
) when more controllers are required. 


  In the i-rate version of \emph{s32b14}
, there is not an initial value input argument. The output is taken directly from the current status of internal controller array of Csound. 
\subsection*{Credits}


 


 


\begin{tabular}{ccc}
Author: Gabriel Maldonado &Italy &December 1998

\end{tabular}



 


 New in Csound version 3.50


 Thanks goes to Rasmus Ekman for pointing out the correct MIDI channel and controller number ranges.
%\hline 


\begin{comment}
\begin{tabular}{lcr}
Previous &Home &Next \\
s16b14 &Up &samphold

\end{tabular}


\end{document}
\end{comment}
