\begin{comment}
\documentclass[10pt]{article}
\usepackage{fullpage, graphicx, url}
\setlength{\parskip}{1ex}
\setlength{\parindent}{0ex}
\title{foscil}
\begin{document}


\begin{tabular}{ccc}
The Alternative Csound Reference Manual & & \\
Previous & &Next

\end{tabular}

%\hline 
\end{comment}
\section{foscil}
foscil�--� A basic frequency modulated oscillator. \subsection*{Description}


  A basic frequency modulated oscillator. 
\subsection*{Syntax}


 ar \textbf{foscil}
 xamp, kcps, xcar, xmod, kndx, ifn [, iphs]
\subsection*{Initialization}


 \emph{ifn}
 -- function table number. Requires a wrap-around guard point. 


 \emph{iphs}
 (optional, default=0) -- initial phase of waveform in table \emph{ifn}
, expressed as a fraction of a cycle (0 to 1). A negative value will cause phase initialization to be skipped. The default value is 0. 
\subsection*{Performance}


 \emph{xamp}
 -- the amplitude of the output signal. 


 \emph{kcps}
 -- a common denominator, in cycles per second, for the carrier and modulating frequencies. 


 \emph{xcar}
 -- a factor that, when multiplied by the \emph{kcps}
 parameter, gives the carrier frequency. 


 \emph{xmod}
 -- a factor that, when multiplied by the \emph{kcps}
 parameter, gives the modulating frequency. 


 \emph{kndx}
 -- the modulation index. 


 \emph{foscil}
 is a composite unit that effectively banks two \emph{oscil}
 opcodes in the familiar Chowning FM setup, wherein the audio-rate output of one generator is used to modulate the frequency input of another (the ``carrier''). Effective carrier frequency = \emph{kcps}
 * \emph{xcar}
, and modulating frequency = \emph{kcps}
 * \emph{xmod}
. For integral values of \emph{xcar}
 and \emph{xmod}
, the perceived fundamental will be the minimum positive value of \emph{kcps}
 * (\emph{xcar}
 -- n * \emph{xmod}
), n = 1,1,2,... The input \emph{kndx}
 is the index of modulation (usually time-varying and ranging 0 to 4 or so) which determines the spread of acoustic energy over the partial positions given by n = 0,1,2,.., etc. \emph{ifn}
 should point to a stored sine wave. Previous to version 3.50, \emph{xcar}
 and \emph{xmod}
 could be k-rate only. 
\subsection*{Examples}


  Here is an example of the foscil opcode. It uses the files \emph{foscil.orc}
 and \emph{foscil.sco}
. 


 \textbf{Example 1. Example of the foscil opcode.}

\begin{lstlisting}
/* foscil.orc */
; Initialize the global variables.
sr = 44100
kr = 4410
ksmps = 10
nchnls = 1

; Instrument #1 - a basic FM waveform.
instr 1
  kamp = 10000
  kcps = 440
  kcar = 600
  kmod = 210
  kndx = 2
  ifn = 1

  a1 foscil kamp, kcps, kcar, kmod, kndx, ifn
  out a1
endin
/* foscil.orc */
        
\end{lstlisting}
\begin{lstlisting}
/* foscil.sco */
; Table #1, a sine wave.
f 1 0 16384 10 1

; Play Instrument #1 for 2 seconds.
i 1 0 2
e
/* foscil.sco */
        
\end{lstlisting}
\subsection*{Credits}


 Example written by Kevin Conder.
%\hline 


\begin{comment}
\begin{tabular}{lcr}
Previous &Home &Next \\
follow2 &Up &foscili

\end{tabular}


\end{document}
\end{comment}
