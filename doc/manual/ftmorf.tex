\begin{comment}
\documentclass[10pt]{article}
\usepackage{fullpage, graphicx, url}
\setlength{\parskip}{1ex}
\setlength{\parindent}{0ex}
\title{ftmorf}
\begin{document}


\begin{tabular}{ccc}
The Alternative Csound Reference Manual & & \\
Previous & &Next

\end{tabular}

%\hline 
\end{comment}
\section{ftmorf}
ftmorf�--� Morphs between multiple ftables as specified in a list. \subsection*{Description}


  Uses an index into a table of ftable numbers to morph between adjacent tables in the list.This morphed function is written into the table referenced by \emph{iresfn}
 on every k-cycle. 
\subsection*{Syntax}


 \textbf{ftmorf}
 kftndx, iftfn, iresfn
\subsection*{Initialization}


 \emph{iftfn}
 -- The ftable function. The list of values are expected to be pre-existing ftable numbers. 


 \emph{iresfn}
 -- Table number of the morphed function 


  The length of all the tables in \emph{iftfn}
 must equal the length of \emph{iresfn}
. 
\subsection*{Performance}


 \emph{kftndx}
 -- the index into the \emph{iftfn}
 table. 


  If \emph{iftfn}
 contains (6, 4, 6, 8, 7, 4): 


 
\begin{itemize}
\item 

 \emph{kftndx=4}
 will write the contents of f7 into \emph{iresfn}
. 

\item 

 \emph{kftndx=4.5}
 will write the average of the contents of f7 and f4 into \emph{iresfn}
. 


\end{itemize}
\subsection*{Examples}


  Here is an example of the ftmorf opcode. It uses the files \emph{ftmorf.orc}
 and \emph{ftmorf.sco}
. 


 \textbf{Example 1. Example of the ftmorf opcode.}

\begin{lstlisting}
/* ftmorf.orc */
sr = 44100
kr = 4410
ksmps = 10
nchnls = 1

instr 1
kndx    line    0, p3, 7
        ftmorf  kndx, 1, 2
asig    oscili  30000, 440, 2
        out     asig
endin
/* ftmorf.orc */
        
\end{lstlisting}
\begin{lstlisting}
/* ftmorf.sco */
f1 0 8 -2 3 4 5 6 7 8 9 10
f2 0 1024 10 1 /*contents of f2 dont matter */
f3 0 1024 10 1
f4 0 1024 10 0 1
f5 0 1024 10 0 0 1
f6 0 1024 10 0 0 0 1
f7 0 1024 10 0 0 0 0 1
f8 0 1024 10 0 0 0 0 0 1
f9 0 1024 10 0 0 0 0 0 0 1
f10 0 1024 10 1 1 1 1 1 1 1

i1 0 10
e
/* ftmorf.sco */
        
\end{lstlisting}
\subsection*{Credits}


 


 


\begin{tabular}{cccc}
Author: William ``Pete'' Moss &University of Texas at Austin &Austin, Texas USA &Jan. 2002

\end{tabular}



 


 New in version 4.18
%\hline 


\begin{comment}
\begin{tabular}{lcr}
Previous &Home &Next \\
ftlptim &Up &ftsave

\end{tabular}


\end{document}
\end{comment}
