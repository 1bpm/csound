\begin{comment}
\documentclass[10pt]{article}
\usepackage{fullpage, graphicx, url}
\setlength{\parskip}{1ex}
\setlength{\parindent}{0ex}
\title{resonr}
\begin{document}


\begin{tabular}{ccc}
The Alternative Csound Reference Manual & & \\
Previous & &Next

\end{tabular}

%\hline 
\end{comment}
\section{resonr}
resonr�--� A bandpass filter with variable frequency response. \subsection*{Description}


  Implementations of a second-order, two-pole two-zero bandpass filter with variable frequency response. 
\subsection*{Syntax}


 ar \textbf{resonr}
 asig, kcf, kbw [, iscl] [, iskip]
\subsection*{Initialization}


  The optional initialization variables for \emph{resonr}
 are identical to the i-time variables for \emph{reson}
. 


 \emph{iscl}
 (optional, default=0) -- coded scaling factor for resonators. A value of 1 signifies a peak response factor of 1, i.e. all frequencies other than \emph{kcf}
 are attenuated in accordance with the (normalized) response curve. A value of 2 raises the response factor so that its overall RMS value equals 1. This intended equalization of input and output power assumes all frequencies are physically present; hence it is most applicable to white noise. A zero value signifies no scaling of the signal, leaving that to some later adjustment (see \emph{balance}
). The default value is 0. 


 \emph{iskip}
 (optional, default=0) -- initial disposition of internal data space. Since filtering incorporates a feedback loop of previous output, the initial status of the storage space used is significant. A zero value will clear the space; a non-zero value will allow previous information to remain. The default value is 0. 
\subsection*{Performance}


 \emph{asig}
 -- input signal to be filtered 


 \emph{kcf}
 -- cutoff or resonant frequency of the filter, measured in Hz 


 \emph{kbw}
 -- bandwidth of the filter (the Hz difference between the upper and lower half-power points) 


 \emph{resonr}
 and \emph{resonz}
 are variations of the classic two-pole bandpass resonator (\emph{reson}
). Both filters have two zeroes in their transfer functions, in addition to the two poles. \emph{resonz}
 has its zeroes located at z = 1 and z = -1. \emph{resonr}
 has its zeroes located at +sqrt(\emph{R}
) and -sqrt(\emph{R}
), where \emph{R}
 is the radius of the poles in the complex z-plane. The addition of zeroes to \emph{resonr}
 and \emph{resonz}
 results in the improved selectivity of the magnitude response of these filters at cutoff frequencies close to 0, at the expense of less selectivity of frequencies above the cutoff peak. 


 \emph{resonr}
 and \emph{resonz}
 are very close to constant-gain as the center frequency is swept, resulting in a more efficient control of the magnitude response than with traditional two-pole resonators such as \emph{reson}
. 


 \emph{resonr}
 and \emph{resonz}
 produce a sound that is considerably different from \emph{reson}
, especially for lower center frequencies; trial and error is the best way of determining which resonator is best suited for a particular application. 
\subsection*{Examples}


  Here is an example of the resonr and resonz opcodes. It uses the files \emph{resonr.orc}
 and \emph{resonr.sco}
. 


 \textbf{Example 1. Example of the resonr and resonz opcodes.}

\begin{lstlisting}
/* resonr.orc */
/* Written by Sean Costello */
 ; Orchestra file for resonant filter sweep of a sawtooth-like waveform. 
  ; The outputs of reson, resonr, and resonz are scaled by coefficients
  ; specified in the score, so that each filter can be heard on its own
  ; from the same instrument.

  sr = 44100
  kr = 4410
  ksmps = 10
  nchnls = 1
  
instr 1
  
  idur     =        p3
  ibegfreq =        p4                              ; beginning of sweep frequency
  iendfreq =        p5                              ; ending of sweep frequency
  ibw      =        p6                              ; bandwidth of filters in Hz
  ifreq    =        p7                              ; frequency of gbuzz that is to be filtered
  iamp     =        p8                              ; amplitude to scale output by
  ires     =        p9                              ; coefficient to scale amount of reson in output
  iresr    =        p10                             ; coefficient to scale amount of resonr in output
  iresz    =        p11                             ; coefficient to scale amount of resonz in output
  
 ; Frequency envelope for reson cutoff
  kfreq    linseg ibegfreq, idur * .5, iendfreq, idur * .5, ibegfreq
  
 ; Amplitude envelope to prevent clicking
  kenv     linseg 0, .1, iamp, idur - .2, iamp, .1, 0
  
 ; Number of harmonics for gbuzz scaled to avoid aliasing
  iharms   =        (sr*.4)/ifreq
  
  asig     gbuzz 1, ifreq, iharms, 1, .9, 1      ; "Sawtooth" waveform
  ain      =        kenv * asig                     ; output scaled by amp envelope
  ares     reson ain, kfreq, ibw, 1
  aresr    resonr ain, kfreq, ibw, 1
  aresz    resonz ain, kfreq, ibw, 1
  
           out ares * ires + aresr * iresr + aresz * iresz
  
endin
/* resonr.orc */
        
\end{lstlisting}
\begin{lstlisting}
/* resonr.sco */
/* Written by Sean Costello */
f1 0 8192 9 1 1 .25                               ; cosine table for gbuzz generator
  
i1  0 10 1 3000 200 100 4000 1 0 0                ; reson  output with bw = 200
i1 10 10 1 3000 200 100 4000 0 1 0                ; resonr output with bw = 200
i1 20 10 1 3000 200 100 4000 0 0 1                ; resonz output with bw = 200
i1 30 10 1 3000  50 200 8000 1 0 0                ; reson  output with bw = 50
i1 40 10 1 3000  50 200 8000 0 1 0                ; resonr output with bw = 50
i1 50 10 1 3000  50 200 8000 0 0 1                ; resonz output with bw = 50
e
/* resonr.sco */
        
\end{lstlisting}
\subsection*{Technical History}


 \emph{resonr}
 and \emph{resonz}
 were originally described in an article by Julius O. Smith and James B. Angell.1 Smith and Angell recommended the \emph{resonz}
 form (zeros at +1 and -1) when computational efficiency was the main concern, as it has one less multiply per sample, while \emph{resonr}
 (zeroes at + and - the square root of the pole radius R) was recommended for situations when a perfectly constant-gain center peak was required. 


  Ken Steiglitz, in a later article 2, demonstrated that \emph{resonz}
 had constant gain at the true peak of the filter, as opposed to \emph{resonr}
, which displayed constant gain at the pole angle. Steiglitz also recommended \emph{resonz}
 for its sharper notches in the gain curve at zero and Nyquist frequency. Steiglitz's recent book 3 features a thorough technical discussion of \emph{reson}
 and \emph{resonz}
, while Dodge and Jerse's textbook 4 illustrates the differences in the response curves of \emph{reson}
 and \emph{resonz}
. 
\subsection*{References}


 


 
\begin{enumerate}
\item 

  Smith, Julius O. and Angell, James B., ``A Constant-Gain Resonator Tuned by a Single Coefficient,'' \emph{Computer Music Journal}
, vol. 6, no. 4, pp. 36-39, Winter 1982. 

\item 

  Steiglitz, Ken, ``A Note on Constant-Gain Digital Resonators,'' \emph{Computer Music Journal}
, vol. 18, no. 4, pp. 8-10, Winter 1994. 

\item 

  Ken Steiglitz, \emph{A Digital Signal Processing Primer, with Applications to Digital Audio and Computer Music}
. Addison-Wesley Publishing Company, Menlo Park, CA, 1996. 

\item 

  Dodge, Charles and Jerse, Thomas A., \emph{Computer Music: Synthesis, Composition, and Performance}
. New York: Schirmer Books, 1997, 2nd edition, pp. 211-214. 


\end{enumerate}
\subsection*{See Also}


 \emph{resonz}

\subsection*{Credits}


 


 


\begin{tabular}{ccc}
Author: Sean Costello &Seattle, Washington &1999

\end{tabular}



 


 New in Csound version 3.55
%\hline 


\begin{comment}
\begin{tabular}{lcr}
Previous &Home &Next \\
resonk &Up &resonx

\end{tabular}


\end{document}
\end{comment}
