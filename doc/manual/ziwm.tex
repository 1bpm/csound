\begin{comment}
\documentclass[10pt]{article}
\usepackage{fullpage, graphicx, url}
\setlength{\parskip}{1ex}
\setlength{\parindent}{0ex}
\title{ziwm}
\begin{document}


\begin{tabular}{ccc}
The Alternative Csound Reference Manual & & \\
Previous & &Next

\end{tabular}

%\hline 
\end{comment}
\section{ziwm}
ziwm�--� Writes to a zk variable to an i-rate variable with mixing. \subsection*{Description}


  Writes to a zk variable to an i-rate variable with mixing. 
\subsection*{Syntax}


 \textbf{ziwm}
 isig, indx [, imix]
\subsection*{Initialization}


 \emph{isig}
 -- initializes the value of the zk location. 


 \emph{indx}
 -- points to the zk location location to which to write. 


 \emph{imix}
 (optional, default=1) -- indicates if mixing should occur. 
\subsection*{Performance}


 \emph{ziwm}
 is a mixing opcode, it adds the signal to the current value of the variable. If no \emph{imix}
 is specified, mixing always occurs. \emph{imix}
 = 0 will cause overwriting like \emph{ziw}
, \emph{zkw}
, and \emph{zaw}
. Any other value will cause mixing. 


 \emph{Caution}
: When using the mixing opcodes \emph{ziwm}
, \emph{zkwm}
, and \emph{zawm}
, care must be taken that the variables mixed to, are zeroed at the end (or start) of each k- or a-cycle. Continuing to add signals to them, can cause their values can drift to astronomical figures. 


  One approach would be to establish certain ranges of zk or za variables to be used for mixing, then use \emph{zkcl}
 or \emph{zacl}
 to clear those ranges. 
\subsection*{Examples}


  Here is an example of the ziwm opcode. It uses the files \emph{ziwm.orc}
 and \emph{ziwm.sco}
. 


 \textbf{Example 1. Example of the ziwm opcode.}

\begin{lstlisting}
/* ziwm.orc */
; Initialize the global variables.
sr = 44100
kr = 4410
ksmps = 10
nchnls = 1

; Initialize the ZAK space.
; Create 1 a-rate variable and 1 k-rate variable.
zakinit 1, 1

; Instrument #1 -- a simple instrument.
instr 1
  ; Add 20.5 to zk variable #1.
  ziwm 20.5, 1
endin

; Instrument #2 -- another simple instrument.
instr 2
  ; Add 15.25 to zk variable #1.
  ziwm 15.25, 1
endin

; Instrument #3 -- prints out zk variable #1.
instr 3
  ; Read zk variable #1 at i-rate.
  i1 zir 1

  ; Print out the value of zk variable #1.
  ; It should be 35.75 (20.5 + 15.25)
  print i1
endin
/* ziwm.orc */
        
\end{lstlisting}
\begin{lstlisting}
/* ziwm.sco */
; Play Instrument #1 for one second.
i 1 0 1
; Play Instrument #2 for one second.
i 2 0 1
; Play Instrument #3 for one second.
i 3 0 1
e
/* ziwm.sco */
        
\end{lstlisting}
\subsection*{See Also}


 \emph{zaw}
, \emph{zawm}
, \emph{ziw}
, \emph{zkw}
, \emph{zkwm}

\subsection*{Credits}


 


 


\begin{tabular}{ccc}
Author: Robin Whittle &Australia &May 1997

\end{tabular}



 


 Example written by Kevin Conder.
%\hline 


\begin{comment}
\begin{tabular}{lcr}
Previous &Home &Next \\
ziw &Up &zkcl

\end{tabular}


\end{document}
\end{comment}
