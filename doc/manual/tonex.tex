\begin{comment}
\documentclass[10pt]{article}
\usepackage{fullpage, graphicx, url}
\setlength{\parskip}{1ex}
\setlength{\parindent}{0ex}
\title{tonex}
\begin{document}


\begin{tabular}{ccc}
The Alternative Csound Reference Manual & & \\
Previous & &Next

\end{tabular}

%\hline 
\end{comment}
\section{tonex}
tonex�--� Emulates a stack of filters using the tone opcode. \subsection*{Description}


 \emph{tonex}
 is equivalent to a filter consisting of more layers of \emph{tone}
 with the same arguments, serially connected. Using a stack of a larger number of filters allows a sharper cutoff. They are faster than using a larger number instances in a Csound orchestra of the old opcodes, because only one initialization and k- cycle are needed at time and the audio loop falls entirely inside the cache memory of processor. 
\subsection*{Syntax}


 ar \textbf{tonex}
 asig, khp [, inumlayer] [, iskip]
\subsection*{Initialization}


 \emph{inumlayer}
 (optional) -- number of elements in the filter stack. Default value is 4. 


 \emph{iskip}
 (optional, default=0) -- initial disposition of internal data space. Since filtering incorporates a feedback loop of previous output, the initial status of the storage space used is significant. A zero value will clear the space; a non-zero value will allow previous information to remain. The default value is 0. 
\subsection*{Performance}


 \emph{asig}
 -- input signal 


 \emph{khp}
 -- the response curve's half-power point. Half power is defined as peak power / root 2. 
\subsection*{See Also}


 \emph{atonex}
, \emph{resonx}

\subsection*{Credits}


 


 


\begin{tabular}{cc}
Author: Gabriel Maldonado (adapted by John ffitch) &Italy

\end{tabular}



 


 New in Csound version 3.49
%\hline 


\begin{comment}
\begin{tabular}{lcr}
Previous &Home &Next \\
tonek &Up &transeg

\end{tabular}


\end{document}
\end{comment}
