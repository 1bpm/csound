\begin{comment}
\documentclass[10pt]{article}
\usepackage{fullpage, graphicx, url}
\setlength{\parskip}{1ex}
\setlength{\parindent}{0ex}
\title{zawm}
\begin{document}


\begin{tabular}{ccc}
The Alternative Csound Reference Manual & & \\
Previous & &Next

\end{tabular}

%\hline 
\end{comment}
\section{zawm}
zawm�--� Writes to a za variable at a-rate with mixing. \subsection*{Description}


  Writes to a za variable at a-rate with mixing. 
\subsection*{Syntax}


 \textbf{zawm}
 asig, kndx [, imix]
\subsection*{Initialization}


 \emph{imix}
 (optional, default=1) -- indicates if mixing should occur. 
\subsection*{Performance}


 \emph{asig}
 -- value to be written to the za location. 


 \emph{kndx}
 -- points to the zk or za location to which to write. 


  These opcodes are fast, and always check that the index is within the range of zk or za space. If not, an error is reported, 0 is returned, and no writing takes place. 


 \emph{zawm}
 is a mixing opcode, it adds the signal to the current value of the variable. If no \emph{imix}
 is specified, mixing always occurs. \emph{imix}
 = 0 will cause overwriting like \emph{ziw}
, \emph{zkw}
, and \emph{zaw}
. Any other value will cause mixing. 


 \emph{Caution}
: When using the mixing opcodes \emph{ziwm}
, \emph{zkwm}
, and \emph{zawm}
, care must be taken that the variables mixed to, are zeroed at the end (or start) of each k- or a-cycle. Continuing to add signals to them, can cause their values can drift to astronomical figures. 


  One approach would be to establish certain ranges of zk or za variables to be used for mixing, then use \emph{zkcl}
 or \emph{zacl}
 to clear those ranges. 
\subsection*{Examples}


  Here is an example of the zawm opcode. It uses the files \emph{zawm.orc}
 and \emph{zawm.sco}
. 


 \textbf{Example 1. Example of the zawm opcode.}

\begin{lstlisting}
/* zawm.orc */
; Initialize the global variables.
sr = 44100
kr = 4410
ksmps = 10
nchnls = 1

; Initialize the ZAK space.
; Create 1 a-rate variable and 1 k-rate variable.
zakinit 1, 1

; Instrument #1 -- a basic instrument.
instr 1
  ; Generate a simple sine waveform.
  asin oscil 15000, 440, 1

  ; Mix the sine waveform with za variable #1.
  zawm asin, 1
endin

; Instrument #2 -- another basic instrument.
instr 2
  ; Generate another waveform with a different frequency.
  asin oscil 15000, 880, 1

  ; Mix this sine waveform with za variable #1.
  zawm asin, 1
endin

; Instrument #3 -- generates audio output.
instr 3
  ; Read za variable #1, containing both waveforms.
  a1 zar 1

  ; Generate the audio output.
  out a1

  ; Clear the za variables, get them ready for 
  ; another pass.
  zacl 0, 1
endin
/* zawm.orc */
        
\end{lstlisting}
\begin{lstlisting}
/* zawm.sco */
; Table #1, a sine wave.
f 1 0 16384 10 1

; Play Instrument #1 for one second.
i 1 0 1
; Play Instrument #2 for one second.
i 2 0 1
; Play Instrument #3 for one second.
i 3 0 1
e
/* zawm.sco */
        
\end{lstlisting}
\subsection*{See Also}


 \emph{zaw}
, \emph{ziw}
, \emph{ziwm}
, \emph{zkw}
, \emph{zkwm}

\subsection*{Credits}


 


 


\begin{tabular}{ccc}
Author: Robin Whittle &Australia &May 1997

\end{tabular}



 


 Example written by Kevin Conder.
%\hline 


\begin{comment}
\begin{tabular}{lcr}
Previous &Home &Next \\
zaw &Up &zfilter2

\end{tabular}


\end{document}
\end{comment}
