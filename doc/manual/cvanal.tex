\begin{comment}
\documentclass[10pt]{article}
\usepackage{fullpage, graphicx, url}
\setlength{\parskip}{1ex}
\setlength{\parindent}{0ex}
\title{cvanal}
\begin{document}


\begin{tabular}{ccc}
The Alternative Csound Reference Manual & & \\
Previous & &Next

\end{tabular}

%\hline 
\end{comment}
\section{cvanal}
cvanal�--� Converts a soundfile into a single Fourier transform frame. \subsection*{Description}


  Impulse Response Fourier Analysis for \emph{convolve}
 operator 
\subsection*{Syntax}


 \textbf{CSound -U cvanal}
 [flags] infilename outfilename
\subsection*{Initialization}


 \emph{cvanal}
 -- converts a soundfile into a single Fourier transform frame. The output file can be used by the \emph{convolve}
 operator to perform Fast Convolution between an input signal and the original impulse response. Analysis is conditioned by the flags below. A space is optional between the flag and its argument. 


 \emph{-s rate}
 -- sampling rate of the audio input file. This will over-ride the srate of the soundfile header, which otherwise applies. If neither is present, the default is 10000. 


 \emph{-c channel}
 -- channel number sought. If omitted, the default is to process all channels. If a value is given, only the selected channel will be processed. 


 \emph{-b begin}
 -- beginning time (in seconds) of the audio segment to be analyzed. The default is 0.0 


 \emph{-d duration}
 -- duration (in seconds) of the audio segment to be analyzed. The default of 0.0 means to the end of the file. 
\subsection*{Examples}


 


 
\begin{lstlisting}
\emph{cvanal}
 asound cvfile
        
\end{lstlisting}


 
 will analyze the soundfile ``asound'' to produce the file ``cvfile'' for the use with \emph{convolve}
. 

  To use data that is not already contained in a soundfile, a soundfile converter that accepts text files may be used to create a standard audio file, e.g., the .DAT format for SOX. This is useful for implementing FIR filters. 
\subsubsection*{Files}


  The output file has a special \emph{convolve}
 header, containing details of the source audio file. The analysis data is stored as ``float'', in rectangular (real/imaginary) form. 


 


\begin{tabular}{cc}
\textbf{Note}
 \\
� &

  The analysis file is \emph{not}
 system independent! Ensure that the original impulse recording/data is retained. If/when required, the analysis file can be recreated. 


\end{tabular}

\subsection*{Credits}


 Author: Greg Sullivan


 Based on algorithm given in \emph{Elements Of Computer Music}
, by F. Richard Moore.
%\hline 


\begin{comment}
\begin{tabular}{lcr}
Previous &Home &Next \\
pvanal &Up &File Queries

\end{tabular}


\end{document}
\end{comment}
