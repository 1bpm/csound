\begin{comment}
\documentclass[10pt]{article}
\usepackage{fullpage, graphicx, url}
\setlength{\parskip}{1ex}
\setlength{\parindent}{0ex}
\title{zir}
\begin{document}


\begin{tabular}{ccc}
The Alternative Csound Reference Manual & & \\
Previous & &Next

\end{tabular}

%\hline 
\end{comment}
\section{zir}
zir�--� Reads from a location in zk space at i-rate. \subsection*{Description}


  Reads from a location in zk space at i-rate. 
\subsection*{Syntax}


 ir \textbf{zir}
 indx
\subsection*{Initialization}


 \emph{indx}
 -- points to the zk location to be read. 
\subsection*{Performance}


 \emph{zir}
 reads the signal at \emph{indx}
 location in zk space. 
\subsection*{Examples}


  Here is an example of the zir opcode. It uses the files \emph{zir.orc}
 and \emph{zir.sco}
. 


 \textbf{Example 1. Example of the zir opcode.}

\begin{lstlisting}
/* zir.orc */
; Initialize the global variables.
sr = 44100
kr = 4410
ksmps = 10
nchnls = 1

; Initialize the ZAK space.
; Create 1 a-rate variable and 1 k-rate variable.
zakinit 1, 1

; Instrument #1 -- a simple instrument.
instr 1
  ; Set the zk variable #1 to 32.594.
  ziw 32.594, 1
endin

; Instrument #2 -- prints out zk variable #1.
instr 2
  ; Read the zk variable #1 at i-rate.
  i1 zir 1

  ; Print out the value of zk variable #1.
  print i1
endin
/* zir.orc */
        
\end{lstlisting}
\begin{lstlisting}
/* zir.sco */
; Play Instrument #1 for one second.
i 1 0 1
; Play Instrument #2 for one second.
i 2 0 1
e
/* zir.sco */
        
\end{lstlisting}
\subsection*{See Also}


 \emph{zar}
, \emph{zarg}
, \emph{zkr}

\subsection*{Credits}


 


 


\begin{tabular}{ccc}
Author: Robin Whittle &Australia &May 1997

\end{tabular}



 


 Example written by Kevin Conder.
%\hline 


\begin{comment}
\begin{tabular}{lcr}
Previous &Home &Next \\
zfilter2 &Up &ziw

\end{tabular}


\end{document}
\end{comment}
