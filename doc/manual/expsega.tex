\begin{comment}
\documentclass[10pt]{article}
\usepackage{fullpage, graphicx, url}
\setlength{\parskip}{1ex}
\setlength{\parindent}{0ex}
\title{expsega}
\begin{document}


\begin{tabular}{ccc}
The Alternative Csound Reference Manual & & \\
Previous & &Next

\end{tabular}

%\hline 
\end{comment}
\section{expsega}
expsega�--� An exponential segment generator operating at a-rate. \subsection*{Description}


  An exponential segment generator operating at a-rate. This unit is almost identical to \emph{expseg}
, but more precise when defining segments with very short durations (i.e., in a percussive attack phase) at audio rate. 
\subsection*{Syntax}


 ar \textbf{expsega}
 ia, idur1, ib [, idur2] [, ic] [...]
\subsection*{Initialization}


 \emph{ia}
 -- starting value. Zero is illegal. 


 \emph{ib}
, \emph{ic}
, etc. -- value after \emph{idur1}
 seconds, etc. must be non-zero and must agree in sign with \emph{ia}
. 


 \emph{idur1}
 -- duration in seconds of first segment. A zero or negative value will cause all initialization to be skipped. 


 \emph{idur2}
, \emph{idur3}
, etc. -- duration in seconds of subsequent segments. A zero or negative value will terminate the initialization process with the preceding point, permitting the last defined line or curve to be continued indefinitely in performance. The default is zero. 
\subsection*{Performance}


  These units generate control or audio signals whose values can pass through two or more specified points. The sum of \emph{dur}
 values may or may not equal the instrument's performance time. A shorter performance will truncate the specified pattern, while a longer one will cause the last defined segment to continue on in the same direction. 
\subsection*{Examples}


  Here is an example of the expsega opcode. It uses the files \emph{expsega.orc}
 and \emph{expsega.sco}
. 


 \textbf{Example 1. Example of the expsega opcode.}

\begin{lstlisting}
/* expsega.orc */
; Initialize the global variables.
sr = 44100
kr = 4410
ksmps = 10
nchnls = 1

; Instrument #1.
instr 1
  ; Define a short percussive amplitude envelope that
  ; goes from 0.01 to 20,000 and back.
  aenv expsega 0.01, 0.1, 20000, 0.1, 0.01

  a1 oscil aenv, 440, 1
  out a1
endin
/* expsega.orc */
        
\end{lstlisting}
\begin{lstlisting}
/* expsega.sco */
; Table #1, a sine wave.
f 1 0 16384 10 1

; Play Instrument #1 for one second.
i 1 0 1
; Play Instrument #1 for one second.
i 1 1 1
; Play Instrument #1 for one second.
i 1 2 1
; Play Instrument #1 for one second.
i 1 3 1
e
/* expsega.sco */
        
\end{lstlisting}
\subsection*{See Also}


 \emph{expseg}
, \emph{expsegr}

\subsection*{Credits}


 Author: Gabriel Maldonado


 Example written by Kevin Conder.


 New in Csound 3.57
%\hline 


\begin{comment}
\begin{tabular}{lcr}
Previous &Home &Next \\
expseg &Up &expsegr

\end{tabular}


\end{document}
\end{comment}
