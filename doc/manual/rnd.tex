\begin{comment}
\documentclass[10pt]{article}
\usepackage{fullpage, graphicx, url}
\setlength{\parskip}{1ex}
\setlength{\parindent}{0ex}
\title{rnd}
\begin{document}


\begin{tabular}{ccc}
The Alternative Csound Reference Manual & & \\
Previous & &Next

\end{tabular}

%\hline 
\end{comment}
\section{rnd}
rnd�--� Returns a random number in a unipolar range. \subsection*{Description}


  Returns a random number in a unipolar range. 
\subsection*{Syntax}


 \textbf{rnd}
(x) (init- or control-rate only)


  Where the argument within the parentheses may be an expression. These value converters sample a global random sequence, but do not reference \emph{seed}
. The result can be a term in a further expression. 
\subsection*{Performance}


  Returns a random number in the unipolar range 0 to \emph{x}
. 
\subsection*{Examples}


  Here is an example of the rnd opcode. It uses the files \emph{rnd.orc}
 and \emph{rnd.sco}
. 


 \textbf{Example 1. Example of the rnd opcode.}

\begin{lstlisting}
/* rnd.orc */
; Initialize the global variables.
sr = 44100
kr = 4410
ksmps = 10
nchnls = 1

; Instrument #1.
instr 1
  ; Generate a random number from 0 to 1.
  i1 = rnd(1)
  print i1
endin
/* rnd.orc */
        
\end{lstlisting}
\begin{lstlisting}
/* rnd.sco */
; Play Instrument #1 for one second.
i 1 0 1
; Play Instrument #1 for one second.
i 1 1 1
e
/* rnd.sco */
        
\end{lstlisting}
 Its output should include lines like this: \begin{lstlisting}
instr 1:  i1 = 0.974
instr 1:  i1 = 0.139
      
\end{lstlisting}
\subsection*{See Also}


 \emph{birnd}

\subsection*{Credits}


 


 


\begin{tabular}{cccc}
Author: Barry L. Vercoe &MIT &Cambridge, Massachussetts &1997

\end{tabular}



 


 Example written by Kevin Conder.
%\hline 


\begin{comment}
\begin{tabular}{lcr}
Previous &Home &Next \\
rms &Up &rnd31

\end{tabular}


\end{document}
\end{comment}
