\begin{comment}
\documentclass[10pt]{article}
\usepackage{fullpage, graphicx, url}
\setlength{\parskip}{1ex}
\setlength{\parindent}{0ex}
\title{FLsavesnap}
\begin{document}


\begin{tabular}{ccc}
The Alternative Csound Reference Manual & & \\
Previous & &Next

\end{tabular}

%\hline 
\end{comment}
\section{FLsavesnap}
FLsavesnap�--� Saves all snapshots currently created into a file. \subsection*{Description}


 \emph{FLsavesnap}
 saves all snapshots currently created (i.e. the entire memory bank) into a file. 
\subsection*{Syntax}


 \textbf{FLsavesnap}
 ``filename''
\subsection*{Initialization}


 \emph{``filename''}
 -- a double-quoted string corresponding to a file to store a bank of snapshots. 
\subsection*{Performance}


 \emph{FLsavesnap}
 saves all snapshots currently created (i.e. the entire memory bank) into a file whose name is \emph{filename}
. Since the file is a text file, snapshot values can also be edited manually by means of a text editor. The format of the data stored in the file is the following (at present time, this could be changed in next Csound version): 


 
\begin{lstlisting}
----------- 0 -----------
FLvalue 0 0 1 0 ""
FLvalue 0 0 1 0 ""
FLvalue 0 0 1 0 ""
FLslider 331.946 80 5000 -1 "frequency of the first oscillator"
FLslider 385.923 80 5000 -1 "frequency of the second oscillator"
FLslider 80 80 5000 -1 "frequency of the third oscillator"
FLcount 0 0 10 0 "this index must point to the location number where snapshot is stored"
FLbutton 0 0 1 0 "Store snapshot to current index"
FLbutton 0 0 1 0 "Save snapshot bank to disk"
FLbutton 0 0 1 0 "Load snapshot bank from disk"
FLbox 0 0 1 0 ""
----------- 1 -----------
FLvalue 0 0 1 0 ""
FLvalue 0 0 1 0 ""
FLvalue 0 0 1 0 ""
FLslider 819.72 80 5000 -1 "frequency of the first oscillator"
FLslider 385.923 80 5000 -1 "frequency of the second oscillator"
FLslider 80 80 5000 -1 "frequency of the third oscillator"
FLcount 1 0 10 0 "this index must point to the location number where snapshot is stored"
FLbutton 0 0 1 0 "Store snapshot to current index"
FLbutton 0 0 1 0 "Save snapshot bank to disk"
FLbutton 0 0 1 0 "Load snapshot bank from disk"
FLbox 0 0 1 0 ""
----------- 2 -----------
..... etc...
----------- 3 -----------
..... etc...
---------------------------
        
\end{lstlisting}


 


  As you can see, each snapshot contain several lines. Each snapshot is separated from previous and next snapshot by a line of this kind: 


 ``-----------�snapshot�Num�-----------''\\ 
 ������
 Then there are several lines containing data. Each of these lines corresponds to a widget. 

  The first field of each line is an unquoted string containing opcode name corresponding to that widget. Second field is a number that expresses current value of a snapshot. In current version, this is the only field that can be modified manually. The third and fourth fields shows minimum and maximum values allowed for that valuator. The fifth field is a special number that indicates if the valuator is linear (value 0), exponential (value -1), or is indexed by a table interpolating values (negative table numbers) or non-interpolating (positive table numbers). The last field is a quoted string with the label of the widget. Last line of the file is always 


 ``---------------------------''\\ 
 ����
. \subsection*{See Also}


 \emph{FLgetsnap}
, \emph{FLloadsnap}
, \emph{FLrun}
, \emph{FLsetsnap}
, \emph{FLupdate}

\subsection*{Credits}


 Author: Gabriel Maldonado


 New in version 4.22
%\hline 


\begin{comment}
\begin{tabular}{lcr}
Previous &Home &Next \\
FLrun &Up &FLscroll

\end{tabular}


\end{document}
\end{comment}
