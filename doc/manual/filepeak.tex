\begin{comment}
\documentclass[10pt]{article}
\usepackage{fullpage, graphicx, url}
\setlength{\parskip}{1ex}
\setlength{\parindent}{0ex}
\title{filepeak}
\begin{document}


\begin{tabular}{ccc}
The Alternative Csound Reference Manual & & \\
Previous & &Next

\end{tabular}

%\hline 
\end{comment}
\section{filepeak}
filepeak�--� Returns the peak absolute value of a sound file. \subsection*{Description}


  Returns the peak absolute value of a sound file. 
\subsection*{Syntax}


 ir \textbf{filepeak}
 ifilcod [, ichnl]
\subsection*{Initialization}


 \emph{ifilcod}
 -- sound file to be queried 


 \emph{ichnl}
 (optional, default=0) -- channel to be used in calculating the peak value. Default is 0. 


 
\begin{itemize}
\item 

 \emph{ichnl}
 = 0 returns peak value of all channels

\item 

 \emph{ichnl}
 $>$ 0 returns peak value of \emph{ichnl}



\end{itemize}
\subsection*{Performance}


 \emph{filepeak}
 returns the peak absolute value of the sound file \emph{ifilcod}
. Currently, \emph{filepeak}
 supports only AIFF-C float files. 
\subsection*{Examples}


  Here is an example of the filepeak opcode. It uses the files \emph{filepeak.orc}
, \emph{filepeak.sco}
, and \emph{mary.wav}
. 


 \textbf{Example 1. Example of the filepeak opcode.}

\begin{lstlisting}
/* filepeak.orc */
; Initialize the global variables.
sr = 44100
kr = 4410
ksmps = 10
nchnls = 1

; Instrument #1.
instr 1
  ; Print out the peak absolute value of the
  ; audio file "mary.wav".
  ipeak filepeak "mary.wav"
  print ipeak
endin
/* filepeak.orc */
        
\end{lstlisting}
\begin{lstlisting}
/* filepeak.sco */
; Play Instrument #1 for 1 second.
i 1 0 1
e
/* filepeak.sco */
        
\end{lstlisting}
 The peak absolute value of the audio file ``mary.wav'' is 0.306902. So \emph{filepeak}
's output should include a line like this: \begin{lstlisting}
instr 1:  ipeak = 0.307
      
\end{lstlisting}
\subsection*{See Also}


 \emph{filelen}
, \emph{filenchnls}
, \emph{filesr}

\subsection*{Credits}


 


 


\begin{tabular}{cc}
Author: Matt Ingalls &July 1999

\end{tabular}



 


 Example written by Kevin Conder.


 New in Csound version 3.57
%\hline 


\begin{comment}
\begin{tabular}{lcr}
Previous &Home &Next \\
filenchnls &Up &filesr

\end{tabular}


\end{document}
\end{comment}
