\begin{comment}
\documentclass[10pt]{article}
\usepackage{fullpage, graphicx, url}
\setlength{\parskip}{1ex}
\setlength{\parindent}{0ex}
\title{sleighbells}
\begin{document}


\begin{tabular}{ccc}
The Alternative Csound Reference Manual & & \\
Previous & &Next

\end{tabular}

%\hline 
\end{comment}
\section{sleighbells}
sleighbells�--� Semi-physical model of a sleighbell sound. \subsection*{Description}


 \emph{sleighbells}
 is a semi-physical model of a sleighbell sound. It is one of the PhISEM percussion opcodes. PhISEM (Physically Informed Stochastic Event Modeling) is an algorithmic approach for simulating collisions of multiple independent sound producing objects. 
\subsection*{Syntax}


 ar \textbf{sleighbells}
 kamp, idettack [, inum] [, idamp] [, imaxshake] [, ifreq] [, ifreq1] [, ifreq2]
\subsection*{Initialization}


 \emph{idettack}
 -- period of time over which all sound is stopped 


 \emph{inum}
 (optional) -- The number of beads, teeth, bells, timbrels, etc. If zero, the default value is 32. 


 \emph{idamp}
 (optional) -- the damping factor, as part of this equation: 


 damping\_amount�=�0.9994�+�(idamp�*�0.002)


  The default \emph{damping\_amount}
 is 0.9994 which means that the default value of \emph{idamp}
 is 0. The maximum \emph{damping\_amount}
 is 1.0 (no damping). This means the maximum value for \emph{idamp}
 is 0.03. 


  The recommended range for \emph{idamp}
 is usually below 75\% of the maximum value. 


 \emph{imaxshake}
 (optional, default=0) -- amount of energy to add back into the system. The value should be in range 0 to 1. 


 \emph{ifreq}
 (optional) -- the main resonant frequency. The default value is 2500. 


 \emph{ifreq1}
 (optional) -- the first resonant frequency. The default value is 5300. 


 \emph{ifreq2}
 (optional) -- the second resonant frequency. The default value is 6500. 
\subsection*{Performance}


 \emph{kamp}
 -- Amplitude of output. Note: As these instruments are stochastic, this is only an approximation. 
\subsection*{Examples}


  Here is an example of the sleighbells opcode. It uses the files \emph{sleighbells.orc}
 and \emph{sleighbells.sco}
. 


 \textbf{Example 1. Example of the sleighbells opcode.}

\begin{lstlisting}
/* sleighbells.orc */
sr = 22050
kr = 2205
ksmps = 10
nchnls = 1

; Instrument #1: An example of sleighbells.
instr 1
  a1 sleighbells 20000, 0.01

  out a1
endin
/* sleighbells.orc */
        
\end{lstlisting}
\begin{lstlisting}
/* sleighbells.sco */
i 1 0.00 0.25
i 1 0.30 0.25
i 1 0.60 0.25
i 1 0.90 0.25
i 1 1.20 0.25
i 1 1.50 0.25
i 1 1.80 0.25
i 1 2.10 0.25
i 1 2.40 0.25
i 1 2.70 0.25
i 1 3.00 0.25
e
/* sleighbells.sco */
        
\end{lstlisting}
\subsection*{See Also}


 \emph{bamboo}
, \emph{dripwater}
, \emph{guiro}
, \emph{tambourine}

\subsection*{Credits}


 


 


\begin{tabular}{cccc}
Author: Perry Cook, part of the PhISEM (Physically Informed Stochastic Event Modeling) &Adapted by John ffitch &University of Bath, Codemist Ltd. &Bath, UK

\end{tabular}



 


 New in Csound version 4.07


 Added notes by Rasmus Ekman on May 2002.
%\hline 


\begin{comment}
\begin{tabular}{lcr}
Previous &Home &Next \\
sininv &Up &slider16

\end{tabular}


\end{document}
\end{comment}
