\begin{comment}
\documentclass[10pt]{article}
\usepackage{fullpage, graphicx, url}
\setlength{\parskip}{1ex}
\setlength{\parindent}{0ex}
\title{moog}
\begin{document}


\begin{tabular}{ccc}
The Alternative Csound Reference Manual & & \\
Previous & &Next

\end{tabular}

%\hline 
\end{comment}
\section{moog}
moog�--� An emulation of a mini-Moog synthesizer. \subsection*{Description}


  An emulation of a mini-Moog synthesizer. 
\subsection*{Syntax}


 ar \textbf{moog}
 kamp, kfreq, kfiltq, kfiltrate, kvibf, kvamp, iafn, iwfn, ivfn
\subsection*{Initialization}


 \emph{iafn, iwfn, ivfn}
 -- three table numbers containing the attack waveform (unlooped), the main looping wave form, and the vibrato waveform. The files \emph{mandpluk.aiff}
 and \emph{impuls20.aiff}
 are suitable for the first two, and a sine wave for the last. 


 


\begin{tabular}{cc}
\textbf{Note}
 \\
� &

  The files ``mandpluk.aiff'' and ``impuls20.aiff'' are also available at \emph{\url{ftp://ftp.cs.bath.ac.uk/pub/dream/documentation/sounds/modelling/}}
. 


\end{tabular}

\subsection*{Performance}


 \emph{kamp}
 -- Amplitude of note. 


 \emph{kfreq}
 -- Frequency of note played. 


 \emph{kfiltq}
 -- Q of the filter, in the range 0.8 to 0.9 


 \emph{kfiltrate}
 -- rate control for the filter in the range 0 to 0.0002 


 \emph{kvibf}
 -- frequency of vibrato in Hertz. Suggested range is 0 to 12 


 \emph{kvamp}
 -- amplitude of the vibrato 
\subsection*{Examples}


  Here is an example of the moog opcode. It uses the files \emph{moog.orc}
, \emph{moog.sco}
, \emph{mandpluk.aiff}
, and \emph{impuls20.aiff}
. 


 \textbf{Example 1. Example of the moog opcode.}

\begin{lstlisting}
/* moog.orc */
; Initialize the global variables.
sr = 22050
kr = 2205
ksmps = 10
nchnls = 1

; Instrument #1.
instr 1
  kamp = 30000
  kfreq = 220
  kfiltq = 0.81
  kfiltrate = 0
  kvibf = 1.4
  kvamp = 2.22
  iafn = 1
  iwfn = 2
  ivfn = 3

  am moog kamp, kfreq, kfiltq, kfiltrate, kvibf, kvamp, iafn, iwfn, ivfn

  ; It tends to get loud, so clip moog's amplitude at 30,000.
  a1 clip am, 2, 30000
  out a1
endin
/* moog.orc */
        
\end{lstlisting}
\begin{lstlisting}
/* moog.sco */
; Table #1: the "mandpluk.aiff" audio file
f 1 0 8192 1 "mandpluk.aiff" 0 0 0
; Table #2: the "impuls20.aiff" audio file
f 2 0 256 1 "impuls20.aiff" 0 0 0
; Table #3: a sine wave
f 3 0 256 10 1

; Play Instrument #1 for three seconds.
i 1 0 3
e
/* moog.sco */
        
\end{lstlisting}
\subsection*{Credits}


 


 


\begin{tabular}{ccc}
Author: John ffitch (after Perry Cook) &University of Bath, Codemist Ltd. &Bath, UK

\end{tabular}



 


 Example written by Kevin Conder.


 New in Csound version 3.47
%\hline 


\begin{comment}
\begin{tabular}{lcr}
Previous &Home &Next \\
mirror &Up &moogvcf

\end{tabular}


\end{document}
\end{comment}
