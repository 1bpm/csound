\begin{comment}
\documentclass[10pt]{article}
\usepackage{fullpage, graphicx, url}
\setlength{\parskip}{1ex}
\setlength{\parindent}{0ex}
\title{specaddm}
\begin{document}


\begin{tabular}{ccc}
The Alternative Csound Reference Manual & & \\
Previous & &Next

\end{tabular}

%\hline 
\end{comment}
\section{specaddm}
specaddm�--� Perform a weighted add of two input spectra. \subsection*{Description}


  Perform a weighted add of two input spectra. 
\subsection*{Syntax}


 wsig \textbf{specaddm}
 wsig1, wsig2 [, imul2]
\subsection*{Initialization}


 \emph{imul2}
 (optional, default=0) -- if non-zero, scale the \emph{wsig2}
 magnitudes before adding. The default value is 0. 
\subsection*{Performance}


 \emph{wsig1}
 -- the first input spectra. 


 \emph{wsig2}
 -- the second input spectra. 


  Do a weighted add of two input spectra. For each channel of the two input spectra, the two magnitudes are combined and written to the output according to: 


 magout�=�mag1in�+�mag2in�*�imul2\\ 
 ������


  The operation is performed whenever the input \emph{wsig1}
 is sensed to be new. This unit will (at Initialization) verify the consistency of the two spectra (equal size, equal period, equal mag types). 
\subsection*{Examples}


 


 
\begin{lstlisting}
  wsig2    \emph{specdiff}
         wsig1               ; sense onsets 
  wsig3    \emph{specfilt}
         wsig2, 2            ; absorb slowly 
           \emph{specdisp}
         wsig2, .1           ; & display both spectra 
           \emph{specdisp}
         wsig3, .1
        
\end{lstlisting}


 
\subsection*{See Also}


 \emph{specdiff}
, \emph{specfilt}
, \emph{spechist}
, \emph{specscal}

%\hline 


\begin{comment}
\begin{tabular}{lcr}
Previous &Home &Next \\
spdist &Up &specdiff

\end{tabular}


\end{document}
\end{comment}
