\begin{comment}
\documentclass[10pt]{article}
\usepackage{fullpage, graphicx, url}
\setlength{\parskip}{1ex}
\setlength{\parindent}{0ex}
\title{lphasor}
\begin{document}


\begin{tabular}{ccc}
The Alternative Csound Reference Manual & & \\
Previous & &Next

\end{tabular}

%\hline 
\end{comment}
\section{lphasor}
lphasor�--� Generates a table index for sample playback \subsection*{Description}


  This opcode can be used to generate table index for sample playback (e.g. tablexkt). 
\subsection*{Syntax}


 ar \textbf{lphasor}
 xtrns [, ilps] [, ilpe] [, imode] [, istrt] [, istor]
\subsection*{Initialization}


 \emph{ilps}
 -- loop start. 


 \emph{ilpe}
 -- loop end (must be greater than \emph{ilps}
 to enable looping). The default value of \emph{ilps}
 and \emph{ilpe}
 is zero. 


 \emph{imode}
 (optional: default = 0) -- loop mode. Allowed values are: 


 
\begin{itemize}
\item 

 \emph{0:}
 no loop

\item 

 \emph{1:}
 forward loop

\item 

 \emph{2:}
 backward loop

\item 

 \emph{3:}
 forward-backward loop


\end{itemize}


 \emph{istrt}
 (optional: default = 0) -- The initial output value (phase). It must be less than \emph{ilpe}
 if looping is enabled, but is allowed to be greater than \emph{ilps}
 (i.e. you can start playback in the middle of the loop). 


 \emph{istor}
 (optional: default = 0) -- skip initialization if set to any non-zero value. 
\subsection*{Performance}


 \emph{ar}
 -- a raw table index in samples (same unit for loop points). Can be used as index with the table opcodes. 


 \emph{xtrns}
 -- transpose factor, expressed as a playback ratio. \emph{ar}
 is incremented by this value, and wraps around loop points. For example, 1.5 means a fifth above, 0.75 means fourth below. It is not allowed to be negative. 
\subsection*{Credits}


 


 


\begin{tabular}{cc}
Author: Istvan Varga &January 2002

\end{tabular}



 


 New in version 4.18


 Updated April 2002 and November 2002 by Istvan Varga
%\hline 


\begin{comment}
\begin{tabular}{lcr}
Previous &Home &Next \\
lpfreson &Up &lpinterp

\end{tabular}


\end{document}
\end{comment}
