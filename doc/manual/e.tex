\begin{comment}
\documentclass[10pt]{article}
\usepackage{fullpage, graphicx, url}
\setlength{\parskip}{1ex}
\setlength{\parindent}{0ex}
\title{e Statement}
\begin{document}


\begin{tabular}{ccc}
The Alternative Csound Reference Manual & & \\
Previous & &Next

\end{tabular}

%\hline 
\end{comment}
\section{e Statement}
e statement�--� This statement may be used to mark the end of the last section of the score. \subsection*{Description}


  This statement may be used to mark the end of the last section of the score. 
\subsection*{Syntax}


 \textbf{e}
 anything
\subsection*{Performance}


  All pfields are ignored. 
\subsubsection*{Special Considerations}


  The \emph{e statement}
 is contextually identical to an \emph{s statement}
. Additionally, the \emph{e statement}
 terminates all signal generation (including indefinite performance) and closes all input and output files. 


  If an \emph{e statement}
 occurs before the end of a score, all subsequent score lines will be ignored. 


  The \emph{e statement}
 is optional in a score file yet to be sorted. If a score file has no \emph{e statement}
, then Sort processing will supply one. 
%\hline 


\begin{comment}
\begin{tabular}{lcr}
Previous &Home &Next \\
b Statement &Up &f Statement (or Function Table Statement)

\end{tabular}


\end{document}
\end{comment}
