\begin{comment}
\documentclass[10pt]{article}
\usepackage{fullpage, graphicx, url}
\setlength{\parskip}{1ex}
\setlength{\parindent}{0ex}
\title{outkc}
\begin{document}


\begin{tabular}{ccc}
The Alternative Csound Reference Manual & & \\
Previous & &Next

\end{tabular}

%\hline 
\end{comment}
\section{outkc}
outkc�--� Sends MIDI controller messages at k-rate. \subsection*{Description}


  Sends MIDI controller messages at k-rate. 
\subsection*{Syntax}


 \textbf{outkc}
 kchn, knum, kvalue, kmin, kmax
\subsection*{Performance}


 \emph{kchn}
 -- MIDI channel number (1-16) 


 \emph{knum}
 -- controller number (0-127 for example 1 = ModWheel; 2 = BreathControl etc.) 


 \emph{kvalue}
 -- floating point value 


 \emph{kmin}
 -- minimum floating point value (converted in MIDI integer value 0) 


 \emph{kmax}
 -- maximum floating point value (converted in MIDI integer value 127 (7 bit)) 


 \emph{outkc}
 (k-rate MIDI controller output) sends controller messages to MIDI OUT device. It works only with MIDI instruments which recognize them. It can drive a different value of a parameter for each note currently active. 


  It can scale the k-value floating-point argument according to the \emph{kmin}
 and \emph{kmax}
 values. For example: set \emph{kmin}
 = 1.0 and \emph{kmax}
 = 2.0. When the \emph{kvalue}
 argument receives a 2.0 value, the opcode will send a 127 value to the MIDI OUT device. When the \emph{kvalue}
 argument receives a 1.0 value, it will send a 0 value. k-rate opcodes send a message each time the MIDI converted value of argument \emph{kvalue}
 changes. 
\subsection*{See Also}


 \emph{outiat}
, \emph{outic14}
, \emph{outic}
, \emph{outipat}
, \emph{outipb}
, \emph{outipc}
, \emph{outkat}
, \emph{outkc14}
, \emph{outkpat}
, \emph{outkpb}
, \emph{outkpc}

\subsection*{Credits}


 


 


\begin{tabular}{cc}
Author: Gabriel Maldonado &Italy

\end{tabular}



 


 New in Csound version 3.47


 Thanks goes to Rasmus Ekman for pointing out the correct MIDI channel and controller number ranges.
%\hline 


\begin{comment}
\begin{tabular}{lcr}
Previous &Home &Next \\
outkat &Up &outkc14

\end{tabular}


\end{document}
\end{comment}
