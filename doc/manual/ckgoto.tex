\begin{comment}
\documentclass[10pt]{article}
\usepackage{fullpage, graphicx, url}
\setlength{\parskip}{1ex}
\setlength{\parindent}{0ex}
\title{ckgoto}
\begin{document}


\begin{tabular}{ccc}
The Alternative Csound Reference Manual & & \\
Previous & &Next

\end{tabular}

%\hline 
\end{comment}
\section{ckgoto}
ckgoto�--� Conditionally transfer control during the p-time passes. \subsection*{Description}


  During the p-time passes only, unconditionally transfer control to the statement labeled by \emph{label}
. 
\subsection*{Syntax}


 \textbf{ckgoto}
 condition, label


  where \emph{label}
 is in the same instrument block and is not an expression, and where \emph{R}
 is one of the Relational operators (\emph{$<$}
,\emph{ =}
, \emph{$<$=}
, \emph{==}
, \emph{!=}
) (and \emph{=}
 for convenience, see also under \emph{Conditional Values}
). 
\subsection*{Examples}


  Here is an example of the ckgoto opcode. It uses the files \emph{ckgoto.orc}
 and \emph{ckgoto.sco}
. 


 \textbf{Example 1. Example of the ckgoto opcode.}

\begin{lstlisting}
/* ckgoto.orc */
; Initialize the global variables.
sr = 44100
kr = 4410
ksmps = 10
nchnls = 1

; Instrument #1.
instr 1
  ; Change kval linearly from 0 to 2 over
  ; the period set by the third p-field.
  kval line 0, p3, 2

  ; If kval is greater than or equal to 1 then play the high note.
  ; If not then play the low note.
  ckgoto (kval >= 1), highnote
    kgoto lownote

highnote:
  kfreq = 880
  goto playit

lownote:
  kfreq = 440
  goto playit

playit:
  ; Print the values of kval and kfreq.
  printks "kval = %f, kfreq = %f\\n", 1, kval, kfreq

  a1 oscil 10000, kfreq, 1
  out a1
endin
/* ckgoto.orc */
        
\end{lstlisting}
\begin{lstlisting}
/* ckgoto.sco */
; Table: a simple sine wave.
f 1 0 32768 10 1

; Play Instrument #1 for two seconds.
i 1 0 2
e
/* ckgoto.sco */
        
\end{lstlisting}
 Its output should include lines like: \begin{lstlisting}
kval = 0.000000, kfreq = 440.000000
kval = 0.999732, kfreq = 440.000000
kval = 1.999639, kfreq = 880.000000
      
\end{lstlisting}
\subsection*{See Also}


 \emph{cggoto}
, \emph{cigoto}
, \emph{cngoto}
, \emph{goto}
, \emph{if}
, \emph{igoto}
, \emph{tigoto}
, \emph{timout}

\subsection*{Credits}


 Added a note by Jim Aikin.


 Example written by Kevin Conder.
%\hline 


\begin{comment}
\begin{tabular}{lcr}
Previous &Home &Next \\
cigoto &Up &clear

\end{tabular}


\end{document}
\end{comment}
