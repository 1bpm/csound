\begin{comment}
\documentclass[10pt]{article}
\usepackage{fullpage, graphicx, url}
\setlength{\parskip}{1ex}
\setlength{\parindent}{0ex}
\title{fof}
\begin{document}


\begin{tabular}{ccc}
The Alternative Csound Reference Manual & & \\
Previous & &Next

\end{tabular}

%\hline 
\end{comment}
\section{fof}
fof�--� Produces sinusoid bursts useful for formant and granular synthesis. \subsection*{Description}


  Audio output is a succession of sinusoid bursts initiated at frequency \emph{xfund}
 with a spectral peak at \emph{xform}
. For \emph{xfund}
 above 25 Hz these bursts produce a speech-like formant with spectral characteristics determined by the k-input parameters. For lower fundamentals this generator provides a special form of granular synthesis. 
\subsection*{Syntax}


 ar \textbf{fof}
 xamp, xfund, xform, koct, kband, kris, kdur, kdec, iolaps, ifna, ifnb, itotdur [, iphs] [, ifmode] [, iskip]
\subsection*{Initialization}


 \emph{iolaps}
 -- number of preallocated spaces needed to hold overlapping burst data. Overlaps are frequency dependent, and the space required depends on the maximum value of \emph{xfund * kdur}
. Can be over-estimated at no computation cost. Uses less than 50 bytes of memory per \emph{iolap}
. 


 \emph{ifna, ifnb}
 -- table numbers of two stored functions. The first is a sine table for sineburst synthesis (size of at least 4096 recommended). The second is a rise shape, used forwards and backwards to shape the sineburst rise and decay; this may be linear (\emph{GEN07}
) or perhaps a sigmoid (\emph{GEN19}
). 


 \emph{itotdur}
 -- total time during which this \emph{fof}
 will be active. Normally set to p3. No new sineburst is created if it cannot complete its \emph{kdur}
 within the remaining \emph{itotdur}
. 


 \emph{iphs}
 (optional, default=0) -- initial phase of the fundamental, expressed as a fraction of a cycle (0 to 1). The default value is 0. 


 \emph{ifmode}
 (optional, default=0) -- formant frequency mode. If zero, each sineburst keeps the \emph{xform}
 frequency it was launched with. If non-zero, each is influenced by \emph{xform}
 continuously. The default value is 0. 


 \emph{iskip}
 (optional, default=0) -- If non-zero, skip initialisation (allows legato use). 
\subsection*{Performance}


 \emph{xamp}
 -- peak amplitude of each sineburst, observed at the true end of its rise pattern. The rise may exceed this value given a large bandwidth (say, Q $<$ 10) and/or when the bursts are overlapping. 


 \emph{xfund}
 -- the fundamental frequency (in Hertz) of the impulses that create new sinebursts. 


 \emph{xform}
 -- the formant frequency, i.e. freq of the sinusoid burst induced by each \emph{xfund}
 impulse. This frequency can be fixed for each burst or can vary continuously (see \emph{ifmode}
). 


 \emph{koct}
 -- octaviation index, normally zero. If greater than zero, lowers the effective \emph{xfund}
 frequency by attenuating odd-numbered sinebursts. Whole numbers are full octaves, fractions transitional. 


 \emph{kband}
 -- the formant bandwidth (at -6dB), expressed in Hz. The bandwidth determines the rate of exponential decay throughout the sineburst, before the enveloping described below is applied. 


 \emph{kris, kdur, kdec}
 -- rise, overall duration, and decay times (in seconds) of the sinusoid burst. These values apply an enveloped duration to each burst, in similar fashion to a Csound \emph{linen}
 generator but with rise and decay shapes derived from the \emph{ifnb}
 input. \emph{kris}
 inversely determines the skirtwidth (at -40 dB) of the induced formant region. \emph{kdur}
 affects the density of sineburst overlaps, and thus the speed of computation. Typical values for vocal imitation are .003,.02,.007. 


  Csound's \emph{fof}
 generator is loosely based on Michael Clarke's C-coding of IRCAM's \emph{CHANT}
 program (Xavier Rodet et al.). Each fof produces a single formant, and the output of four or more of these can be summed to produce a rich vocal imitation. \emph{fof}
 synthesis is a special form of granular synthesis, and this implementation aids transformation between vocal imitation and granular textures. Computation speed depends on \emph{kdur, xfund}
, and the density of any overlaps. 
\subsection*{Examples}


  Here is an example of the fof opcode. It uses the files \emph{fof.orc}
 and \emph{fof.sco}
. 


 \textbf{Example 1. Example of the fof opcode.}

\begin{lstlisting}
/* fof.orc */
/* Adapted from 1401.orc by Michael Clarke */
; Initialize the global variables.
sr = 44100
kr = 4410
ksmps = 10
nchnls = 1

; Instrument #1.
instr 1
  ; Combine five formants together to create 
  ; an alto-"a" sound.

  ; Values common to all of the formants.
  kfund init 261.659
  koct init 0
  kris init 0.003
  kdur init 0.02
  kdec init 0.007
  iolaps = 14850
  ifna = 1
  ifnb = 2
  itotdur = p3

  ; First formant.
  k1amp = ampdb(0)
  k1form init 800
  k1band init 80

  ; Second formant.
  k2amp = ampdb(-4)
  k2form init 1150
  k2band init 90

  ; Third formant.
  k3amp = ampdb(-20)
  k3form init 2800
  k3band init 120

  ; Fourth formant.
  k4amp = ampdb(-36)
  k4form init 3500
  k4band init 130

  ; Fifth formant.
  k5amp = ampdb(-60)
  k5form init 4950
  k5band init 140

  a1 fof k1amp, kfund, k1form, koct, k1band, kris, \
         kdur, kdec, iolaps, ifna, ifnb, itotdur
  a2 fof k2amp, kfund, k2form, koct, k2band, kris, \
         kdur, kdec, iolaps, ifna, ifnb, itotdur
  a3 fof k3amp, kfund, k3form, koct, k3band, kris, \
         kdur, kdec, iolaps, ifna, ifnb, itotdur
  a4 fof k4amp, kfund, k4form, koct, k4band, kris, \
         kdur, kdec, iolaps, ifna, ifnb, itotdur
  a5 fof k5amp, kfund, k5form, koct, k5band, kris, \
         kdur, kdec, iolaps, ifna, ifnb, itotdur

  ; Combine all of the formants together.
  out (a1+a2+a3+a4+a5) * 16384
endin
/* fof.orc */
        
\end{lstlisting}
\begin{lstlisting}
/* fof.sco */
/* Adapted from 1401.sco by Michael Clarke */
; Table #1, a sine wave.
f 1 0 4096 10 1
; Table #2.
f 2 0 1024 19 0.5 0.5 270 0.5

; Play Instrument #1 for three seconds.
i 1 0 3
e
/* fof.sco */
        
\end{lstlisting}
 The formant values for the alto-''a'' sound were taken from the \emph{Formant Values Appendix}
. \subsection*{See Also}


 \emph{fof2}
, \emph{Formant Values Appendix}

%\hline 


\begin{comment}
\begin{tabular}{lcr}
Previous &Home &Next \\
fmwurlie &Up &fof2

\end{tabular}


\end{document}
\end{comment}
