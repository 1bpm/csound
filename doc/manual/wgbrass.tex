\begin{comment}
\documentclass[10pt]{article}
\usepackage{fullpage, graphicx, url}
\setlength{\parskip}{1ex}
\setlength{\parindent}{0ex}
\title{wgbrass}
\begin{document}


\begin{tabular}{ccc}
The Alternative Csound Reference Manual & & \\
Previous & &Next

\end{tabular}

%\hline 
\end{comment}
\section{wgbrass}
wgbrass�--� Creates a tone related to a brass instrument. \subsection*{Description}


  Audio output is a tone related to a brass instrument, using a physical model developed from Perry Cook, but re-coded for Csound. 
\subsection*{Syntax}


 ar \textbf{wgbrass}
 kamp, kfreq, ktens, iatt, kvibf, kvamp, ifn [, iminfreq]
\subsection*{Initialization}


 \emph{iatt}
 -- time taken to reach full pressure 


 \emph{ifn}
 -- table of shape of vibrato, usually a sine table, created by a function 


 \emph{iminfreq}
 -- lowest frequency at which the instrument will play. If it is omitted it is taken to be the same as the initial \emph{kfreq}
. If \emph{iminfreq}
 is negative, initialization will be skipped. 
\subsection*{Performance}


  A note is played on a brass-like instrument, with the arguments as below. 


 \emph{kamp}
 -- Amplitude of note. 


 \emph{kfreq}
 -- Frequency of note played. 


 \emph{ktens}
 -- lip tension of the player. Suggested value is about 0.4 


 \emph{kvibf}
 -- frequency of vibrato in Hertz. Suggested range is 0 to 12 


 \emph{kvamp}
 -- amplitude of the vibrato 


 


\begin{tabular}{cc}
Warning &\textbf{NOTE}
 \\
� &

  This is rather poor, and at present uncontrolled. Needs revision, and possibly more parameters. 


\end{tabular}

\subsection*{Examples}


  Here is an example of the wgbrass opcode. It uses the files \emph{wgbrass.orc}
 and \emph{wgbrass.sco}
. 


 \textbf{Example 1. Example of the wgbrass opcode.}

\begin{lstlisting}
/* wgbrass.orc */
; Initialize the global variables.
sr = 44100
kr = 4410
ksmps = 10
nchnls = 1

; Instrument #1.
instr 1
  kamp = 31129.60
  kfreq = 440
  ktens = 0.4
  iatt = 0.1
  kvibf = 6.137
  ifn = 1

  ; Create an amplitude envelope for the vibrato.
  kvamp line 0, p3, 0.5

  a1 wgbrass kamp, kfreq, ktens, iatt, kvibf, kvamp, ifn
  out a1
endin
/* wgbrass.orc */
        
\end{lstlisting}
\begin{lstlisting}
/* wgbrass.sco */
; Table #1, a sine wave.
f 1 0 128 10 1

; Play Instrument #1 for one second.
i 1 0 1
e
/* wgbrass.sco */
        
\end{lstlisting}
\subsection*{Credits}


 


 


\begin{tabular}{ccc}
Author: John ffitch (after Perry Cook) &University of Bath, Codemist Ltd. &Bath, UK

\end{tabular}



 


 New in Csound version 3.47
%\hline 


\begin{comment}
\begin{tabular}{lcr}
Previous &Home &Next \\
wgbowedbar &Up &wgclar

\end{tabular}


\end{document}
\end{comment}
