\begin{comment}
\documentclass[10pt]{article}
\usepackage{fullpage, graphicx, url}
\setlength{\parskip}{1ex}
\setlength{\parindent}{0ex}
\title{lfo}
\begin{document}


\begin{tabular}{ccc}
The Alternative Csound Reference Manual & & \\
Previous & &Next

\end{tabular}

%\hline 
\end{comment}
\section{lfo}
lfo�--� A low frequency oscillator of various shapes. \subsection*{Description}


  A low frequency oscillator of various shapes. 
\subsection*{Syntax}


 kr \textbf{lfo}
 kamp, kcps [, itype]


 ar \textbf{lfo}
 kamp, kcps [, itype]
\subsection*{Initialization}


 \emph{itype}
 (optional, default=0) -- determine the waveform of the oscillator. Default is 0. 


 
\begin{itemize}
\item 

 \emph{itype}
 = 0 - sine

\item 

 \emph{itype}
 = 1 - triangles

\item 

 \emph{itype}
 = 2 - square (bipolar)

\item 

 \emph{itype}
 = 3 - square (unipolar)

\item 

 \emph{itype}
 = 4 - saw-tooth

\item 

 \emph{itype}
 = 5 - saw-tooth(down)


\end{itemize}


  The sine wave is implemented as a 4096 table and linear interpolation. The others are calculated. 
\subsection*{Performance}


 \emph{kamp}
 -- amplitude of output 


 \emph{kcps}
 -- frequency of oscillator 
\subsection*{Examples}


  Here is an example of the lfo opcode. It uses the files \emph{lfo.orc}
 and \emph{lfo.sco}
. 


 \textbf{Example 1. Example of the lfo opcode.}

\begin{lstlisting}
/* lfo.orc */
; Initialize the global variables.
sr = 44100
kr = 4410
ksmps = 10
nchnls = 1

; Instrument #1.
instr 1
  kamp = 10
  kcps = 5
  itype = 4

  k1 lfo kamp, kcps, itype
  ar oscil p4, p5+k1, 1
  out ar
endin
/* lfo.orc */
        
\end{lstlisting}
\begin{lstlisting}
/* lfo.sco */
; Table #1: an ordinary sine wave.
f 1 0 32768 10 1

; p4 = amplitude of the output signal.
; p5 = frequency (in cycles per second) of the output signal.
; Play Instrument #1 for two seconds.
i 1 0 2 10000 220
e
/* lfo.sco */
        
\end{lstlisting}
\subsection*{Credits}


 


 


\begin{tabular}{cccc}
Author: John ffitch &University of Bath/Codemist Ltd. &Bath, UK &November 1998

\end{tabular}



 


 New in Csound version 3.491
%\hline 


\begin{comment}
\begin{tabular}{lcr}
Previous &Home &Next \\
kweibull &Up &limit

\end{tabular}


\end{document}
\end{comment}
