\begin{comment}
\documentclass[10pt]{article}
\usepackage{fullpage, graphicx, url}
\setlength{\parskip}{1ex}
\setlength{\parindent}{0ex}
\title{Csound64}
\begin{document}


\begin{tabular}{ccc}
The Alternative Csound Reference Manual & & \\
Previous & &Next

\end{tabular}

%\hline 
\end{comment}
\section{Csound64}


  Csound64 is a version of Csound that uses 64-bit DOUBLE's internally to do processing versus regular Csound's 32-bit FLOAT's. This larger resolution for processing internally yields a much ``cleaner'' sound but at the expense of extended processing time. Because it does require much longer to process, Csound64 is typicaly used after a work is finished for a final production run. 


 \textbf{Notes On Using Csound64. }



 
\begin{enumerate}
\item 

 hetro files generated for Csound will work with Csound64.

\item 

 PVOC-EX analysis and pvanal files generated for Csound will not work with Csound64. For Csound64, use of pvanal and pvoc opcodes are not currently supported. If your work file uses pvoc, use Csound. (This is a temporary issue relating to older file formats and is currently being addressed and worked on.)

\item 

 lpanal files generated for Csound will not work with Csound64. For Csound64, use of lpanal and lpc opcodes are not currently supported. If your work file uses lpc, use Csound. (This is a temporary issue relating to older file formats and is currently being addressed and worked on.)

\item 

 cvanal files generated for Csound will not work with Csound64. To generated cv files usable by Csound64, use the following command line: 


 csound64�-U�cvanal�\\ 
 ����������
 instead of either of the following: 

 csound�-U�cvanal\\ 
 cvanal\\ 
 ����������


\end{enumerate}


  This will generate a 64-bit cv file. If you were working with 32-bit Csound and using a 32-bit cv file, the cv file will not work with Csound64. When you switch to Csound64, you will need to use a 64-bit generated cv file. 
%\hline 


\begin{comment}
\begin{tabular}{lcr}
Previous &Home &Next \\
SoundFont2 File Format &� &Quick Reference

\end{tabular}


\end{document}
\end{comment}
