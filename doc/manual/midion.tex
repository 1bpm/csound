\begin{comment}
\documentclass[10pt]{article}
\usepackage{fullpage, graphicx, url}
\setlength{\parskip}{1ex}
\setlength{\parindent}{0ex}
\title{midion}
\begin{document}


\begin{tabular}{ccc}
The Alternative Csound Reference Manual & & \\
Previous & &Next

\end{tabular}

%\hline 
\end{comment}
\section{midion}
midion�--� Plays MIDI notes. \subsection*{Description}


  Plays MIDI notes. 
\subsection*{Syntax}


 \textbf{midion}
 kchn, knum, kvel
\subsection*{Performance}


 \emph{kchn}
 -- MIDI channel number (1-16) 


 \emph{knum}
 -- note number (0-127) 


 \emph{kvel}
 -- velocity (0-127) 


 \emph{midion}
 (k-rate note on) plays MIDI notes with current \emph{kchn}
, \emph{knum}
 and \emph{kvel}
. These arguments can be varied at k-rate. Each time the MIDI converted value of any of these arguments changes, last MIDI note played by current instance of \emph{midion}
 is immediately turned off and a new note with the new argument values is activated. This opcode, as well as \emph{moscil}
, can generate very complex melodic textures if controlled by complex k-rate signals. 


  Any number of \emph{midion}
 opcodes can appear in the same Csound instrument, allowing a counterpoint-style polyphony within a single instrument. 
\subsection*{See Also}


 \emph{moscil}

\subsection*{Credits}


 


 


\begin{tabular}{ccc}
Author: Gabriel Maldonado &Italy &May 1997

\end{tabular}



 


 Thanks goes to Rasmus Ekman for pointing out the correct MIDI channel and controller number ranges.
%\hline 


\begin{comment}
\begin{tabular}{lcr}
Previous &Home &Next \\
midinoteonpch &Up &midion2

\end{tabular}


\end{document}
\end{comment}
