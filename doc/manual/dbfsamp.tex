\begin{comment}
\documentclass[10pt]{article}
\usepackage{fullpage, graphicx, url}
\setlength{\parskip}{1ex}
\setlength{\parindent}{0ex}
\title{dbfsamp}
\begin{document}


\begin{tabular}{ccc}
The Alternative Csound Reference Manual & & \\
Previous & &Next

\end{tabular}

%\hline 
\end{comment}
\section{dbfsamp}
dbfsamp�--� Returns the decibel equivalent of the raw amplitude x, relative to full scale amplitude. \subsection*{Description}


  Returns the decibel equivalent of the raw amplitude x, relative to full scale amplitude. Full scale is assumed to be 16 bit. New is Csound version 4.10. 
\subsection*{Syntax}


 \textbf{dbfsamp}
(x) (init-rate or control-rate args only)
\subsection*{Examples}


  Here is an example of the dbfsamp opcode. It uses the files \emph{dbfsamp.orc}
 and \emph{dbfsamp.sco}
. 


 \textbf{Example 1. Example of the dbfsamp opcode.}

\begin{lstlisting}
/* dbfsamp.orc */
; Initialize the global variables.
sr = 44100
kr = 4410
ksmps = 10
nchnls = 1

; Instrument #1.
instr 1
  iamp = 30000
  idb = dbfsamp(iamp)

  print idb
endin
/* dbfsamp.orc */
        
\end{lstlisting}
\begin{lstlisting}
/* dbfsamp.sco */
; Play Instrument #1 for one second.
i 1 0 1
e
/* dbfsamp.sco */
        
\end{lstlisting}
 Its output should include lines like this: \begin{lstlisting}
instr 1:  idb = -0.767
      
\end{lstlisting}
\subsection*{See Also}


 \emph{ampdb}
, \emph{ampdbfs}
, \emph{dbamp}

\subsection*{Credits}


 Example written by Kevin Conder.
%\hline 


\begin{comment}
\begin{tabular}{lcr}
Previous &Home &Next \\
dbamp &Up &dcblock

\end{tabular}


\end{document}
\end{comment}
