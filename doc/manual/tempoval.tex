\begin{comment}
\documentclass[10pt]{article}
\usepackage{fullpage, graphicx, url}
\setlength{\parskip}{1ex}
\setlength{\parindent}{0ex}
\title{tempoval}
\begin{document}


\begin{tabular}{ccc}
The Alternative Csound Reference Manual & & \\
Previous & &Next

\end{tabular}

%\hline 
\end{comment}
\section{tempoval}
tempoval�--� Reads the current value of the tempo. \subsection*{Description}


  Reads the current value of the tempo. 
\subsection*{Syntax}


 kr \textbf{tempoval}

\subsection*{Performance}


 \emph{kr}
 -- the value of the tempo. If you use a positive value with the \emph{-t command-line flag}
, \emph{tempoval}
 returns the percentage increase/decrease from the original tempo of 60 beats per minute. If you don't, its value will be 60 (for 60 beats per minute). 
\subsection*{Examples}


  Here is an example of the tempoval opcode. Remember, it only works if you use the \emph{-t}
 flag with Csound. It uses the files \emph{tempoval.orc}
 and \emph{tempoval.sco}
. 


 \textbf{Example 1. Example of the tempoval opcode.}

\begin{lstlisting}
/* tempoval.orc */
; Initialize the global variables.
sr = 44100
kr = 4410
ksmps = 10
nchnls = 1

; Instrument #1.
instr 1
  ; Adjust the tempo to 120 beats per minute.
  tempo 120, 60

  ; Get the tempo value.
  kval tempoval

  printks "kval = %f\\n", 0.1, kval
endin
/* tempoval.orc */
        
\end{lstlisting}
\begin{lstlisting}
/* tempoval.sco */
; Play Instrument #1 for one second.
i 1 0 1
e
/* tempoval.sco */
        
\end{lstlisting}
 Since 120 beats per minute is a 50\% increase over the original 60 beats per minute, its output should include lines like: \begin{lstlisting}
kval = 0.500000
      
\end{lstlisting}
\subsection*{See Also}


 \emph{tempo}

\subsection*{Credits}


 Example written by Kevin Conder.


 New in version 4.15


 December 2002. Thanks to Drake Wilson for pointing out unclear documentation.
%\hline 


\begin{comment}
\begin{tabular}{lcr}
Previous &Home &Next \\
tempo &Up &tigoto

\end{tabular}


\end{document}
\end{comment}
