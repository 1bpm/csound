\begin{comment}
\documentclass[10pt]{article}
\usepackage{fullpage, graphicx, url}
\setlength{\parskip}{1ex}
\setlength{\parindent}{0ex}
\title{FLprintk2}
\begin{document}


\begin{tabular}{ccc}
The Alternative Csound Reference Manual & & \\
Previous & &Next

\end{tabular}

%\hline 
\end{comment}
\section{FLprintk2}
FLprintk2�--� A FLTK opcode that prints a new value every time a control-rate variable changes. \subsection*{Description}


 \emph{FLprintk2}
 is similar to \emph{FLprintk}
 but shows a k-rate variable's value only when it changes. 
\subsection*{Syntax}


 \textbf{FLprintk2}
 kval, idisp
\subsection*{Initialization}


 \emph{idisp}
 -- a handle value that was output from a previous instance of the \emph{FLvalue}
 opcode to display the current value of the current valuator in the \emph{FLvalue}
 widget itself. If the user doesn't want to use this feature that displays current values, it must be set to a negative number by the user. 
\subsection*{Performance}


 \emph{kval}
 -- k-rate signal to be displayed. 


 \emph{FLprintk2}
 is similar to \emph{FLprintk}
, but shows the k-rate variable's value only each time it changes. Useful for monitoring MIDI control changes when using sliders. It should be used for debugging purposes only, since it slows-down performance. 
\subsection*{See Also}


 \emph{FLbox}
, \emph{FLbutBank}
, \emph{FLbutton}
, \emph{FLprintk}
, \emph{FLvalue}

\subsection*{Credits}


 Author: Gabriel Maldonado


 New in version 4.22
%\hline 


\begin{comment}
\begin{tabular}{lcr}
Previous &Home &Next \\
FLprintk &Up &FLroller

\end{tabular}


\end{document}
\end{comment}
