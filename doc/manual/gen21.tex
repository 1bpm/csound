\begin{comment}
\documentclass[10pt]{article}
\usepackage{fullpage, graphicx, url}
\setlength{\parskip}{1ex}
\setlength{\parindent}{0ex}
\title{GEN21}
\begin{document}


\begin{tabular}{ccc}
The Alternative Csound Reference Manual & & \\
Previous & &Next

\end{tabular}

%\hline 
\end{comment}
\section{GEN21}
GEN21�--� Generates tables of different random distributions. \subsection*{Description}


  This generates tables of different random distributions. (See also \emph{betarand}
, \emph{bexprnd}
, \emph{cauchy}
, \emph{exprand}
, \emph{gauss}
, \emph{linrand}
, \emph{pcauchy}
, \emph{poisson}
, \emph{trirand}
, \emph{unirand}
, and \emph{weibull}
) 
\subsection*{Syntax}


 \textbf{f}
 \# time size 21 type level [arg1 [arg2]]
\subsection*{Initialization}


 \emph{time}
 and \emph{size}
 are the usual GEN function arguments. \emph{level}
 defines the amplitude. Note that GEN21 is not self-normalizing as are most other GEN functions. \emph{type}
 defines the distribution to be used as follow: 


 
\begin{itemize}
\item 

 1 = Uniform (positive numbers only)

\item 

 2 = Linear (positive numbers only)

\item 

 3 = Triangular (positive and negative numbers)

\item 

 4 = Exponential (positive numbers only)

\item 

 5 = Biexponential (positive and negative numbers)

\item 

 6 = Gaussian (positive and negative numbers)

\item 

 7 = Cauchy (positive and negative numbers)

\item 

 8 = Positive Cauchy (positive numbers only)

\item 

 9 = Beta (positive numbers only)

\item 

 10 = Weibull (positive numbers only)

\item 

 11 = Poisson (positive numbers only)


\end{itemize}
 Of all these cases only 9 (Beta) and 10 (Weibull) need extra arguments. Beta needs two arguments and Weibull one. \subsection*{Examples}


 


 
\begin{lstlisting}
\emph{f}
1 0 1024 21 1       ; Uniform (white noise)
\emph{f}
1 0 1024 21 6       ; Gaussian
\emph{f}
1 0 1024 21 9 1 1 2 ; Beta (note that level precedes arguments)
\emph{f}
1 0 1024 21 10 1 2  ; Weibull
        
\end{lstlisting}


 
 All of the above additions were designed by the author between May and December 1994, under the supervision of Dr. Richard Boulanger. \subsection*{Credits}


 


 


\begin{tabular}{ccc}
Author: Paris Smaragdis &MIT, Cambridge &1995

\end{tabular}



 


 


 


\begin{tabular}{ccc}
Author: John ffitch &University of Bath/Codemist Ltd. &Bath, UK

\end{tabular}



 


 New in Csound version 3.2
%\hline 


\begin{comment}
\begin{tabular}{lcr}
Previous &Home &Next \\
GEN20 &Up &GEN22

\end{tabular}


\end{document}
\end{comment}
