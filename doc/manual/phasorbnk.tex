\begin{comment}
\documentclass[10pt]{article}
\usepackage{fullpage, graphicx, url}
\setlength{\parskip}{1ex}
\setlength{\parindent}{0ex}
\title{phasorbnk}
\begin{document}


\begin{tabular}{ccc}
The Alternative Csound Reference Manual & & \\
Previous & &Next

\end{tabular}

%\hline 
\end{comment}
\section{phasorbnk}
phasorbnk�--� Produce an arbitrary number of normalized moving phase values. \subsection*{Description}


  Produce an arbitrary number of normalized moving phase values, accessable by an index. 
\subsection*{Syntax}


 ar \textbf{phasorbnk}
 xcps, kndx, icnt [, iphs]


 kr \textbf{phasorbnk}
 kcps, kndx, icnt [, iphs]
\subsection*{Initialization}


 \emph{icnt}
 -- maximum number of phasors to be used. 


 \emph{iphs}
 -- initial phase, expressed as a fraction of a cycle (0 to 1). If -1 initialization is skipped. If \emph{iphas}
$>$1 each phasor will be initialized with a random value. 
\subsection*{Performance}


 \emph{kndx}
 -- index value to access individual phasors 


  For each independent phasor, an internal phase is successively accumulated in accordance with the \emph{kcps}
 or \emph{xcps}
 frequency to produce a moving phase value, normalized to lie in the range 0 $<$= phs $<$ 1. Each individual phasor is accessed by index \emph{kndx}
. 


  This phasor bank can be used inside a k-rate loop to generate multiple independent voices, or together with the \emph{adsynt}
 opcode to change parameters in the tables used by \emph{adsynt}
. 
\subsection*{Examples}


  Here is an example of the phasorbnk opcode. It uses the files \emph{phasorbnk.orc}
 and \emph{phasorbnk.sco}
. 


 \textbf{Example 1. Example of the phasorbnk opcode.}

\begin{lstlisting}
/* phasorbnk.orc */
; Initialize the global variables.
sr = 44100
kr = 4410
ksmps = 10
nchnls = 1

; Generate a sinewave table.
giwave ftgen 1, 0, 1024, 10, 1 

; Instrument #1
instr 1
  ; Generate 10 voices.
  icnt = 10 
  ; Empty the output buffer.
  asum = 0 
  ; Reset the loop index.
  kindex = 0 

; This loop is executed every k-cycle.
loop: 
  ; Generate non-harmonic partials.
  kcps = (kindex+1)*100+30 
  ; Get the phase for each voice.
  aphas phasorbnk kcps, kindex, icnt 
  ; Read the wave from the table.
  asig table aphas, giwave, 1 
  ; Accumulate the audio output.
  asum = asum + asig 

  ; Increment the index.
  kindex = kindex + 1

  ; Perform the loop until the index (kindex) reaches 
  ; the counter value (icnt).
  if (kindex < icnt) kgoto loop 

  out asum*3000
endin
/* phasorbnk.orc */
        
\end{lstlisting}
\begin{lstlisting}
/* phasorbnk.sco */
; Play Instrument #1 for two seconds.
i 1 0 2
e
/* phasorbnk.sco */
        
\end{lstlisting}
 Generate multiple voices with independent partials. This example is better with \emph{adsynt}
. See also the example under \emph{adsynt}
, for k-rate use of \emph{phasorbnk}
. \subsection*{Credits}


 


 


\begin{tabular}{ccc}
Author: Peter Neub\"acker &Munich, Germany &August 1999

\end{tabular}



 


 New in Csound version 3.58
%\hline 


\begin{comment}
\begin{tabular}{lcr}
Previous &Home &Next \\
phasor &Up &pinkish

\end{tabular}


\end{document}
\end{comment}
