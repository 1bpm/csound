\begin{comment}
\documentclass[10pt]{article}
\usepackage{fullpage, graphicx, url}
\setlength{\parskip}{1ex}
\setlength{\parindent}{0ex}
\title{schedule}
\begin{document}


\begin{tabular}{ccc}
The Alternative Csound Reference Manual & & \\
Previous & &Next

\end{tabular}

%\hline 
\end{comment}
\section{schedule}
schedule�--� Adds a new score event. \subsection*{Description}


  Adds a new score event. 
\subsection*{Syntax}


 \textbf{schedule}
 insnum, iwhen, idur [, ip4] [, ip5] [...]


 \textbf{schedule}
 ``insname'', iwhen, idur [, ip4] [, ip5] [...]
\subsection*{Initialization}


 \emph{insnum}
 -- instrument number. Equivalent to p1 in a score \emph{i statement}
. \emph{insnum}
 must be a number greater than the number of the calling instrument. 


 \emph{``insname''}
 -- A string (in double-quotes) representing a named instrument. 


 \emph{iwhen}
 -- start time of the new event. Equivalent to p2 in a score \emph{i statement}
. \emph{iwhen}
 must be nonnegative. If \emph{iwhen}
 is zero, \emph{insum}
 must be greater than or equal to the p1 of the current instrument. 


 \emph{idur}
 -- duration of event. Equivalent to p3 in a score \emph{i statement}
. 


 \emph{ip4, ip5, ...}
 -- Equivalent to p4, p5, etc., in a score \emph{i statement}
. 
\subsection*{Performance}


 \emph{ktrigger}
 -- trigger value for new event 


 \emph{schedule}
 adds a new score event. The arguments, including options, are the same as in a score. The \emph{iwhen}
 time (p2) is measured from the time of this event. 


  If the duration is zero or negative the new event is of MIDI type, and inherits the release sub-event from the scheduling instruction. 
\subsection*{Examples}


  Here is an example of the schedule opcode. It uses the files \emph{schedule.orc}
 and \emph{schedule.sco}
. 


 \textbf{Example 1. Example of the schedule opcode.}

\begin{lstlisting}
/* schedule.orc */
; Initialize the global variables.
sr = 44100
kr = 4410
ksmps = 10
nchnls = 1

; Instrument #1 - oscillator with a high note.
instr 1
  ; Play Instrument #2 at the same time.
  schedule 2, 0, p3

  ; Play a high note.
  a1 oscils 10000, 880, 1
  out a1
endin

; Instrument #2 - oscillator with a low note.
instr 2
  ; Play a low note.
  a1 oscils 10000, 220, 1
  out a1
endin
/* schedule.orc */
        
\end{lstlisting}
\begin{lstlisting}
/* schedule.sco */
; Table #1, a sine wave.
f 1 0 16384 10 1

; Play Instrument #1 for half a second.
i 1 0 0.5
; Play Instrument #1 for half a second.
i 1 1 0.5
e
/* schedule.sco */
        
\end{lstlisting}
\subsection*{See Also}


 \emph{schedwhen}

\subsection*{Credits}


 


 


\begin{tabular}{cccc}
Author: John ffitch &University of Bath/Codemist Ltd. &Bath, UK &November 1998

\end{tabular}



 


 Example written by Kevin Conder.


 New in Csound version 3.491


 Based on work by Gabriel Maldonado


 Thanks goes to David Gladstein, for clarifying the \emph{iwhen}
 parameter.
%\hline 


\begin{comment}
\begin{tabular}{lcr}
Previous &Home &Next \\
schedkwhennamed &Up &schedwhen

\end{tabular}


\end{document}
\end{comment}
