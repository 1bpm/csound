\begin{comment}
\documentclass[10pt]{article}
\usepackage{fullpage, graphicx, url}
\setlength{\parskip}{1ex}
\setlength{\parindent}{0ex}
\title{spechist}
\begin{document}


\begin{tabular}{ccc}
The Alternative Csound Reference Manual & & \\
Previous & &Next

\end{tabular}

%\hline 
\end{comment}
\section{spechist}
spechist�--� Accumulates the values of successive spectral frames. \subsection*{Description}


  Accumulates the values of successive spectral frames. 
\subsection*{Syntax}


 wsig \textbf{spechist}
 wsigin
\subsection*{Performance}


 \emph{wsigin}
 -- the input spectra. 


  Accumulates the values of successive spectral frames. At each new frame of \emph{wsigin}
, the accumulations-to-date in each magnitude track are written to the output spectrum. This unit thus provides a running \emph{histogram}
 of spectral distribution. 
\subsection*{Examples}


 


 
\begin{lstlisting}
  wsig2    \emph{specdiff}
         wsig1               ; sense onsets 
  wsig3    \emph{specfilt}
         wsig2, 2            ; absorb slowly 
           \emph{specdisp}
         wsig2, .1           ; & display both spectra 
           \emph{specdisp}
         wsig3, .1
        
\end{lstlisting}


 
\subsection*{See Also}


 \emph{specaddm}
, \emph{specdiff}
, \emph{specfilt}
, \emph{specscal}

%\hline 


\begin{comment}
\begin{tabular}{lcr}
Previous &Home &Next \\
specfilt &Up &specptrk

\end{tabular}


\end{document}
\end{comment}
