\begin{comment}
\documentclass[10pt]{article}
\usepackage{fullpage, graphicx, url}
\setlength{\parskip}{1ex}
\setlength{\parindent}{0ex}
\title{atonek}
\begin{document}


\begin{tabular}{ccc}
The Alternative Csound Reference Manual & & \\
Previous & &Next

\end{tabular}

%\hline 
\end{comment}
\section{atonek}
atonek�--� A notch filter whose transfer functions are the complements of the tone opcode. \subsection*{Description}


  A notch filter whose transfer functions are the complements of the tone opcode. 
\subsection*{Syntax}


 kr \textbf{atonek}
 ksig, khp [, iskip]
\subsection*{Initialization}


 \emph{iskip}
 (optional, default=0) -- initial disposition of internal data space. Since filtering incorporates a feedback loop of previous output, the initial status of the storage space used is significant. A zero value will clear the space; a non-zero value will allow previous information to remain. The default value is 0. 
\subsection*{Performance}


 \emph{kr}
 -- the output signal at control-rate. 


 \emph{ksig}
 -- the input signal at control-rate. 


 \emph{khp}
 -- the response curve's half-power point, in Hertz. Half power is defined as peak power / root 2. 


 \emph{atonek}
 is a filter whose transfer functions is the complement of \emph{tonek}
. \emph{atonek}
 is thus a form of high-pass filter whose transfer functions represent the ``filtered out'' aspects of their complements. However, power scaling is not normalized in \emph{atonek}
 but remains the true complement of the corresponding unit. 
\subsection*{See Also}


 \emph{areson}
, \emph{aresonk}
, \emph{atone}
, \emph{port}
, \emph{portk}
, \emph{reson}
, \emph{resonk}
, \emph{tone}
, \emph{tonek}

%\hline 


\begin{comment}
\begin{tabular}{lcr}
Previous &Home &Next \\
atone &Up &atonex

\end{tabular}


\end{document}
\end{comment}
