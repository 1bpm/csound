\begin{comment}
\documentclass[10pt]{article}
\usepackage{fullpage, graphicx, url}
\setlength{\parskip}{1ex}
\setlength{\parindent}{0ex}
\title{s Statement}
\begin{document}


\begin{tabular}{ccc}
The Alternative Csound Reference Manual & & \\
Previous & &Next

\end{tabular}

%\hline 
\end{comment}
\section{s Statement}
s�--� Marks the end of a section. \subsection*{Description}


  The \emph{s statement}
 marks the end of a section. 
\subsection*{Syntax}


 \textbf{s}
 anything
\subsection*{Initialization}


  All p-fields are ignored. 
\subsection*{Performance}


  Sorting of the \emph{i statement}
, \emph{f statement}
 and \emph{a statement}
 by action time is done section by section. 


  Time warping for the \emph{t statement}
 is done section by section. 


  All action times within a section are relative to its beginning. A section statement establishes a new relative time of 0, but has no other reinitializing effects (e.g. stored function tables are preserved across section boundaries). 


  A section is considered complete when all action times and finite durations have been satisfied (i.e., the ``length'' of a section is determined by the last occurring action or turn-off). A section can be extended by the use of an \emph{f0 statement}
. 


  A section ending automatically invokes a Purge of inactive instrument and data spaces. 


 


\begin{tabular}{cc}
\textbf{Note}
 \\
� &

 


 
\begin{itemize}
\item 

  Since score statements are processed section by section, the amount of memory required depends on the maximum number of score statements in a section. Memory allocation is dynamic, and the user will be informed as extra memory blocks are requested during score processing. 

\item 

  For the end of the final section of a score, the \emph{s statement}
 is optional; the \emph{e statement}
 may be used instead. 


\end{itemize}


\end{tabular}

%\hline 
\end{comment}


\begin{comment}
\begin{tabular}{lcr}
Previous &Home &Next \\
r Statement (Repeat Statement) &Up &t Statement (Tempo Statement)

\end{tabular}


\end{document}
\end{comment}
