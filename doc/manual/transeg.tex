\begin{comment}
\documentclass[10pt]{article}
\usepackage{fullpage, graphicx, url}
\setlength{\parskip}{1ex}
\setlength{\parindent}{0ex}
\title{transeg}
\begin{document}


\begin{tabular}{ccc}
The Alternative Csound Reference Manual & & \\
Previous & &Next

\end{tabular}

%\hline 
\end{comment}
\section{transeg}
transeg�--� Constructs a user-definable envelope. \subsection*{Description}


  Constructs a user-definable envelope. 
\subsection*{Syntax}


 ar \textbf{transeg}
 ia, idur, itype, ib [, idur2] [, itype] [, ic] ...


 kr \textbf{transeg}
 ia, idur, itype, ib [, idur2] [, itype] [, ic] ...
\subsection*{Initialization}


 \emph{ia}
 -- starting value. 


 \emph{ib, ic,}
 etc. -- value after \emph{idur}
 seconds. 


 \emph{idur, idur2,}
 etc. -- duration in seconds of segment 


 \emph{itype, itype2,}
 etc. -- if 0, a straight line is produced. If non-zero, then \emph{transeg}
 creates the following curve, for \emph{n}
 steps: 


 ibeg�+�(ivalue�-�ibeg)�*�(1�-�exp(�i*itype/(n-1)�))�/�(1�-�exp(itype))\\ 
 ������
\subsection*{Performance}


  If \emph{itype}
 $>$ 0, there is a slowly rising, fast decaying (convex) curve, while if \emph{itype}
 $<$ 0, the curve is fast rising, slowly decaying (concave). See also \emph{GEN16}
. 
\subsection*{Credits}


 


 


\begin{tabular}{cccc}
Author: John ffitch &University of Bath, Codemist. Ltd. &Bath, UK &October 2000

\end{tabular}



 


 New in Csound version 4.09


 Thanks goes to Matt Gerassimoff for pointing out the correct command syntax.
%\hline 


\begin{comment}
\begin{tabular}{lcr}
Previous &Home &Next \\
tonex &Up &trigger

\end{tabular}


\end{document}
\end{comment}
