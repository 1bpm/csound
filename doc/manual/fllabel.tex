\begin{comment}
\documentclass[10pt]{article}
\usepackage{fullpage, graphicx, url}
\setlength{\parskip}{1ex}
\setlength{\parindent}{0ex}
\title{FLlabel}
\begin{document}


\begin{tabular}{ccc}
The Alternative Csound Reference Manual & & \\
Previous & &Next

\end{tabular}

%\hline 
\end{comment}
\section{FLlabel}
FLlabel�--� A FLTK opcode that modifies the appearance of a text label. \subsection*{Description}


  Modifies a set of parameters related to the text label appearence of a widget (i.e. size, font, alignment and color of corresponding text). 
\subsection*{Syntax}


 \textbf{FLlabel}
 isize, ifont, ialign, ired, igreen, iblue
\subsection*{Initialization}


 \emph{isize}
 -- size of the font of the target widget. Normal values are in the order of 15. Greater numbers enlarge font size, while smaller numbers reduce it. 


 \emph{ifont}
 -- sets the the font type of the label of a widget. 


  Legal values for ifont argument are: 


 
\begin{itemize}
\item 

 1 - Helvetica (same as Arial under Windows)

\item 

 2 - Helvetica Bold

\item 

 3 - Helvetica Italic

\item 

 4 - Helvetica Bold Italic

\item 

 5 - Courier

\item 

 6 - Courier Bold

\item 

 7 - Courier Italic

\item 

 8 - Courier Bold Italic

\item 

 9 - Times

\item 

 10 - Times Bold

\item 

 11 - Times Italic

\item 

 12 - Times Bold Italic

\item 

 13 - Symbol

\item 

 14 - Screen

\item 

 15 - Screen Bold

\item 

 16 - Dingbats


\end{itemize}


 \emph{ialign}
 -- sets the alignment of the label text of the widget. 


  Legal values for ialign argument are: 


 
\begin{itemize}
\item 

 1 - align center

\item 

 2 - align top

\item 

 3 - align bottom

\item 

 4 - align left

\item 

 5 - align right

\item 

 6 - align top-left

\item 

 7 - align top-right

\item 

 8 - align bottom-left

\item 

 9 - align bottom-right


\end{itemize}


 \emph{ired}
 -- The red color of the target widget. The range for each RGB component is 0-255 


 \emph{igreen}
 -- The green color of the target widget. The range for each RGB component is 0-255 


 \emph{iblue}
 -- The blue color of the target widget. The range for each RGB component is 0-255 
\subsection*{Performance}


 \emph{FLlabel}
 modifies a set of parameters related to the text label appearance of a widget, i.e. size, font, alignment and color of corresponding text. This opcode affects (almost) all widgets defined next its location. A user can put several instances of \emph{FLlabel}
 in front of each widget he intends to modify. However, to modify a particular widget, it is better to use the opcode belonging to the second type (i.e. those containing the \emph{ihandle}
 argument). 


  The influence of \emph{FLlabel}
 on the next widget can be turned off by using -1 as its only argument. \emph{FLlabel}
 is designed to modify text attributes of a group of related widgets. 
\subsection*{See Also}


 \emph{FLcolor}
, \emph{FLcolor2}
, \emph{FLhide}
, \emph{FLsetAlign}
, \emph{FLsetBox}
, \emph{FLsetColor}
, \emph{FLsetColor2}
, \emph{FLsetFont}
, \emph{FLsetPosition}
, \emph{FLsetSize}
, \emph{FLsetText}
, \emph{FLsetTextColor}
, \emph{FLsetTextSize}
, \emph{FLsetTextType}
, \emph{FLsetVal\_i}
, \emph{FLsetVal}
, \emph{FLshow}

\subsection*{Credits}


 Author: Gabriel Maldonado


 New in version 4.22
%\hline 


\begin{comment}
\begin{tabular}{lcr}
Previous &Home &Next \\
FLknob &Up &FLloadsnap

\end{tabular}


\end{document}
\end{comment}
