\begin{comment}
\documentclass[10pt]{article}
\usepackage{fullpage, graphicx, url}
\setlength{\parskip}{1ex}
\setlength{\parindent}{0ex}
\title{printks}
\begin{document}


\begin{tabular}{ccc}
The Alternative Csound Reference Manual & & \\
Previous & &Next

\end{tabular}

%\hline 
\end{comment}
\section{printks}
printks�--� Prints at k-rate using a printf() style syntax. \subsection*{Description}


  Prints at k-rate using a printf() style syntax. 
\subsection*{Syntax}


 \textbf{printks}
 ``string'', itime [, kval1] [, kval2] [...]
\subsection*{Initialization}


 \emph{``string''}
 -- the text string to be printed. Can be up to 8192 characters and must be in double quotes. 


 \emph{itime}
 -- time in seconds between printings. 
\subsection*{Performance}


 \emph{kval1, kval2, ...}
 (optional) -- The k-rate values to be printed. These are specified in \emph{``string''}
 with the standard C value specifier (\%f, \%d, etc.) in the order given. 


  In Csound version 4.23, you can use as many \emph{kval}
 variables as you like. In versions prior to 4.23, you must specify 4 and only 4 kvals (using 0 for unused kvals). 


 \emph{printks}
 prints numbers and text which can be i-time or k-rate values. \emph{printks}
 is highly flexible, and if used together with cursor positioning codes, could be used to write specific values to locations in the screen as the Csound processing proceeds. 


  A special mode of operation allows this \emph{printks}
 to convert \emph{kval1}
 input parameter into a 0 to 255 value and to use it as the first character to be printed. This enables a Csound program to send arbitrary characters to the console. To achieve this, make the first character of the string a \# and then, if desired continue with normal text and format specifiers. 


  This opcode can be run on every k-cycle it is run in the instrument. To every accomplish this, set \emph{itime}
 to 0. 


  When \emph{itime}
 is not 0, the opcode print on the first k-cycle it is called, and subsequently when every \emph{itime}
 period has elapsed. The time cycles start from the time the opcode is initialized - typically the initialization of the instrument. 
\subsection*{Print Output Formatting}


  All standard C language printf() control characters may be used. For example, if \emph{kval1}
 = 153.26789 then some common formatting options are: 


 
\begin{enumerate}
\item 

 \%f prints with full precision: 153.26789

\item 

 \%5.2f prints: 153.26

\item 

 \%d prints integers-only: 153

\item 

 \%c treats \emph{kval1}
 as an ascii character code. 


\end{enumerate}


  In addition to all the printf() codes, printks supports these useful character codes: 


 


\begin{tabular}{|c|c|c|c|c|c|c|c|}
%\hline 
printks CodeCharacter Code & & & & & & & \\
 %\hline 
$\backslash$$\backslash$r, $\backslash$$\backslash$R, \%r, or \%Rreturn character ($\backslash$r) &$\backslash$$\backslash$n, $\backslash$$\backslash$N, \%n, \%Nnewline character ($\backslash$n) &$\backslash$$\backslash$t, $\backslash$$\backslash$T, \%t, or \%Ttab character ($\backslash$t) &\%!semicolon character (;) This was needed because a ``;'' is interpreted as an comment. &\^{}escape character (0x1B) &\^{} \^{}caret character (\^{}) &\&\#732;ESC[ (escape+[ is the escape sequence for ANSI consoles) &\&\#732;\&\#732;tilde (\&\#732;) \\
 %\hline 

\end{tabular}



 


  For more information about printf() formatting, consult any C language documentation. 


 


\begin{tabular}{cc}
\textbf{Note}
 \\
� &

  Prior to version 4.23, only the \%f format code was supported. 


\end{tabular}

\subsection*{Examples}


  Here is an example of the printks opcode. It uses the files \emph{printks.orc}
 and \emph{printks.sco}
. 


 \textbf{Example 1. Example of the printks opcode.}

\begin{lstlisting}
/* printks.orc */
; Initialize the global variables.
sr = 44100
kr = 44100
ksmps = 1
nchnls = 1

; Instrument #1.
instr 1
  ; Change a value linearly from 0 to 100,
  ; over the period defined by p3.
  kup line 0, p3, 100
  ; Change a value linearly from 30 to 10, 
  ; over the period defined by p3.
  kdown line 30, p3, 10

  ; Print the value of kup and kdown, once per second.
  printks "kup = %f, kdown = %f\\n", 1, kup, kdown
endin
/* printks.orc */
        
\end{lstlisting}
\begin{lstlisting}
/* printks.sco */
; Play Instrument #1 for 5 seconds.
i 1 0 5
e
/* printks.sco */
        
\end{lstlisting}
 Its output should include lines like this: \begin{lstlisting}
kup = 0.000000, kdown = 30.000000
kup = 20.010843, kdown = 25.962524
kup = 40.029991, kdown = 21.925049
kup = 60.049141, kdown = 17.887573
kup = 79.933266, kdown = 13.872493
      
\end{lstlisting}
\subsection*{See Also}


 \emph{printk2}
 and \emph{printk}

\subsection*{Credits}


 


 


\begin{tabular}{ccc}
Author: Robin Whittle &Australia &May 1997

\end{tabular}



 


 Example written by Kevin Conder.


 Thanks goes to Luis Jure for pointing out a mistake with the \emph{itime}
 parameter.


 Thanks to Matt Ingalls, updated the documentation for version 4.23.
%\hline 


\begin{comment}
\begin{tabular}{lcr}
Previous &Home &Next \\
printk2 &Up &prints

\end{tabular}


\end{document}
\end{comment}
