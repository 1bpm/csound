\begin{comment}
\documentclass[10pt]{article}
\usepackage{fullpage, graphicx, url}
\setlength{\parskip}{1ex}
\setlength{\parindent}{0ex}
\title{tablexseg}
\begin{document}


\begin{tabular}{ccc}
The Alternative Csound Reference Manual & & \\
Previous & &Next

\end{tabular}

%\hline 
\end{comment}
\section{tablexseg}
tablexseg�--� Creates a new function table by making exponential segments between values in stored function tables. \subsection*{Description}


 \emph{tablexseg}
 is like \emph{expseg}
 but interpolate between values in a stored function tables. The result is a new function table passed internally to any following \emph{vpvoc}
 which occurs before a subsequent \emph{tablexseg}
 (much like \emph{lpread}
/\emph{lpreson}
 pairs work). The uses of these are described below under \emph{vpvoc}
. 
\subsection*{Syntax}


 \textbf{tablexseg}
 ifn1, idur1, ifn2 [, idur2] [, ifn3] [...]
\subsection*{Initialization}


 \emph{ifn1}
, \emph{ifn2}
, \emph{ifn3}
, etc. -- function table numbers. \emph{ifn1}
, \emph{ifn2}
, and so on, must be the same size. 


 \emph{idur1}
, \emph{idur2}
, etc. -- durations during which interpolation from one table to the next will take place. 
\subsection*{See Also}


 \emph{pvbufread}
, \emph{pvcross}
, \emph{pvinterp}
, \emph{pvread}
, \emph{tableseg}

\subsection*{Credits}


 


 


\begin{tabular}{ccc}
Author: Richard Karpen &Seattle, WA USA &1997

\end{tabular}



 
%\hline 


\begin{comment}
\begin{tabular}{lcr}
Previous &Home &Next \\
tablexkt &Up &tambourine

\end{tabular}


\end{document}
\end{comment}
